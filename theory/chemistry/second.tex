%
%  set up at April 4th, 2009
%
%
%
\chapter{Second Quantization Methods}
%
%
%
The second Quantization methods is one of the powerful tools used in
relativistic quantum theory and quantum field theory, where the
physical problems that actually involves creation or destruction of
particles, photons, etc. In a majority of the applications of the
second quantization techniques to quantum-chemical problems, no
electrons or other particles are created or destroyed. Thus, this
method usually serve merely as a convenient and operationally useful
approach. However, this methods deeply simplifies the discussion about 
the system involving many identical interacting particles; e.g., the electrons
in the molecules.



%%%%%%%%%%%%%%%%%%%%%%%%%%%%%%%%%%%%%%%%%%%%%%%%%%%%
\section{General Discussion}
%
%
%
Now let's define some operator which is called ``creation operator'':
$\hat{r}^{+}$. This operator is used to generate an electron which is on orbital
of $\varphi_{r}$. Here we note that compared with traditional notion which
requires the label of electron, the creation operator does not need it anymore.
Hence we can generally express the Slater determinant of $\Phi$ as:
\begin{equation}
 \label{second_quantization_general_discussion:1}
\Phi = \frac{1}{\sqrt[2]{n!}} \left | \begin{array}{cccc}
  \varphi_{1}(1) & \varphi_{2}(1) & \cdots & \varphi_{n}(1) \\
  \varphi_{1}(2) & \varphi_{2}(2) & \cdots & \varphi_{n}(2) \\
  \cdots & \cdots & \cdots & \cdots                        \\
  \varphi_{1}(n) & \varphi_{2}(n) & \cdots & \varphi_{n}(n)
\end{array} \right | =
\hat{n}^{+}\cdots\hat{3}^{+}\hat{2}^{+}\hat{1}^{+}\ket{vac}
\end{equation}
Again we follow the convention that to use a, b, c etc. to designate the virtual
orbitals, the i, j, k etc. to refer to the occupied orbitals; and p,
q, r etc. to specify the general orbitals. What's more, we note that
$\hat{2}^{+}\hat{1}^{+}\ket{vac}$ denotes that firstly to generate an electron
in the $\varphi_{1}$, then to generate another electron in $\varphi_{2}$. 

Compared with the traditional Slater determinant expression in
(\ref{second_quantization_general_discussion:1}), the most important thing is
that the Slater determinant satisfies the Pauli principle that if we exchange
two electrons in the wave function, then the wave function should change its
sign. Hence how can we express it in the second quantization?

For the creation operator, it generally follows the rule that:
\begin{equation}
  \label{second_quantization_general_discussion:2}
\hat{r}^{+}\hat{s}^{+} + \hat{s}^{+}\hat{r}^{+} = [\hat{r}^{+}\hat{s}^{+}]_{+}
= 0
\end{equation}
We note that it gives the Pauli principle. Let's generally consider some wave
function in second quantization: $\Phi
=\hat{r}^{+}\cdots\hat{s}^{+}\hat{t}^{+}\hat{u}^{+}\ket{vac}$, 

%%%%%%%%%%%%%%%%%%%%%%%%%%%%%%%%%%%%%%%%%%%%%%%%%%%%




%%% Local Variables: 
%%% mode: latex
%%% TeX-master: "../../main"
%%% End: 
