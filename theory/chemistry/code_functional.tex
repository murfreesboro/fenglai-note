%%
%% code_functional.tex
%% 
%% Made by liufenglai
%% Login   <liufenglai@phoenix>
%% 
%% Started on  Mon Nov 23 12:04:05 2009 liufenglai
%%

%%%%%%%%%%%%%%%%%%%%%%%%%%%%%%%%%%%%%%%%%%%%%%%%%%%%%%%%%%%%%%%%%%%%%%%%%%%%%%
\chapter{How to Code The Exchange-Correlation Energy}
%
%
%
%
%

%%%%%%%%%%%%%%%%%%%%%%%%%%%%%%%%%%%%%%%%%%%%%%%%%%%%%%%%%%%%%%%%%%%%%%%%%%%%%%
\section{Introduction}
\label{sec:XC_functional}
%
%
%
%
Now in this chapter, we are going to give a discussion that how to
code the conceptual XC functional so that we can calculate the energy,
first and second derivatives etc. within DFT framework. For
simplicity, let's restricted the discussion within the ground state
DFT calculation first.

In the ground state DFT calculation, compared with the traditional
wave function method; there are two things we needed to get the
information of the system we study:
\begin{align}
 \label{eq:XC_functional.1}
\text{wave function} &\Leftrightarrow \text{electron density} \nonumber
\\
\text{operator on the wave function} &\Leftrightarrow \text{functional
  on the density}
\end{align}
The density here we are talking about, is the electron density from
the non-interacting system which is in one to one mapping relationship
with the real system. Through the imaginary Kohn-Sham orbitals (see
\ref{DFTI:2} for more information), we can construct the electron
density for the real system.

On the other hand, how can we derive the electron density? We have
Kohn-sham equation, which is similar with Hatree-Fock equation (more
details about the KS equation please see \ref{DFTI:4}), through that
equation the trial density firstly generated by guess will be renewed,
then the newer electron density will be brought into the whole
function again until the self-consistent condition is met. Finally, by
the result electron density we can evaluate all the system information
we need.

In the Kohn-Sham equation in (\ref{DFTIeq:12}), we have an item which
contains the exchange and correlation energy, its form has been
detailed discussion in the above content of this chapter. This is the
energy functional which operating on the density, and from the above
content we know that it's explicit form should be derived. Then if
both of the explicit form of XC functional and the trial density is
known. then we can finally get the result density through KS equation.

Finally, there's a question left that how to construct the trial
density, or more properly to say, the trial Kohn-Sham orbitals? In
principle, we can find enormous ways to do this, but in practice we
construct the trial density from the traditional AO basis set
functions, in other words;  the GTO or STO functions:
\begin{align}
  \label{eq:XC_functional.2}
\phi_{i} &= \sum_{j}d_{ij}\chi_{j} \Rightarrow \nonumber \\
\varphi_{i} &= \sum_{j}^{n}c_{ij}\phi_{i} \Rightarrow \nonumber \\
\rho_{trial} &= \sum_{i}^{occ}\varphi_{i}^{*}\varphi_{j} \nonumber \\
&=
\sum_{i}^{occ}\sum_{j}^{n}\sum_{k}^{n}c_{ij}^{*}c_{ik}\phi_{j}^{*}\phi_{k}
\nonumber \\
&= \sum_{j}^{n}\sum_{k}^{n}P_{jk}\phi_{j}^{*}\phi_{k}
\end{align}
Here, the $\chi$ is the basic functions to form the basis set functions
$\phi$, the $\chi$ can be GTO or STO; but mostly we use GTO. Here in
this step, the basis set functions are same between the wave function
method and the new DFT method. Then based on the basis set functions
of $\phi$, we can construct the trial Kohn-Sham orbitals of $\varphi$,
so finally we can get the trial density. Here above the $P_{jk}$ is
the density matrix for the basis function of $\phi_{j}$ and
$\phi_{k}$. 

furthermore, if we compare the HF equation and the KS equation, we can
find out that the only difference btween them is that there's a new
item of $E_{XC}$ in KS equation (here the point is aiming at the
hybrid functionals because it contains the HF exchange part, for pure
functional method such as BLYP; there's no HF exchange part), and all
the other terms are kept to be same. Hence let's proceed to discuss
the formation of $E_{XC}$ in details.


%%%%%%%%%%%%%%%%%%%%%%%%%%%%%%%%%%%%%%%%%%%%%%%%%%%%%%%%%%%%%%%%%%%%%%%%%%%%%%
\section{Formation of $E_{XC}$ term for ground states}
\label{sec:formation_XC_functional_ground}
%
%
%
%
%
Now let's start from some general form of GGA functional:
\begin{equation}
\label{eq:XC_functional.3}
  E_{XC} = \int  f \Big(\rho_{\alpha}, \rho_{\beta},
  \gamma_{\alpha\alpha}, \gamma_{\beta\beta},
 \gamma_{\alpha\beta}\Big) d^{3}r
\end{equation}
Where we have:
\begin{align}
  \gamma_{\alpha\alpha} &= |\nabla\rho_{\alpha}|^{2} \nonumber \\
  \gamma_{\beta\beta}  &= |\nabla\rho_{\beta}|^{2} \nonumber \\
  \gamma_{\alpha\beta} &=\nabla\rho_{\alpha} \cdot \nabla\rho_{\beta}
\end{align}
It's potential has been derived in (\ref{eq:functional:36}), it is:
\begin{equation}
  \label{eq:XC_functional.4}
  V_{\alpha}^{XC} = \frac{\partial f}{\partial \rho_{\alpha}} - 2\nabla\cdot \left(
    \frac{\partial f} {\partial \gamma_{\alpha\alpha}}
   \nabla\rho_{\alpha}\right) - \nabla\cdot \left( \frac{\partial f}
    {\partial \gamma_{\alpha\beta}} \nabla\rho_{\beta} \right) 
\end{equation}
Their form has been gotten in previous papers\cite{CPL_1992_6_557,johnson:5612}.

Now let's discuss the formation of electron density. Generally we
consider the unrestricted case, where the alpha electron and beta
electron do not need to share the same spatial orbitals (the
restricted type can be seen as the special case of unrestricted
method). Hence we have:
\begin{align}
  \label{eq:XC_functional.5}
  \rho_{\alpha} &=
  \sum_{j}^{n}\sum_{k}^{n}P^{\alpha}_{jk}\phi_{j}^{*}\phi_{k}
  \nonumber
  \\
  \rho_{\beta} &=
  \sum_{j}^{n}\sum_{k}^{n}P^{\beta}_{jk}\phi_{j}^{*}\phi_{k}
\end{align}

Then we can get the XC energy matrix element based on the basis set
functions $\phi_{\mu}$ and $\phi_{\nu}$:
\begin{align}
  \label{eq:XC_functional.6}
  F^{\alpha}_{\mu\nu} &=
 \bra{\phi_{\mu}}\hat{V}^{\alpha}_{XC}\ket{\phi_{\nu}} \nonumber \\
  &= \int\phi^{*}_{\mu}(r)\hat{V}^{\alpha}_{XC}\phi_{\nu}(r) d^{3}r
  \nonumber \\
  &= \int\hat{V}^{\alpha}_{XC}\phi^{*}_{\mu}(r)\phi_{\nu}(r) d^{3}r
\end{align}
Here the $\hat{V}^{\alpha}_{XC}$ is actually some function expression based on
the electron density $\rho^{\alpha}$, hence we can move it ahead of
the basis functions. For the beta electron we can get the similar result.

Interestingly, we can use another way to get the result in
(\ref{eq:XC_functional.6}):
\begin{align}
  \label{eq:XC_functional.9}
E^{\alpha}_{XC} &= \sum_{\mu\nu}P^{\alpha}_{\mu\nu}F^{\alpha}_{\mu\nu}
\Rightarrow \nonumber \\
F^{\alpha}_{\mu\nu} &= \frac{\partial E^{\alpha}_{XC}}{\partial P^{\alpha}_{\mu\nu}}
\nonumber \\
&=
\int \frac{\partial E^{\alpha}_{XC}}{\partial
  \rho^{\alpha}}\frac{\partial \rho^{\alpha}}{\partial P^{\alpha}_{\mu\nu}} d^{3}r
\nonumber \\
&= \int \hat{V}^{\alpha}_{XC} \phi^{*}_{\mu}(r)\phi_{\nu}(r) d^{3}r
\end{align}
Here we have used the relation defined in the (\ref{eq:XC_functional.5}).

In evaluating the (\ref{eq:XC_functional.6}), we will meet some
difficulty in calculating the gradient part in
(\ref{eq:XC_functional.4}); however, we have some simple method to
avoid it. Let's take a gradient part as example:
\begin{equation}
  \label{eq:XC_functional.7}
  \begin{split}
    & \int \phi^{*}_{\mu}(r) \Bigg\{2\nabla\cdot \left( \frac{\partial
        f} {\partial \gamma_{\alpha\alpha}}
      \nabla\rho_{\alpha}\right)\Bigg\}\phi_{\nu}(r) d^{3}r  \\
    &=\int \Bigg\{2\nabla\cdot \left( \frac{\partial f} {\partial
        \gamma_{\alpha\alpha}}
      \nabla\rho_{\alpha}\right)\Bigg\}\phi^{*}_{\mu}(r) \phi_{\nu}(r)
    d^{3}r  \\
    &= 2\int \phi^{*}_{\mu}(r) \phi_{\nu}(r)\Bigg\{ \nabla \cdot\left(
      \frac{\partial f} {\partial \gamma_{\alpha\alpha}}
      \nabla\rho_{\alpha}\right) \Bigg\} \\
    &= 2\phi^{*}_{\mu}(r) \phi_{\nu}(r)\left. \left( \frac{\partial f}
        {\partial \gamma_{\alpha\alpha}}
        \nabla\rho_{\alpha}\right)\right|^{+\infty}_{-\infty} - 2\int
    \left( \frac{\partial f} {\partial \gamma_{\alpha\alpha}}
      \nabla\rho_{\alpha}\right)\Big\{\nabla\cdot(\phi^{*}_{\mu}(r) \phi_{\nu}(r))\Big\} \\
    &= - 2\int \left( \frac{\partial f} {\partial
        \gamma_{\alpha\alpha}}
      \nabla\rho_{\alpha}\right)\Big\{\nabla\cdot(\phi^{*}_{\mu}(r)
    \phi_{\nu}(r))\Big\}
  \end{split}
\end{equation}
Here in this derivation, we have used the integration by parts, and
since the wave function should be zero at the $+\infty$ and $-\infty$,
hence it's clear the first term is going zero. Then we get the final
result. 

In terms of the transformation in (\ref{eq:XC_functional.7}), we can
finally reach the result:
\begin{equation}
  \label{eq:XC_functional.8}
  \begin{split}
      F^{\alpha}_{\mu\nu} &= \int
  \phi^{*}_{\mu}(r)\hat{V}^{\alpha}_{XC}\phi_{\nu}(r) d^{3}r  \\
  &= \int \frac{\partial f}{\partial
      \rho_{\alpha}}\phi^{*}_{\mu}(r) \phi_{\nu}(r)d^{3}r + \\
  &\int\Bigg\{2\left( \frac{\partial f} {\partial \gamma_{\alpha\alpha}}
    \nabla\rho_{\alpha}\right) + \left( \frac{\partial f} {\partial
      \gamma_{\alpha\beta}} \nabla\rho_{\beta} \right)
  \Bigg\}\nabla\cdot(\phi^{*}_{\mu}(r) \phi_{\nu}(r))d^{3}r
  \end{split}
\end{equation}
This is what we have got for calculating the ground state Kohn-Sham
equation by the GGA.

%%%%%%%%%%%%%%%%%%%%%%%%%%%%%%%%%%%%%%%%%%%%%%%%%%%%%%%%%%%%%%%%%%%%%%%%%%%%%%
\section{Formation of $E_{XC}$ for TDDFT}
%
%
%
%


%%%%%%%%%%%%%%%%%%%%%%%%%%%%%%%%%%%%%%%%%%%%%%%%%%%%%%%%%%%%%%%%%%%%%%%%%%%%%%



