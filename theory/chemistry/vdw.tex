%
% set up on June 2013
%
\chapter{Van der Waals Interaction in Density Functional Theory}
%
%
The dispersion or Van der Waals interaction is classical in Chemistry. 
Given two widely separated systems, an instantaneous dipole moment on 
one system induces a dipole moment on the other. The attraction between 
these moments results in an interaction is expressed as following:
\begin{equation}
 \label{VDW_DFT_eq:1}
 E_{VDW} = -\frac{C_{6}}{R^{6}}
\end{equation}
where $C_{6}$ is some constant whose value depends on the systems
involved, and $R$ is the distance between the two separated systems.

%
% let's state that in density functional theory 
% it's hard to include VDW
%

\section{XDM Model}
%
%

\subsection{Oringinal Idea for XDM Model}
%
%
%
The XDM model is promoted by A. D. Becke and his colleagues. It's 
trying to provide a clear model to explain the Van der Waals interaction
in Density functional theory. Basically, it trys to give an answer to
the following question, that How do ``instantaneous'' dipole moments arise 
in systems which may have a zero permanent dipole moment? For example, 
in the noble gas system? XDM model suggests that the ``instantaneous'' 
dipole moments actually arise from the assymetrical exchange hole.

Consider an electron with $\sigma$ spin in an molecular system. As it moves 
through the system it is accompanied by an exchange hole whose shape depends 
on the electron’s instantaneous position $r_{1}$. The hole is given by the
expression(This is constructed based on the Slater determinant. 
see \ref{DHF_in_density_matrices} for more information):
\begin{equation}
\label{XDM_exchange_hole}
 h_{X\sigma}(r_{1},r_{2}) = -\frac{
\sum_{ij}\varphi_{i\sigma}(r_{1})\varphi_{i\sigma}(r_{2})\varphi_{j\sigma}(r_{1})
\varphi_{j\sigma}(r_{2})}{\rho_{\sigma}(r_{1})}
\end{equation}
where $i,j$ are both occupied electron orbitals. Here $r_{2}$ defines the shape 
of the hole and $r_{1}$ is called the ``reference'' point. The spin exchange 
energy $E_{X}$ is related to the exchange hole by:
\begin{equation}
 \label{XDM_exchange_energy}
E_{X\sigma} = \frac{1}{2}\int dr_{1}dr_{2} \rho_{\sigma}(r_{1}) 
\frac{h_{X\sigma}(r_{1},r_{2})}{r_{12}}
\end{equation}

The exchange-hole definition enables us to visualize the effects of self-interaction 
correction and exchange. When an electron is at $r_{1}$, the hole measures the depletion 
of probability with respect to the total electron density of finding another same-spin 
electron at $r_{2}$. The probability of finding
another same-spin electron at $r_{1} = r_{2}$ is completely extinguished because of the 
Pauli principle. In general, the depletion of the electron density because of the Fermi hole
is same with the integration of electron density itself:
\begin{equation}
 \label{XDM_eq:1}
\int dr_{2} h_{X\sigma}(r_{1},r_{2}) = -\int dr_{1} \rho(r_{1}) = -1
\end{equation}
Therefore, the electron density plus its Fermi hole has zero charge overall.

However, the hole is not, in general, spherically symmetric around $r_{1}$. Only in a uniform 
electron gas it has spherical symmetry. Thus the electron density plus its Fermi hole,
although it's zero charge overall, since it's not symmetrical therefore it will produce
a dipole moment for the reference point of $r_{1}$. We note that such induced dipole moment
is not affecting total energy. Since in equation \ref{XDM_exchange_hole}, the hole only
senses the spherical average around $r_{1}$ (see \ref{hole_function_pair_distribution} 
for more details that why the hole function is symmetrical); therefore such assymetrical
deformation of the hole does not alter the exchange enegry.

Based on the above analysis, we can define some ``operator'' to evaluate the dipole 
moment triggered by the aspherical exchange hole:
\begin{equation}
 \label{XDM_model}
d_{X\sigma}(\bm{r_{1}}) = 
\frac{
\sum_{ij}\varphi_{i\sigma}(r_{1})\varphi_{j\sigma}(r_{1})}{\rho_{\sigma}(r_{1})}
\left[ \int dr_{2} \varphi_{i\sigma}(r_{2})
\varphi_{j\sigma}(r_{2}) \bm{r_{2}} \right] - \bm{r_{1}}
\end{equation}
This term is used to describe the difference between position of $\bm{r_{1}}$ and
the average exchange hole at $\bm{r_{1}}$. 

Based on this operator form, the induced dipole could be expressed as:
\begin{equation}
\label{XDM_eq:0}
 \langle d_{X\sigma}\rangle = \int \rho_{\sigma}(r_{1})d_{X\sigma}(\bm{r_{1}}) dr_{1}
\end{equation}
This is the hypothesis for the whole XDM model.

Now let's derive the induced dipole for multiple systems from such assumption. 
Suggest that we have two  separated systems A and B and they are far away from 
each other ($R$ is large enough)
comparing with their size. If the system B is approaching to A then there will be
an induced electric field generated because of the assymetrical deformation of the 
exchange hole. This electric field could be expressed as\footnote{This expression 
could be got from the classic induced dipole moment expression}:
\begin{equation}
 \label{XDM_eq:2}
\bm{E} = \frac{(3\bm{d}_{X\sigma}\cdot \bm{R})\bm{R}}{R^{5}} - 
\frac{\bm{d}_{X\sigma}}{R^{3}}
\end{equation}
Here $\bm{d}_{X\sigma}$ is defined in \ref{XDM_model}.

For system B, let's assume that it's Polarizability is $\alpha_{B}$ The corresponding 
induced diploe moment is:
\begin{equation}
\label{XDM_eq:3}
 \bm{d}_{ind} = \alpha_{B} \bm{E}
\end{equation}

Then the dipole energy on the point of $\bm{r_{1}}$could be expressed as:
\begin{equation}
 \label{XDM_eq:4}
V = \frac{\bm{d}_{X\sigma}\cdotp \bm{d}_{ind}}{R^{3}} - 
\frac{3 (\bm{d}_{X\sigma}\cdotp \bm{R})(\bm{d}_{ind}\cdotp \bm{R})}{R^{5}}
\end{equation}

Now we can bring \ref{XDM_eq:2} and \ref{XDM_eq:3} into the \ref{XDM_eq:4} and omit
these terms with $R^{10}$, we can get:
\begin{equation}
 \label{XDM_eq:5}
 V= -\alpha_{B}\frac{\bm{d}_{X\sigma}\cdotp \bm{d}_{X\sigma}}{R^{6}}
-3\alpha_{B}\frac{(\bm{d}_{X\sigma}\cdotp \bm{R})^{2}}{R^{8}}
\end{equation}
 
The average dipole moment energy could be got by integrating over the angles:
\begin{align}
\label{XDM_eq:6}
 V^{avg} &= -\frac{\alpha_{B}}{4\pi}\int^{\pi}_{0} \int^{2\pi}_{0} \sin\theta d\theta d\phi 
\left[ \frac{\bm{d}_{X\sigma}\cdotp \bm{d}_{X\sigma}}{R^{6}}
+3\alpha_{B}\frac{(\bm{d}_{X\sigma}\cdotp \bm{R})^{2}}{R^{8}}\right]   \nonumber \\
&= -2\frac{\alpha_{B}d_{X\sigma}^{2}}{R^{6}}
\end{align}
By integrating over $r_{1}$ (the reference point), we can have:
\begin{equation}
\label{XDM_eq:7}
 U_{\sigma} = \int \rho(r_{1}) V^{avg} dr_{1} = 
-2\frac{\alpha_{B}\langle d_{X\sigma}^{2}\rangle}{R^{6}}
\end{equation}
Where $\langle d_{X\sigma}^{2}\rangle$ is
 \begin{equation}
 \langle d_{X\sigma}^{2}\rangle = \int \rho_{\sigma}(r_{1})d_{X\sigma}(\bm{r_{1}})\cdotp 
d_{X\sigma}(\bm{r_{1}}) dr_{1}
\end{equation}
comparing with \ref{XDM_eq:0}.

So far the dipole moment is only given for one spin state, for spin-resolved state
it has:
\begin{equation}
 \label{XDM_eq:8}
\langle d_{X}^{2} \rangle = \langle d_{X\alpha}^{2}\rangle + \langle d_{X\beta}^{2}\rangle
\end{equation}
therefore the spin-resolved energy is expressed as:
\begin{equation}
 \label{XDM_eq:9}
U = -2\frac{\alpha_{B}\langle d_{X}^{2}\rangle}{R^{6}}
\end{equation}
and the coefficient of $C_{6}$ in \ref{VDW_DFT_eq:1} is given as:
\begin{equation}
 C_{6} = -2\alpha_{B}\langle d_{X}^{2}\rangle
\end{equation}

However, in the original paper Becke etc. found that \ref{XDM_eq:9} yields four times larger result 
compared with experimental data, therefore, they take a heuristic approach to modify \ref{XDM_eq:9}
into:
\begin{equation}
 \label{XDM_eq:10}
U = -\frac{\alpha_{B}\langle d_{X}^{2}\rangle}{2R^{6}}
\end{equation}
so the corresponding $C_{6}$ is changing to:
\begin{equation}
\label{XDM_eq:12}
 C_{6} = -\frac{1}{2}\alpha_{B}\langle d_{X}^{2}\rangle
\end{equation}


Expression \ref{XDM_eq:10} describes the dipole energy for B induced from A, and symmetrically;
we also have dipole energy for A induced from B; which is:
\begin{equation}
 \label{XDM_eq:11}
U_{AB} = -\frac{\alpha_{A}\langle d_{BX}^{2}\rangle}{2R^{6}}
\end{equation}
If A and B are different system, then how can we count \ref{XDM_eq:10} and \ref{XDM_eq:11} 
together?

For this case, Becke etc. defined the $C_{6}$ parameter as:
\begin{equation}
 \frac{2}{C_{6}} = \frac{1}{C_{6,AB}} + \frac{1}{C_{6,BA}}
\end{equation}
By bring \ref{XDM_eq:11} into this expression, we can have the final $C_{6}$ expression:
\begin{equation}
 \label{XDM_eq:13}
C_{6} = \frac{\alpha_{A}\alpha_{B}\langle d_{AX}^{2}\rangle\langle d_{BX}^{2}\rangle}
{\alpha_{A}\langle d_{BX}^{2}\rangle + \alpha_{B}\langle d_{AX}^{2}\rangle}
\end{equation}



 



