%
%
%
%
%
%
\chapter{Perturbation Treatment in Molecular Quantum Mechanics}
%
%
%
%
%%%%%%%%%%%%%%%%%%%%%%%%%%%%%%%%%%%%%%%%%%%%%%%%%%%%%%%%%%%%%%%%%%%%%%%%%%%%%%%%%%%%%%%%%%%%%%%%%
\section{Introduction}
%
% what does the perturbation treatment do in the quantum chemistry
%
%
In quantum chemistry, the perturbation treatment is usually employed
to tackle with the weak interactions in the molecular systems. Such
weak interaction can be in variety kinds, such as the vibration of
atom skeleton, or the magnetic susceptibility and
shielding\cite{stevens:550}; or even the time dependent external
field change etc. In general, such physical quantity are very small
so that it's suitable to be the perturbation operator, and the
perturbation treatment can usually yield good results.

%%%%%%%%%%%%%%%%%%%%%%%%%%%%%%%%%%%%%%%%%%%%%%%%%%%%%%%%%%%%%%%%%%%%%%%%%%%%%%%%%%%%%%%%%%%%%%%%%
\section{The perturbed Hatree-Fock equation in differential form}
%
%  1  start from the Fock equation, the orbital is involved. we
%     do not consider its expansion
%  2  perturbation expansion for the orbital and the orbital energy
%
%
Firstly let's derive the general procedure for the perturbation
treatment in the Hatree-Fock framework\cite{PhysRev.118.167,
Peng_H_W, RevModPhys.32.455}.

As what has been shown in the Hatree-Fock chapter, the general
Hatree-Fock equation in molecular orbital form can be expressed as:
\begin{multline}\label{PTIMQMeq:1}
\hat{H}_{1}(1)\varphi_{i}(1) + \sum_{j}\left[
\left\{\int\varphi^{*}_{j}(2)\varphi_{j}(2)d\tau(2)\frac{1}{r_{12}}\right\}\varphi_{i}(1)
\right] \\
 -
\sum_{j}\left[
\left\{\int\varphi^{*}_{j}(2)\varphi_{i}(2)d\tau(2)\frac{1}{r_{12}}\right\}\varphi_{j}(1)
\right] \\
 = \epsilon_{i}\varphi_{i}(1)
\end{multline}
Here the $j$ goes over all the occupied orbitals. Here for
simplicity we only consider the differential form of Hatree-Fock
equation, the involvement of basis sets are left for the future.

Suggest there's some perturbation in the Fock operator, and it's
assumed to be the one electron operator-where such assumption can
satisfy most of the perturbation cases. In the presence of the
perturbed operator, the molecular orbitals and the its energy levels
will make corresponding change; we can express such changes in
progressive way:
\begin{align}\label{PTIMQMeq:2}
\varphi_{i} &= \varphi^{(0)}_{i} + \lambda\varphi^{(1)}_{i} +
\lambda^{2}\varphi^{(2)}_{i} + \cdots \nonumber \\
\epsilon_{i} &= \epsilon^{(0)}_{i} + \lambda\epsilon^{(1)}_{i} +
\lambda^{2}\epsilon^{(2)}_{i} + \cdots
\end{align}
The new Fock operator can be expressed as:
\begin{equation}\label{PTIMQMeq:3}
\hat{F}^{'} = \hat{F} + \lambda \hat{V}
\end{equation}
Now we go to see how to get the perturbed HF equations.

%%%%%%%%%%%%%%%%%%%%%%%%%%%%%%%%%%%%%%%%%%%%%%%%%%%%%%%%%%%%%%%%%%%%%%%%%%%%%%%%%%%%%%%%%%%%%%%%%
\subsection{First order Hatree Fock perturbed equation}
%
%   to get the first order Fock equation
%
By taking the (\ref{PTIMQMeq:2}) and (\ref{PTIMQMeq:3}) into the
original Hatree-Fock equation (\ref{PTIMQMeq:1}), then by gathering
the terms with same power of $\lambda$, we can get a series of
perturbation equations. The equation scales with $\lambda^{0}$ is:
\begin{equation}\label{PTIMQMeq:4}
\hat{F}\varphi^{(0)}_{i} = \epsilon^{(0)}_{i}\varphi^{(0)}_{i}
\end{equation}
That's the equation for the zero order approximation.

By gathering the terms related to the $\lambda$, we can get the
first order approximation:
\begin{multline}\label{PTIMQMeq:6}
\hat{H}_{1}\varphi^{(1)}_{i} + \hat{V}\varphi^{(0)}_{i} \\
+\sum_{j}\Bigg\{
\int\varphi^{*(0)}_{j}\varphi^{(1)}_{j}d\tau\frac{1}{r_{12}}\varphi^{(0)}_{i}
+
\int\varphi^{*(1)}_{j}\varphi^{(0)}_{j}d\tau\frac{1}{r_{12}}\varphi^{(0)}_{i}
+
\int\varphi^{*(0)}_{j}\varphi^{(0)}_{j}d\tau\frac{1}{r_{12}}\varphi^{(1)}_{i}
\Bigg\} \\
-\sum_{j}\Bigg\{
\int\varphi^{*(0)}_{j}\varphi^{(1)}_{i}d\tau\frac{1}{r_{12}}\varphi^{(0)}_{j}
+
\int\varphi^{*(1)}_{j}\varphi^{(0)}_{i}d\tau\frac{1}{r_{12}}\varphi^{(0)}_{j}
+
\int\varphi^{*(0)}_{j}\varphi^{(0)}_{i}d\tau\frac{1}{r_{12}}\varphi^{(1)}_{j}
\Bigg\} \\
=\epsilon^{(0)}_{i}\varphi^{(1)}_{i} +
\epsilon^{(1)}_{i}\varphi^{(0)}_{i}
\end{multline}
However, we can arrange the above equation into the more convenient
form:
\begin{multline}\label{PTIMQMeq:5}
\hat{F}\varphi^{(1)}_{i} - \epsilon^{(0)}_{i}\varphi^{(1)}_{i} =
\epsilon^{(1)}_{i}\varphi^{(0)}_{i} - \hat{V}\varphi^{(0)}_{i} \\
-\sum_{j}\Bigg\{
\int\varphi^{*(0)}_{j}\varphi^{(1)}_{j}d\tau\frac{1}{r_{12}}\varphi^{(0)}_{i}
+
\int\varphi^{*(1)}_{j}\varphi^{(0)}_{j}d\tau\frac{1}{r_{12}}\varphi^{(0)}_{i}
\\
-
\int\varphi^{*(1)}_{j}\varphi^{(0)}_{i}d\tau\frac{1}{r_{12}}\varphi^{(0)}_{j}
-
\int\varphi^{*(0)}_{j}\varphi^{(0)}_{i}d\tau\frac{1}{r_{12}}\varphi^{(1)}_{j}
\Bigg\}
\end{multline}
Here the items related to the $\varphi^{(1)}_{i}$ are put to the
left of the equation, and the others related to the
$\varphi^{(0)}_{i}$ are put to the right side. Furthermore the
$\hat{F}\varphi^{(1)}_{i}$ is:
\begin{multline}\label{PTIMQMeq:18}
\hat{F}\varphi^{(1)}_{i} = \hat{H}_{1}\varphi^{(1)}_{i} +
\sum_{j}\left[
\left\{\int\varphi^{*(0)}_{j}\varphi^{(0)}_{j}d\tau\frac{1}{r_{12}}\right\}
\varphi^{(1)}_{i}
\right. \\
 -
\left.
\left\{\int\varphi^{*(0)}_{j}\varphi^{(1)}_{i}d\tau\frac{1}{r_{12}}\right\}
\varphi^{(0)}_{j} \right]
\end{multline}
Compared with the (\ref{PTIMQMeq:1}) we have omitted all the
explicit electron labels for simplicity. Besides, the higher order
equations can be obtained in a like manner, but here we concentrate
on the discussion of first order perturbation equation.

By multiply the equation with $\varphi^{*(0)}_{i}$, we can get the
expression for the first order energy for the molecular orbitals:
\begin{multline}\label{PTIMQMeq:13}
\epsilon^{(1)}_{i}=
\int\varphi^{*(0)}_{i}\hat{V}\varphi^{(0)}_{i}d\tau + \\
\sum_{j}\Bigg\{
\int\varphi^{*(0)}_{j}\varphi^{(1)}_{j}d\tau\frac{1}{r_{12}}
    \varphi^{*(0)}_{i}\varphi^{(0)}_{i}d\tau +
\int\varphi^{*(1)}_{j}\varphi^{(0)}_{j}d\tau\frac{1}{r_{12}}
    \varphi^{*(0)}_{i}\varphi^{(0)}_{i}d\tau
\\
-
\int\varphi^{*(1)}_{j}\varphi^{(0)}_{i}d\tau\frac{1}{r_{12}}
    \varphi^{*(0)}_{i}\varphi^{(0)}_{j}d\tau
-
\int\varphi^{*(0)}_{j}\varphi^{(0)}_{i}d\tau\frac{1}{r_{12}}
    \varphi^{*(0)}_{i}\varphi^{(1)}_{j}d\tau
\Bigg\}
\end{multline}


%%%%%%%%%%%%%%%%%%%%%%%%%%%%%%%%%%%%%%%%%%%%%%%%%%%%%%%%%%%%%%%%%%%%%%%%%%%%%%%%%%%%%%%%%%%%%%%%%
\subsection{Relations between the perturbed molecular
orbitals}\label{PTIMQM:1}
%
%   consideration about the relations between the wave functions
%
Next we will consider the relations between the $\varphi^{(0)}_{i}$,
$\varphi^{(1)}_{i}$ etc. For the $\varphi^{(0)}_{i}$, it's known
that it equals to the unperturbed HF orbitals:
\begin{align}\label{}
\varphi^{(0)}_{i} &= \varphi_{i}  \nonumber \\
\epsilon^{(0)}_{i} &= \epsilon_{i}
\end{align}
Furthermore, these orbitals can be safely assumed to be orthogonal
with each other:
\begin{equation}\label{}
\int \varphi^{*(0)}_{i}\varphi^{(0)}_{j} d\tau = \delta_{ij}
\end{equation}

For the $\varphi^{(0)}_{i}$, $\varphi^{(1)}_{i}$, we can prove that
they are orthogonal with each other:
\begin{equation}\label{PTIMQMeq:19}
\int \varphi^{*(0)}_{i}\varphi^{(1)}_{i} d\tau = 0
\end{equation}
Such relation can be obtained via the normalization condition of the
$\varphi_{i}$. Suggest that the $\varphi_{i}$ is approximated at the
first order, then we have:
\begin{align}\label{}
\int \varphi^{*}_{i}\varphi_{i} d\tau &= 1  \Rightarrow \nonumber \\
\int \varphi^{*(0)}_{i}\varphi^{(1)}_{i} d\tau  + \int
\varphi^{*(1)}_{i}\varphi^{(0)}_{i} d\tau &= 0
\end{align}
Since that we have:
\begin{equation}\label{}
\int \varphi^{*(0)}_{i}\varphi^{(1)}_{i} d\tau  = \left\{\int
\varphi^{*(1)}_{i}\varphi^{(0)}_{i} d\tau\right\}^{*}
\end{equation}
Then the integral should be some imaginary number so that it can be
set to zero without hurting the generality. Therefore the
orthogonality condition has been proved.

For the $\varphi^{(0)}_{i}$, $\varphi^{(1)}_{k}$ where $k\neq i$, we
can have the relation that:
\begin{equation}\label{PTIMQMeq:9}
\int \varphi^{*(0)}_{i}\varphi^{(1)}_{k} d\tau  + \int
\varphi^{*(1)}_{i}\varphi^{(0)}_{k} d\tau = 0
\end{equation}
This can be achieved through the (\ref{PTIMQMeq:5}). If we take
complex conjugation of the equation in (\ref{PTIMQMeq:5}) and
multiply the $\varphi^{(0)}_{k}$ then to integrate, we can get some
equation:
\begin{multline}\label{PTIMQMeq:7}
\langle\varphi^{(1)}_{i}|\hat{F}|\varphi^{(0)}_{k}\rangle -
\epsilon^{(0)}_{i}\langle\varphi^{(1)}_{i}|\varphi^{(0)}_{k}\rangle
=
- \langle\varphi^{(0)}_{i}|\hat{V}|\varphi^{(0)}_{k}\rangle \\
-\sum_{j}\Bigg\{
\left(\varphi^{(1)}_{j}\varphi^{(0)}_{j}|\varphi^{(0)}_{i}\varphi^{(0)}_{k}\right)
+
\left(\varphi^{(0)}_{j}\varphi^{(1)}_{j}|\varphi^{(0)}_{i}\varphi^{(0)}_{k}\right)
\\
-
\left(\varphi^{(0)}_{i}\varphi^{(1)}_{j}|\varphi^{(0)}_{j}\varphi^{(0)}_{k}\right)
-
\left(\varphi^{(0)}_{i}\varphi^{(0)}_{j}|\varphi^{(1)}_{j}\varphi^{(0)}_{k}\right)
\Bigg\}
\end{multline}
Here we use the Dirac notion to express the single electron
integral, on the other hand the double electron integral is shorten
as:
\begin{equation}\label{}
\left(\varphi^{(1)}_{j}\varphi^{(0)}_{j}|\varphi^{(0)}_{i}\varphi^{(0)}_{k}\right)
= \int \int
\varphi^{*(1)}_{j}(1)\varphi^{(0)}_{j}(1)\frac{1}{r_{12}}
\varphi^{*(0)}_{i}(2)\varphi^{(0)}_{k}(2)d\tau_{1}d\tau_{2}
\end{equation}
Such abbreviation is to make the derivation as clear as possible.

On the other hand, if the equation in (\ref{PTIMQMeq:5}) in on
$\varphi^{(1)}_{k}$ and we multiply the $\varphi^{*(0)}_{i}$ then to
make integration, we can get similar equation with
(\ref{PTIMQMeq:7}):
\begin{multline}\label{PTIMQMeq:8}
\langle\varphi^{(0)}_{i}|\hat{F}|\varphi^{(1)}_{k}\rangle -
\epsilon^{(0)}_{k}\langle\varphi^{(0)}_{i}|\varphi^{(1)}_{k}\rangle
=
- \langle\varphi^{(0)}_{i}|\hat{V}|\varphi^{(0)}_{k}\rangle \\
-\sum_{j}\Bigg\{
\left(\varphi^{(0)}_{j}\varphi^{(1)}_{j}|\varphi^{(0)}_{i}\varphi^{(0)}_{k}\right)
+
\left(\varphi^{(1)}_{j}\varphi^{(0)}_{j}|\varphi^{(0)}_{i}\varphi^{(0)}_{k}\right)
\\
-
\left(\varphi^{(1)}_{j}\varphi^{(0)}_{k}|\varphi^{(0)}_{i}\varphi^{(0)}_{j}\right)
-
\left(\varphi^{(0)}_{j}\varphi^{(0)}_{k}|\varphi^{(0)}_{i}\varphi^{(1)}_{j}\right)
\Bigg\}
\end{multline}
Hence it's easy to see that the right part of (\ref{PTIMQMeq:7}) is
same with the right side of (\ref{PTIMQMeq:8}), therefore we have:
\begin{align}\label{PTIMQMeq:27}
\langle\varphi^{(0)}_{i}|\hat{F}|\varphi^{(1)}_{k}\rangle -
\epsilon^{(0)}_{k}\langle\varphi^{(0)}_{i}|\varphi^{(1)}_{k}\rangle
&=\langle\varphi^{(1)}_{i}|\hat{F}|\varphi^{(0)}_{k}\rangle -
\epsilon^{(0)}_{i}\langle\varphi^{(1)}_{i}|\varphi^{(0)}_{k}\rangle
\nonumber \\
(\epsilon^{(0)}_{i}-\epsilon^{(0)}_{k})
\langle\varphi^{(0)}_{i}|\varphi^{(1)}_{k}\rangle &=
(\epsilon^{(0)}_{k}-\epsilon^{(0)}_{i})
\langle\varphi^{(1)}_{i}|\varphi^{(0)}_{k}\rangle
\end{align}
Then we can get the conclusion in the (\ref{PTIMQMeq:9}).

Nevertheless, in such derivation if the energy level is degenerated
(which means that we have $\epsilon^{(0)}_{k}=\epsilon^{(0)}_{i}$
for the $k\neq i$); we can not get the conclusion of
(\ref{PTIMQMeq:9}) directly from the (\ref{PTIMQMeq:27}). However,
there is another way to derive the relation shown in the
(\ref{PTIMQMeq:9}), which is similar to the way we got the relation
in (\ref{PTIMQMeq:19}).

Once again we use the normalization condition for the $\varphi_{i}$.
Suggest that the $\varphi_{i}$ is approximated at the first order,
then we have:
\begin{align}\label{}
\int \varphi^{*}_{i}\varphi_{k} d\tau = 0 \quad  i\neq k &
\Rightarrow \nonumber \\
\int (\varphi^{*(0)}_{i} + \lambda\varphi^{*(1)}_{i})
(\varphi^{(0)}_{k} + \lambda\varphi^{(1)}_{k}) d\tau &=0 \nonumber
\\
\int\varphi^{*(0)}_{i}\varphi^{(0)}_{k} d\tau  + \lambda(\int
\varphi^{*(0)}_{i}\varphi^{(1)}_{k} d\tau + \int
\varphi^{*(1)}_{i}\varphi^{(0)}_{k} d\tau) + O(\lambda^{2}) &= 0
\end{align}

Because the unperturbed orbitals are orthogonal with each other,
thus $\int\varphi^{*(0)}_{i}\varphi^{(0)}_{k} d\tau = 0$.
Furthermore, since $\lambda$ is some arbitrary number, therefore it
requires that:
\begin{equation}
\int \varphi^{*(0)}_{i}\varphi^{(1)}_{k} d\tau + \int
\varphi^{*(1)}_{i}\varphi^{(0)}_{k} d\tau = 0
\end{equation}
This is just the conclusion in the (\ref{PTIMQMeq:9}). We note that
this proof does not refer to the energy degeneracy so that the
conclusion in the (\ref{PTIMQMeq:9}) is holding in all cases.

All in all, we can get the relations between the first order
correlated orbital of $\varphi^{(1)}_{i}$ and the unperturbed
orbital of $\varphi^{(0)}_{j}$:
\begin{equation}\label{}
\int \varphi^{*(0)}_{i}\varphi^{(1)}_{j} d\tau + \int
\varphi^{*(1)}_{i}\varphi^{(0)}_{j} d\tau = 0
\end{equation}
Here the label of $i$ and $j$ can be any kind of orbitals.

%%%%%%%%%%%%%%%%%%%%%%%%%%%%%%%%%%%%%%%%%%%%%%%%%%%%%%%%%%%%%%%%%%%%%%%%%%%%%%%%%%%%%%%%%%%%%%%%%
\subsection{The first order total energy expression}
%
% consider the total energy expression under the perturbation framework
%
Next we switch to the total energy derivation. According to the
results we have gotten in the HF chapter, the total energy for an
arbitrary $n$ electrons system is:
\begin{multline}\label{PTIMQMeq:10}
E =
\sum_{i}^{n}\langle\varphi_{i}(1)|\hat{H}_{1}(1)|\varphi_{i}(1)\rangle
+
\\
\frac{1}{2}\sum_{i}\sum_{j} \left\{
\int\int\varphi^{*}_{i}(1)\varphi_{i}(1)\frac{1}{r_{12}}
\varphi^{*}_{j}(2)\varphi_{j}(2)d\tau_{1}d\tau_{2}- \right. \\
\left.
\int\int\varphi^{*}_{i}(1)\varphi_{j}(1)\frac{1}{r_{12}}
\varphi^{*}_{j}(2)\varphi_{i}(2)d\tau_{1}d\tau_{2} \right\}
\end{multline}

Similarly, the total energy can be also expanded according to the
order of $\lambda$:
\begin{equation}\label{}
E = E^{(0)} + \lambda E^{(1)} + \lambda^{2}E^{(2)} + \cdots
\end{equation}
Now by taking the expansion for $\varphi$ defined in
(\ref{PTIMQMeq:2}) as well as the total energy expansion into the
(\ref{PTIMQMeq:10}), we can get the perturbation equations for $E$.

The zero order approximation for the total energy equation is:
\begin{multline}\label{PTIMQMeq:11}
E^{(0)} =
\sum_{i}^{n}\langle\varphi^{(0)}_{i}|\hat{H}_{1}|\varphi^{(0)}_{i}\rangle
+
\\
\frac{1}{2}\sum_{i}\sum_{j} \left\{
\int\int\varphi^{*(0)}_{i}\varphi^{(0)}_{i}\frac{1}{r_{12}}
\varphi^{*(0)}_{j}\varphi^{(0)}_{j}d\tau_{1}d\tau_{2}- \right. \\
\left. \int\int\varphi^{*(0)}_{i}\varphi^{(0)}_{j}\frac{1}{r_{12}}
\varphi^{*(0)}_{j}\varphi^{(0)}_{i}d\tau_{1}d\tau_{2} \right\}
\end{multline}
On the other hand, by using the orbital energy we can express it as:
\begin{multline}\label{PTIMQMeq:12}
E^{(0)} = \sum_{i}^{n}\epsilon_{i}^{(0)} -
\frac{1}{2}\sum_{i}\sum_{j} \left\{
\int\int\varphi^{*(0)}_{i}\varphi^{(0)}_{i}\frac{1}{r_{12}}
\varphi^{*(0)}_{j}\varphi^{(0)}_{j}d\tau_{1}d\tau_{2}- \right. \\
\left. \int\int\varphi^{*(0)}_{i}\varphi^{(0)}_{j}\frac{1}{r_{12}}
\varphi^{*(0)}_{j}\varphi^{(0)}_{i}d\tau_{1}d\tau_{2} \right\}
\end{multline}

For the first order approximation, it scales with $\lambda$ and we
can express this term as:
\begin{multline}\label{PTIMQMeq:14}
E^{(1)} =\sum_{i}^{n}
\Bigg\{\langle\varphi^{(1)}_{i}|\hat{H}_{1}|\varphi^{(0)}_{i}\rangle
      +\langle\varphi^{(0)}_{i}|\hat{H}_{1}|\varphi^{(1)}_{i}\rangle +
       \langle\varphi^{(0)}_{i}|\hat{V}|\varphi^{(0)}_{i}\rangle\Bigg\}
\\
+\frac{1}{2}\sum_{i}\sum_{j}\Bigg\{
\left(\varphi^{(1)}_{i}\varphi^{(0)}_{i}|\varphi^{(0)}_{j}\varphi^{(0)}_{j}\right)
+
\left(\varphi^{(0)}_{i}\varphi^{(1)}_{i}|\varphi^{(0)}_{j}\varphi^{(0)}_{j}\right)
\\
+
\left(\varphi^{(0)}_{i}\varphi^{(0)}_{i}|\varphi^{(1)}_{j}\varphi^{(0)}_{j}\right)
+
\left(\varphi^{(0)}_{i}\varphi^{(0)}_{i}|\varphi^{(0)}_{j}\varphi^{(1)}_{j}\right)
\Bigg\} \\
-\frac{1}{2}\sum_{i}\sum_{j}\Bigg\{
\left(\varphi^{(1)}_{i}\varphi^{(0)}_{j}|\varphi^{(0)}_{j}\varphi^{(0)}_{i}\right)
+
\left(\varphi^{(0)}_{i}\varphi^{(1)}_{j}|\varphi^{(0)}_{j}\varphi^{(0)}_{i}\right)
\\
+
\left(\varphi^{(0)}_{i}\varphi^{(0)}_{j}|\varphi^{(1)}_{j}\varphi^{(0)}_{i}\right)
+
\left(\varphi^{(0)}_{i}\varphi^{(0)}_{j}|\varphi^{(0)}_{j}\varphi^{(1)}_{i}\right)
\Bigg\} \\
\end{multline}

This expression is too much complicated. Therefore, by using the
relation we have got before, we try to reduce it into some simpler
form. Firstly, from the (\ref{PTIMQMeq:13}) we can see that the two
electrons integral can be expressed as:
\begin{multline}\label{PTIMQMeq:16}
\sum_{i}\sum_{j}\Bigg\{
\left(\varphi^{(0)}_{j}\varphi^{(1)}_{j}|\varphi^{(0)}_{i}\varphi^{(0)}_{i}\right)
+
\left(\varphi^{(1)}_{j}\varphi^{(0)}_{j}|\varphi^{(0)}_{i}\varphi^{(0)}_{i}\right)
\\
-
\left(\varphi^{(1)}_{j}\varphi^{(0)}_{i}|\varphi^{(0)}_{i}\varphi^{(0)}_{j}\right)
-
\left(\varphi^{(0)}_{j}\varphi^{(0)}_{i}|\varphi^{(0)}_{i}\varphi^{(1)}_{j}\right)
\Bigg\} \\
= \sum_{i}\Bigg\{\epsilon^{(1)}_{i} -
\langle\varphi^{(0)}_{i}|\hat{V}|\varphi^{(0)}_{i}\rangle\Bigg\}
\end{multline}
By exchanging the label of $i$ and $j$ on the right side of the
above equation (such operation will change nothing), we can get:
\begin{multline}\label{PTIMQMeq:15}
\sum_{i}\sum_{j}\Bigg\{
\left(\varphi^{(0)}_{i}\varphi^{(1)}_{i}|\varphi^{(0)}_{j}\varphi^{(0)}_{j}\right)
+
\left(\varphi^{(1)}_{i}\varphi^{(0)}_{i}|\varphi^{(0)}_{j}\varphi^{(0)}_{j}\right)
\\
-
\left(\varphi^{(1)}_{i}\varphi^{(0)}_{j}|\varphi^{(0)}_{j}\varphi^{(0)}_{i}\right)
-
\left(\varphi^{(0)}_{i}\varphi^{(0)}_{j}|\varphi^{(0)}_{j}\varphi^{(1)}_{i}\right)
\Bigg\} \\
= \sum_{i}\Bigg\{\epsilon^{(1)}_{i} -
\langle\varphi^{(0)}_{i}|\hat{V}|\varphi^{(0)}_{i}\rangle\Bigg\}
\end{multline}
That's half part of the two electron integrals in the
(\ref{PTIMQMeq:14}).

On the other hand, in the (\ref{PTIMQMeq:16}) we can exchange the
electron label in the two electron integrals, that is to make:
\begin{equation}\label{}
(ii|jj) \Rightarrow (jj|ii)
\end{equation}
Certainly such operation will not change the integrals. Thus we can
transform the (\ref{PTIMQMeq:16}) into:
\begin{multline}\label{PTIMQMeq:17}
\sum_{i}\sum_{j}\Bigg\{
\left(\varphi^{(0)}_{i}\varphi^{(0)}_{i}|\varphi^{(0)}_{j}\varphi^{(1)}_{j}\right)
+
\left(\varphi^{(0)}_{i}\varphi^{(0)}_{i}|\varphi^{(1)}_{j}\varphi^{(0)}_{j}\right)
\\
-
\left(\varphi^{(0)}_{i}\varphi^{(0)}_{j}|\varphi^{(1)}_{j}\varphi^{(0)}_{i}\right)
-
\left(\varphi^{(0)}_{i}\varphi^{(1)}_{j}|\varphi^{(0)}_{j}\varphi^{(0)}_{i}\right)
\Bigg\} \\
= \sum_{i}\Bigg\{\epsilon^{(1)}_{i} -
\langle\varphi^{(0)}_{i}|\hat{V}|\varphi^{(0)}_{i}\rangle\Bigg\}
\end{multline}
Now it can see this is the other half part of the two electron
integrals in the (\ref{PTIMQMeq:14}).

All in all, by the expression in (\ref{PTIMQMeq:15}) and
(\ref{PTIMQMeq:17}) the (\ref{PTIMQMeq:14}) can be transformed into:
\begin{equation}\label{PTIMQMeq:22}
E^{(1)} =\sum_{i}^{n} \left\{\epsilon^{(1)}_{i} +
\langle\varphi^{(1)}_{i}|\hat{H}_{1}|\varphi^{(0)}_{i}\rangle
      +\langle\varphi^{(0)}_{i}|\hat{H}_{1}|\varphi^{(1)}_{i}\rangle\right\}
\end{equation}

so far we have gotten some simpler expression for the $E^{(1)}$,
however; the concrete expression for the $\epsilon^{(1)}_{i}$ is not
known yet. Therefore we make to make additional transformation.
First, let's evaluate the integrals below according to the
(\ref{PTIMQMeq:18}):
\begin{multline}\label{PTIMQMeq:20}
\sum_{i}\Bigg\{\langle\varphi^{(0)}_{i}|\hat{F}|\varphi^{(1)}_{i}\rangle
+\langle\varphi^{(1)}_{i}|\hat{F}|\varphi^{(0)}_{i}\rangle\Bigg\}
= \\
\sum_{i}\Bigg\{
\langle\varphi^{(0)}_{i}|\hat{H}_{1}|\varphi^{(1)}_{i}\rangle +
\sum_{j}\Big[
\left(\varphi^{(0)}_{j}\varphi^{(0)}_{j}|\varphi^{(0)}_{i}\varphi^{(1)}_{i}\right)
-
\left(\varphi^{(0)}_{j}\varphi^{(1)}_{i}|\varphi^{(0)}_{i}\varphi^{(0)}_{j}\right)
\Big] \Bigg\} \\
+ \sum_{i}\Bigg\{
\langle\varphi^{(1)}_{i}|\hat{H}_{1}|\varphi^{(0)}_{i}\rangle +
\sum_{j}\Big[
\left(\varphi^{(0)}_{j}\varphi^{(0)}_{j}|\varphi^{(1)}_{i}\varphi^{(0)}_{i}\right)
-
\left(\varphi^{(1)}_{i}\varphi^{(0)}_{j}|\varphi^{(0)}_{j}\varphi^{(0)}_{i}\right)
\Big] \Bigg\}
\end{multline}
Because of the orthogonality between the $\varphi^{(0)}_{i}$ and
$\varphi^{(1)}_{i}$ defined in (\ref{PTIMQMeq:19}), we have:
\begin{equation}\label{}
\langle\varphi^{(0)}_{i}|\hat{F}|\varphi^{(1)}_{i}\rangle =
\epsilon^{(0)}_{i}\langle\varphi^{(0)}_{i}|\varphi^{(1)}_{i}\rangle
= 0
\end{equation}
Therefore the equation of (\ref{PTIMQMeq:20}) can be rearranged
into:
\begin{multline}\label{PTIMQMeq:20}
\sum_{i}\Bigg\{
\langle\varphi^{(0)}_{i}|\hat{H}_{1}|\varphi^{(1)}_{i}\rangle +
\langle\varphi^{(1)}_{i}|\hat{H}_{1}|\varphi^{(0)}_{i}\rangle
\Bigg\}= \\
-\sum_{i}\sum_{j}\Bigg\{
\left(\varphi^{(0)}_{j}\varphi^{(0)}_{j}|\varphi^{(0)}_{i}\varphi^{(1)}_{i}\right)
-
\left(\varphi^{(0)}_{j}\varphi^{(1)}_{i}|\varphi^{(0)}_{i}\varphi^{(0)}_{j}\right)
+ \\
\left(\varphi^{(0)}_{j}\varphi^{(0)}_{j}|\varphi^{(1)}_{i}\varphi^{(0)}_{i}\right)
-
\left(\varphi^{(1)}_{i}\varphi^{(0)}_{j}|\varphi^{(0)}_{j}\varphi^{(0)}_{i}\right)
\Bigg\}
\end{multline}
Here we can exchange the label of $i$, $j$ and then exchange the
electron label, such operation will keep the integrals invariant.
Then we can have:
\begin{multline}\label{PTIMQMeq:20}
\sum_{i}\Bigg\{
\langle\varphi^{(0)}_{i}|\hat{H}_{1}|\varphi^{(1)}_{i}\rangle +
\langle\varphi^{(1)}_{i}|\hat{H}_{1}|\varphi^{(0)}_{i}\rangle
\Bigg\}= \\
-\sum_{i}\sum_{j}\Bigg\{
\left(\varphi^{(0)}_{j}\varphi^{(1)}_{j}|\varphi^{(0)}_{i}\varphi^{(0)}_{i}\right)
-
\left(\varphi^{(0)}_{j}\varphi^{(0)}_{i}|\varphi^{(0)}_{i}\varphi^{(1)}_{j}\right)
+ \\
\left(\varphi^{(1)}_{j}\varphi^{(0)}_{j}|\varphi^{(0)}_{i}\varphi^{(0)}_{i}\right)
-
\left(\varphi^{(0)}_{i}\varphi^{(0)}_{j}|\varphi^{(1)}_{j}\varphi^{(0)}_{i}\right)
\Bigg\}
\end{multline}
For the integral of
$\left(\varphi^{(0)}_{i}\varphi^{(0)}_{j}|\varphi^{(1)}_{j}\varphi^{(0)}_{i}\right)$,
if we exchange the electron label again, it will be:
\begin{equation}\label{}
\left(\varphi^{(0)}_{i}\varphi^{(0)}_{j}|\varphi^{(1)}_{j}\varphi^{(0)}_{i}\right)=
\left(\varphi^{(1)}_{j}\varphi^{(0)}_{i}|\varphi^{(0)}_{i}\varphi^{(0)}_{j}\right)
\end{equation}

Then finally the (\ref{PTIMQMeq:20}) will be:
\begin{multline}\label{PTIMQMeq:21}
\sum_{i}\Bigg\{
\langle\varphi^{(0)}_{i}|\hat{H}_{1}|\varphi^{(1)}_{i}\rangle +
\langle\varphi^{(1)}_{i}|\hat{H}_{1}|\varphi^{(0)}_{i}\rangle
\Bigg\}= \\
-\sum_{i}\sum_{j}\Bigg\{
\left(\varphi^{(0)}_{j}\varphi^{(1)}_{j}|\varphi^{(0)}_{i}\varphi^{(0)}_{i}\right)
-
\left(\varphi^{(0)}_{j}\varphi^{(0)}_{i}|\varphi^{(0)}_{i}\varphi^{(1)}_{j}\right)
+ \\
\left(\varphi^{(1)}_{j}\varphi^{(0)}_{j}|\varphi^{(0)}_{i}\varphi^{(0)}_{i}\right)
-
\left(\varphi^{(1)}_{j}\varphi^{(0)}_{i}|\varphi^{(0)}_{i}\varphi^{(0)}_{j}\right)
\Bigg\}
\end{multline}

Now we can transform the (\ref{PTIMQMeq:22}) by using the conclusion
we get from the (\ref{PTIMQMeq:21}), as well as the expression for
the $\epsilon^{(1)}_{i}$ in (\ref{PTIMQMeq:13}). Consequently, it
turns out that $E^{(1)}$ finally is:
\begin{equation}\label{}
E^{(1)} = \sum_{i}
\langle\varphi^{(0)}_{i}|\hat{V}|\varphi^{(0)}_{i}\rangle
\end{equation}


%%%%%%%%%%%%%%%%%%%%%%%%%%%%%%%%%%%%%%%%%%%%%%%%%%%%%%%%%%%%%%%%%%%%%%%%%%%%%%%%%%%%%%%%%%%%%%%%%
\subsection{Matrix form of the perturbed Hatree-Fock equation}
%
%
%
%
In the above content, we have derived the first order perturbed
Hatree-Fock equation in a very detailed way. However, since in
quantum chemistry we usually use the matrix form of the Hatree-Fock
equation in terms of the basis sets functions; thus it's necessary
to rebuild the perturbed equations into the matrix form based on the
above achievement\cite{stevens:550}.


%%%%%%%%%%%%%%%%%%%%%%%%%%%%%%%%%%%%%%%%%%%%%%%%%%%%%%%%%%%%%%%%%%%%%%%%%%%%%%%%%%%%%%%%%%%%%%%%%
\subsection{General procedures}
%
%  general preparation before the derivation,
%  specify the ground states and HF equation as well as how to express the
%  first order correlated molecular orbitals
%
Now we begin to investigate how to make a perturbation treatment on
the base of the linear combination of basis sets functions. In this
case, the unperturbed molecular orbitals is usually expressed as:
\begin{equation}\label{PTIMQMeq:23}
\varphi_{q} = \sum_{p}c_{pq}\chi_{p}
\end{equation}
Where the $\chi_{p}$ is referred as the basis sets functions. Here
we follow the convention that to use $i, j$ etc. to designate the
occupied orbitals, $a, b$ etc. to designate the virtual orbitals and
to use the $p, q$ etc. to refer to the general orbitals. Here it's
better to reform it into the matrix form, that is:
\begin{equation}\label{PTIMQMeq:30}
\varphi_{q} =\begin{bmatrix}
               \chi_{1} & \chi_{2} & \cdots & \chi_{n} \\
             \end{bmatrix}
             \begin{bmatrix}
               c_{1q} \\
               c_{2q} \\
               \cdots \\
               c_{nq} \\
             \end{bmatrix}
\end{equation}
According to the representation theory \ref{REPRESENTATION:1}, such
expression is called coefficients $C$ representation based on the
selected basis functions space of $\chi$.

Now we assume that there's one to one correspondence between the
coefficients representation and the total wave function (so far in
my opinion this assumption does not hurt the generality, in other
words such assumption always holds true). Then by taking the vector
form of the molecular orbitals in (\ref{PTIMQMeq:30}) back into the
original Hatree-Fock equation (\ref{PTIMQMeq:1}), and to multiply
with $\psi_{i}^{*}(1)$ then make integration; we can get the
corresponding matrix form:
\begin{equation}\label{PTIMQMeq:24}
FC_{q} = SC_{q}\epsilon_{q}
\end{equation}
Furthermore, since we can use some unitary matrix to drop the
overlap matrix, then the Hatree-Fock equation can finally be:
\begin{equation}\label{PTIMQMeq:34}
FC_{q} = \epsilon C_{q}
\end{equation}
This equation has been fully discussed in the Hatree-Fock chapter
\ref{HFT}. Here $C_{q}$ denotes the vector of $c_{pq}$ in the
(\ref{PTIMQMeq:23}). The $\epsilon$ is some $n\times n$ diagonal
matrix which characterizes the orbital energy. Later in the content,
we will go on the discussion focusing on the (\ref{PTIMQMeq:34}).

Moreover, the total energy is given as:
\begin{equation}\label{PTIMQMeq:25}
E = \frac{1}{2}\sum_{q}(C_{q}^{+}FC_{q} + \epsilon_{q})
\end{equation}
The (\ref{PTIMQMeq:25}) can be get directly from the addition
between (\ref{PTIMQMeq:11}) and (\ref{PTIMQMeq:12}). For simplicity,
The orbitals can still assumed to be orthogonal with each other,
that means $\langle\varphi_{p}|\varphi_{q}\rangle = \delta_{pq}$.

Now we assume that there's some perturbation in the whole system so
that the it causes corresponding change in the molecular orbital,
the orbital energy and the total energy:
\begin{align}\label{}
\hat{F}^{'} &= \hat{F} + \lambda \hat{V} \Rightarrow \nonumber \\
\varphi_{q} &= \varphi^{(0)}_{q} + \lambda\varphi^{(1)}_{q} +
\lambda^{2}\varphi^{(2)}_{q} + \cdots \nonumber \\
\epsilon_{q} &= \epsilon^{(0)}_{q} + \lambda\epsilon^{(1)}_{q} +
\lambda^{2}\epsilon^{(2)}_{q} + \cdots \nonumber \\
E &= E^{(0)} + \lambda E^{(1)} + \lambda^{2}E^{(2)} + \cdots
\end{align}
We note that this is same with the derivation in the last section.
Here in a like manner we still concentrate on the first order
perturbation effects.

Furthermore, how to express the first order molecular orbital? Since
that all the molecular orbitals are expressed on the space
constituted by the basis sets, thus it's natural to be represented
by the perturbed coefficients:
\begin{align}\label{PTIMQMeq:26}
\varphi^{(0)}_{q} &= \sum_{p}c^{(0)}_{pq}\chi_{p} \nonumber \\
\varphi^{(1)}_{q} &= \sum_{p}c^{(1)}_{pq}\chi_{p}
\end{align}
Here the label of $p$ goes over all the orbitals, including the
occupied ones and the virtual ones. Next, we are going to consider
the relations between the molecular orbitals.



%%%%%%%%%%%%%%%%%%%%%%%%%%%%%%%%%%%%%%%%%%%%%%%%%%%%%%%%%%%%%%%%%%%%%%%%%%%%%%%%%%%%%%%%%%%%%%%%%
\subsection{Relation for the perturbed orbitals}
%
%  derive the relation between the perturbed orbitals, in terms of the
%  basis sets functions
%
%
In the section of (\ref{PTIMQM:1}), we have derived some general
relation between the $\varphi^{(1)}_{p}$ and $\varphi^{(0)}_{q}$:
\begin{equation}\label{PTIMQMeq:31}
\int\varphi^{*(1)}_{p}\varphi^{(0)}_{q}d\tau
+\int\varphi^{*(0)}_{p}\varphi^{(1)}_{q}d\tau = 0
\end{equation}
Furthermore, for the case that $p=q$; we have proved that the
integral can be set to $0$:
\begin{equation}\label{}
\int\varphi^{*(1)}_{p}\varphi^{(0)}_{q}d\tau = 0
\end{equation}

By taking the expression of (\ref{PTIMQMeq:26}) into the
(\ref{PTIMQMeq:31}), we can get:
\begin{align}\label{PTIMQMeq:33}
\sum_{r}\sum_{s}(c^{*(1)}_{rp}c^{(0)}_{sq} +
c^{*(0)}_{rp}c^{(1)}_{sq})\langle\chi_{r}|\chi_{s}\rangle &= 0
\Rightarrow\nonumber \\
C^{+(1)}_{p}SC^{(0)}_{q} + C^{+(0)}_{p}SC^{(1)}_{q} &= 0
\end{align}
However, since that in the (\ref{PTIMQMeq:34}) we have ``drop'' the
overlap matrix by multiplying some unitary matrix; that means we
have:
\begin{equation}\label{}
C^{'}_{q} = UC_{q} \quad U^{+}SU = I \Rightarrow
C^{+(1)'}_{p}SC^{(0)'}_{q} = C^{+(1)}_{p}U^{+}SUC^{(0)}_{q} =
C^{+(1)}_{p}C^{(0)}_{q}
\end{equation}
Therefore the (\ref{PTIMQMeq:33}) can be transformed as:
\begin{equation}\label{PTIMQMeq:35}
C^{+(1)}_{p}C^{(0)}_{q} + C^{+(0)}_{p}C^{(1)}_{q} = 0
\end{equation}

Now let's consider the expectation value for an single electron
operator of $\hat{A}$. Based on the perturbed molecular orbital
expression, the expectation value is:
\begin{equation}\label{}
\sum_{i=1}^{n}\langle\varphi_{i}|\hat{A}|\varphi_{i}\rangle =
\sum_{i=1}^{n}\langle\varphi^{(0)}_{i}|\hat{A}|\varphi^{(0)}_{i}\rangle
+ \lambda\sum_{i=1}^{n}\Bigg\{
\langle\varphi^{(1)}_{i}|\hat{A}|\varphi^{(0)}_{i}\rangle +
\langle\varphi^{(0)}_{i}|\hat{A}|\varphi^{(1)}_{i}\rangle \Bigg\}
\end{equation}
Here we suggest there's $n$ occupied orbitals and totally $m$ basis
sets. The expectation is going over all the occupied orbitals. The
expectation value for the $\hat{A}$ is approximated at the first
order.

On the other hand, we can expand the expectation value also in the
progressive order:
\begin{equation}\label{}
\langle A \rangle = \langle A \rangle^{(0)} + \lambda\langle A
\rangle^{(1)} + \cdots
\end{equation}
Obviously the zero order approximation is:
\begin{equation}\label{}
\langle A \rangle^{(0)} =
\sum_{i=1}^{n}\langle\varphi^{(0)}_{i}|\hat{A}|\varphi^{(0)}_{i}\rangle
\end{equation}
And the first order approximation is:
\begin{equation}\label{}
\langle A \rangle^{(1)} = \sum_{i=1}^{n}\Bigg\{
\langle\varphi^{(1)}_{i}|\hat{A}|\varphi^{(0)}_{i}\rangle +
\langle\varphi^{(0)}_{i}|\hat{A}|\varphi^{(1)}_{i}\rangle \Bigg\}
\end{equation}

Now let's further evaluate the first order approximation. By taking
the basis sets into the equation above, it transforms into:
\begin{align}\label{}
\langle A \rangle^{}(1) &= \sum_{i=1}^{n}c^{*(1)}_{ki}c^{(0)}_{ji}
\end{align}


Now it comes up with a question that how to express the relations
for the perturbed orbitals in terms of the basis sets functions?

By taking the expansion of (\ref{PTIMQMeq:23}) into the above
relation, we have:
\begin{equation}\label{}
\sum_{p}c^{*(1)}_{pi}\langle\varphi^{(0)}_{p}|\varphi^{(0)}_{j}\rangle
+
\sum_{q}c^{(1)}_{qj}\langle\varphi^{(0)}_{i}|\varphi^{(0)}_{q}\rangle
= 0
\end{equation}
Since that the unperturbed orbitals satisfy the orthogonal relation,
thus $\langle\varphi^{(0)}_{p}|\varphi^{(0)}_{q}\rangle =
\delta_{pq}$. Hence we have:
\begin{equation}\label{PTIMQMeq:28}
c^{*(1)}_{ji} + c^{(1)}_{ij} = 0
\end{equation}
Generally if $i=j$, then $c^{*(1)}_{ii} + c^{(1)}_{ii} = 0$; that
means the $c^{(1)}_{ii}$ is some imaginary number so that it can be
safely set to $0$.

Now let's go to see how to use the conclusion in (\ref{PTIMQMeq:28})
to get the further information about the density. In the
perturbation treatment, the density can be expressed as:
\begin{equation}\label{}
\rho = \rho^{(0)} + \lambda\rho^{(1)} +\lambda^{2}\rho^{(2)} +
\cdots
\end{equation}
For the orbitals approximated at the first order, according to the
 (\ref{PTIMQMeq:26}) we can express it as:
\begin{align}\label{}
\varphi_{i} &= \varphi^{(0)}_{i} +
\lambda\sum_{p}c^{(1)}_{pi}\varphi^{(0)}_{p} \nonumber \\
\varphi^{*}_{i} &= \varphi^{*(0)}_{i} +
\lambda\sum_{q}c^{*(1)}_{qi}\varphi^{*(0)}_{q}
\end{align}
Then the electron density can be expressed as (suggest that there
are $m$ orbitals, each one is occupied with one electron; the others
are virtual orbitals):
\begin{align}\label{}
\rho &= \rho^{(0)} + \lambda\rho^{(1)} \Rightarrow \nonumber \\
\sum_{i=1}^{m}\varphi^{*}_{i}\varphi_{i}d\tau &=
\sum_{i=1}^{m}\varphi^{*(0)}_{i}\varphi^{(0)}_{i}d\tau + \nonumber \\
&\lambda\Bigg\{\sum_{i=1}^{m}\sum_{p}c^{(1)}_{pi}
\varphi^{*(0)}_{i}\varphi^{(0)}_{p}d\tau  +
\sum_{i=1}^{m}\sum_{q}c^{*(1)}_{qi}
\varphi^{*(0)}_{q}\varphi^{(0)}_{i}d\tau \Bigg\}
\end{align}

Here it's known that the unperturbed density $\rho^{(0)}$ is:
\begin{equation}\label{}
\rho^{(0)} = \sum_{i=1}^{m}\varphi^{*(0)}_{i}\varphi^{(0)}_{i}d\tau
\end{equation}
Therefore the first perturbed density $\rho^{(1)}$ is:
\begin{equation}\label{PTIMQMeq:29}
\rho^{(1)} = \sum_{i=1}^{m}\sum_{p}c^{(1)}_{pi}
\varphi^{*(0)}_{i}\varphi^{(0)}_{p}d\tau  +
\sum_{i=1}^{m}\sum_{q}c^{*(1)}_{qi}
\varphi^{*(0)}_{q}\varphi^{(0)}_{i}d\tau
\end{equation}
Here the label of $i$ goes over all the occupied orbitals, and the
label of $p$, $q$ go over all the orbitals (occupied type as well as
the virtual orbitals).

Now by using the relation in the (\ref{PTIMQMeq:28}), we can further
simplify the expression for the $\rho^{(1)}$. By dividing the sum in
(\ref{PTIMQMeq:29}) into the occupied type and occupied-virtual
type; we can have (suggesting that there are $n$ basis sets so that
totally $n$ orbitals generated in the Hatree-Fock equation):
\begin{multline}\label{}
\rho^{(1)} = \sum_{i=1}^{m}\sum_{j=1}^{m}c^{(1)}_{ji}
\varphi^{*(0)}_{i}\varphi^{(0)}_{j}d\tau +
\sum_{i=1}^{m}\sum_{a=m+1}^{n}c^{(1)}_{ai}
\varphi^{*(0)}_{i}\varphi^{(0)}_{a}d\tau  + \\
\sum_{i=1}^{m}\sum_{j=1}^{m}c^{*(1)}_{ji}
\varphi^{*(0)}_{j}\varphi^{(0)}_{i}d\tau +
\sum_{i=1}^{m}\sum_{a=m+1}^{n}c^{*(1)}_{ai}
\varphi^{*(0)}_{a}\varphi^{(0)}_{i}d\tau
\end{multline}




%%%%%%%%%%%%%%%%%%%%%%%%%%%%%%%%%%%%%%%%%%%%%%%%%%%%%%%%%%%%%%%%%%%%%%%%%%%%%%%%%%%%%%%%%%%%%%%%%


%%% Local Variables: 
%%% mode: latex
%%% TeX-master: "../../main"
%%% End: 
