%%%%%%%%%%%%%%%%%%%%%%%%%%%%%%%%%%%%%%%%%%%%%%%%%%
%
%  This chapter is now in good condition
%
%%%%%%%%%%%%%%%%%%%%%%%%%%%%%%%%%%%%%%%%%%%%%%%%%%


% change log:
% setting up around Feb. 2008
% revised at June, 25th, 2009
% 1  add the reference to this part
% 2  before the HF discussion, correct some bugs
% 3  rewrite the HF part discussion
% revised at July, 5th, 2009
% correct the exchange hole discussion
% revised at July, 8th, 2009
% revise the exchange hole discussion by giving more clear idea
% revised at July, 25th, 2009
% to make the hole function discussion more clear
% revised at September 1th, 2009
% add the discussion from Barth's paper so that to make the hole
% function derivation much more clear
% revised at Feb 7th, 2011
% really understand the physical meaning of hole function
%
%


\chapter{Density matrices}
%
% this part used to present the base for the density matrices.
%
%
%
%
%%%%%%%%%%%%%%%%%%%%%%%%%%%%%%%%%%%%%%%%%%%%%%%%%%%%%%%%%%%%%%%%%%%%%%%%%%%%
\section{Introduction}
%
% why we have to discuss the density matrices?
%
%
In quantum mechanics, density matrices is generally considered as some
extension to the concept of wave functions\cite{Landau}.  However, in
this part; we concentrate on the discussion of density matrices for
different purpose.

The traditional algorithm devised in the quantum mechanics, is mainly
focused on the concept of wave functions. From the HF single
determinant to the configurations in the post-HF methods (here also
including the perturbation theory), the core of the framework is all
around the question that how to form a wave function to solve the
problem.

However, in this chapter we are going to prove that the use of wave
functions is not indispensable, potentially we can use the density to
replace it and get the same result. This important idea is some kind
of foundation of the popular DFT theory, and this idea will be
unveiled in this part\footnote{reference can see Lowdin's
  paper\cite{Lowdin1,Lowdin2,Lowdin3}, Weitao Yang's
  book\cite{weitaoYang}, Mr. Tang's book \cite{aoqingTang} etc. Here
  we find that the materials provided by U von
  Barth\cite{2004PhST..109....9V} is most inspiring for understanding
  the DFT origin.}.


%%%%%%%%%%%%%%%%%%%%%%%%%%%%%%%%%%%%%%%%%%%%%%%%%%%%%%%%%%%%%%%%%%%%%%%%%%%%
\section{Density matrices}
%
% 1 what's the relation between density matrices and project operator?
% actually they are the same definition.
% 2 reduced density matrices. main discussion point
%   2.1 the reduced density matrices only depend on the coordinates
%       and spin
%   2.2 the character of the reduced density matrices,
%       also compared with the density matrices
%   2.3 give the example of first order and second order ones
%   2.4 discuss the diagonal elements for first order and second
%       order ones, and its physical meaning
%   2.5 discussion to the operators: they only depends on coordinates.
%       thus it is no need to introduce 4n variables
%       to express the expectation values for these operators. so
%       potentially we may use the density matrices to express the
%       expectation values.
%   2.6 Hamiltonian and dipole moment example
%   2.7 also variation process can be done to the reduced density matrices
%   2.8 eigen values for the reduced density matrices
%
\subsection{Definition of density matrices}
\label{DODM_in_density_matrices}
% definition. and several discussion to it
%
Suggest that the $\Psi(x_{1}, x_{2}, \cdots, x_{n})$ is some $n$
particles wave function. Here, the $x_{i}$ is the arguments for the
electron $i$ in the total wave functions, they include the
coordinates $x,y,z$ of electron $i$ and its spin:
\begin{equation}\label{}
  x_{i} \Leftrightarrow (x, y, z, s) \quad \text{for electron $i$}
\end{equation}
Then the $\Psi^{*}(x_{1}, x_{2}, \cdots, x_{n})\Psi(x_{1}, x_{2},
\cdots, x_{n})$ is associated with the probability distribution for
the electrons in this system. Here we can define a more general form:
\begin{equation}\label{DMeq:1}
  \gamma_{n}(x^{'}_{1}, x^{'}_{2}, \cdots, x^{'}_{n}, x_{1}, x_{2},
  \cdots, x_{n}) = \Psi^{*}(x^{'}_{1}, x^{'}_{2}, \cdots,
  x^{'}_{n})\Psi(x_{1}, x_{2}, \cdots, x_{n})
\end{equation}
Here the two sets of independent quantities $x^{'}_{1}, x^{'}_{2},
\cdots, x^{'}_{n}$ and $x_{1}, x_{2}, \cdots, x_{n}$ can be thought
of as two sets of indices. If the two sets are identical with each
other, it will lead to the density probability. Actually, we can see
that the definition in (\ref{DMeq:1}) is identical with density
operator of $\ket{\Psi}\bra{\Psi}$. The form in (\ref{DMeq:1}) is
$\ket{\Psi}\bra{\Psi}$'s position representation.

Now let's discuss the characters related to density matrices:
\begin{theorem}\label{}
$\gamma_{n}^{+} = \gamma_{n}$. So $\gamma_{n}$ is hermitian.
\end{theorem}

\begin{proof}
\begin{equation}\label{}
\gamma_{n}^{+} = (\ket{\Psi}\bra{\Psi})^{+} = \ket{\Psi}\bra{\Psi} =
\gamma_{n}
\end{equation}
Hence the operator of $\gamma_{n}$ is hermitian. \qedhere
\end{proof}

\begin{theorem}\label{}
For any $\Phi$, $\langle\Phi|\gamma_{n}|\Phi\rangle \geq 0$.
\end{theorem}

\begin{proof}
\begin{equation}\label{}
\langle\Phi|\gamma_{n}|\Phi\rangle = \langle\Phi|\Psi\rangle^{2}
\geq 0
\end{equation}
\qedhere
\end{proof}

\begin{theorem}\label{}
$\gamma_{n}^{2} = \gamma_{n}$.
\end{theorem}

\begin{proof}
\begin{equation}\label{}
\gamma_{n}^{2} = \ket{\Psi}\bra{\Psi}\Psi\rangle\bra{\Psi} =
\ket{\Psi}\bra{\Psi} = \gamma_{n}
\end{equation}
So $\gamma_{n}$ is idempotent. \qedhere
\end{proof}

Here there's something need to note, that if the $\ket{\Psi}$ is
some complete sets: $\ket{\Psi_{i}}$ ($i = 1, 2, \cdots$); then from
the closure relation we can get that:
\begin{equation}\label{}
\sum_{i}\ket{\Psi_{i}}\bra{\Psi_{i}} = I
\end{equation}

\begin{theorem}\label{}
$\gamma_{n}\hat{B} = \hat{B}\gamma_{n}$.
If $\hat{B}$ is hermitian.
\end{theorem}

\begin{proof}
Suggest that we have some arbitrary wave function of $\Phi$:
\begin{equation}\label{}
\langle\Phi|\gamma_{n}\hat{B}|\Phi\rangle =
\langle\Phi\ket{\Psi}\bra{\Psi}\hat{B}|\Phi\rangle =
\bra{\Psi}\hat{B}|\Phi\rangle\langle\Phi\ket{\Psi} =
\bra{\Phi}\hat{B}|\Psi\rangle\langle\Psi\ket{\Phi} =
\langle\Phi|\hat{B}\gamma_{n}|\Phi\rangle
\end{equation}
So $\gamma_{n}$ is able to exchange with any hermitian operators.
\qedhere
\end{proof}

Finally, let's prove some very important equation in quantum
chemistry:
\begin{theorem}\label{DM:3}
For any arbitrary  $\gamma_{n}$, we have:
\begin{equation}\label{}
i \hbar\frac{\partial \gamma_{n}}{\partial t} = [\hat{H},
\gamma_{n}]
\end{equation}
\end{theorem}

\begin{proof}
Suggest that $\gamma_{n} = \ket{\Psi}\bra{\Psi}$, now let's start
from the Schrodinger equation:
\begin{align}\label{}
\hat{H}\ket{\Psi} &= i \hbar \frac{\partial \ket{\Psi}}{\partial t}
\nonumber \\
\bra{\Psi}\hat{H} &= -i \hbar \frac{\partial \bra{\Psi}}{\partial t}
\end{align}
Hence we have:
\begin{equation}\label{}
\begin{split}
  i \hbar\frac{\partial \gamma_{n}}{\partial t} &=
   i \hbar \frac{\partial \ket{\Psi}\bra{\Psi}}{\partial t}\\
    &=
    i \hbar \frac{\partial \ket{\Psi}}{\partial t}\bra{\Psi} +
    \ket{\Psi}i \hbar \frac{\partial \bra{\Psi}}{\partial t} \\
    &= (\hat{H}\ket{\Psi})\bra{\Psi} - \ket{\Psi}(\bra{\Psi}\hat{H})
    \\
    &= [\hat{H}, \gamma_{n}]
\end{split}
\end{equation}
 \qedhere
\end{proof}

%%%%%%%%%%%%%%%%%%%%%%%%%%%%%%%%%%%%%%%%%%%%%%%%%%%%%%%%%%%%%%%%%%%%%%%%%%%%
\section{Reduced density matrices}
%
%

%%%%%%%%%%%%%%%%%%%%%%%%%%%%%%%%%%%%%%%%%%%%%%%%%%%%%%%%%%%%%%%%%%%%%%%%%%%%
\subsection{Definition}
%
%
In quantum chemistry the most useful form is the reduced density
matrices, which is deriving from the common density matrices.

The Pth order of reduced density matrices is defined as:
\begin{multline}\label{DMeq:2}
  \gamma^{p}(x^{'}_{1}, x^{'}_{2}, \cdots, x^{'}_{p}, x_{1}, x_{2},
  \cdots, x_{p}) \\
  = C^{p}_{n}\int \Psi^{*}(x^{'}_{1}, x^{'}_{2}, \cdots, x^{'}_{p},
  x_{p+1},
  \cdots, x_{n})* \\
  \Psi(x_{1}, x_{2}, \cdots, x_{p}, x_{p+1}, \cdots,
  x_{n})dx_{p+1}\cdots dx_{n}
\end{multline}
Here in this definition, we make integration to the electrons of $p+1,
p+2, \cdots, n$ so that to ensure that they are not included in the
expression of the Pth order of reduced density matrices. On the other
hand, since that the electrons are indistinguishable with each other,
there are $C^{p}_{n}$ Pth order reduced density matrices formed by any
$p$ electrons picking up from the total $n$ electrons, they are same
with each other; thus we have to multiply a coefficient of
$C^{p}_{n}$.

Hence we can see that the Pth order of reduced density matrices is not
relied on the specific electrons, its is only depending on the
coordinates and the spin on this specific location:
\begin{equation}\label{}
  x_{i} \Leftrightarrow (\mathbf{r}, s)
\end{equation}
Here in the Pth order of reduced density matrices, the expression
depends on $2P$ independent coordinates (two sets of indices are
included for $x_{i}$ and $x_{i}^{'}$), and the spin state in the
specific location of $\mathbf{r}$ or $\mathbf{r^{'}}$.

Now let's go to see the characters related to the reduced density
matrices.

%%%%%%%%%%%%%%%%%%%%%%%%%%%%%%%%%%%%%%%%%%%%%%%%%%%%%%%%%%%%%%%%%%%%%%%%%%%%
\subsection{Characters of reduced density matrices}
%
%
%
First, the reduced density matrices still preserve the hermitian
property:
\begin{equation}\label{}
  \gamma^{p}(x^{'}_{1}, x^{'}_{2}, \cdots, x^{'}_{p}, x_{1}, x_{2},
  \cdots, x_{p}) = \gamma^{p*}(x_{1}, x_{2}, \cdots,
  x_{p},x^{'}_{1}, x^{'}_{2}, \cdots, x^{'}_{p})
\end{equation}
On the other hand, it's diagonal element is not equal to $1$ any more,
but amount to $C^{p}_{n}$:
\begin{equation}\label{}
  tr(\gamma^{p}) = C^{p}_{n}
\end{equation}

Next, the reduced density matrices are antisymmetric in each set of
indices:
\begin{equation}\label{}
  \gamma^{p}(x^{'}_{1}, x^{'}_{2}, \cdots, x^{'}_{p}, x_{1}, x_{2},
  \cdots, x_{p}) = -\gamma^{p}(x^{'}_{2}, x^{'}_{1},  \cdots,
  x^{'}_{p}, x_{1}, x_{2}, \cdots, x_{p})
\end{equation}
This property ensures that the Pauli principle is still preserved in
the reduced density matrices, thus we can expected an ``hole''
around each position of \vect{r}:
\begin{equation}\label{}
  \gamma^{p}(x^{'}_{1}, x^{'}_{1}, \cdots, x^{'}_{p}, x_{1}, x_{2},
  \cdots, x_{p}) = 0
\end{equation}
That's the famous ``Fermi'' hole to describe the antisymmetry
requirement.

We can also derive the lower order of reduced density matrices from
the higher order one, that is:
\begin{multline}\label{}
  \gamma^{p-1}(x^{'}_{1}, x^{'}_{2}, \cdots, x^{'}_{p-1}, x_{1},
  x_{2},
  \cdots, x_{p-1}) \\
  = \frac{p}{n-p+1}\int \gamma^{p}(x^{'}_{1}, x^{'}_{2}, \cdots,
  x^{'}_{p-1}, x^{'}_{p},
  x_{1}, x_{2}, \cdots, x_{p-1}, x_{p})_{x_{p}=x^{'}_{p}}
  dx_{p}
\end{multline}
That's some interesting property, and can be directly gotten from
the definition of the reduced density matrices. Since that we can
not derive $n-1$ particles wave function from the $n$ particles wave
function. Here we can do this is just because that the independent
variables used is only the coordinates and spin, it's not related to
the electron itself.

As some kind of hermitian operator, the reduced density matrices
also has it's eigen values and eigen functions. For the
$\gamma^{1}$, its eigen function is:
\begin{equation}\label{}
  \int \gamma^{1}(x^{'}_{1},x_{1})\psi^{*}_{i}(x^{'}_{1})dx^{'}_{1} =
  n_{i}\psi_{i}(x_{1})
\end{equation}
Here the $\psi_{i}(x_{1})$ is some single electron orbital, and it's
called natural spin orbital, the $n_{i}$ is called occupation
number. Similarly, for the $\gamma^{2}$, its eigen function is:
\begin{equation}\label{}
  \int \gamma^{2}(x^{'}_{1},x^{'}_{2}, x_{1},
  x_{2})\theta_{i}(x_{1},x_{2})dx_{1}dx_{2} =
  g_{i}\theta^{*}_{i}(x^{'}_{1}, x^{'}_{2})
\end{equation}
Here the $\theta_{i}$ is called natural germinal, and $g_{i}$ is the
occupation number.

The reduced density matrices can be expressed by its eigen functions
as well as its eigen values. Since that for an arbitrary operator of
$\hat{A}$, we have:
\begin{align}\label{}
  \hat{A}\ket{\psi_{i}} &= a_{i}\ket{\psi_{i}} \Rightarrow \nonumber \\
  \hat{A}\ket{\psi_{i}}\bra{\psi_{i}} &=
  a_{i}\ket{\psi_{i}}\bra{\psi_{i}}
  \Rightarrow \nonumber \\
  \hat{A} &= \hat{A}\sum_{i}(\ket{\psi_{i}}\bra{\psi_{i}}) =
  \sum_{i}a_{i}\ket{\psi_{i}}\bra{\psi_{i}}
\end{align}
Thus $\hat{A}$ can be expressed by its set of eigen functions and
eigen values.

Therefore, for the $\gamma^{1}$, if its set of eigen functions has
been found, we can write:
\begin{equation}\label{}
  \gamma^{1} = \sum_{i}n_{i}\psi^{*}_{i}(x^{'}_{1})\psi_{i}(x_{1})
\end{equation}


The benefit to introduce the reduced density matrices, is that such
concept greatly reduces the arguments while to determine the states
of the quantum system. In the traditional quantum mechanics, to
determine the wave functions for the common chemical system it's
necessary to know up to $4n$ variables, $n$ is the number of
electrons. For each electron, all its coordinates and spin state is
needed in specifying the wave functions. On the contrary, in the
reduced density matrices only $4$ variables needed to know to
describe it, that's the coordinates plus the spin state.

On the other hand, we can see that the operators commonly used is
actually expressed on the base of the coordinates, for instance, the
kinetic operator, nucleus-electron operator and the electron
repulsion operator; thus the Hamiltonian for the ordinary chemical
system is working on the coordinates. that means, potentially
there's some possibility to express the system states in a way that
is to avoid the use of $4n$ variables wave functions.



%%%%%%%%%%%%%%%%%%%%%%%%%%%%%%%%%%%%%%%%%%%%%%%%%%%%%%%%%%%%%%%%%%%%%%%%%%%%
\subsection{First order and second order of reduced density matrices}
\label{RDM1_and_EDM2}
%
%
%
Now we are going to present some examples of reduced density
matrices. In quantum chemistry, since we only use single electron
operator and double electron operator; thus it's appropriate to
discuss the expression for first order reduced density matrices and
second order reduced density matrices. The first order reduced
density matrices is:
\begin{align}\label{}
  \gamma^{1}(x^{'}_{1}, x_{1}) &= n\int \Psi^{*}(x^{'}_{1}, x_{2},
  \cdots, x_{p}, x_{p+1},
  \cdots, x_{n})* \nonumber \\
  & \Psi(x_{1}, x_{2}, \cdots, x_{p}, x_{p+1}, \cdots,
  x_{n})dx_{2}dx_{3}\cdots dx_{n}
\end{align}
The second order one is:
\begin{align}\label{}
  \gamma^{2}(x^{'}_{1},x^{'}_{2}, x_{1},x_{2}) &= \frac{n(n-1)}{2}\int
  \Psi^{*}(x^{'}_{1}, x^{'}_{2}, \cdots, x_{p}, x_{p+1},
  \cdots, x_{n})* \nonumber \\
  & \Psi(x_{1}, x_{2}, \cdots, x_{p}, x_{p+1}, \cdots,
  x_{n})dx_{3}dx_{4}\cdots dx_{n}
\end{align}

Here, their diagonal elements have some important physical
interpretations. For the $\gamma^{1}(x_{1}, x_{1})dv$ is the number
of electrons $\times$ the probability of finding an electron within
the volume of $dv$ around the position of \vect{r_{1}} having the
spin of $s$ (here we should remember that $x_{i} \Leftrightarrow
(\mathbf{r_{i}}, s)$), when all the other electrons have arbitrary
position and spin states. It's easy to see that the
$\gamma^{1}(x_{1}, x_{1})dv$ is just the electron density of
$\rho_{\sigma}(\bm{r_{1}})dv$ with spin state of $\sigma$.

Similarly, the $\gamma^{2}(x_{1},x_{2},x_{1},x_{2})dv_{1}dv_{2}$ is
the number of electrons pairs $\times$ the probability of finding an
electron within the volume of $dv_{1}$ around the position of
\vect{r_{1}} having the spin state of $s_{1}$; and the other within
the volume of $dv_{2}$ around the position of \vect{r_{2}} having
the spin state of $s_{2}$; when all the other electrons have
arbitrary position and spin states. Compared with the electron
density of $\rho_{\sigma}(\bm{r_{1}})$, here we can define such
electron density for two electrons as ``pair density'', which is
labeled as $\rho_{\sigma\sigma^{'}}(\bm{r_{1}}, \bm{r_{2}})$, in
which at $\bm{r_{1}}$ it has spin state of $\sigma$, at $\bm{r_{2}}$
it has spin state of $\sigma^{'}$.

What's more, the physical interpretation for the Pth order of
reduced density matrices can be got in the same way.

On the other hand, in literature we can see that there's another way
to get the expressions for electron density and the pair density.
For the electron density, we can define the ``density operator'' of
$\hat{n}_{\sigma}(\bm{r})$ as:
\begin{equation}\label{DMeq:41}
\hat{n}_{\sigma}(\bm{r}) = \sum_{i=1}^{n}\delta(\bm{r} -
\bm{r_{i}})\delta_{\sigma\sigma_{i}}
\end{equation}
From the discussion for the delta function, we know that:
\begin{equation}\label{DMeq:43}
\int f(x^{'})\delta(x^{'}-x) = f(x)
\end{equation}
Hence for the $\langle\Psi|\hat{n}_{\sigma}(\bm{r})|\Psi\rangle$
(the integration is over for each of $\bm{r_{i}}$ and $\sigma_{i}$),
it just equals to the diagonal form of first order reduced density
matrices:
\begin{equation}\label{}
\langle\Psi|\hat{n}_{\sigma}(\bm{r})|\Psi\rangle =
\gamma^{1}(x^{'}_{1}, x_{1})_{x^{'}_{1} = x_{1}} =\rho_{\sigma}(\bm{r})
\end{equation}
It's equivalent to the electron density for all electrons with spin
state of $\sigma$ in location of $\bm{r}$.

How can we understand this form? Now let's suggest that the many
particle wave function of $\Psi(\bm{r_{1}}, \sigma_{1}, \bm{r_{2}},
\sigma_{2}, \cdots, \bm{r_{n}}, \sigma_{n})$ is composed by slater
determinant same with (\ref{HFTeq:45}):
\begin{equation}\label{HFTeq:45}
\Psi =\frac{1}{\sqrt{n!}}
\begin{vmatrix}
\phi_{1}(\bm{r_{1}})\sigma(1) &  \phi_{2}(\bm{r_{1}})\sigma(1) & \cdots & \phi_{n}(\bm{r_{1}})\sigma(1)  \\
\phi_{1}(\bm{r_{2}})\sigma(2) &  \phi_{2}(\bm{r_{2}})\sigma(2) & \cdots & \phi_{n}(\bm{r_{2}})\sigma(2)  \\
\cdots               &  \cdots               & \cdots & \cdots                  \\
\phi_{1}(\bm{r_{n}})\sigma(n) &  \phi_{2}(\bm{r_{n}})\sigma(n) & \cdots & \phi_{n}(\bm{r_{n}})\sigma(n)  \\
\end{vmatrix}
\end{equation}

Then let's evaluate the
$\langle\Psi|\hat{n}_{\sigma}(\bm{r})|\Psi\rangle$:
\begin{equation}\label{}
\begin{split}
\langle\Psi|\hat{n}_{\sigma}(\bm{r})|\Psi\rangle &=
\sum_{i=1}^{n}\langle\Psi|\delta(\bm{r} -
\bm{r_{i}})\delta_{\sigma\sigma_{i}}|\Psi\rangle \\
    &= \frac{1}{n!}\sum_{i=1}^{n}\sum_{p=1}^{n}\int
\phi^{*}_{p}(\bm{r_{i}})\sigma(i)\delta(\bm{r} -
\bm{r_{i}})\phi_{p}(\bm{r_{i}})\sigma(i)\delta_{\sigma\sigma_{i}}
d\bm{r_{i}}d\sigma_{i}\\
&\langle\triangle^{(p, \bm{r_{i}})}|\triangle^{(p,
\bm{r_{i}})}\rangle \\
&=\frac{1}{n}\sum_{i=1}^{n}\sum_{p=1}^{n}\int
\phi^{*}_{p}(\bm{r_{i}})\sigma(i)\delta(\bm{r} -
\bm{r_{i}})\phi_{p}(\bm{r_{i}})\sigma(i)\delta_{\sigma\sigma_{i}}
d\bm{r_{i}}d\sigma_{i} \\
&=\sum_{p=1}^{n}\int \phi^{*}_{p\sigma}(\bm{r^{'}})\delta(\bm{r} -
\bm{r^{'}})\phi_{p\sigma}(\bm{r^{'}})
d\bm{r^{'}} \\
&=\sum_{p=1}^{n}\phi^{*}_{p\sigma}(\bm{r})\phi_{p\sigma}(\bm{r})
\quad \\
&= \rho_{\sigma}(\bm{r})
\end{split}
\end{equation}
Furthermore, it's clear that such derivation can be naturally
extended to the general form of wave functions which is absent with
the single particle restriction.

What's more, we can introduce the operator to generate the pair
density:
\begin{equation}\label{DMeq:42}
\hat{n}_{\sigma\sigma^{'}}(\bm{r}, \bm{r^{'}}) = \sum_{i\neq
j}^{n}\delta(\bm{r} - \bm{r_{i}})\delta_{\sigma\sigma_{i}}
\delta(\bm{r^{'}} - \bm{r_{j}})\delta_{\sigma^{'}\sigma_{j}}
\end{equation}
It's just some natural extension from the (\ref{DMeq:41}). Hence we
can get:
\begin{equation}\label{}
\langle\Psi|\hat{n}_{\sigma\sigma^{'}}(\bm{r},
\bm{r^{'}})|\Psi\rangle = \rho_{\sigma\sigma^{'}}(\bm{r},
\bm{r^{'}})
\end{equation}

Now let's give some discussion for the pair density of
$\rho_{\sigma\sigma^{'}}(\bm{r}, \bm{r^{'}})$. Firstly, since the
$\bm{r}$ and $\bm{r^{'}}$ are arbitrarily defined, since the
exchange between the two index will cause no change; hence we have:
\begin{equation}\label{}
\rho_{\sigma\sigma^{'}}(\bm{r}, \bm{r^{'}}) =
\rho_{\sigma\sigma^{'}}(\bm{r^{'}}, \bm{r})
\end{equation}
The pair density is symmetric. On the other hand, since the pair
density operator is some positive operator
($\hat{n}_{\sigma\sigma^{'}}(\bm{r}, \bm{r^{'}}) \geq 0$), hence we
have the pair density must be larger or equal to zero:
\begin{equation}\label{}
\rho_{\sigma\sigma^{'}}(\bm{r}, \bm{r^{'}}) \geq 0
\end{equation}
Finally, we note that the pair density should also obey the Pauli
principle, that for the case $\bm{r} = \bm{r^{'}}$, the pair density
is zero:
\begin{equation}\label{}
\rho_{\sigma\sigma^{'}}(\bm{r}, \bm{r}) = 0
\end{equation}


%%%%%%%%%%%%%%%%%%%%%%%%%%%%%%%%%%%%%%%%%%%%%%%%%%%%%%%%%%%%%%%%%%%%%%%%%%%%
\subsection{To express the Schrodinger equation via reduced density
  matrices}
%
%
%
%
%
Now let's try to rewrite the Schrodinger equation by the reduced
density matrices. we assume that the Hamiltonian operator for the
ordinary chemical system is:
\begin{equation}\label{DMeq:3}
  \hat{H} = \left\{\sum_{i=1}^{n}(\frac{-1}{2}\times\nabla_{i}^2)+
    \sum_{i=1}^{n}\sum_{\alpha=1}^{N}(\frac{-1}{r_{i\alpha}})+
    \sum\sum_{i<j}(\frac{1}{r_{ij}}) \right\}
\end{equation}
First we bring forward some conventions in treating the reduced
density matrices. The operators are considered to work only on the
unprimed coordinates $x_{i}, x_{j}$ etc., but not on the $x_{i}^{'},
x_{j}^{'}$ etc.; it's just after the operation we put $x_{i} =
x_{i}^{'}$, $x_{j} = x_{j}^{'}$ etc. to get the diagonal elements.

For the kinetic operator, we have:
\begin{align}\label{DMeq:4}
  \langle T\rangle &=
  \bra{\Psi}\sum_{i=1}^{n}(\frac{-1}{2}\times\nabla_{i}^2)\ket{\Psi}
  \nonumber \\
  &=n\int \Psi^{*}(x^{'}_{1}, x_{2}, \cdots,
  x_{n})(\frac{-1}{2}\times\nabla_{1}^2)\Psi(x_{1}, x_{2}, \cdots,
  x_{n})d\tau \nonumber \\
  &=\int \left[\left(\frac{-1}{2}\times\nabla^2\right)
    \gamma^{1}(x^{'}_{1}, x_{1})\right]_{x_{1} = x_{1}^{'}} dx_{1}
  \nonumber \\
  &=\int \left[\left(\frac{-1}{2}\times\nabla^2\right)
    \gamma^{1}(\bm{r^{'}_{1}}, \bm{r_{1}})\right]_{\bm{r_{1}} =
    \bm{r_{1}^{'}}}d\bm{r_{1}}
\end{align}
Here we integrate the spin state contained in the index of $x_{1}$ and
$x^{'}_{1}$ since that the kinetic operator does not contain the spin
operator. Finally, we can see that the kinetic energy is totally
expressed by the first order reduced density matrices.

The nucleus-electron is much simpler:
\begin{align}\label{DMeq:5}
  \langle V_{ne}\rangle &=
  \bra{\Psi}\left(\sum_{i=1}^{n}\sum_{\alpha=1}^{N}(\frac{-1}{r_{i\alpha}}
    )\right)\ket{\Psi}
  \nonumber \\
  &=\sum_{\alpha=1}^{N} n\int \Psi^{*}(x^{'}_{1}, x_{2}, \cdots,
  x_{n})(\frac{-1}{r_{x_{1}\alpha}})\Psi(x_{1}, x_{2}, \cdots,
  x_{n})d\tau \nonumber \\
  &=\sum_{\alpha=1}^{N} n\int
  (\frac{-1}{r_{x_{1}\alpha}})\Psi^{*}(x^{'}_{1}, x_{2}, \cdots,
  x_{n})\Psi(x_{1}, x_{2}, \cdots,
  x_{n})d\tau \nonumber \\
  &=\sum_{\alpha=1}^{N}\int
  \left(\frac{-1}{r_{x_{1}\alpha}}\right)\left[ \gamma^{1}(x^{'}_{1},
    x_{1})\right]_{x_{1} = x_{1}^{'}} dx_{1}
  \nonumber \\
  &=\sum_{\alpha=1}^{N}\int
  \left[\left(\frac{-1}{r_{x_{1}\alpha}}\right) \rho(x_{1})\right]
  dx_{1} \nonumber \\
  &=\sum_{\alpha=1}^{N}\int
  \left[\left(\frac{-1}{r_{x_{1}\alpha}}\right) \rho(\bm{r_{1}})\right]
  d\bm{r_{1}}
\end{align}
Hence we can see that the nucleus-electron energy is purely working on
the electron density.

Finally for the electron repulsion term, we have:
\begin{align}\label{DMeq:6}
  \langle V_{ee}\rangle &=
  \bra{\Psi}\left(\sum\sum_{i<j}(\frac{1}{r_{ij}})\right)\ket{\Psi}
  \nonumber \\
  &= C_{n}^{2}\int \Psi^{*}(x^{'}_{1}, x^{'}_{2}, \cdots,
  x_{n})(\frac{1}{r_{12}})\Psi(x_{1}, x_{2}, \cdots,
  x_{n})d\tau \nonumber \\
  &=C_{n}^{2}\int (\frac{1}{r_{12}})\Psi^{*}(x^{'}_{1}, x^{'}_{2}, \cdots,
  x_{n})\Psi(x_{1}, x_{2}, \cdots,
  x_{n})d\tau \nonumber \\
  &=\int \left(\frac{1}{r_{12}}\right)\left[
    \gamma^{2}(x^{'}_{1},x^{'}_{2}, x_{1},x_{2})\right]_{x_{1} =
    x_{1}^{'}, x_{2} = x_{2}^{'}} dx_{1}dx_{2}
  \nonumber \\
  &=\int \left[\left(\frac{1}{r_{12}}\right)
    \gamma^{2}(x_{1},x_{2})\right] dx_{1}dx_{2} \nonumber \\
  &= \int \left[\left(\frac{1}{r_{12}}\right)
    \rho(\bm{r_{1}},\bm{r_{2}})\right] d\bm{r_{1}}d\bm{r_{2}}
\end{align}
Hence to evaluate the electron repulsion energy we only need the pair
density. Since that the two position of \vect{r_{1}} and \vect{r_{2}}
correlated with each other (including the Fermi effects and the
correlation effects), we can see that this term implicitly contains
the Pauli requirement and the correlation effect.

It's interesting to point out that since the first order of reduced
density matrix can be expressed by the second order one, thus for any
quantum system; if we can know the exact expression for the second
order of reduced density matrices, we can get the exact energy of the
whole system.

All in all, the energy based on the reduced density matrices can be
expressed as:
\begin{multline}\label{DMeq:21}
  E = \int \left[\left(\frac{-1}{2}\times\nabla^2\right)
    \gamma^{1}(\bm{r^{'}_{1}}, \bm{r_{1}})\right]_{\bm{r_{1}} =
    \bm{r_{1}^{'}}} d\bm{r_{1}} + \sum_{\alpha=1}^{N}\int
  \left[\left(\frac{-1}{r_{x_{1}\alpha}}\right) \rho(\bm{r_{1}})\right]
  d\bm{r_{1}} + \\
  \int \left[\left(\frac{1}{r_{12}}\right)
    \rho(\bm{r_{1}}, \bm{r_{2}})\right]d\bm{r_{1}}d\bm{r_{2}}
\end{multline}


Next, we will consider the operator for the electric moment \vect{D}:
\begin{equation}\label{}
  \mathbf{D} = \mathbf{e}\sum_{i}\mathbf{r_{i}}
\end{equation}
Here the $i$ is the electron label. Similar to the nucleus-electron
term, we can finally get the expression by the reduced density
matrices:
\begin{equation}\label{}
  \langle\mathbf{D}\rangle = \mathbf{e}\int
  \mathbf{r_{1}}\gamma(x_{1})dx_{1} = \mathbf{e}\int
  \mathbf{r_{1}}\rho(\bm{r_{1}})d\bm{r_{1}}
\end{equation}
We can see that in the expression above, the expectation value of the
dipole moment is only depending on the electron density as well as the
position in the space.

Finally, we can see that we can use the variation process to determine
the best reduced density matrices which give the lowest energy of
Hamiltonian operator. However, there's some theoretical difficulties
related to this step, yet have not been solved. In the following
content, we can see what's the difficulties behind it.

%%%%%%%%%%%%%%%%%%%%%%%%%%%%%%%%%%%%%%%%%%%%%%%%%%%%%%%%%%%%%%%%%%%%%%%%%%%%%
\section{Spinless reduced density matrices}
% 1 expression for the spinless reduced density matrices 2 examples of
% first order and second order ones 3 expectation values are same 4
% the spin density matrices
%
%
Since that many operators do not involve the spin state, such as the
common Hamiltonian operator shown in the (\ref{DMeq:3}); therefore
it's necessary for us to consider the spinless reduced density
matrices.

The spinless reduced density matrices, is to integrate the spin state
in the expression of Pth order of reduced density matrices:
\begin{multline}\label{DMeq:11}
  \gamma^{p}(\mathbf{r}^{'}_{1}s_{1}, \mathbf{r}^{'}_{2}s_{2}, \cdots,
  \mathbf{r}^{'}_{p}s_{p}, \mathbf{r}_{1}s_{1}, \mathbf{r}_{2}s_{2},
  \cdots, \mathbf{r}_{p}s_{p}) \\
  \shoveleft{= C^{p}_{n}\int \Psi^{*}(\mathbf{r}^{'}_{1}s_{1},
    \mathbf{r}^{'}_{2}s_{2}, \cdots, \mathbf{r}^{'}_{p}s_{p},
    \mathbf{r}_{p+1}s_{p+1},
    \cdots, \mathbf{r}_{n}s_{n})*} \\
  \Psi(\mathbf{r}_{1}s_{1}, \mathbf{r}_{2}s_{2}, \cdots,
  \mathbf{r}_{p}s_{p}, \mathbf{r}_{p+1}s_{p+1}, \cdots,
  \mathbf{r}_{n}s_{n}) \\
  ds_{1}ds_{2}\cdots ds_{p}d\mathbf{r}_{p+1}ds_{p+1}\cdots
  d\mathbf{r}_{n}ds_{n}
\end{multline}
Here we can see that for the two indices of $x_{i}$ and
$x^{'}_{i}$, they share the same spin states; this is because that the
wave functions with different spin will automatically orthogonal to
each other, thus for the spinless density matrices, only the
contribution from the same spin state can be left here.

Hence, the first order of spinless reduced density matrices is:
\begin{multline}\label{}
  \gamma^{1}(\mathbf{r}^{'}_{1}, \mathbf{r}_{1}) \\
  \shoveleft{= n \int \Psi^{*}(\mathbf{r}^{'}_{1}s_{1},
    \mathbf{r}_{2}s_{2}, \cdots, \mathbf{r}_{n}s_{n})*
    \Psi(\mathbf{r}_{1}s_{1}, \mathbf{r}_{2}s_{2}, \cdots,
    \mathbf{r}_{n}s_{n}) } \\
  ds_{1}d\mathbf{r}_{2}ds_{2}\cdots d\mathbf{r}_{n}ds_{n}
\end{multline}

The second order of spinless reduced density matrices is:
\begin{multline}\label{}
  \gamma^{2}(\mathbf{r}^{'}_{1}, \mathbf{r}^{'}_{2},
  \mathbf{r}_{1}, \mathbf{r}_{2}) \\
  \shoveleft{= C^{2}_{n}\int \Psi^{*}(\mathbf{r}^{'}_{1}s_{1},
    \mathbf{r}^{'}_{2}s_{2}, \mathbf{r}_{3}s_{3}, \cdots,
    \mathbf{r}_{n}s_{n})* \Psi(\mathbf{r}_{1}s_{1},
    \mathbf{r}_{2}s_{2}, \mathbf{r}_{3}s_{3}, \cdots,
    \mathbf{r}_{n}s_{n}) }\\
  ds_{1}ds_{2}d\mathbf{r}_{3}ds_{3}\cdots d\mathbf{r}_{n}ds_{n}
\end{multline}

Here we can see that the expression for the spinless ones are similar
to their general form. For the operators which does not involve in the
spin state, it's only needed to integrate the spin state in the
expectation expression; therefore by changing the indices of $x_{i}$
and $x^{'}_{i}$ into the $\mathbf{r}_{i}$ and $\mathbf{r}^{'}_{i}$; we
can arrive at the expression for the spinless ones.

Sometimes it's convenient to have the spinless reduced density
matrices decomposed into the polarized components, which arising from
the different spin state. For the first order of reduced density
matrices, we have:
\begin{equation}\label{}
  \gamma^{1}(\mathbf{r}^{'}_{1}, \mathbf{r}_{1}) =
  \gamma^{1}_{\alpha\alpha}(\mathbf{r}^{'}_{1}, \mathbf{r}_{1}) +
  \gamma^{1}_{\beta\beta}(\mathbf{r}^{'}_{1}, \mathbf{r}_{1})
\end{equation}
Thus the first order of reduced density matrices can be regard as the
direct sum between the $\alpha$ part and the $\beta$ part.

For the second order ones, it's a bit of complex; since that for the
location of \vect{r_{1}}, \vect{r_{2}}, \vect{r^{'}_{1}} and
\vect{r^{'}_{2}} we may have different spin state corresponding to
them. Generally to say, the second order ones can be split into:
\begin{align}\label{}
  \gamma^{2}(\mathbf{r}^{'}_{1},\mathbf{r}^{'}_{2},
  \mathbf{r}_{1},\mathbf{r}_{2}) &= \nonumber \\
  &\gamma^{2}_{\alpha\alpha,\alpha\alpha}(\mathbf{r}^{'}_{1},\mathbf{r}^{'}_{2},
  \mathbf{r}_{1},\mathbf{r}_{2}) +
  \gamma^{2}_{\beta\beta,\beta\beta}(\mathbf{r}^{'}_{1},\mathbf{r}^{'}_{2},
  \mathbf{r}_{1},\mathbf{r}_{2}) +\nonumber \\
  &\gamma^{2}_{\alpha\beta,\alpha\beta}(\mathbf{r}^{'}_{1},\mathbf{r}^{'}_{2},
  \mathbf{r}_{1},\mathbf{r}_{2}) +
  \gamma^{2}_{\beta\alpha,\beta\alpha}(\mathbf{r}^{'}_{1},\mathbf{r}^{'}_{2},
  \mathbf{r}_{1},\mathbf{r}_{2})
\end{align}
The other terms, such as $\gamma^{2}_{\alpha\alpha,\beta\alpha}$
etc. does not exist. This is because the spin state for the
$\mathbf{r}_{i}$ and $\mathbf{r}^{'}_{i}$ should be same. So only
such four terms left.

%%%%%%%%%%%%%%%%%%%%%%%%%%%%%%%%%%%%%%%%%%%%%%%%%%%%%%%%%%%%%%%%%%%%%%%%%%%%%%
\section{Hole function derivation}
%
% 1 hole function portrays the relations between electron density and
% electron pair density
% 2 get the hole function expression
% 3 define the exchange and correlation holes
% 4 the importance of the exchange and correlation holes
%
In the above content, we have raised the concept of electron density
and electron pair density, which are represented by $\rho(\bm{r_{1}})$
and $\rho(\bm{r_{1}},\bm{r_{2}})$, respectively; now in this section
we are going to further investigate some relation between them, which
is called ``hole function''.

%%%%%%%%%%%%%%%%%%%%%%%%%%%%%%%%%%%%%%%%%%%%%%%%%%%%%%%%%%%%%%%%%%%%%%%%%%%%%%
\subsection{Derive hole function from density operator}
\label{DHF_in_density_matrices}
%
%
%
%
%
Firstly, let's start from the density operator as well as the pair
density operator\footnote{This derivation is taken from the reference
  written by Barth\cite{2004PhST..109....9V}}:
\begin{equation}
\label{DMeq:44}
\begin{split}
  \int \hat{n}_{\sigma\sigma^{'}}(\bm{r}, \bm{r^{'}})d^{3}\bm{r^{'}} &=
  \sum_{i\neq j}^{n}\delta(\bm{r} - \bm{r_{i}})
  \delta_{\sigma\sigma_{i}}\delta_{\sigma^{'}\sigma_{j}}
  \int\delta(\bm{r^{'}} - \bm{r_{j}})d^{3}\bm{r^{'}} \\
  &= \sum_{i\neq j}^{n}\delta(\bm{r} -
  \bm{r_{i}})\delta_{\sigma\sigma_{i}}
  \delta_{\sigma^{'}\sigma_{j}} \\
  &= \sum_{i}^{n}\sum_{j}^{n}\delta(\bm{r} -
  \bm{r_{i}})\delta_{\sigma\sigma_{i}}
  \delta_{\sigma^{'}\sigma_{j}} - \sum_{i}^{n}\delta(\bm{r} -
  \bm{r_{i}})\delta_{\sigma\sigma_{i}}
  \delta_{\sigma^{'}\sigma_{i}} \\
  &= \sum_{i}^{n}\delta(\bm{r} -
  \bm{r_{i}})\delta_{\sigma\sigma_{i}}
  \sum_{j}^{n}\delta_{\sigma^{'}\sigma_{j}} -
  \hat{n}_{\sigma}\delta_{\sigma\sigma^{'}} \\
  &= \hat{n}_{\sigma}\hat{N}_{\sigma^{'}} -
  \hat{n}_{\sigma}\delta_{\sigma\sigma^{'}}
\end{split}
\end{equation}
Now let's give some explanation to the above derivation. Such
derivation is totally mathematical by starting from the definition of
the density operator as well as the pair density operator, so it
guarantees the generality of the result.

The general idea is to set up some connection between the pair density
operator and the density operator, hence we make integration over the
$\hat{n}_{\sigma\sigma^{'}}(\bm{r}, \bm{r^{'}})$. From the property of
delta function, we arrive at some expression with two sum symbol with
$i\neq j$. Then we try to convert the expression to make it closer to
the form of density operator, so that to get the final result. In the
above derivation, we have:
\begin{equation}
 \begin{split}
  \hat{n}_{\sigma} &= \sum_{i}^{n}\delta(\bm{r} -
  \bm{r_{i}})\delta_{\sigma\sigma_{i}} \\
  \hat{N}_{\sigma^{'}} &= \sum_{j}^{n}\delta_{\sigma^{'}\sigma_{j}}
 \end{split}
\end{equation}

Now by using the (\ref{DMeq:44}) we can further evaluate the relation
between the pair density and the density\footnote{Here we should go 
to Page 12 of reference\cite{2004PhST..109....9V} for more explanations. 
The $n_{\sigma^{'}}$ in the final result needs more consideration. Here
we just follow his derivation.}:
\begin{equation}
\begin{split}
  \bra{\Psi}\int\hat{n}_{\sigma\sigma^{'}}(\bm{r},
  \bm{r^{'}})d^{3}\bm{r^{'}} \ket{\Psi} &=
  \bra{\Psi}\hat{n}_{\sigma}\hat{N}_{\sigma^{'}} -
  \hat{n}_{\sigma}\delta_{\sigma\sigma^{'}}\ket{\Psi} \\
&= \rho_{\sigma}(\bm{r})n_{\sigma^{'}} -
\rho_{\sigma}(\bm{r})\delta_{\sigma\sigma^{'}}
\end{split}
\label{DMeq:45}
\end{equation}


For the pair density, we have:
\begin{align}
  \label{DMeq:46}
  \bra{\Psi}\int\hat{n}_{\sigma\sigma^{'}}(\bm{r},
  \bm{r^{'}})d^{3}\bm{r^{'}} \ket{\Psi} &= \int
  \bra{\Psi}\hat{n}_{\sigma\sigma^{'}}(\bm{r},
  \bm{r^{'}}) \ket{\Psi} d^{3}\bm{r^{'}} \nonumber \\
  &= \int \rho_{\sigma\sigma^{'}}(\bm{r}, \bm{r^{'}}) d^{3}\bm{r^{'}}
\end{align}
Hence the (\ref{DMeq:45}) is converted into:
\begin{equation}
  \label{DMeq:47}
  \int \rho_{\sigma\sigma^{'}}(\bm{r}, \bm{r^{'}}) d^{3}\bm{r^{'}} =
\rho_{\sigma}(\bm{r})n_{\sigma^{'}} -
\rho_{\sigma}(\bm{r})\delta_{\sigma\sigma^{'}}  
\end{equation}

Furthermore, if we express the pair density as:
\begin{equation}
  \label{DMeq:48}
  \rho_{\sigma\sigma^{'}}(\bm{r}, \bm{r^{'}}) =
  \rho_{\sigma}(\bm{r})\rho_{\sigma^{'}}(\bm{r^{'}})
  g_{\sigma\sigma^{'}}(\bm{r}, \bm{r^{'}})
\end{equation}
Here the function of $g$ characterizes the correlation between the two
spots in the $\bm{r}$ and $\bm{r^{'}}$. By using this expression, we
can see that (\ref{DMeq:45}) finally turns into:
\begin{equation}
  \label{DMeq:49}
  \begin{split}
    \rho_{\sigma}(\bm{r})\int\rho_{\sigma^{'}}(\bm{r^{'}})
  g_{\sigma\sigma^{'}}(\bm{r}, \bm{r^{'}})d^{3}\bm{r^{'}} &=
\rho_{\sigma}(\bm{r})n_{\sigma^{'}} -
\rho_{\sigma}(\bm{r})\delta_{\sigma\sigma^{'}} \Rightarrow \\
\int\rho_{\sigma^{'}}(\bm{r^{'}})
  g_{\sigma\sigma^{'}}(\bm{r}, \bm{r^{'}})d^{3}\bm{r^{'}} &=
n_{\sigma^{'}} - \delta_{\sigma\sigma^{'}}
  \end{split}
\end{equation}

By further consideration that
\begin{equation}
  \label{DMeq:50}
  n_{\sigma^{'}} = \int \rho_{\sigma^{'}}(\bm{r^{'}}) d^{3}\bm{r^{'}}
\end{equation}

Then we can arrive at the final conclusion:
\begin{equation}
  \label{DMeq:51}
  \int\rho_{\sigma^{'}}(\bm{r^{'}})
  (g_{\sigma\sigma^{'}}(\bm{r}, \bm{r^{'}}) - 1)d^{3}\bm{r^{'}} =
- \delta_{\sigma\sigma^{'}}
\end{equation}

Now let's consider the case that $\sigma = \sigma^{'}$, then we
have:
\begin{equation}\label{DMeq:52}
\int\rho_{\sigma}(\bm{r^{'}})
(g_{\sigma}(\bm{r}, \bm{r^{'}}) -
1)d^{3}\bm{r^{'}} = - 1
\end{equation}

On the other hand, if we consider the case that $\sigma \neq
\sigma^{'}$, then the (\ref{DMeq:51}) is:
\begin{equation}
\label{DMeq:53}
\int\rho_{\sigma^{'}}(\bm{r^{'}})
  (g_{\sigma\sigma^{'}}(\bm{r}, \bm{r^{'}}) - 1)d^{3}\bm{r^{'}} =
0
\end{equation}

Finally, let's consider the spinless pair density where:
\begin{align}\label{DMeq:54}
\rho(\bm{r}, \bm{r^{'}}) &=
\sum_{\sigma\sigma^{'}}\rho_{\sigma\sigma^{'}}(\bm{r}, \bm{r^{'}})
\nonumber \\
&=\sum_{\sigma\sigma^{'}}\rho_{\sigma}(\bm{r})
\rho_{\sigma^{'}}(\bm{r^{'}})g_{\sigma\sigma^{'}}(\bm{r},
\bm{r^{'}}) \nonumber \\
&=\rho(\bm{r}) \rho(\bm{r^{'}})g(\bm{r}, \bm{r^{'}})
\end{align}

Hence we apply the summation over all the spin states for the
(\ref{DMeq:51}), the result turns into:
\begin{equation}\label{DMeq:55}
  \int\rho(\bm{r^{'}})
  (g(\bm{r}, \bm{r^{'}}) - 1)d^{3}\bm{r^{'}} =
- \sum_{\sigma\sigma^{'}}\delta_{\sigma\sigma^{'}} = -1
\end{equation}
This result is called ``sum rule'' for the hole function.

Now how can we understand these results physically? Let's
back to the pair density to use the 
\ref{DMeq:48}, now we can have:
\begin{equation}
 \begin{split}
 \int \rho(\bm{r}, \bm{r^{'}}) d \bm{r^{'}}
&= \int \rho(\bm{r})\rho(\bm{r^{'}})
  g(\bm{r}, \bm{r^{'}}) d \bm{r^{'}} \\
&= \int \rho(\bm{r})\rho(\bm{r^{'}})
  (g(\bm{r}, \bm{r^{'}}) - 1 + 1) d \bm{r^{'}} \\
&=\int \rho(\bm{r})\rho(\bm{r^{'}})d \bm{r^{'}} +
\int \rho(\bm{r})\rho(\bm{r^{'}})
  (g(\bm{r}, \bm{r^{'}}) - 1) d \bm{r^{'}} \\
&= \rho(\bm{r})n-\rho(\bm{r}) \\
&= \rho(\bm{r})(n-1)
 \end{split}
 \label{DM_hole_function_explanation_1}
\end{equation}

On the other hand, if we neglect the $g(\bm{r}, \bm{r^{'}})$ in the pair
density, which means; we set it to $1$; then physically the total density will
be the direct product of the density pieces. Then we have:
\begin{equation}
\begin{split}
  \int \rho(\bm{r}, \bm{r^{'}}) d \bm{r^{'}}
&= \int \rho(\bm{r})\rho(\bm{r^{'}}) d \bm{r^{'}} \\
&=\rho(\bm{r})\int \rho(\bm{r^{'}}) d \bm{r^{'}} \\
&= \rho(\bm{r})n
\end{split}
\label{DM_hole_function_explanation_2}
\end{equation}

The results show in the (\ref{DM_hole_function_explanation_1}) and
(\ref{DM_hole_function_explanation_2}) reveal the physical meaning of the hole
function; and more precisely; the physical meaning of $g(\bm{r}, \bm{r^{'}})$
in the pair density. \textbf{It's because of the $g(\bm{r}, \bm{r^{'}})$, the
overall probability that we can find electrons in the $\bm{r^{'}}$ decrease to
$n-1$ compared to the completely uncorrelated system, which is $n$. It seems
that the reference point of $\bm{r}$ is digging some ``hole'' that to
``retain'' some electron density so that there's depletion of electron
density at $\bm{r^{'}}$.} This is the physical meaning of the hole function.

This hole is also called ``exchange-correlation'' hole in the density
functional theory. Later we can see, that the hole describing the exchange
effects are integrated to $-1$, and the correlation hole integrates to $0$.
However, it does not mean that \ref{DMeq:52} is the exchange hole and
\ref{DMeq:53} is the correlation hole! The exchange energy, as what we can 
see; is always originated from the same spin wave functions; however; the
correlation energy, is both contributed from same spin wave functions and
opposite spin wave functions. Hence the \ref{DMeq:53} is only a part of 
correlation hole, that's what we need to pay attention to.





%%%%%%%%%%%%%%%%%%%%%%%%%%%%%%%%%%%%%%%%%%%%%%%%%%%%%%%%%%%%%%%%%%%%%%%%%%%%%%
\subsection{Another derivation for the hole function}
%
%
%
%
%
Let's consider the spinless density matrices for the first order and
second order:
\begin{equation}\label{DMeq:12}
  \gamma^{1}(\bm{r_{1}^{'}}, \bm{r_{1}}) = \dfrac{2}{n-1}\int
  \gamma^{2}(\bm{r_{1}^{'}},\bm{r_{2}},
  \bm{r_{1}},\bm{r_{2}})d\bm{r_{2}}
\end{equation}
If we take the diagonal form, the expression above will be some
relation between the electron density and the pair density:
\begin{equation}
\rho_{1}(\bm{r_{1}}) = \dfrac{2}{n-1}\int
\rho_{2}(\bm{r_{1}},\bm{r_{2}})d\bm{r_{2}}
\end{equation}
Here we have to remember some important fact that in (\ref{DMeq:12})
the variables of $\bm{r_{1}}$ and $\bm{r_{2}}$ should be always
different since that they are representing different electrons (see
the definition of second order reduced density matrices, we multiply
some constant of $C^{2}_{n}$).

Now let's go to see how to express the pair density. In the
(\ref{DMeq:48}) we partition the density as:
\begin{equation}\label{}
\rho_{2}(\bm{r_{1}},\bm{r_{2}}) = \rho(\bm{r_{1}})
\rho(\bm{r_{2}})g(\bm{r_{1}}, \bm{r_{2}})
\end{equation}
This is for the spinless case. However, in this expression we can
see that the variable of $\bm{r_{1}}$ and $\bm{r_{2}}$ are
symmetric, they may equal to each other; hence in such pair density
expression we have to multiply the factor of $\frac{1}{2}$ to
prevent double counting:
\begin{equation}\label{DMeq:14}
\rho_{2}(\bm{r_{1}},\bm{r_{2}}) = \frac{1}{2}\rho(\bm{r_{1}})
\rho(\bm{r_{2}})g(\bm{r_{1}}, \bm{r_{2}})
\end{equation}
Then we can apply this expression into the (\ref{DMeq:48}).

Now we can combine the (\ref{DMeq:14}) and (\ref{DMeq:12}) together,
we will have:
\begin{align}
  \frac{n-1}{2}\rho(\bm{r_{1}}) &= \int
  \frac{1}{2}\rho(\bm{r_{1}})\rho(\bm{r_{2}})g(\bm{r_{1}}, \bm{r_{2}})
  d\bm{r_{2}}  \nonumber \\
  n-1 &= \int \rho(\bm{r_{2}})g(\bm{r_{1}}, \bm{r_{2}}) d\bm{r_{2}}
\end{align}

Since that $\int \rho(\bm{r_{2}})d\bm{r_{2}} = n$, $n$ is the number
of electrons; we can arrive at the final result:
\begin{equation}
  \label{DMeq:15}
  \int\rho(\bm{r_{2}})(g(\bm{r_{1}}, \bm{r_{2}}) - 1) d\bm{r_{2}} = -1
\end{equation}
If we write the $g(\bm{r_{1}}, \bm{r_{2}}) - 1$ as 
$\widetilde{h}(\bm{r_{1}}, \bm{r_{2}})$, then we can have:
\begin{equation}
   \label{DM_xc_hole_sum_rule}
  \int\rho(\bm{r_{2}})\widetilde{h}(\bm{r_{1}}, \bm{r_{2}})
 d\bm{r_{2}} = -1
\end{equation}

This is the sum rule which we have derived from some more general
formula (see \ref{DMeq:55}). It demonstrates the average interactions
between the reference location of $\bm{r_{1}}$ and all the other
positions of $\bm{r_{2}}$ for the electron density.

The concept of exchange and correlation holes possess some kind of
importance in the density functional theory. Because that such
concepts demonstrated what kind of behavior the density should be. If
the correct density functionals are employed, the result density
should obey the rules mentioned above.

%%%%%%%%%%%%%%%%%%%%%%%%%%%%%%%%%%%%%%%%%%%%%%%%%%%%%%%%%%%%%%%%%%%%%%%%%%%%%
\section{The N-representability of the reduced density matrices}
%
% 1 sufficient and necessary condition of the reduced density matrices
% 2 why the sufficient condition is vital for the variation process 3
% the result for the gamma_{1} 4 the state for the gamma_{2}
%
%
From the discussion above, we can see that the reduced density
matrices actually is some kind of contraction for the wave
functions. Inside the reduced density matrices, some information
beneath the wave functions has been wrapped up so that we can not
trace it any more. However, in this section we can see that this has
brought some difficulties for further extending the use of reduced
density matrices.

For an arbitrary Nth order of reduced density matrices of
$\gamma^{n}$, if we can find some wave functions contracting to it;
then we can get $\gamma^{1}$ and $\gamma^{2}$ from the $\gamma^{n}$,
finally the variation process can be utilized on the total energy
based on the $\gamma^{1}$ and $\gamma^{2}$.

Here we note that the above condition actually is necessary to apply
the variation principle. If a set of $\gamma^{n}$ can be found that
they are all contracting from the wave functions of the
corresponding system; then the variation process can be strike up to
find the best reduced density matrices which is to minimize the
energy of the system. Furthermore, the result reduced density
matrices can also be used to evaluate the other system properties.
Since that all the reduced density matrices are originated from the
wave functions, this guarantees that such result is able to
correspond to the true wave functions of the system. Hence, to find
out the condition that the requirement for the $\gamma^{n}$ to be
contracted from the wave function is something vital.

So far what we have discussed is the sufficient condition to get
$\gamma^{1}$ and $\gamma^{2}$, the necessary condition for the
$\gamma^{1}$ and $\gamma^{2}$ is that whether $\gamma^{1}$ and
$\gamma^{2}$ can be contracted from the $\gamma^{n}$. Here it's said
that $\gamma^{1}$ and $\gamma^{2}$ satisfy the sufficient condition
will lead to an energy higher than the true energy, and the
$\gamma^{1}$ and $\gamma^{2}$ satisfy the necessary condition will
lead to the energy lower than the true energy\cite{weitaoYang}. If
the $\gamma^{1}$ and $\gamma^{2}$ satisfies both of the sufficient
condition and the necessary condition, it's said that $\gamma^{1}$
and $\gamma^{2}$ are N-representable. Hence, we can make use of such
$\gamma^{1}$ and $\gamma^{2}$ to get the system energy by variation
process.

The condition for the $\gamma^{1}$ to be N-representable has been
found, that if $\bra{\psi_{i}}$ is some normalized spin orbital
state, that $\gamma^{1}$ is N-representable as long as to satisfy
the condition below:
\begin{equation}\label{}
  0 \leq \int
  \psi_{i}^{*}(x_{1}^{'})\gamma^{1}(x_{1}^{'},x_{1})\psi_{i}(x_{1})dx_{1}dx_{1}^{'}
  \leq 1
\end{equation}
This is equivalent to the requirement that eigenvalue for the
$\gamma^{1}$ (the occupation number) of $n_{i}$ should fall into the
range of $0$ to $1$:
\begin{equation}\label{}
  0 \leq n_{i} \leq 1
\end{equation}

However, it's difficult to get the N-representable condition for the
$\gamma^{2}$; even the sufficient condition for the $\gamma^{2}$ is
still too difficult so that it's not reachable yet. Therefore, so
far it's impossible to make variation process on the energy
expression based on the reduced density matrices.


%%%%%%%%%%%%%%%%%%%%%%%%%%%%%%%%%%%%%%%%%%%%%%%%%%%%%%%%%%%%%%%%%%%%%%%%%%%
\section{Hatree-Fock theory in reduced density matrices form}
% 1 how to express the density matrices based on the HF wave function
% 2 idempotent character 3 energy expression
%
\subsection{Expression of the reduced density matrices}
%
%
In the above content, the reduced density matrices has been discussed
generally, now in this section we will see how to express the reduced
density matrices within the Hatree-Fock framework.

First, on the base of the single electron hypothesis; the first
order of reduced density matrices can be directly gotten from the
definition (here we note that by following the above definition, the
label of $x_{i}$ denotes some combination of spin variable and the
coordinate variable; so the spin part is not explicitly given)\label{DM:1}:
\begin{align}\label{DMeq:10}
  \gamma^{1}(x^{'}_{1},x_{1})
      &= \frac{1}{n!}\times n\int \Psi^{*}(x^{'}_{1}, x_{2},
  \cdots, x_{p}, x_{p+1},
  \cdots, x_{n})* \nonumber \\
  & \Psi(x_{1}, x_{2}, \cdots, x_{p}, x_{p+1}, \cdots,
  x_{n})dx_{2}dx_{3}\cdots dx_{n} \nonumber \\
      &= \frac{1}{(n-1)!}\int \begin{vmatrix}
       \psi^{*}_{1}(x^{'}_{1}) & \psi^{*}_{2}(x^{'}_{1}) & \cdots & \psi^{*}_{n}(x^{'}_{1}) \\
       \psi^{*}_{1}(x_{2}) & \psi^{*}_{2}(x_{2}) & \cdots & \psi^{*}_{n}(x_{2}) \\
       \cdots & \cdots & \cdots & \cdots \\
       \psi^{*}_{1}(x_{n}) & \psi^{*}_{2}(x_{n}) & \cdots & \psi^{*}_{n}(x_{n}) \\
     \end{vmatrix}\times \nonumber \\
     &\begin{vmatrix}
       \psi_{1}(x_{1}) & \psi_{1}(x_{2}) & \cdots & \psi_{1}(x_{n}) \\
       \psi_{2}(x_{1}) & \psi_{2}(x_{2}) & \cdots & \psi_{2}(x_{n}) \\
       \cdots & \cdots & \cdots & \cdots \\
       \psi_{n}(x_{1}) & \psi_{n}(x_{2}) & \cdots & \psi_{n}(x_{n}) \\
     \end{vmatrix} dx_{2}dx_{3}\cdots dx_{n} \nonumber \\
  &=\sum_{i=1}^{n}\psi^{*}_{i}(x^{'}_{1})\psi_{i}(x_{1})
\end{align}
Here $\varphi_{i}(x^{'}_{1})$ indicates that the $i$th orbital's
value on the coordinate of $x^{'}_{1}$(here the spin state is
obviously same for the $\mathbf{r^{'}_{1}}$ and
$\mathbf{\bm{r_{1}}}$, else both of the orbitals will be orthogonal
with each other).

Similarly, we can get the Pth order of reduced density matrices:
\begin{multline}\label{DMeq:7}
  \gamma^{p}(x^{'}_{1}, x^{'}_{2}, \cdots, x^{'}_{p}, x_{1}, x_{2},
  \cdots, x_{p}) \\
  = \frac{1}{p!}
  \begin{vmatrix}
    \gamma^{1}(x^{'}_{1},x_{1}) & \gamma^{1}(x^{'}_{1},x_{2}) & \cdots
    &
    \gamma^{1}(x^{'}_{1},x_{p}) \\
    \gamma^{1}(x^{'}_{2},x_{1}) & \gamma^{1}(x^{'}_{2},x_{2}) & \cdots
    &
    \gamma^{1}(x^{'}_{2},x_{p}) \\
    \cdots & \cdots & \cdots &
    \cdots \\
    \gamma^{1}(x^{'}_{p},x_{1}) & \gamma^{1}(x^{'}_{p},x_{2}) & \cdots
    &
    \gamma^{1}(x^{'}_{p},x_{p}) \\
  \end{vmatrix}
\end{multline}
It's interesting to see that in the single electron approximation
that all the higher order of density matrices are constructed by the
first order ones. Actually this is some inherent requirement by the
single electron approximation.

Now we will give the example of the second order of reduced density
matrices to see that how we can get the expression of (\ref{DMeq:7}).

First, we can suggest that in the $\Psi^{*}$ the electron $1$ always
resides on position of $x_{1}^{'}$ (in spin state of $s_{1}$), and
the electron $2$ resides on the position of $x_{2}^{'}$ (in spin
state of $s_{2}$); on the other hand, in the $\Psi$ the electron $1$
and $2$ reside on the position of $x_{1}$ and $x_{2}$, respectively.
However, since that in the reduced density matrices the electron
label does not exist; thus here it's only ``representative'' to show
that in such position we have electron here but without caring that
which one it is.

Now we begin the job. Firstly, for any multi-electron wave functions
under Hatree-Fock framework, we can all have:
\begin{equation}\label{}
  \Psi = \dfrac{1}{\sqrt{n}}\sum^{n}_{1 \leq i<j \leq n}\left|
    \begin{array}{cc}
      \psi_{i}(1) & \psi_{j}(1) \\
      \psi_{i}(2) & \psi_{j}(2) \\
    \end{array}
  \right|
  \Delta^{
    i,\,1 \choose
    j,\,2 }
  (3,4,\cdots,n)
\end{equation}
This is from the (\ref{HFTeq:16}). However, for deriving the
(\ref{DMeq:7}) it's better to rewrite it into the form that:
\begin{equation}\label{}
  \Psi = \dfrac{1}{\sqrt{n}}\sum^{n}_{1 \leq i<j \leq n}\left|
    \begin{array}{cc}
      \psi_{i}(1) & \psi_{i}(2) \\
      \psi_{j}(1) & \psi_{j}(2) \\
    \end{array}
  \right|
  \Delta^{
    i,\,1 \choose
    j,\,2 }
  (3,4,\cdots,n)
\end{equation}

If we replace the electron label with the position variables of
$x_{i}$ and $x_{i}^{'}$, we can see that the second order reduced
density matrices can be expressed as:
\begin{multline}\label{}
  \gamma^{2}(x^{'}_{1},x^{'}_{2},x_{1},x_{2}) = \\
  \shoveleft{C^{2}_{n}\frac{1}{n!}\times\int \sum^{n}_{1 \leq i<j \leq n}
     \left|
      \begin{array}{cc}
        \psi^{*}_{i}(x^{'}_{1}) &  \psi^{*}_{j}(x^{'}_{1})  \\
        \psi^{*}_{i}(x^{'}_{2}) & \psi^{*}_{j}(x^{'}_{2})   \\
      \end{array}
    \right| \Delta^{ i,\,1 \choose j,\,2 }
    (3,4,\cdots,n)\times} \\
  \sum^{n}_{1 \leq i^{'}<j^{'} \leq n}\left|
    \begin{array}{cc}
      \psi_{i^{'}}(x_{1}) & \psi_{i^{'}}(x_{2}) \\
      \psi_{j^{'}}(x_{1}) & \psi_{j^{'}}(x_{2}) \\
    \end{array}
  \right| \Delta^{ i^{'},\,1 \choose j^{'},\,2 } (3,4,\cdots,n)
  dx_{3}x_{4}\cdots dx_{n}
\end{multline}
Here in this expression, we can see that if the label of $i$ and $j$
have been fixed in the bra, the $i^{'}$ and $j^{'}$ should have the
relation that $i=i^{'}$, $j=j^{'}$ (or $i=j^{'}$, $j=i^{'}$),
respectively; else the integration for the complement minor of
$\Delta$ will be zero due to the orthogonal condition. Thus such
analysis leads to an very important conclusion:
\begin{equation}\label{DMeq:8}
  \gamma^{2} = \frac{1}{2!}\sum^{n}_{1 \leq i<j \leq
    n}
\left|A^{*}\begin{pmatrix}
             x^{'}_{1} & x^{'}_{2} \\
             i & j \\
           \end{pmatrix}
\right|
\left|A\begin{pmatrix}
             i & j \\
             x_{1} & x_{2} \\
           \end{pmatrix}
\right|
\end{equation}
Here we use the $A$ to represent the determinant:
\begin{equation}\label{}
A\begin{pmatrix}
             i & j \\
             x^{'}_{1} & x^{'}_{2} \\
           \end{pmatrix} = \left[
      \begin{array}{cc}
        \psi_{i}(x_{1}) &  \psi_{i}(x_{2})  \\
        \psi_{j}(x_{1}) &  \psi_{j}(x_{2})   \\
      \end{array}
    \right]
\end{equation}
The first row indicates the label for the orbital it changes as the
row of matrix changes; the second row is the variable of $x_{i}$,
which is composed by $x_{i} = (\mathbf{r_{i}}, s_{i})$ ($s_{i}$ is
the spin variable); it changes as the matrix column changes. Hence,
the orbitals are changing in rows, while the variables of $x_{i}$ is
changing in column format in the matrix. Finally, $A^{*}$ indicates
that the matrix element is in bra space.

The conclusion in the (\ref{DMeq:8}) can be similarly expanded into
the Pth order of reduced density matrices:
\begin{equation}\label{DMeq:9}
  \gamma^{p} = \frac{1}{p!}\sum^{n}_{1 \leq i_{1}<i_{2}<\cdots<i_{p}
    \leq n}\left|A^{*}\begin{pmatrix}
                        x^{'}_{1} & x^{'}_{2} & \cdots & x^{'}_{p} \\
                        i_{1} & i_{2} & \cdots & i_{p} \\
                      \end{pmatrix}
\right| \left|A\begin{pmatrix}
                        i_{1} & i_{2} & \cdots & i_{p} \\
                        x_{1} & x_{2} & \cdots & x_{p} \\
                      \end{pmatrix}
\right|
\end{equation}
Here the derivation for the Pth order of reduced density matrices is
same with the second order one.

In this step, the most important thing for us to do is to eliminate
the integration operation. So far the in the (\ref{DMeq:9}), we can
drop the quadrature symbol. the next step is to make the
(\ref{DMeq:9}) come close to the (\ref{DMeq:7}). According to the
Cauchy-binet formula in the linear algebra, we generally have:
\begin{align}\label{}
&\sum^{n}_{1 \leq i_{1}<i_{2}<\cdots<i_{p}
    \leq n}\left|A^{*}\begin{pmatrix}
                        x^{'}_{1} & x^{'}_{2} & \cdots & x^{'}_{p} \\
                        i_{1} & i_{2} & \cdots & i_{p} \\
                      \end{pmatrix}
\right| \left|A\begin{pmatrix}
                        i_{1} & i_{2} & \cdots & i_{p} \\
                        x_{1} & x_{2} & \cdots & x_{p} \\
                      \end{pmatrix}
\right| \nonumber \\
&=    \left|A^{*}A\begin{pmatrix}
                        x^{'}_{1} & x^{'}_{2} & \cdots & x_{p} \\
                        x_{1} & x_{2} & \cdots & x_{p} \\
                      \end{pmatrix}
\right|
\end{align}
Therefore, we have:
\begin{equation}\label{DMeq:22}
  \gamma^{p} = \frac{1}{p!}\left|A^{*}A\begin{pmatrix}
                        x^{'}_{1} & x^{'}_{2} & \cdots & x_{p} \\
                        x_{1} & x_{2} & \cdots & x_{p} \\
                      \end{pmatrix}
\right|
\end{equation}
So finally we have drop the summation symbol.

It's easy to prove that (\ref{DMeq:22}) is just equal to the
(\ref{DMeq:7}). Here in the (\ref{DMeq:22}) the results means that
the variables of $x_{i}$ ($x^{'}_{i}$) is selected to be fixed (from
$1$ to $p$); while the label of orbital is free to change (from $1$
to $n$); therefore it comes to the result of (\ref{DMeq:7}).

Finally it's interesting to compare the form of reduced density
matrices in (\ref{DMeq:10}) with the ``density matrix'' defined in the
HF chapter (see (\ref{HFT:5})). Why they have the same name? Do they
represent the same thing?

In ((\ref{HFT:5})), the density matrix of $P_{\mu\nu}$ is defined
as:
\begin{equation}
  \label{DMeq:27}
P_{\mu\nu} = \sum^{occ}_{j=1}c^{*}_{\mu j}c_{\nu j}
\end{equation}

Then the first order of reduced density matrices is:
\begin{align}
    \label{DMeq:28}
    \gamma^{1}(x_{1}^{'}, x_{1}) &=
\sum_{i=1}^{occ}\psi^{*}_{i}(x^{'}_{1})\psi_{i}(x_{1}) \quad
\underrightarrow{\psi_{i}(x) = \sum_{\mu}c_{\mu i}\phi_{\mu}(x)}
\nonumber \\
&= \sum_{i=1}^{occ}\left( \sum_{\mu}c^{*}_{\mu i}\phi^{*}_{\mu}(x^{'}_{1}) \right)
\left( \sum_{\nu}c_{\nu i}\phi_{\nu}(x_{1}) \right) \nonumber \\
&=\sum_{\mu}\sum_{\nu}P_{\mu\nu}\phi^{*}_{\mu}(x^{'}_{1})\phi_{\nu}(x_{1})
\end{align}

From the (\ref{DMeq:28}) we can see that in the selected basis
functions of $\phi(x)$, if the $P_{\mu\nu}$ is given then the
first order of density matrices is fixed (for all the $\mu$ and
$\nu$). On the contrary, if the $\gamma^{1}(x_{1}^{'}, x_{1})$ is
known (here below we assume that the basis functions are orthogonal
with each other, such relation can always be made true); then
through the normalization condition (for the arbitrary pair of $\mu$
and $\nu$):
\begin{equation}
  \label{DMeq:29}
  \int \phi^{*}_{\nu}(x_{1})\gamma^{1}(x_{1}^{'},
  x_{1})\phi_{\mu}(x^{'}_{1})dx^{'}_{1}dx_{1} = P_{\mu\nu}
\end{equation}
The $P_{\mu\nu}$ is also fixed. Hence, we say that the density
matrix of $P_{\mu\nu}$ is identical to the first order of
density matrices of $\gamma^{1}(x_{1}^{'},x_{1})$.

%%%%%%%%%%%%%%%%%%%%%%%%%%%%%%%%%%%%%%%%%%%%%%%%%%%%%%%%%%%%%%%%%%%%%%%%%%%
\subsection{Character for the reduced density matrices in HF framework}
%
%
%
%
Now we turn back to see the physical meaning of (\ref{DMeq:7}). We
can see that in the Hatree-Fock framework, all the higher order of
reduced density matrices are only depending on the first order ones.
As a matter of fact, this is some kind of one-to-one mapping
relationship between the Hatree-Fock framework and the expression of
(\ref{DMeq:7})\cite{weitaoYang}. If the single electron
approximation and slater determinants expression are introduced, the
expression of reduced density matrices will turn to (\ref{DMeq:7});
on the other hand, if we already have the reduced density matrices
in expression of (\ref{DMeq:7}), we will arrive at the single
electron approximation and slater determinants (here, the proof has
been omitted here). Therefore, it can say that (\ref{DMeq:7})
actually is another way to represent the Hatree-Fock framework.

For the $\gamma^{1}$, from the expression of (\ref{DMeq:10}) we can
see that:
\begin{align}\label{DMeq:23}
&\int\gamma^{1}(x^{'}_{1},x^{''}_{1})
\gamma^{1}(x^{''}_{1},x_{1})dx^{''}_{1} \nonumber
\\
&=\int \sum_{i=1}^{n}\psi^{*}_{i}(x^{'}_{1})\psi_{i}(x^{''}_{1})
\sum_{j=1}^{n}\psi^{*}_{j}(x^{''}_{1})\psi_{j}(x_{1})dx^{''}_{1}
\nonumber
\\
&=\sum_{i=1}^{n}\sum_{j=1}^{n}
\psi^{*}_{i}(x^{'}_{1})\psi_{j}(x_{1})
\int\psi_{i}(x^{''}_{1})
\psi^{*}_{j}(x^{''}_{1})dx^{''}_{1} \nonumber
\\
&=\sum_{i=1}^{n}\sum_{j=1}^{n}
\psi^{*}_{i}(x^{'}_{1})\psi_{j}(x_{1})\delta_{ij} \nonumber \\
&=\sum_{i=1}^{n} \psi^{*}_{i}(x^{'}_{1})\psi_{i}(x_{1}) =
\gamma^{1}(x^{'}_{1},x_{1})
\end{align}
This reflects that the $\gamma^{1}$ has the idempotent character.
Such character is originated from the orthogonality of the orbital
space. On the other hand, it's easy to see that
\begin{equation}\label{DMeq:25}
  \int \gamma^{1}\psi_{i}(x_{1})dx_{1} = \psi^{'}_{i}(x^{'}_{1})
\end{equation}
Here we note that the occupation number for the $\psi^{'}_{i}$
equals to $1$. From the definition of the reduced density matrices
in (\ref{DMeq:10}), we know that they are all composed by the
occupied orbitals so this conclusion is natural.

Finally, from the explanation of (\ref{DMeq:25}) it's easy to prove
that:
\begin{equation}\label{DMeq:24}
Tr(\gamma^{1}) =
\int\gamma^{1}(x^{'}_{1},x_{1})_{x^{'}_{1}=x_{1}}dx_{1} = n
\end{equation}
Such meaning is clear for there are totally $n$ electrons so that
the electron density should be $n$.


%%%%%%%%%%%%%%%%%%%%%%%%%%%%%%%%%%%%%%%%%%%%%%%%%%%%%%%%%%%%%%%%%%%%%%%%%%%
\subsection{Energy expression for the reduced density matrices
in HF framework}\label{expression_RDM_HF_framework}
%
%
%
%
Finally, let's bring the expression of (\ref{DMeq:10}) and
(\ref{DMeq:7}) into (\ref{DMeq:21}); then we can express total
energy for the reduced density matrices via the single electron
approximation. However, firstly let's make some changes to the
$\gamma^{2}$:
\begin{align}\label{}
\gamma^{2}(x^{'}_{1}, x^{'}_{2}, x_{1}, x_{2})_{x^{'}_{1}=x_{1},
x^{'}_{2}=x_{2}} &= \frac{1}{2}
  \begin{vmatrix}
    \gamma^{1}(x^{'}_{1},x_{1}) & \gamma^{1}(x^{'}_{1},x_{2})  \\
    \gamma^{1}(x^{'}_{2},x_{1}) & \gamma^{1}(x^{'}_{1},x_{1})  \\
  \end{vmatrix}_{x^{'}_{1}=x_{1},
x^{'}_{2}=x_{2}} \nonumber \\
  &=\frac{1}{2}(\gamma^{1}(x_{1},x_{1})\gamma^{1}(x_{1},x_{1}) -
  \gamma^{1}(x_{1},x_{2})\gamma^{1}(x_{2},x_{1}))
\end{align}


Hence, the energy expression in the (\ref{DMeq:21}) will finally becomes:
\begin{multline}\label{DMeq:26}
  E = \int \left[\left(\frac{-1}{2}\times\nabla^2\right)
    \gamma^{1}(x^{'}_{1}, x_{1})\right]_{x_{1} = x_{1}^{'}} dx_{1} +
  \sum_{\alpha=1}^{N}\int
  \left[\left(\frac{-1}{r_{x_{1}\alpha}}\right) \rho(x_{1})\right]
  dx_{1} + \\
  \frac{1}{2}\int \left[\left(\frac{1}{r_{12}}\right)
    \left(\gamma^{1}(x_{1},x_{1})\gamma^{1}(x_{2},x_{2}) -
  \gamma^{1}(x_{1},x_{2})\gamma^{1}(x_{2},x_{1})\right)\right] dx_{1}dx_{2}
\end{multline}


In the expression of (\ref{DMeq:26}), the coulomb energy term and the
exchange energy term is fully characterized. For the coulomb term:
\begin{equation}
  \label{DMeq:30}
  E_{coulomb} =  \frac{1}{2}\int \frac{1}{r_{12}}
  \gamma^{1}(x_{1},x_{1})\gamma^{1}(x_{2},x_{2})dx_{1}dx_{2} =
  \frac{1}{2}\int \frac{1}{r_{12}} \rho(x_{1})\rho(x_{2})dx_{1}dx_{2}
\end{equation}
It's clearly the classic coulomb energy expression between the
electron density of $\rho(x_{1})$ and $\rho(x_{2})$. Here the $\rho$
is the total density in given point:
\begin{equation}
  \label{DMeq:31}
  \rho(x_{1}) = \rho^{\alpha}(x_{1}) + \rho^{\beta}(x_{1})
\end{equation}

The exchange energy part, however; is a bit of complicated:
\begin{equation}
  E_{exchange} =  -\frac{1}{2}\int \frac{1}{r_{12}}
  \gamma^{1}(x_{1},x_{2})\gamma^{1}(x_{2},x_{1})dx_{1}dx_{2}
\end{equation}

Since that $\gamma^{1}(x_{1},x_{2}) = \gamma^{1*}(x_{2},x_{1})$
according to the (\ref{DMeq:10}), then this term becomes:
\begin{equation}
  \label{DMeq:32}
  E_{exchange} =  -\frac{1}{2}\int \frac{1}{r_{12}}
  |\gamma^{1}(x_{1},x_{2})|^{2}dx_{1}dx_{2}
\end{equation}
such expression invoke us to define the ``exchange density'', which
is related to two different position $x_{1}$ and $x_{2}$. Below we use
the subscript of $1$ to indicate it's the first order density
matrices:
\begin{equation}\label{}
\rho_{1}(x_{1},x_{2}) = \gamma^{1}(x_{1},x_{2})
\end{equation}
Since it is related to two locations, it has some ``non-local''
character. Furthermore, in the (\ref{DM:1}), we have stated that the
spin variable has been integrated into the variable of $x_{i}$, then
here it's clear that both of the two location; $x_{1}$ and $x_{2}$
should share the same spin, else the integral will go zero. Hence we
have:
\begin{align}
  \label{DMeq:33}
  E_{exchange} &=  -\frac{1}{2}\int \frac{1}{r_{12}}
  \left\{|\rho^{\alpha}_{1}(x_{1},x_{2})|^{2} +
  |\rho^{\beta}_{1}(x_{1},x_{2})|^{2}\right\}dx_{1}dx_{2}
\end{align}

Therefore, we can further express the (\ref{DMeq:26}) as:
\begin{multline}
\label{DMeq:34}
  E = \int \left[\left(\frac{-1}{2}\times\nabla^2\right)
    \gamma^{1}(x^{'}_{1}, x_{1})\right]_{x_{1} = x_{1}^{'}} dx_{1} +
  \sum_{\alpha=1}^{N}\int
  \left[\left(\frac{-1}{r_{x_{1}\alpha}}\right) \rho(x_{1})\right]
  dx_{1} + \\
\frac{1}{2}\int \frac{1}{r_{12}} \rho(x_{1})\rho(x_{2})dx_{1}dx_{2}
- \frac{1}{2}\int \frac{1}{r_{12}}
  \Bigg\{|\rho^{\alpha}_{1}(x_{1},x_{2})|^{2} +
         |\rho^{\beta}_{1}(x_{1},x_{2})|^{2}\Bigg\}dx_{1}dx_{2}
\end{multline}

If the configuration is in close shell (RHF case), then we have:
\begin{equation}
  \label{DMeq:35}
  \rho^{\alpha}_{1}(x_{1},x_{2})=
  \rho^{\beta}_{1}(x_{1},x_{2})=
\frac{1}{2}\rho_{1}(x_{1},x_{2})
\end{equation}

Then the exchange energy becomes:
\begin{equation}
  \label{DMeq:36}
    E_{exchange} =  -\frac{1}{4}\int \frac{1}{r_{12}}
  |\rho_{1}(x_{1},x_{2})|^{2}dx_{1}dx_{2}
\end{equation}


%%%%%%%%%%%%%%%%%%%%%%%%%%%%%%%%%%%%%%%%%%%%%%%%%%%%%%%%%%%%%%%%%%%%%%%%%%%
\subsection{Exchange hole description}\label{DM:2}
%
%
%
%
From the discussion about the Hatree-Fock framework, we have known
that the single electron approximation have just brought in the full
exchange effects so that to make the approximated wave function
satisfy the Pauli principle. However, for the correlation effects,
the single electron approximation has completely neglected it; which
is clear by comparing the energy expression in (\ref{DMeq:34}) and
the pair density shown in (\ref{DM_xc_hole_sum_rule}). Let's rewrite 
the exchange energy in the last section into some new form: 
\begin{equation}
\begin{split}
E_{exchange} &=  -\frac{1}{2}\int \frac{1}{r_{12}}
  \gamma^{1}(x_{1},x_{2})\gamma^{1}(x_{2},x_{1})dx_{1}dx_{2} \\
&= -\frac{1}{2}\int dx_{1}dx_{2} \rho(x_{1})\rho(x_{2})
\frac{1}{r_{12}}
\frac{\gamma^{1}(x_{1},x_{2})\gamma^{1}(x_{2},x_{1})}{\rho(x_{1})\rho(x_{2})}
\end{split}
\end{equation}

Here we declare that the term inside the $\int dx_{2}$ contains the exchange
hole $\widetilde{h}(x_{1}, x_{2})$\footnote{Actually this declaration is more
clear we we get to the adiabatic connection in the density functional theory
part, see \ref{Adiabatic_connections} and the expression for XC functional 
\ref{DFTIeq:29}}:
\begin{equation}\label{DMeq:37}
\widetilde{h}(x_{1}, x_{2})
 =\widetilde{h}_{x}(x_{1}, x_{2}) = -\frac{1}{2}
\frac{\rho_{1}(x_{1},x_{2})\rho_{1}(x_{2},x_{1})}{\rho(x_{1})\rho(x_{2})}
\end{equation}

Now let's evaluate the average effects for the exchange hole:
\begin{align}\label{DMeq:38}
\int\rho(x_{2})\widetilde{h}_{x}(x_{1}, x_{2})dx_{2} &= -\frac{1}{2}\int
\frac{\rho_{1}(x_{1},x_{2})\rho_{1}(x_{2},x_{1})}{\rho(x_{1})}dx_{2}
\nonumber \\
&=-\frac{1}{2}\int\frac{1}{\rho(x_{1})}
\sum_{i=1}^{occ}\psi^{*}_{i}(x_{1})\psi_{i}(x_{2})
\sum_{j=1}^{occ}\psi^{*}_{j}(x_{2})\psi_{j}(x_{1})
dx_{2} \nonumber \\
&=-\frac{1}{2}\frac{1}{\rho(x_{1})}
\sum_{i=1}^{occ}\sum_{j=1}^{occ}\psi^{*}_{i}(x_{1})\psi_{j}(x_{1})
\int\psi^{*}_{j}(x_{2})\psi_{i}(x_{2})dx_{2} \nonumber \\
&= -\frac{1}{\rho(x_{1})}
\sum_{i=1}^{occ}\sum_{j=1}^{occ}\psi^{*}_{i}(x_{1})\psi_{j}(x_{1})
\delta_{ij} \nonumber \\
&= -\frac{1}{\rho(x_{1})}\rho(x_{1}) \nonumber \\
&= -1
\end{align}

Here in the above derivation we have used the relation that:
\begin{equation}\label{}
\int\psi^{*}_{i}(x_{2})\psi_{j}(x_{2})dx_{2} = 2\delta_{ij}
\end{equation}
Here all of the alpha and beta electrons are counted, since we are 
sticking to the spinless form. 

Now we have arrived at some important conclusion that the average
effects for the exchange hole equal to $-1$. this holds true for any
single Slater determinant wave functions. The \ref{DMeq:38} means, if we
imagine that there's an electron fixing at the location of
$\bm{r_{1}}$, then if we counting the exchange effects for all the
other locations of $\bm{r_{2}}$ (which means to cause depletion of electron 
density at the $\bm{r_{2}}$), it will be $-1$; which is
equivalent to one electron.


%%%%%%%%%%%%%%%%%%%%%%%%%%%%%%%%%%%%%%%%%%%%%%%%%%%%%%%%%%%%%%%%%%%%%%%%%%%




%%% Local Variables:
%%% mode: latex
%%% TeX-master: "../../main"
%%% End:
