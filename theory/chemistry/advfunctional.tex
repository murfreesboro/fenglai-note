%
%  set up at March 9th, 2011
%
%
%
\chapter{Advanced Discussion for Functionals}
%
%
%


%%%%%%%%%%%%%%%%%%%%%%%%%%%%%%%%%%%%%%%%%%%%%%%%%%%%%%%%%%%%%%%%%%%%%%%%%%%%%%
\section{$\lambda$ dependent functionals}
%
% this part is coded in q-chem, from weitao yang's original paper
%

The MCY functional, is actually based on the idea that how to find a way to
construct the adiabatic connection path:
\begin{equation}
 E_{XC}[\rho] = \int_{0}^{1} W_{\lambda}[\rho] d \lambda
\label{MCY_functional_eq:1}
\end{equation}
If the $W_{\lambda}[\rho]$ is known, then we can know how to calculate the
exchange correlation energy.

MCY functional starts from the Pade form:
\begin{equation}
 W_{\lambda}[\rho] = a + \frac{\lambda b}{1 + \lambda c}
\label{MCY_functional_eq:2}
\end{equation}

Through integral by parts we can get that:
\begin{equation}
 E_{XC}[\rho] = a + \frac{b}{c}\left(1 - \frac{\ln
(1+c)}{c}\right)
\label{MCY_functional_eq:3}
\end{equation}
Hence the unknown $E_{xc}$ is then determined by the $a$, $b$ and $c$.

According to the already known exact information for the adiabatic connection
path, we have:
\begin{itemize}
 \item As $\lambda \rightarrow 0$, then $W_{\lambda}[\rho]\rightarrow E_{x}$.
The $E_{x}$ is the exact exchange energy.
\item  As $\lambda \rightarrow 0$, then
$\frac{\partial W_{\lambda}[\rho]}{\partial \lambda}\rightarrow
2 E_{c}^{GL}$. This is the known ``Gorling-Levy'' functional expression.
\end{itemize}

According to the (\ref{MCY_functional_eq:2}), we can know that as $\lambda
\rightarrow 0$, the $W_{\lambda}[\rho] = a$. Hence the functional of $a$ just
characterizes the exact exchange energy in this functional. On the other hand,
for the one electron system; we have $E_{xc} = a = E_{x}$ so that to cancel the
SIE error. So this functional is also SIE free functional.
 
The next step is to associate the $b$ functional with the ``Gorling-Levy''
expression. According to the (\ref{MCY_functional_eq:2}), assume that $a$, $b$
and $c$ are independent of $\lambda$ variable, we can have:
\begin{align}
\frac{\partial W_{\lambda}[\rho]}{\partial \lambda} 
&= \left. \frac{b(1+\lambda c) - \lambda cb}{(1+\lambda c)^{2}}\right |_{\lambda
= 0} \nonumber \\
&= b
\label{MCY_functional_eq:4} 
\end{align}
Hence $b$ is related to the initial slope for $\lambda = 0$ on the adiabatic
connection path, then we can have:
\begin{equation}
b = 2\lim_{\lambda \rightarrow 0}E_{c}[\rho_{1/\lambda}]
\label{MCY_functional_eq:5}  
\end{equation}
Now to find a proper form of $b$ is necessary.

In order to retain the SIE free functional, it requires that the selected
correlation functional should be vanished in one electron system.  Hence they
choose the TPSS correlation functional, then we can have such expression:
\begin{equation}
b =2 \lim_{\lambda \rightarrow 0}E_{tpss}[\rho_{1/\lambda}]
\label{MCY_functional_eq:6}   
\end{equation}

Now it's only the last component of $c$ need to be determined. For the
adiabatic connection path, we already have the initial point and initial
slope, which is characterized by $a$ and $b$ respectively; to determine the
adiabatic connection path we only need another point on this path, that is:
\begin{equation}
  W_{\lambda}[\rho] =  W_{\lambda_{p}}[\rho]
\label{MCY_functional_eq:7}   
\end{equation}
So that to determine the component $c$. 

Therefore, as $\lambda = \lambda_{p}$, according to the
(\ref{MCY_functional_eq:2}) the $c$ is expressed as:
\begin{equation}
c =\frac{W_{\lambda_{p}}[\rho] - \lambda_{p}b - a}{
\lambda_{p}(a - W_{\lambda_{p}}[\rho])}
\label{MCY_functional_eq:8}    
\end{equation}
Now we finished the work to produce the MCY functional.

On the other hand, from the given Fortran file of xc\_getv.F; we can see that
the subroutine scf\_adiabatic is used to calculate the MCY total energy and Fock
matrix components. In this program, it's rather strange to see that $b$ is
instead some composite functional, which is expressed as:
\begin{equation}
 b = \frac{-a^{2} + aE_{blyp} - \lambda_{p}E_{tpss}E_{blyp}}{
\lambda_{p}a(a-E_{blyp})}
\end{equation}
Here the $E_{tpss}$ is the energy for scaled $\lambda$-dependent TPSS
correlation functional, which is characterized in (\ref{MCY_functional_eq:6});
the $E_{blyp}$ is the energy for the $\lambda$-dependent BLYP functional, which
is just the $W_{\lambda_{p}}[\rho]$ in the (\ref{MCY_functional_eq:7}). On the
other hand, the components of $a$ and $c$ are still same.

What's more, the exchange-correlation functional, is also quite different from
the expression (\ref{MCY_functional_eq:3}); in the codes the total energy is
expressed as:
\begin{equation}
 E_{xc} = \frac{ab}{c} + \frac{a(c-b)}{c^{2}}\ln(1+c) 
\end{equation}
This is different from what we have known from the published paper.

Thanks and best wishes,

fenglai  
 
 







%%% Local Variables: 
%%% mode: latex
%%% TeX-master: "../../main"
%%% End: 
