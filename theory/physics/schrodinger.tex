%
% revised at March 31th, 2009
% modify the first section, to specify the details of time dependent
% system
% revise the discussion of second section.
%
% revised at Aug 10th, 2009
% here according to the above content we have delete the
% expression for CSCCO.
%
\chapter{Schrodinger equation}
%
%  detailed describe the Schrodinger equation
%  1  Schrodinger equation, formation,meaning
%  2  variational principle
%  3  operators commuted with Hamiltonian
%  4  CSCCO
%
%
%


%%%%%%%%%%%%%%%%%%%%%%%%%%%%%%%%%%%%%%%%%%%%%%%%%%%%%%%%%%%%%%%%%%%%%%%%%%%
\section{Schrodinger equation}\label{SE:5}
%
% whole function and the time independent function form
% discuss the hamiltonian that does not contain time
%
After talking so much, let's formally introduce the most important
equation in quantum  mechanics:
\begin{equation}\label{}
\hat{H}\Psi(\bm{r},t) = i \hbar \frac{\partial
\Psi(\bm{r},t)}{\partial t}
\end{equation}
the $\hat{H}$ is the Hamiltonian operator, which contains the
kinetic operator and the potential energy operator, therefore it can
be expressed as:
\begin{equation}\label{SEeq:1}
\left\{-\frac{\hbar^{2}}{2m}\nabla^{2} +
V(\bm{r},t)\right\}\Psi(\bm{r},t) = i \hbar \frac{\partial
\Psi(\bm{r},t)}{\partial t}
\end{equation}

For Schrodinger equation, we have some axiom to correspond it:
\begin{axiom}\label{axiom6}
\textbf{The quantum state of $\ket{\Psi(t)}$ is determined by
Schrodinger equation:
\begin{equation}\label{}
\hat{H}\ket{\Psi(t)} = i \hbar \frac{\partial
\ket{\Psi(t)}}{\partial t}
\end{equation}
Where the $\hat{H}$ is the Hamiltonian operator: $\hat{H} =
\hat{H}(\hei{r}, \hei{p}, t)$. }
\end{axiom}


If the Hamiltonian operator does not contain the $t$ ($V(r,t)
\rightarrow V(r)$, actually most of the circumstance we meet in
quantum mechanics are belonging to this kind), the wave function can
be separated into $\Psi(\bm{r},t) = \Psi(\bm{r})f(t)$. Here the
$\Psi(\bm{r})$ express the quantum state in the initial state:
$t=0$. Then by bringing this $\Psi(\bm{r},t)$ into the
(\ref{SEeq:1}), we can get:
\begin{align}\label{SEeq:17}
i \hbar \Psi(\bm{r})\frac{\partial f(t)}{\partial t} &=
\hat{H}\Psi(\bm{r})f(t) \nonumber \\
i \hbar \frac{\partial f(t)}{\partial t}  &=
\bra{\Psi(\bm{r})}\hat{H}\ket{\Psi(\bm{r})}f(t) \Rightarrow \nonumber \\
f(t) &=
e^{\frac{-i\bra{\Psi(\bm{r})}\hat{H}\ket{\Psi(\bm{r})}t}{\hbar}}
\end{align}
So finally the time dependent wave function of $\Psi(\bm{r},t)$ is:
\begin{equation}\label{}
\Psi(\bm{r},t) = \Psi(\bm{r})e^{-i\overline{H}t/\hbar}
\end{equation}
Here the $\overline{H}$ denotes the average value for the
Hamiltonian operator.

On the other hand, in the following content (see the discussion
related to the commutation with Hamiltonian operator), we can see
that if the Hamiltonian operator does not contain the $t$, then the
energy is some conserved physical quantity; which means, the energy
does not change as $t$ varies. Hence in the (\ref{SEeq:17}) we have:
\begin{equation}\label{SEeq:19}
\begin{split}
  i \hbar \frac{1}{f(t)}\frac{\partial f(t)}{\partial t}
  &= E  \Rightarrow\\
   \hat{H}\Psi(\bm{r}) &= E\Psi(\bm{r})
\end{split}
\end{equation}
Hence we can have the eigen function for the Hamiltonian:
\begin{equation}\label{SEeq:2}
\hat{H}\psi_{i}(\bm{r}) = E_{i}\psi_{i}(\bm{r}) \quad i=1,2,\cdots
\end{equation}
Here the $\psi_{i}(\bm{r})$ is some eigen states for the Hamiltonian
operator, and the corresponding $E_{i}$ is the eigen values-energy
of the system. $\psi_{i}$ is called stationary state of this system,
and the lowest energy state is called ground state. Furthermore, we
can get the time evolution of each eigen state of $\psi_{i}(\bm{r})$
by repeating the procedure in (\ref{SEeq:17}):
\begin{equation}\label{}
\psi_{i}(r, t) = \psi_{i}(r)e^{-iE_{i}t/\hbar}
\end{equation}
Here we can see that if the initial state is in of the eigen states,
then the system will keep the original energy as time evolves. On
the other hand, based on this complete sets, we can also express an
arbitrary state of $\Psi(\bm{r},t)$ into linear combination of
$\psi_{i}(\bm{r},t)$:
\begin{align}\label{}
\Psi(\bm{r},t) &= \sum_{i}a_{i}\psi_{i}(\bm{r},t) \nonumber \\
&=\sum_{i}a_{i}\psi_{i}(\bm{r})e^{-iE_{i}t/\hbar}
\end{align}
Here it's not some stationary state at any give time of $t$.
However, the change if time does not alter the component of $a_{i}$;
since that $|a_{i}e^{-iE_{i}t/\hbar}|^{2} = |a_{i}|^{2}$.

%%%%%%%%%%%%%%%%%%%%%%%%%%%%%%%%%%%%%%%%%%%%%%%%%%%%%%%%%%%%%%%%%%%%%%%%%%%
\section{Some basic concept in quantum chemistry}\label{SE:4}
%
%  how to understand quantum chemistry ? in general
%
%
In the quantum chemistry, we always do not consider the time
variation; thus in the following content we with to give some
discussion about the function in (\ref{SEeq:2}).

For the system described through the (\ref{SEeq:2}), it's actually
some kind of isolated system since that there's neither materials
nor energy floating inside or outside the corresponding system. For
such system, since the time should be equivalent, it's impossible to
distinguish the past from the future. Thus the Hamiltonian for a
isolated system should not contain the time explicitly.

Most of the chemical system we apply for quantum simulation belong
to this type. For example, if we construct a chemical reaction
system; it's indeed a isolated system; since there's no materials
exchanged or external fields existed. On the Born-Oppenheimer
approximation, the quantum states for the system is some function of
the nuclear coordinates. Hence, for some configuration or
conformation it may corresponds to higher or lower energy state
(here we only consider the ground states), so we can get some n
dimensional PES. For such PES, we can identify the energy as new
bonds forming or old bonds breaking, so that to get more
understanding about the chemical reaction in adiabatic condition.

Sometimes the system is not in the ground state, by some external
perturbation the system may excite onto some higher states; this is
called quantum transition; which plays crucial role in quantum
chemistry. If the external perturbation is in infrared wave band,
the molecule's rotation and vibration states may be got excited so
as to produce the IR spectrum; which is very important in chemistry
to identity the characters of this molecule. If the external
perturbation is some ultraviolet rays then the UV spectrum may be
got excited or even the electrons may be excited onto the higher
state to form the photochemistry reaction.

In the following chapter (see the perturbation treatment for the
time-dependent system), we can see that we always use the
perturbation treatment to deal with such time dependent case.

%%%%%%%%%%%%%%%%%%%%%%%%%%%%%%%%%%%%%%%%%%%%%%%%%%%%%%%%%%%%%%%%%%%%%%%%%%%%%%%
\section{Commutation with Hamiltonian operator}\label{SE:3}
%
% contains two aspects:
%   1   the equation deduction, dA/dt
%   2   the correlation between conserved operator
%       and energy degeneration
%

So far we have known that Hamiltonian operator possesses an very
important position in quantum mechanics, thus what about the
operators which commutate with the Hamiltonian operator? Does they
have some special character?

Now we suggest an hermite operator of $\hat{A}$:
\begin{equation}
\overline{A(t)} = \bra{\psi(t)}\hat{A}\ket{\psi(t)}
\end{equation}

So with respect to the time, the differentiation of
$\overline{A(t)}$ is:
\begin{eqnarray}\label{SEeq:3}
% \nonumber to remove numbering (before each equation)
  \frac{d \overline{A(t)}}{dt} &=& \bra{\frac{\partial \psi(t)}{\partial t}}\hat{A}\ket{\psi(t)}
  + \bra{\psi(t)}\frac{\partial\hat{A}}{\partial t}\ket{\psi(t)} +
  \bra{\psi(t)}\hat{A}\ket{\frac{\partial \psi(t)}{\partial t}} \nonumber \\
    &=& \bra{\frac{\hat{H} \psi(t)}{-i\hbar}}\hat{A}\ket{\psi(t)}
  + \bra{\psi(t)}\frac{\partial\hat{A}}{\partial t}\ket{\psi(t)} +
  \bra{\psi(t)}\hat{A}\ket{\frac{\hat{H} \psi(t)}{i\hbar}}  \nonumber \\
    &=& -\frac{1}{i\hbar}\bra{\psi(t)}\hat{H}\hat{A}\ket{\psi(t)}
  + \bra{\psi(t)}\frac{\partial\hat{A}}{\partial t}\ket{\psi(t)}
  + \frac{1}{i\hbar}\bra{\psi(t)}\hat{A}\hat{H}\ket{\psi(t)}  \nonumber \\
    &=& \frac{1}{i\hbar}\bra{\psi(t)}[\hat{A},\hat{H}]\ket{\psi(t)}
    + \bra{\psi(t)}\frac{\partial\hat{A}}{\partial t}\ket{\psi(t)}  \nonumber \\
    &=& \bra{\psi(t)}\frac{1}{i\hbar}[\hat{A},\hat{H}]+\frac{\partial\hat{A}}{\partial t}\ket{\psi(t)}
\end{eqnarray}

Here, since we acquiesce that $\psi \Leftrightarrow \ket{\psi} $,
and $\psi^{*} \Leftrightarrow \bra{\psi} $, so the Hamiltonian on
the bra leads to $\hat{H}\psi^{*} = -i\hbar \frac{\partial
\psi^{*}}{\partial t}$.

In the expression of (\ref{SEeq:3}), if $\hat{A}$ does not contain
the time of $t$, the term of $\frac{\partial \hat{A}}{\partial t} =
0$. Furthermore, if $[\hat{A},\hat{H}] = 0$; it leads to the result
of $\frac{d \overline{A(t)}}{dt} = 0$. This means that the
expectation value of $\hat{A}$ does not change as the time varies,
such kind of physical quantity is called ``conserved'' physical
quantity.

As what we note before, if an operator of $\hat{A}$ commutes with
$\hat{H}$, then they may share the same eigen states. However, it
does not mean that they ``must'' share the same eigen states. The
examples shown in the \ref{transformation_in_representation} is an
good illustration to this case.

%%%%%%%%%%%%%%%%%%%%%%%%%%%%%%%%%%%%%%%%%%%%%%%%%%%%%%%%%%%%%%%%%%%%%
\section{Complete set Of commutating conserved
operators}\label{SE:1}
%
% why we have to introduce the CSCCO?
% the existence of the CSCCO, proof; in a general form
%
%
%
In the section \ref{CSCO_in_operator}, we have introduced the
concept of ``CSCO''. However, since that for most of the situation
we have the energy as conserved quantity; then we always choose the
CSCO which contains the $\hat{H}$; this group is called ``complete
set Of commutating conserved operators''; in the future, they will
be abbreviated as ``CSCCO''.

%%%%%%%%%%%%%%%%%%%%%%%%%%%%%%%%%%%%%%%%%%%%%%%%%%%%%%%%%%%%%%%%%%%%%%%%%%%%%%%%%%%
\section{Variation principle}\label{SE:2}
%
% 1  equivalence between variation principle and the schrodinger equation,proof
%    two points: first, the wave functions are the stationary points;
%                second, the trial wave functions are the upper limit of the energy
% 2  how to understand its physical meaning, how to construct a variation process?
% 3  Ritz variation method
% 4  linear variation method
%
 For the time independent quantum system, we have the
schrodinger equation:
\begin{equation}\label{SEeq:5}
\hat{H}\Psi = E\Psi
\end{equation}
Where the $E$ is the energy of the system. In the following content,
we will prove that this equation is equivalent to the variation
principle.

In the variation process, the energy of the quantum system and the
corresponding wave functions can be achieved through the variation
of the integral: $\bra{\Psi}\hat{H}\ket{\Psi}$, under the constraint
condition of $\langle\Psi|\Psi\rangle = 1$. Here the $\Psi$ is the
trial wave function for the system.

The whole process can be expressed as:
\begin{equation}\label{SEeq:6}
\delta [\bra{\Psi}\hat{H}\ket{\Psi} - \lambda
\langle\Psi|\Psi\rangle] = 0
\end{equation}

Now we begin to prove it. First, we can demonstrate that through
(\ref{SEeq:6}) we can get (\ref{SEeq:5}).
\begin{align}\label{}
 \delta[\bra{\Psi}\hat{H}\ket{\Psi} - \lambda
\langle\Psi|\Psi\rangle] &= 0 \nonumber \\
\left\{\bra{\delta\Psi}\hat{H}\ket{\Psi} - \lambda
\langle\delta\Psi|\Psi\rangle \right\} +
\left\{\bra{\Psi}\hat{H}\delta\ket{\Psi} - \lambda
\langle\Psi|\delta\Psi\rangle\right\} &= 0 \nonumber \\
\delta\langle\Psi|\left\{\hat{H}\ket{\Psi} - \lambda
\ket{\Psi}\right\} + \left\{\bra{\Psi}\hat{H} - \lambda
\langle\Psi|\right\}\delta|\Psi\rangle &= 0
\end{align}
Because that $\delta\langle\Psi|$ and $\delta|\Psi\rangle$ vary
independently, so we can have two functions:
\begin{equation} \label{}
\left\{
\begin{aligned}
\hat{H}\ket{\Psi} - \lambda\ket{\Psi}    &= 0   \\
\bra{\Psi}\hat{H} - \lambda \langle\Psi| &= 0
\end{aligned}
\right.
\end{equation}
Since $\hat{H} = \hat{H}^{+}$, the two functions are equivalent to
each other; and that's the Schrodinger equation of (\ref{SEeq:5}).

Second, we can prove that through \ref{SEeq:5} we can get
(\ref{SEeq:6}).

For the $\Psi$ in (\ref{SEeq:5}), we can make an infinitesimal
change:
\begin{eqnarray}
% \nonumber to remove numbering (before each equation)
  \ket{\Psi}^{'} &=& \ket{\Psi} + \delta\ket{\Psi} \nonumber \\
  \bra{\Psi}^{'} &=& \bra{\Psi} + \delta\bra{\Psi}
\end{eqnarray}
Since $\langle\Psi|\Psi\rangle = 1$, this condition should be
retained during the change process(that means, the infinitesimal
change to the wave function will not alter its orthogonality); thus
we can have:
\begin{align}\label{}
\langle\Psi^{'}|\Psi^{'}\rangle &= 1  \Rightarrow \nonumber \\
\langle\delta\Psi|\Psi\rangle + \langle\Psi|\delta\Psi\rangle +
\langle\delta\Psi|\delta\Psi\rangle &= 0
\end{align}
Under this condition, the change to the system energy is:
\begin{align}\label{}
E + \delta E &= \bra{\Psi^{'}}\hat{H}\ket{\Psi^{'}} \Rightarrow \nonumber \\
\delta E &= \bra{\delta\Psi}\hat{H}\ket{\Psi} +
\bra{\Psi}\hat{H}\ket{\delta\Psi} +
\bra{\delta\Psi}\hat{H}\ket{\delta\Psi} \nonumber \\
&= E\left(\langle\delta\Psi|\Psi\rangle +
\langle\Psi|\delta\Psi\rangle\right) +
\bra{\delta\Psi}\hat{H}\ket{\delta\Psi} \nonumber \\
&=\bra{\delta\Psi}\hat{H}\ket{\delta\Psi} -
E\langle\delta\Psi|\delta\Psi\rangle
\end{align}
Because in the above result, it shows that the $\delta E$ is higher
infinitesimal change of $\delta\Psi$; thus the $\delta E = 0$ as the
$\Psi$ varies. In other words, it means that the real wave function
of $\Psi$ makes the integral of $\bra{\Psi}\hat{H}\ket{\Psi}$ to
take stationary point, under the constraint of
$\langle\Psi|\Psi\rangle = 1$. That's really what the (\ref{SEeq:6})
has shown out.

We can understand this situation just like this: the wave functions
is similar to the variable of $x$, and the integral of
$\bra{\Psi}\hat{H}\ket{\Psi}$ is just like $f(x)$; surely the $x$
has to satisfy some condition; which is similar with
$\langle\Psi|\Psi\rangle = 1$. As $x$ varies, the $f(x)$ may take
different values; but if $x$ is the real ``answer'', it will
definitely make $f^{'}(x) = 0$ (this is corresponding to that
$\delta f(x) = 0$). That's what this variation process expresses.

So far we have proved that the wave functions are the stationary
points of the integral of $\bra{\Psi}\hat{H}\ket{\Psi}$; now there
comes a question: it's a maximum or minimum?

Now we prove that the trial functions will yield the upper limit of
the system energy.

Suggest that we have a group of wave functions of $\Psi_{i}$
($i=1,2,\cdots$), they are the eigen states for some CSCCO; now we
use it to expand some trial wave function of $\Psi^{'}$:
\begin{equation}\label{SEeq:18}
\Psi^{'} = \sum_{i}a_{i}\Psi_{i}
\end{equation}
Thus the average energy for the trial wave functions is:
\begin{align}\label{}
\frac{\bra{\Psi^{'}}\hat{H}\ket{\Psi^{'}}}{\langle\Psi^{'}|\Psi^{'}\rangle}
&=\frac{\sum_{i}\sum_{j}a^{*}_{i}a_{j}\bra{\Psi_{i}}\hat{H}\ket{\Psi_{j}}}
{\sum_{i}\sum_{j}a^{*}_{i}a_{j}\langle\Psi_{i}|\Psi_{j}\rangle}
\nonumber \\
&=\frac{\sum_{i}E_{i}|a_{i}|^{2}\langle\Psi_{i}|\Psi_{i}\rangle}
{\sum_{i}|a_{i}|^{2}\langle\Psi_{i}|\Psi_{i}\rangle} \nonumber \\
&=\frac{\sum_{i}E_{i}|a_{i}|^{2}}{\sum_{i}|a_{i}|^{2}} \nonumber \\
&\geq E_{0}\frac{\sum_{i}|a_{i}|^{2}}{\sum_{i}|a_{i}|^{2}} \nonumber \\
&\geq E_{0}
\end{align}

Therefore, the energy through variation process provide an upper
limit to the real energy of the system.

Finally, how can we understand the variation principle? To some
extent, this principle enlarges our understanding to the Schrodinger
equation and the corresponding wave functions. The "real" wave
functions(they constitute a function space) make the system energy
to a minimum, while other trial wave functions(they also constitute
a function space) also can give the approximation to the wave
functions and the corresponding physical quantities (including the
energy of the system), and if the approximation is better, the
physical quantities (trial functions gives) will be more better.

How to construct the trial wave functions? Actually, there are many
ways; according to the objects we are study. In the book by levine
etc.\cite{levine, aoqingTang}, they provides a lots of examples.
Thus the detailed part is omitted here.

On the other hand, in the variation process we always use the form
suggested by Ritz, called Ritz variation process.

In Ritz variation process, first we suggest to take the trial wave
function as:
\begin{equation}\label{}
\Psi = \Psi(q, c_{1}, c_{2}, \cdots)
\end{equation}
Here the $q$ is the coordinates, and $c_{i}$ is the arguments which
needed to be specified in the variation process. Here, we have to
approximation of the energy as:
\begin{equation}\label{}
E = \frac{\bra{\Psi}\hat{H}\ket{\Psi}}{\langle\Psi|\Psi\rangle}
=E(c_{1}, c_{2}, \cdots)
\end{equation}
The expression of the energy depends on the arguments of $c_{i}$.
According to the variation principle, the real wave function should
be the minimum of the energy; so this can be expressed as:
\begin{equation}\label{SEeq:7}
\frac{\partial E}{\partial c_{i}} = 0 \quad (i=1,2,\cdots)
\end{equation}
Through (\ref{SEeq:7}) we can get the approximated wave functions
and the energy.

In quantum chemistry, actually we always use the variation principle
in the Ritz expression. In the following content, we may further
introduce the linear variation process (it also belongs to Ritz
variation process); that is the core of the configuration
interaction method in quantum chemistry.

%%%%%%%%%%%%%%%%%%%%%%%%%%%%%%%%%%%%%%%%%%%%%%%%%%%%%%%%%%%%%%%%%%%%%%%%%%%%%%%%%%%%%%%%%%%%%%%%%%%%%%%
\section{Local probability conservation}
\label{lpc_in_schrodinger}
%
% 1  derivation for the continuity equation
% 2  the physical meaning of the terms
%
So far the Schrodinger equation we have introduced is the
non-relativistic kind, that means there's no particles generating or
annihilating as the system evolves. This implies that the total
probability density for the whole space is some constant and does
not change as time varies. So we have express this relation by the
equation below:
\begin{equation}\label{SEeq:15}
\frac{d}{dt}\int^{+\infty}_{-\infty}|\Psi(r,t)|^{2}dr = 0
\end{equation}

From the schrodinger equation of (\ref{SEeq:1}), we can also prove
this point.

Now let's prove the above equation. By taking complex conjugating
form of the (\ref{SEeq:1}), we can have:
\begin{equation}\label{SEeq:8}
\left(-\frac{\hbar^{2}}{2m}\nabla^{2} + V \right)\Psi^{*}(r,t) = -i
\hbar \frac{\partial \Psi^{*}(r,t)}{\partial t}
\end{equation}
Here we note that the $V$ is some real operator then $V = V^{*}$.

By multiplying the $\Psi^{*}$ to the left side of Schrodinger
equation, we can have:
\begin{equation}\label{SEeq:9}
\Psi^{*}(r,t)\left(-\frac{\hbar^{2}}{2m}\nabla^{2}\right)\Psi(r,t) +
\Psi^{*}(r,t) V \Psi(r,t) = i \hbar
\Psi^{*}(r,t)\frac{\partial}{\partial t}\Psi(r,t)
\end{equation}

Similarly for the (\ref{SEeq:8}), we can have:
\begin{equation}\label{SEeq:10}
\Psi(r,t)\left(-\frac{\hbar^{2}}{2m}\nabla^{2}\right)\Psi^{*}(r,t) +
\Psi(r,t) V \Psi^{*}(r,t) = -i \hbar
\Psi(r,t)\frac{\partial}{\partial t}\Psi^{*}(r,t)
\end{equation}

If we make (\ref{SEeq:9}) subtract (\ref{SEeq:10}), we can finally
get:
\begin{align}\label{SEeq:11}
i \hbar\frac{\partial}{\partial t}\Psi^{*}(r,t)\Psi(r,t) &=
-\frac{\hbar^{2}}{2m}\left(\Psi^{*}(r,t)\nabla^{2}\Psi(r,t)
-\Psi(r,t)\nabla^{2}\Psi^{*}(r,t)\right) \nonumber \\
&=-\frac{\hbar^{2}}{2m}\left(\Psi^{*}\nabla^{2}\Psi -
\nabla\Psi^{*}\nabla\Psi + \nabla\Psi^{*}\nabla\Psi
-\Psi\nabla^{2}\Psi^{*}\right) \nonumber \\
&=-\frac{\hbar^{2}}{2m}\nabla\left(\Psi^{*}(r,t)\nabla\Psi(r,t)
-\Psi(r,t)\nabla\Psi^{*}(r,t)\right)
\end{align}
Here it notes that in the above derivation since $V$ is some real
operator then we can have $\Psi^{*}(r,t) V \Psi(r,t) = \Psi(r,t) V
\Psi^{*}(r,t)$.

Since that it has $-i\hbar\nabla = \hei{p}$, then we can rewrite the
(\ref{SEeq:11}) into some form which has more clear physical
meaning:
\begin{equation}\label{SEeq:12}
\frac{\partial}{\partial t}\Psi^{*}(r,t)\Psi(r,t) =
-\frac{1}{2m}\nabla(\Psi^{*}(r,t)\hei{p}\Psi(r,t)
-\Psi(r,t)\hei{p}\Psi^{*}(r,t))
\end{equation}
If we integrate the above equation in a close space V, then
according to the Gauss theorem, we can have such transformation:
\begin{multline}\label{}
-\frac{1}{2m}\int_{V}\nabla(\Psi^{*}(r,t)\hei{p}\Psi(r,t)
-\Psi(r,t)\hei{p}\Psi^{*}(r,t))dv = \\
-\frac{1}{2m}\oint_{S}(\Psi^{*}(r,t)\hei{p}\Psi(r,t)
-\Psi(r,t)\hei{p}\Psi^{*}(r,t))\cdot ds
\end{multline}
then by integration over the close space V the (\ref{SEeq:12})
finally is:
\begin{multline}\label{SEeq:13}
\frac{\partial}{\partial t}\int_{V}\Psi^{*}(r,t)\Psi(r,t)dv = \\
-\frac{1}{2m}\oint_{S}(\Psi^{*}(r,t)\hei{p}\Psi(r,t)
-\Psi(r,t)\hei{p}\Psi^{*}(r,t))\cdot ds
\end{multline}

For getting a much clear expression, we can make that:
\begin{eqnarray}
% \nonumber to remove numbering (before each equation)
  \Psi^{*}(r,t)\Psi(r,t) &=& \rho \nonumber  \\
  \Psi^{*}(r,t)\hei{p}\Psi(r,t) -\Psi(r,t)\hei{p}\Psi^{*}(r,t) &=& 2m\mathbf{j}
\end{eqnarray}
Here we note that Since \heit{p} is some vectoring operator, then
the $\mathbf{j}$ is also some vector. Then based on the above
transformation the (\ref{SEeq:13}) is finally to be:
\begin{equation}\label{SEeq:14}
\frac{d}{dt}\int_{V}\rho dv =-\oint_{S}\mathbf{j}\cdot ds
\end{equation}

In the (\ref{SEeq:14}) since the left side express the total
probability to find the particle inside the volume of $V$, then the
$\mathbf{j}$ retains the meaning of "current density"; which means
that in some unit time through the area of $ds$ how much probability
flowing into the close volume of $V$. It can be proved that if let
$V\rightarrow\infty$, then the right side of (\ref{SEeq:14}) will
vanish, then we can get the (\ref{SEeq:15}).


It's worthy to note here that the probability has localization
character. If the probability in some space has been reduced, then
in somewhere else there's must have probability increased.
Therefore, the probability is locally conservative.

%%%%%%%%%%%%%%%%%%%%%%%%%%%%%%%%%%%%%%%%%%%%%%%%%%%%%%%%%%%%%%%%%%%%%%%%%%%%%%%%%%%%%%%%%%%%%%%%%%%%%%%
\section{Analog with Newton laws}
%
% 1  derive the relation about velocity
% 2  derive the acceleration operator
%
%
Now we begin to do some interesting job, that to make some
comparison between the Schrodinger equation and the classical Newton
laws.

The first thing involved into the consideration is that how to
express the velocity of \heit{v} (it's a vector operator!). We begin
with the initial expression of $v = \frac{dr}{dt}$, then from the
equation of (\ref{SEeq:3}), we can have:
\begin{equation}\label{}
\frac{d \overline{\hei{r}(t)}}{dt} =
\bra{\psi(t)}\frac{1}{i\hbar}[\hei{r},\hat{H}]+\frac{\partial\hei{r}}{\partial
t}\ket{\psi(t)}
\end{equation}
Therefore, we can arrive at:
\begin{equation}\label{SEeq:16}
\frac{d \hei{r}(t)}{dt} = \frac{1}{i\hbar}[\hei{r},\hat{H}]
\end{equation}
Here we note that the second term is vanished because that the
\heit{r} does not explicitly contain the time factor.

Since that in the $\hat{H}$, $V$ is some function of $\hei{r}$ so
that it's able to commuted with \heit{r}. Then for the kinetic terms
we have proved that:
\begin{equation}\label{}
[\hei{r}, \hei{p}\cdot\hei{p}] =-2i\hbar\hei{p}
\end{equation}

Thus for the (\ref{SEeq:16}) we can have:
\begin{align}\label{}
\frac{d \hei{r}(t)}{dt} &=
\frac{1}{i\hbar}\times-\frac{1}{2m}[\hei{r}, \hei{p}\cdot\hei{p}]
\nonumber \\
&=\frac{\hei{p}}{m}
\end{align}
Finally it's astonishing that we get the same relation between
$\hei{p}$ and $\hei{v}$.

However, it understand the meaning of $\hei{v}$ in quantum mechanics
is difficult, because the quantum particles do not retain the
trajectory, therefore; it's safe to consider the velocity only as
the differentiation of the average value for the $\hei{r}$ with
respect to the time of $t$.

Second, let's go to see how to express the acceleration operator,
which is defined as $\hei{a} = \frac{d\hei{v}}{dt}$. Similarly, we
can get:
\begin{align}\label{}
\hei{a} &= \frac{1}{i\hbar}[\hei{v},\hat{H}] \nonumber \\
&= \frac{1}{im\hbar}[\hei{p},\hat{V}] \nonumber \\
&=\frac{1}{im\hbar} i\hbar (\nabla\hat{V})
\end{align}
Here in this derivation we have used the relation defined in
(\ref{OPERATORMOREeq:10}).

Then finally we can get:
\begin{equation}\label{}
m \hei{a} = \nabla\hat{V}
\end{equation}
It has the same form with its counterpart in the classical
mechanics.

%%%%%%%%%%%%%%%%%%%%%%%%%%%%%%%%%%%%%%%%%%%%%%%%%%%%%%%%%%%%%%%%%%%%%%%%%%%%%%%%%%%%%%%%%%%%%%%%%%%%%%%
\section{Hellmann-Feynman theorem}
%
% 1  what's the meaning of HF theorem
% 2  proof
%
%
%
Once the set of eigen functions for the $\hat{H}$ has been gotten,
then we can evaluate the expectation values for a variety of
physical quantities. The HF theorem demonstrates that such gauging
process can be simply done by avoiding to use the concrete form of
wave functions.

This theorem is very important in the quantum chemistry while
calculating the gradient around some reference coordinates. More
discussion can see the following chapters.

Suggest that $\hat{H}$ contains some parameter of $\lambda$, and we
have gotten the eigen functions of $\Psi_{n}$ and corresponding
energy of $E_{n}$ for this system ($n = 1, 2, \cdots$). Then the HF
theorem demonstrates that:
\begin{equation}\label{}
\frac{\partial E_{n}}{\partial \lambda} =
\left\langle\Psi_{n}|\frac{\partial \hat{H}}{\partial
\lambda}|\Psi_{n}\right\rangle
\end{equation}

Now let's prove it. According to the Schrodinger equation, we have:
\begin{equation}\label{}
\hat{H}\Psi_{n} = E_{n}\Psi_{n}
\end{equation}
Then to differentiate the above equation with respect to the
$\lambda$, we can get:
\begin{equation}\label{}
\frac{\partial \hat{H}}{\partial \lambda}\Psi_{n} +
\hat{H}\frac{\partial \Psi_{n}}{\partial \lambda} = \frac{\partial
E_{n}}{\partial \lambda}\Psi_{n} +E_{n}\frac{\partial
\Psi_{n}}{\partial \lambda}
\end{equation}
By multiplying the $\Psi^{*}_{n}$ and making integration, we can
have:
\begin{equation}\label{}
\left\langle\Psi_{n}|\frac{\partial \hat{H}}{\partial
\lambda}|\Psi_{n}\right\rangle +
\left\langle\Psi_{n}|\hat{H}|\frac{\partial \Psi_{n}}{\partial
\lambda}\right\rangle = \left\langle\Psi_{n}|\frac{\partial
E_{n}}{\partial \lambda}|\Psi_{n}\right\rangle +
E_{n}\left\langle\Psi_{n}|\frac{\partial \Psi_{n}}{\partial
\lambda}\right\rangle
\end{equation}

Since that $\hat{H}$ is hermitian, then we can have:
\begin{equation}\label{}
\left\langle\Psi_{n}|\hat{H}|\frac{\partial \Psi_{n}}{\partial
\lambda}\right\rangle = E_{n}\left\langle\Psi_{n}|\frac{\partial
\Psi_{n}}{\partial \lambda}\right\rangle
\end{equation}

Therefore, it finally leads to the HF theorem:
\begin{equation}\label{}
\frac{\partial E_{n}}{\partial \lambda} =
\left\langle\Psi_{n}|\frac{\partial \hat{H}}{\partial
\lambda}|\Psi_{n}\right\rangle
\end{equation}



%%%%%%%%%%%%%%%%%%%%%%%%%%%%%%%%%%%%%%%%%%%%%%%%%%%%%%%%%%%%%%%%%%%%%%%%%%%%%%%%%%%%%%%%%%%%%%%%%%%%%%%



%%% Local Variables:
%%% mode: latex
%%% TeX-master: "../../main"
%%% End:
