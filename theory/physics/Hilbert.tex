%
% OK
%
% revised at Jan. 29th; 2009
% revised at Feb. 29th; 2009
% made the discussion of the complete sets more clear
%
% revised on Mar. 22th, 2009
% to use the environment of theorem and law to improve it
%
% revised at July 26th, 2009
% add the discussion of basis functions as examples
% to understand the Hilbert space
%
% revised at Aug. 1st, 2009
% revise the discussion between the relation of wave functions
% and the Hilbert space.
% now it's seems satisfied.
%
%
%

%%%%%%%%%%%%%%%%%%%%%%%%%%%%%%%%%%%%%%%%%%%%%%%%%%%%%%%%%%%%%%%%%%
% important general outline for this chapter:
% 1   Hilbert space is the vector space added
%     the additional inner product rules
% 2   like the vector space, it also has the linear independent
%     relations
% 3   relations between Hilbert space and wave functions
% 4   Bra and Ket

%%%%%%%%%%%%%%%%%%%%%%%%%%%%%%%%%%%%%%%%%%%%%%%%%%%%%%%%%%%%%%%%%%

\chapter{Hilbert space}
%
%
%
\section{Definition of the Hilbert space}
%
% 1  relations between vector space and the Hilbert space
%     Hilbert space is more general, functions can be contained
% 2  rules of the inner product, and its meaning
% 3  the definition of the Hilbert space
% 4  difference between Hilbert space and the common vector space
%


The Hilbert space is a kind of extension to the common vector space.
In the Hilbert space, the element can be the common vectors, or the
functions, or anything which satisfy the definition of Hilbert
space. In the last chapter, the wave functions for the infinite well,
the wave functions for the free particle motion etc. are all the
examples for the Hilbert space. Here we do not intend to give more
examples, they can be found in the common mathematical book or book
related to quantum mechanics. Below we will all use the ``vector'' to
signal the common vectors in the vector space or the elements in the
Hilbert space, its meaning will be judged to be clear by the context.

The Hilbert space has all the operations in the common vector space.
Besides the normal operations, the Hilbert space has some operation
called inner product; that is to pick up two vectors from the space,
we always has a number(real or complex) corresponds to it:
\begin{equation}\label{}
(\Phi,\Psi) = c
\end{equation}
Here the $\Phi$ and $\Psi$ denotes two vectors in the Hilbert space.
In quantum mechanics, the inner product between wave functions is
always expressed as:
\begin{equation}\label{}
\int\Phi^{*}\Psi d\tau
\end{equation}
Where the $\Phi^{*}$ is conjugated with the $\Phi$. In vector space,
that is corresponding to the common dot product between two vectors.

The inner product is highly important in the quantum mechanics. In
the following content, it can see that this is the base for the
``bra'' and ``ket'', on the other hand; the probability character of
the wave functions(also the average eigen values), the orthogonality
etc. are expressed by such inner product between the wave functions
and its conjugated form.

Then we introduce the definition of Hilbert space. If any of the
vectors satisfy the rules below; they are configuring the Hilbert
space (here $a, b$ is used to indicate some arbitrary number and the
capital Greek letter is the vector from the given space):
\begin{eqnarray}
\label{Hilberteq:2}
% \nonumber to remove numbering (before each equation)
  \Phi + \Psi &=& \Psi + \Phi \nonumber \\
  \Phi + (\Psi + \Omega) &=& (\Phi + \Psi) + \Omega \nonumber \\
  \Phi + 0 &=& \Phi  \quad
  \text{we have zero vector in the space} \nonumber \\
  \Phi + \Psi &=& 0  \quad
  \text{for any $\Phi$, such $\Psi$ exists} \nonumber \\
  \Phi 1 &=& \Phi  \quad
  \text{any vector multiply number 1 is still itself} \nonumber \\
  (\Phi a)b &=& \Phi(ab) \nonumber \\
    \Phi (a+b) &=& \Phi a+ \Phi b \nonumber \\
      (\Phi + \Psi)a &=& \Phi a + \Psi a \nonumber \\
  (\Phi,\Psi) &=& (\Psi, \Phi)^{*} \nonumber \\
  (\Phi,\Psi + \Omega) &=&  (\Phi,\Omega) + (\Phi,\Omega)\nonumber \\
  (\Phi,\Psi a) &=& (\Phi,\Psi)a \nonumber \\
  (\Psi,\Psi) & \geq& 0 \qquad
  \text{for any $\Psi$, if it equals to 0, the $\Psi = 0$ }
\end{eqnarray}
Here we note that for the operation between number and the vector,
it should remember that:
\begin{equation}\label{}
\Phi a = a\Phi
\end{equation}
The relations defined in the (\ref{Hilberteq:2}) are some common
extension from the vector space. Besides, it can be easily seen that
the wave functions for a specific system satisfy all the conditions
above. So the function space is a kind of special Hilbert space.

Here in the Hilbert space two vectors are orthogonal if their inner
product is 0. The inner product of $(\Psi,\Psi)$, which can be
abbreviated as $|\Psi|^{2}$; represents the ``length'' of this
vector; in quantum mechanics, it's directly connected to the
probability distribution of the wave functions; therefore it
requires that for any wave function we should have $|\Psi|^{2} = 1$.

In common vector space, we always refer to some finite number of
vectors, however, the Hilbert space can be infinite; and the vectors
can be continuously changed (such as the case in the free particles,
if $\hei{p}$ changes in the continuous way; the wave functions also
changes coordinately); or discontinuously changed (see the case of the
deep infinite well; where the wave functions depends on a natural
number of $n$ inside its expression). However, since that in most of
the cases the discrete case and the continuous case are sharing the
same results, so for simplicity in the following content we only
concentrate on the discussion on the finite number of the vectors in
the Hilbert space.

%%%%%%%%%%%%%%%%%%%%%%%%%%%%%%%%%%%%%%%%%%%%%%%%%%%%%%%%%%%%%%%%%%%%%%%%%%%%%%
\section{Linear independent vectors}
\label{LIV_in_Hilbert}
%
% introduce the linear independent vectors in the vector space, two points:
%   1  the definition of the linear independent vectors
%      how to understand it. bridge the orthogonality with
%      the linear independent condition
%   2  Schmidt method
%   3  complete sets introduction, identical sets and dimension
%   4  why we introduce the Hilbert space
%   3  degeneracy phenomenon, related to the eigen values
% same in the Hilbert space, but the element form are more complicated
%

In the Hilbert space, if we have $n$ vectors:
\begin{equation}\label{Hilberteq:1}
\sum^{n}_{i=1}a_{i}\Psi_{i} = 0
\end{equation}
This equation is only satisfied as long as all the $a_{i} = 0
(i=1,2,\cdots, n)$; then it can say that such $n$ vectors are linear
independent.

From this definition, we can get some important character for the
linear independent vectors in the Hilbert space:
\begin{theorem}
if the given $n$ arbitrary vectors are not linear independent, then
these $n$ vectors can not be orthogonal with each other.
\end{theorem}

\begin{proof}
  Suggest that we have some linear combination of these $n$ linear
  dependent arbitrary vectors equals to zero, and suggest that the
  coefficient for the $\Phi$ is not zero:
\begin{align}\label{}
  a_{1}\Phi + \sum_{i=2}^{n}a_{i}\Psi_{i} &= 0 \Rightarrow \nonumber
  \\
  \sum_{i=2}^{n}a_{i}\Psi_{i} &=-a_{1}\Phi
\end{align}
Now to multiply the above equation with $\Phi^{*}$ and make
integration; then we have (here we suggest that $(\Phi, \Phi) = 1$,
but in fact it can be any number but without hurting the proof here):
\begin{equation}\label{Hilberteq:4}
\sum_{i=2}^{n}a_{i}(\Phi,\Psi_{i}) = -a_{1}
\end{equation}
Here if all the vectors are orthogonal with each other, that implies
we have:
\begin{equation}\label{}
\left\{
  \begin{array}{ll}
    (\Phi,\Psi_{i}) = 0, & i=2,3,\cdots, n \\
    (\Psi_{i},\Psi_{j}) = 0, & i \neq j
  \end{array}
\right.
\end{equation}
Therefore it can see that the left side (\ref{Hilberteq:4}) must be
zero due to the orthogonality condition, yet the right side is some
non-zero number of $-a_{1}$; that brings contradiction. Hence there
must be at least two vectors in the Hilbert space that they are not
orthogonal with each other. \qedhere
\end{proof}

This simple derivation reveals some important fact that if the vectors
in the Hilbert space are all orthogonal with each other then they must
be linear independent. However, we note that the reverse proposition
is not correct. For $n$ linear independent vectors in the Hilbert
space they are not needed to be orthogonal with each other. Let's take
the vectors as an example, the vector of $\alpha_{1} = (1, 0, 0)$,
$\alpha_{2} = (0, 1, 0)$ and $\alpha_{3} = (0, 1, 1)$ are three linear
independent vectors; but we do not have $(\alpha_{2}, \alpha_{3}) =
0$.

However, for any linear independent vectors, there's always has an way
to make them into the ``basis vectors'' (in the function space we
usually call them as ``basis functions''), in which the vectors are
normalized and orthogonal to each other. Here we introduce the Schmidt
method which transform the linear independent vectors of $\alpha_{1},
\alpha_{2}, \cdots, \alpha_{n}$ into the basis vectors of $\beta_{1},
\beta_{2}, \cdots, \beta_{n}$:
\begin{itemize}
  \item First to normalize the $\alpha_{1}$, so that we have:
\begin{equation}\label{}
\beta_{1} = \frac{\alpha_{1}}{\sqrt{|\alpha_{1}|^{2}}}
\end{equation}
  \item Second to make $\beta_{2}^{'}$ as:
\begin{equation}\label{}
\beta_{2}^{'} = \alpha_{2} - a_{12}\beta_{1}
\end{equation}
$a_{12}$ is some undetermined coefficients. Suggest that we have
orthogonal condition between the $\beta_{1}$ and $\beta_{2}^{'}$,
thus we can determine the $a_{12}$ from this condition:
\begin{align}\label{}
(\beta_{1}, \beta_{2}^{'}) &= 0 \Rightarrow \nonumber \\
a_{12} &= \frac{(\alpha_{1},\alpha_{2})}{\sqrt{|\alpha_{1}|^{2}}}
\nonumber \\
a_{12} &= (\beta_{1},\alpha_{2})
\end{align}
Finally we can achieve the $\beta_{2}$ as:
\begin{equation}\label{}
\beta_{2} = \frac{\beta_{2}^{'}}{\sqrt{|\beta_{2}^{'}|^{2}}}
\end{equation}
  \item Similarly we can have $\beta_{3}^{'}$ as:
\begin{equation}\label{}
\beta_{3}^{'} = \alpha_{3} - a_{13}\beta_{1} - a_{23}\beta_{2}
\end{equation}
By the similar process we can get the coefficients of $a_{13}$ and
$a_{23}$; then finally we can have:
\begin{equation}\label{}
\beta_{3} = \frac{\beta_{3}^{'}}{\sqrt{|\beta_{3}^{'}|^{2}}}
\end{equation}
\item For the arbitrary $\beta_{m}$ ($m \leq n$); likely we can
have:
\begin{equation}\label{}
\beta_{m}^{'} = \alpha_{m} - \sum_{i=1}^{m-1}a_{im}\beta_{i}
\end{equation}
Since that each pair of $\beta_{i}$ and $\beta_{j}$ are orthogonal
for different $i$ and $j$; thus we can get $m-1$ functions to
determine the unknown parameter of $a_{im}$. Finally, we can have:
\begin{equation}\label{}
\beta_{m} = \frac{\beta_{m}^{'}}{\sqrt{|\beta_{m}^{'}|^{2}}}
\end{equation}
\end{itemize}

From the above simple derivation, we can introduce the concept of
``complete sets'', which it is identical to the whole space configured
by the given basis vectors. We note that in the given Hilbert space an
arbitrary vector of $\Psi$ can be always expressed by this complete
sets:
\begin{equation}\label{Hilberteq:5}
\Psi = \sum^{n}_{i=1}a_{i}\Psi_{i}
\end{equation}
Hence, all the other vectors in the given Hilbert space can be
expressed via such chosen complete sets.

It's easy to prove that such complete sets is not unique, there are
variety of complete sets can be used to express the same Hilbert
space; however, it can prove that they are identical with each other
(see the chapter of representation for more details). On the other
hand, the dimension of the space is some fixed number among the
identical complete sets (for detailed proof see the
book\cite{XingLinKe}, PP $7$).

The most important character for the Hilbert space is perhaps
reflected by the (\ref{Hilberteq:5}), where it's actually some kind
of abstraction of the superposition principle in the quantum
mechanics. Based on this property, the only thing need to do in the
quantum mechanics is to search a complete set for the Hilbert space
which corresponds to the investigating system, then we can get any
information form this sets.

Finally, let's conclude our derivation into some theorem which is
called ``perfectibility theorem''. That is:
\begin{theorem}
\label{Hilbert:1}
if $\Psi_{i}$ ($i=1,2,\cdots, n$) is a group of
normalized vectors in Hilbert space where they are orthogonal with
each other; then the four propositions below are identical with each
other:
\begin{itemize}
  \item $\Psi_{i}$ is the complete sets where any other vectors can
  be expressed from this sets;
  \item $\Psi = \sum^{n}_{i=1}(\Psi_{i}, \Psi)\Psi_{i}$ is true for
  any arbitrary vectors in the given Hilbert space;
  \item $(\Psi, \Phi) = \sum^{n}_{i=1}(\Psi, \Psi_{i})(\Psi_{i},
  \Phi)$ holds true for any vectors of $\Phi$ and $\Psi$ in the
  given Hilbert space;
  \item $|\Psi|^{2} = \sum^{n}_{i=1}|(\Psi_{i}, \Psi)|^{2}$ holds
  true for any arbitrary vector of $\Psi$.
\end{itemize}
\end{theorem}

The proof for this theorem is direct and simple so here we omitted
it (if necessary, it can be found at book by XinLin
Ke\cite{XingLinKe}, PP. 9).


%%%%%%%%%%%%%%%%%%%%%%%%%%%%%%%%%%%%%%%%%%%%%%%%%%%%%%%%%%%%%%%%%%%%%%%%%%%%%%%%%%

\section{Axiom for quantum mechanics in terms of Hilbert space}
\label{sec:AFQMITOHS_in_Hilbert}

As what we have been demonstrated, the quantum mechanics can be
considered as some axiomatic system which is logically based on
several axioms; from these axioms we can build the whole methodology
in quantum mechanics so that to tackle down any problems in this
realm.

The first axiom is related to the Hilbert space, that is:
\begin{axiom}\label{axiom1}
\textbf{The system state in quantum mechanics is described by a
vector in
  Hilbert space. Any two vectors which is different only by some
  complex factor is same with each other.}
\end{axiom}


%%%%%%%%%%%%%%%%%%%%%%%%%%%%%%%%%%%%%%%%%%%%%%%%%%%%%%%%%%%%%%%%%%%%%%%%%%%%%%%%%%
\section{The relationship between wave functions and Hilbert space}
\label{TRBWFAHS_in_Hilbert}
%
% 1 Hilbert space is some abstraction of the wave functions, wave functions
%   are only some functions which composed into a space
% 2 the variables, x, y, z, t, and the group character
% 3 the probability character
% 4 superposition character
% 5  what the Hilbert space composed of?
%
%
%
In the chapter \ref{basic_chapter}, where we have demonstrated that
the wave function is expressed by the form of $\Psi(\bm{r}, t)$ which
is the function of coordinates and time. Hence, here how can we
understand the relation between such wave functions and the Hilbert
space?

The wave functions indicated in the \ref{basic_chapter}, is actually
the solutions to the Schrodinger equation. For one specific
Schrodinger equation, it always gives a group of functions as the
``solution in cluster'', such cluster is labeled by some arguments.
For example, the clustered solution in free particle system is
controlled by the argument of $\hei{p}$ in (\ref{BASICeq:13}), in
infinite well system the clustered solution is controlled by the $n$
in (\ref{ASPIIDW_in_basic}). As the $\hei{p}$ or the $n$ varies, then
the functions is transforming from one to another in the group. That's
the feature of ``solution in cluster''.

Furthermore, for some given one function in the group, it's the
function of the coordinates and time; namely $x, y, z$ in each time of
$t$ (here the spin variables are temporarily omitted for
simplicity). For each group of certain variables of $x, y, z, t$; the
wave function has some value corresponding to it and different
function in the group usually gives different value for the fixing $x,
y, z, t$. Physically, different wave functions is used to
characterize different state of the corresponding system; for example,
in quantum chemistry calculation, different wave functions always
occupy different energy state, so they are used to describe different
energy state of the chemical system.

One of important feature of the wave functions, is that its modulus
characterizes the probability distribution of the system. This
character brings on an important trait in the Hilbert space, that the
``length'' of the vectors in the Hilbert space has no physical
meaning, only the normalized vectors are meaningful, and the
integration of the modulus over whole space equals to $1$:
$(\Psi,\Psi) = 1$. Thus any vectors in the Hilbert space which
different by no more than a constant will be same in portraying the
system state. However, there also exists some functions which can not
normalized, for example; the wave functions for the free
particle. However, such functions also retain important physical
meaning, in the content below we will present more discussion to this
type of wave functions.

Another important feature of the wave functions, is that they
satisfy the superposition principle; which can be directly observed
from the Schrodinger equation. On the other hand, it can see that
the superposition principle is the foundation of the Hilbert space;
since from the (\ref{Hilberteq:5}) any vector in the space can be
expressed as the linear combination of the vectors in the complete
sets. In a sense, the property in (\ref{Hilberteq:5}) in Hilbert
space really depicts the essence behind the superposition of the
wave functions.

Physically, the Hilbert space can be viewed as the abstraction of
the wave functions in the quantum mechanics. The wave functions is
some concrete function (for example, it can be plane waves, gaussian
functions etc.), while in the Hilbert space we abstracted them as
some ``vector'' and discuss their general properties. Therefore, the
Hilbert space can be seen as some extension for the function
space\cite{Coden}. Such abstraction does not only bring us
convenience, but also actually be indispensable. In the
non-relativistic quantum mechanics, as we can see in the following
chapter; the spin phenomenon is introduced as some additional
``hypothesis'' where we do not have correspondence in the
macroscopical world. Hence, for the discussion of the spin we can
not write down the specific form of operators then to get the
concrete form of wave functions, thus the operator is only
abstractly introduced and as well the wave function (see the chapter
of spin). However, such wave functions can be also expressed by the
``vectors'' in the Hilbert space so that we can still discuss its
properties.

Now let's concentrate on another aspect that what kind of space the
Hilbert space including?

Generally to say, nearly all the wave functions are actually composed
into the normal function space. The function space is some group of
functions (such functions can be discretely varied or continuously
varied, for example; in the \ref{ASPIIDW_in_basic} the functions are
discretely varied from each to another by the variable of $n$), and any
two functions satisfy the definition in (\ref{Hilberteq:2}), more
specifically; the functions should be square-integrable:
\begin{equation}
  \int |\Psi|^{2} d\tau = 1
\end{equation}
So the last item in the (\ref{Hilberteq:2}) is more restricted to be
$(\Psi, \Psi) = 1$. In physics, since the functions characterizes
the distribution of the particle in given state, so it requires that
the functions should be continuously differential, uniform etc. Here
the wave functions in the (\ref{ASPIIDW_in_basic}) is one example of
the function space.

However, the function space is only one part of the Hilbert space, the
Hilbert space also contains some ``irregular space'' in which the wave
functions are not square-integrable. For example, the plane wave
functions we have derived in the (\ref{ASCWFFFP_in_basic}). The
integration of such wave function will lead to infinity. However, it
also satisfy the definition in the (\ref{Hilberteq:2}), so it's also a
kind of space belong to Hilbert type.

In the following content, we will give some examples of the Hilbert
space to see how to bridge their the general properties with the
concrete functions.


%%%%%%%%%%%%%%%%%%%%%%%%%%%%%%%%%%%%%%%%%%%%%%%%%%%%%%%%%%%%%%%%%%%%%%%%%%%%%%%%%%
\section{Some examples}
\label{sec:SE_in_Hilbert_space}

Now let's give some examples to demonstrate the above discussion for
the Hilbert space. The examples is all related to the function space,
and can be divided into two classes: one is discrete orthogonal basis
functions in common function space, the other is continuous orthogonal
basis functions which is not square integrable. From the discussion
below, finally we will draw some similarity between them\footnote{This
  part of discussion is taken from \cite{Coden}}.


%%%%%%%%%%%%%%%%%%%%%%%%%%%%%%%%%%%%%%%%%%%%%%%%%%%%%%%%%%%%%%%%%%%%%%%%%%%%%

\subsection{Discrete orthogonal basis functions}
\label{sec:DOBF_in_Hilbert_space}

Now consider a group of wave functions, they are corresponding to some
given $\hat{H}$ (for example, the wave functions for infinite well in
the \ref{ASPIIDW_in_basic}). These functions can be generally
expressed as:
\begin{equation}
\label{Hilberteq:11}
  \mu_{1}(\mathbf{r}) \quad  \mu_{2}(\mathbf{r}) \quad  \mu_{3}(\mathbf{r}) \cdots
\mu_{n}(\mathbf{r})
\end{equation}

We assume that such $n$ functions are orthogonal with each other:
\begin{equation}
  \int \mu_{i}(\mathbf{r}) \mu_{j}(\mathbf{r}) d^{3}\mathbf{r} = \delta_{ij}
\end{equation}

Therefore, according to the (\ref{Hilberteq:5}), for any quantum
states of $\Phi(\mathbf{r})$ corresponding to the system, it can be
expressed as:
\begin{equation}
\label{Hilberteq:3}
  \Phi(\mathbf{r}) = \sum_{i=1}^{n}c_{i}\mu_{i}(\mathbf{r})
\end{equation}

$c_{i}$ is given by:
\begin{align}
  (\mu_{i}(\mathbf{r}), \Phi(\mathbf{r})) &=
  \sum_{j=1}^{n}c_{j}(\mu_{i}(\mathbf{r}), \mu_{j}(\mathbf{r})) \nonumber
  \\
  &=\sum_{j=1}^{n}c_{j}\delta_{ij} \nonumber \\
  &=c_{i}
\end{align}

Therefore, we have:
\begin{equation}
  \Phi(\mathbf{r}) = \sum_{i=1}^{n}(\mu_{i}(\mathbf{r}),
  \Phi(\mathbf{r}))\mu_{i}(\mathbf{r})
\end{equation}

For two different quantum states of $\Phi(\mathbf{r})$ and
$\Psi(\mathbf{r})$, where they are expressed as:
\begin{align}
  \Phi(\mathbf{r}) &= \sum_{i=1}^{n}a_{i}\mu_{i}(\mathbf{r}) \nonumber \\
  \Psi(\mathbf{r}) &= \sum_{i=1}^{n}b_{i}\mu_{i}(\mathbf{r})
\end{align}

Their inner product can be given as:
\begin{align}
\label{Hilberteq:13}
  (\Phi(\mathbf{r}), \Psi(\mathbf{r})) &=
  \sum_{i=1}^{n}\sum_{j=1}^{n}a^{*}_{i}b_{j} (\mu_{i}(\mathbf{r}),
  \mu_{j}(\mathbf{r})) \nonumber \\
  &=  \sum_{i=1}^{n}\sum_{j=1}^{n}a^{*}_{i}b_{j}\delta_{ij} \nonumber \\
  &=  \sum_{i=1}^{n}a^{*}_{i}b_{i}
\end{align}

%%%%%%%%%%%%%%%%%%%%%%%%%%%%%%%%%%%%%%%%%%%%%%%%%%%%%%%%%%%%%%%%%%%%%%%%%%%%%

\subsection{Closure relation}
\label{sec:CR_in_Hilbert}

Now let's set up another relation which is identical to the definition
of complete sets defined in the section of \ref{LIV_in_Hilbert}. This
is called closure relation.

From the (\ref{Hilberteq:3}), we can write:
\begin{align}
\label{Hilberteq:15}
  \Phi(\mathbf{r}) &= \sum_{i=1}^{n}c_{i}\mu_{i}(\mathbf{r}) \nonumber
  \\
&= \sum_{i=1}^{n}(\mu_{i}(\mathbf{r}),
  \Phi(\mathbf{r}))\mu_{i}(\mathbf{r}) \nonumber \\
&= \sum_{i=1}^{n}\left[ \int
\mu_{i}(\mathbf{r}^{'})\Phi(\mathbf{r}^{'})d^{3}\mathbf{r}^{'}
\right]\mu_{i}(\mathbf{r}) \quad
\underrightarrow{\text{interchange $\sum$ and $\int$}} \nonumber \\
&= \int
\left[ \sum_{i=1}^{n}
\mu_{i}(\mathbf{r}^{'})\mu_{i}(\mathbf{r})\right]
\Phi(\mathbf{r}^{'})d^{3}\mathbf{r}^{'} \nonumber \\
&= \int F(\mathbf{r}, \mathbf{r}^{'})\Phi(\mathbf{r}^{'})d^{3}\mathbf{r}^{'}
\end{align}

Hence we have:
\begin{equation}
  \label{Hilberteq:6}
  F(\mathbf{r}, \mathbf{r}^{'}) = \sum_{i=1}^{n}
\mu_{i}(\mathbf{r}^{'})\mu_{i}(\mathbf{r}) = \delta (\mathbf{r}
-\mathbf{r}^{'})
\end{equation}

The relation in the (\ref{Hilberteq:6}) is the closure relation. On
the other hand, we can prove that from the (\ref{Hilberteq:6}) we can
also derive the (\ref{Hilberteq:3})\cite{Coden}, so that to prove it's
identical to the definition of complete sets.

%%%%%%%%%%%%%%%%%%%%%%%%%%%%%%%%%%%%%%%%%%%%%%%%%%%%%%%%%%%%%%%%%%%%%%%%%%%%%

\subsection{Plane wave functions as continuous basis functions}
\label{sec:PWF_in_Hilbert}

In section of (\ref{sec:HTEAAWFFFP_in_basic}), we have introduced the
free particle wave function as well as the wave packets. From the
(\ref{BASICeq:19}) and (\ref{BASICeq:20}), they can be generally
expressed as:
\begin{align}
  \label{Hilberteq:7}
\Phi(x) &=
\frac{1}{\sqrt[2]{2\pi}}\int_{-\infty}^{+\infty}\Psi(p)e^{ipx}dp
\nonumber \\
\Psi(p) &=
\frac{1}{\sqrt[2]{2\pi}}\int_{-\infty}^{+\infty}\Phi(x)e^{-ipx}dx
\end{align}
Here we have omit the irrelevant time part and restrict to the X
direction for simplicity.

If we write the plane wave function of $\nu_{p}(x)$ as:
\begin{equation}
  \label{Hilberteq:8}
  \nu_{p}(x) = \frac{1}{\sqrt[2]{2\pi}}e^{ipx}
\end{equation}

Then the (\ref{Hilberteq:7}) can be converted to:
\begin{align}
  \label{Hilberteq:9}
\Phi(x) &= \int_{-\infty}^{+\infty}\Psi(p)\nu_{p}(x)dp
\nonumber \\
\Psi(p) &= \int_{-\infty}^{+\infty}\Phi(x)\nu_{p}^{*}(x)dx = (v_{p}(x), \Phi(x))
\end{align}

As $p$ continuously varied, since $|\nu_{p}(x)|^{2} = \frac{1}{2\pi}$,
thus $\nu_{p}(x)$ is not square integrable. However, we can use the
delta function to express their integration that:
\begin{equation}
  \label{Hilberteq:16}
  \int \nu_{p}^{*}(x)\nu_{p}(x) dx = \delta(x)
\end{equation}
The delta function is detailed discussed in the appendix.

Now let's make some comparison between the (\ref{Hilberteq:11}) and
(\ref{Hilberteq:8}), we can see many similarities between the
$\nu_{p}(x)$ and the discrete basis functions.:
\begin{align}
  \label{Hilberteq:10}
p &\Leftrightarrow i \nonumber \\
x &\Leftrightarrow \bm{r} \nonumber \\
\nu_{p}(x) &\Leftrightarrow \mu_{i}(\bm{r}) \nonumber \\
\Phi(x) &\Leftrightarrow \Phi(\bm{r})
\end{align}

Then we can see the relation in the (\ref{Hilberteq:3}) just
equivalent to the (\ref{Hilberteq:9}). So the $c_{i}$ in the
(\ref{Hilberteq:3}) equivalent to the $\Psi(p)$ in the
(\ref{Hilberteq:9}).

Then if we have two wave packets, $\Phi_{1}(x)$ and $\Phi_{2}(x)$,
then we can evaluate their inner product as:
\begin{align}
  \label{Hilberteq:12}
  (\Phi_{1}(x), \Phi_{2}(x)) &= \int_{-\infty}^{+\infty}
  \int_{-\infty}^{+\infty}\int_{-\infty}^{+\infty}
  \Psi^{*}_{1}(p)\nu^{*}_{p}(x)\Psi_{2}(p^{'})\nu_{p^{'}}(x)dpdp^{'}dx
  \nonumber \\
  &= \int_{-\infty}^{+\infty}
  \int_{-\infty}^{+\infty}\int_{-\infty}^{+\infty}
  \Psi^{*}_{1}(p), \Psi_{2}(p^{'})\delta(p-p^{'})dpdp^{'}
  \nonumber \\
  &= \int_{-\infty}^{+\infty}\Psi^{*}_{1}(p), \Psi_{2}(p)dp
\end{align}
So finally the result reveals that it's equivalent to the
(\ref{Hilberteq:13}). Here we have imposed the orthogonality of the
continuous basis functions.

Finally, let's reveal how to express the closure relation in the
(\ref{sec:CR_in_Hilbert}), which is due to the orthogonality for the
plane wave functions:
\begin{align}
  \label{Hilberteq:14}
\int \nu_{p}^{*}(x)\nu_{p^{'}}(x)dx = \delta(p - p^{'}) \nonumber \\
\int \nu_{p}^{*}(x)\nu_{p}(x^{'})dp = \delta(x - x^{'})
\end{align}
This is equivalent to the closure relation in the
(\ref{sec:CR_in_Hilbert}).



%%%%%%%%%%%%%%%%%%%%%%%%%%%%%%%%%%%%%%%%%%%%%%%%%%%%%%%%%%%%%%%%%%%%%%%%%%%%%

\section{Subspace}
%
% some general discussion for the subspace
%
%
Sometimes we have encounter such circumstance that in the given
complete sets, there's a small group of vectors which is also
composed into some Hilbert space. This is called ``subspace''
compared with the original Hilbert space.

In quantum mechanics, such subspace is very common. For example, in
the following content we will introduce the eigen function (see the
chapter of ``General Discussion to Operator''):
\begin{equation}\label{}
\hat{A}\Psi_{i} = a_{i}\Psi_{i}
\end{equation}
$\hat{A}$ is some operator. Here we may meet such case that for a
group of $\Psi_{ij}$ ($j=1,2, \cdots, m$) they all give the eigen
value of $a_{i}$. Such $m$ wave functions just compose into some
subspace inside the complete set $\Psi_{ij}$ ($i=1,2,\cdots, n$).

It's easy to prove that if the subspace is in $m$ dimension, then we
can find another $n-m$ vectors to form the whole complete sets for
$\Psi_{ij}$ ($i=1,2,\cdots, n$) together with the $\Psi_{ij}$
($j=1,2, \cdots, m$). The demonstration is direct and
straightforward (see \cite{XingLinKe}, PP. 12). therefore, no matter
whether the subspace exists, the dimension of the complete sets are
fixed.




%%%%%%%%%%%%%%%%%%%%%%%%%%%%%%%%%%%%%%%%%%%%%%%%%%%%%%%%%%%%%%%%%%%%%%%%%%%%%
\section{Bra and Ket}
%
%  1  why we introduce the bra and ket
%  2  definition
%  3  one to one correspondent relation
%  4  different with complex number
%
In the above definition, it can see that the inner product is related
to the multiplication order between two vectors in Hilbert space. That
is: $(\Phi,\Psi) = (\Psi, \Phi)^{*}$.  Furthermore, it's easy to prove
that $(a\Phi,\Psi) = a^{*}(\Phi,\Psi)$, which is conjugated with
$(\Phi,a\Psi) = (\Phi,\Psi)a$. Thus even for the same vector it's
different between the left position and the right position in the
inner product.

Moreover, in the study of quantum mechanics it's found that to
distinguish the left position and the right position is necessary,
so we introduce the ``bra'' and ``ket'' to represent the left
position and right position, respectively. Mathematically the bra is
written as $\langle\Psi|$, and the ket is written as $|\Psi\rangle$.
They are independent vector space, but coordinate with each other.
It was introduced by Dirac, so this symbolic representation is
called Dirac symbol.

Simply to say, the bra and ket are some dual space, but they are
independent with each other. However, how to understand the relation
between the Hilbert space and the space of ``bra'' and ``ket''? In
short, the bra and ket can be viewed as some extention to the Hilbert
space, just to decompose the original single space into some dual
space for the sake of quantum mechanics. Therefore, in bra and ket the
elements should satisfy the same rules in the Hilbert space:
\begin{eqnarray}
% \nonumber to remove numbering (before each equation)
  |\Psi\rangle + |\Phi\rangle&=& |\Phi\rangle + |\Psi\rangle \nonumber \\
  |\chi\rangle + (|\Psi\rangle + |\Phi\rangle) &=&
  (|\chi\rangle + |\Psi\rangle) + |\Phi\rangle \nonumber \\
  |\Psi\rangle + |O\rangle &=& |\Psi\rangle \nonumber \\
  |\Psi\rangle1 &=& |\Psi\rangle \nonumber\\
  |\Psi\rangle + |\Phi\rangle &=& |O\rangle \nonumber\\
  |\Psi\rangle (a+b) &=& |\Psi\rangle a + |\Psi\rangle b \nonumber\\
  (|\Psi\rangle a)b &=& |\Psi\rangle (ab) \nonumber\\
  (|\Psi\rangle + |\Phi\rangle)a &=& |\Psi\rangle a + |\Phi\rangle a
\end{eqnarray}
It's note that the space of bra satisfy the same rules.

For the bra and ket, it's the inner product that connect them
together, that the inner product is defined as the multiplication
between bra and ket: $\langle\Phi|\Psi\rangle$. hence, the bra and ket
has to comply with the rules related to the inner product:
\begin{eqnarray}
% \nonumber to remove numbering (before each equation)
  \langle\Phi|\Psi\rangle &=& \langle\Psi|\Phi\rangle^{*}      \nonumber\\
  \langle\Omega|(|\Phi\rangle+|\Psi\rangle) &=&
  \langle\Omega|\Phi\rangle + \langle\Omega|\Psi\rangle \nonumber \\
  (\langle\Phi|+\langle\Psi|)|\Omega\rangle &=&
  \langle\Phi|\Omega\rangle + \langle\Psi|\Omega\rangle \nonumber \\
  \langle\Phi|(|\Psi\rangle a) &=& \langle\Phi|\Psi\rangle a \nonumber \\
  (a\langle\Phi|)|\Psi\rangle  &=& a\langle\Phi|\Psi\rangle  \nonumber \\
  \langle\Psi|\Psi\rangle&\geq& 0 \qquad \text{for any vector in space }
\end{eqnarray}

From the definition above, it's easy to see that the bra space is
one to one mapping with the ket space. For an arbitrary \kett{\Psi},
there hast have an \brat{\Psi}; makes $\langle\Psi|\Psi\rangle = 1$,
and only one \brat{\Psi} satisfies this correspondence. Furthermore,
the operations on the \kett{\Psi} is same to \brat{\Psi}. The only
correspondence between \kett{\Psi} and \brat{\Psi} is, the
$c\ket{\Psi}$ is corresponding to $c^{*}\bra{\Psi}$ but not
$c\bra{\Psi}$.

Such correspondence between bra and ket, makes us to call them as
``conjugated vectors''. They are similar to the conjugated complex
number such as $a+ib$ and $a-ib$, however; there's essential
distinction between them( to read the book by Dirac \cite{Dirac} for
more further details, PP 17).

For the conjugated vectors, it's impossible to differentiate the
real part and the imaginary part in the \kett{\Psi} and \brat{\Psi}.
For example, if we have $x=a+ib$, $y=a-ib$; thus $a= \frac{x+y}{2}$
and $b = \frac{x-y}{2i}$. Whereas for the \kett{\Psi} and
\brat{\Psi}, they are in different space, and independent with each
other; \kett{\Psi} $+$ \brat{\Psi} has no meaning, only their
multiplication is meaningful. Thus we called them ``conjugated
vectors''.

Because of the close relation between \kett{\Psi} and \brat{\Psi},
we introduce transposing operation to transform them to each other:
$\ket{\Psi}^{*} = \bra{\Psi}$. Since the bra and ket symbol is
obviously different with the complex number, we still use the
conjugated symbol.

Finally we note that usually we always use the \kett{\Psi} to
represent the wave function.

%%%%%%%%%%%%%%%%%%%%%%%%%%%%%%%%%%%%%%%%%%%%%%%%%%%%%%%%%%%%%%%%%%%%%%%%%%%%%


%%% Local Variables:
%%% mode: latex
%%% TeX-master: "../../main"
%%% End:

% LocalWords:  contious
