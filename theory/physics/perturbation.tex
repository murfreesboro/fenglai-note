%
% firstly set up at Feb. 11th, 2009
%
% revised at 26th, March; to add the new contents:
%    some cases for time independent perturbation;
%    degenerate perturbation function;
%    time dependent case;
%
\chapter{Perturbation Treatment in Quantum Mechanics}
%
%
%
%
%%%%%%%%%%%%%%%%%%%%%%%%%%%%%%%%%%%%%%%%%%%%%%%%%%%%%%%%%%%%%%%%%%%%%%%%%%%%%%%%%%%%%%%%%%%%%%
\section{Introduction}
%
%  1  what's the perturbation theory?
%  2  what is it for?
%  3  general idea
%  4  two perturbation methods. time dependent and time independent
%
In quantum mechanics, except for a few very simple system (such as
the free particle system or the hydrogen atom system), it's usually
impossible to derive the exact wave functions by directly solving
the Schrodinger equation. Hence, the approximated methods for
solution is necessary in dealing with many practical systems. Among
these methods, the perturbation treatment is some powerful and
popular candidate.

As an example, now let's give some general idea about the
perturbation treatment for the time independent case. Suppose that
the Hamiltonian for a given system is the $\hat{H}$, and the
corresponding time independent Schrodinger equation is:
\begin{equation}\label{}
\hat{H}\Psi = E\Psi
\end{equation}
However, the $\Psi$ is impossible to solve directly. On the other
hand, if the $\hat{H}$ can be split into two parts:
\begin{equation}\label{PTIQMeq:4}
\hat{H} = \hat{H}_{0} + \hat{H}^{'}
\end{equation}
Here the eigen states for the $\hat{H}_{0}$ is easy to derive.
Moreover, the $\hat{H}^{'}$ is very small compared with the
$\hat{H}_{0}$ so that $\hat{H}^{'}$ only makes small contribution in
forming the eigen states for the $\hat{H}$, thus the exact eigen
states for the $\hat{H}$ should be very close to the eigen states
for the $\hat{H}_{0}$. In a sense, we can derive the approximated
wave function for the $\hat{H}$ by starting from the eigen states
given by $\hat{H}_{0}$. By considering the progressive effects from
$\hat{H}^{'}$ on the eigen states of $\hat{H}_{0}$, the approximated
wave function is able to come closer and closer to the exact wave
functions.

Usually in the perturbation treatment, there are two general
methods\cite{Dirac}, one is to take the perturbation as some
``modifications'' to the ground states, it will cause motion change
for the system (that means, the wave function is getting change
under the perturbation). This method usually deal with the changes
for the ground state, and the perturbation is independent time. the
other method, is to take the stationary state for the unperturbed
system, and investigate it's changes as time varies. Therefore, this
method is usually used for the time dependent perturbation.

%%%%%%%%%%%%%%%%%%%%%%%%%%%%%%%%%%%%%%%%%%%%%%%%%%%
\section{Time Independent Perturbation
Theory}\label{Time_independent_perturbation_theory}
%
% here we generally consider the perturbation treatment in the
% time independent case
% what is it for?
%
At the beginning, let's go to investigate the perturbation treatment
to the time independent case. Such time independent treatment can be
used to study many practical problems in the quantum mechanics, such
as the coupling between the spin and the orbitals, time independent
external fields etc. Moreover, it's worthy to point out that in
quantum chemistry, the many body perturbation theory methods are
actually derived from the time independent perturbation theory.

%%%%%%%%%%%%%%%%%%%%%%%%%%%%%%%%%%%%%%%%%%%%%%%%%%%%
\subsection{Non-Degenerate Case}
%
% 1  Hamiltonian expansion, discussion about the lambda
% 2  expansion over the wave functions and the energy
% 3  general expression related to the lambda, get a series of
%    perturbed equations
% 4  zero order approximation
% 5  first order
% 6  second order
%
Suggest that for a given system, the Hamiltonian can be expressed
as:
\begin{equation}\label{PTIQMeq:1}
\hat{H} = \hat{H}_{0} + \lambda\hat{H}^{'}
\end{equation}
And the corresponding time independent Schrodinger equation is:
\begin{equation}\label{PTIQMeq:2}
\hat{H}\Phi = E\Phi
\end{equation}
Here the $\hat{H}^{'}$ is very small so that there's only slight
difference between the $\hat{H}$ and $\hat{H}_{0}$. Compared with
(\ref{PTIQMeq:6}), we add some real parameter of $\lambda$ to the
perturbation operator so that the wave functions and the energy can
be expanded according to the order of the $\lambda$ (therefore, this
$\lambda\hat{H}^{'}$ should be small enough so that to guarantee the
feasibility of expansion). Physically this parameter of $\lambda$ is
arbitrary and meaningless, as a kind of standard treatment we use it
only for mathematical clarity\footnote{Here the perturbation method
used here is called ``Rayleigh-Schrodinger'' expansion}.

Now suppose that for the $\hat{H}_{0}$ we can get its complete set
of eigen states and the eigen values:
\begin{equation}\label{PTIQMeq:3}
\hat{H}_{0}\Psi^{0}_{n} = E^{0}_{n}\Psi^{0}_{n} \quad n = 1,
2,\cdots
\end{equation}
We assume that the energy level is not degenerate: $E^{0}_{n} \neq
E^{0}_{m}$ if $n \neq m$. This is the simplest case in the time
independent perturbation treatment, thus we call it as
``non-degenerate case''.

After defining the eigen states for the $\hat{H}_{0}$, we can expand
the wave function of $\Phi$ and the corresponding energy of $E$ in
stepping order of $\lambda$:
\begin{align}\label{PTIQMeq:5}
\Phi &= \Phi^{0} + \lambda\Phi^{1} + \lambda^{2}\Phi^{2} + \cdots
\nonumber \\
E &= E^{0} + \lambda E^{1} + \lambda^{2}E^{2} + \cdots
\end{align}
Here $\Phi^{1}$ denotes the first order modification to the
$\Phi^{0}$, $\Phi^{2}$ denotes the second order modification to the
$\Phi^{0}$ etc. Since $\lambda$ is some small value so that
$\Phi^{i}$ is much smaller than the $\Phi^{i-1}$. Therefore the
expansion can be converged at $i \rightarrow \infty$.

Let's bring the (\ref{PTIQMeq:5}) into the (\ref{PTIQMeq:2}):
\begin{multline}\label{PTIQMeq:6}
(\hat{H}_{0} + \lambda\hat{H}^{'})(\Phi^{0} + \lambda\Phi^{1} +
\lambda^{2}\Phi^{2} + \cdots) = \\
(E^{0} + \lambda E^{1} + \lambda^{2}E^{2} + \cdots)(\Phi^{0} +
\lambda\Phi^{1} + \lambda^{2}\Phi^{2} + \cdots)
\end{multline}
By gathering the terms with same power of $\lambda$ on both side of
the (\ref{PTIQMeq:6}), we can get a series of perturbation
equations:
\begin{align}\label{PTIQMeq:7}
\lambda^{0} &: \hat{H}_{0}\Phi^{0} = E^{0}\Phi^{0} \nonumber \\
\lambda^{1} &: (\hat{H}_{0} - E^{0})\Phi^{1} =
(E^{1} - \hat{H}^{'})\Phi^{0} \nonumber \\
\lambda^{2} &: (\hat{H}_{0} - E^{0})\Phi^{2} =
(E^{1} - \hat{H}^{'})\Phi^{1} +E^{2}\Phi^{0}  \nonumber \\
&\cdots\cdots
\end{align}

\subsubsection{Zero Order Approximation}
Zero order approximation means that there's no perturbation to the
system: that's same with $\lambda = 0$ in (\ref{PTIQMeq:1}) so that
it retreats back to the case defined in (\ref{PTIQMeq:3}). Suppose
that the system is in energy level of $k$, then we have:
\begin{align}\label{PTIQMeq:9}
\Phi^{0}_{k} &= \Psi^{0}_{k} \nonumber \\
E^{0} &= E^{0}_{k}
\end{align}


\subsubsection{First Order Approximation}\label{1st_approximation_WT}
In the first order approximation, it's able to construct the
$\Phi^{1}$ by expanding over the set of eigen states of
$\Psi^{0}_{n}$:
\begin{equation}\label{}
\Phi^{1} = \sum_{n}a^{1}_{n}\Psi^{0}_{n}
\end{equation}
Physically, the $\Phi^{1}$ also compose some complete set wave
functions which is ``slightly modified'' from the set of
$\Psi^{0}_{n}$ ($n=1,2,\cdots$). Therefore for each given energy
level of $k$, we have:
\begin{equation}\label{PTIQMeq:8}
\Phi^{1}_{k} = \sum_{n}a^{1}_{nk}\Psi^{0}_{n} \quad k=1,2,\cdots
\end{equation}
In practice, $\Phi^{1}_{k}$ express the ``modification'' to the
$\Psi^{0}_{k}$ by the perturbation of $\hat{H}^{'}$. Due to the
$\hat{H}^{'}$, the orthogonal condition between the $\Psi^{0}_{n}$
is broken; instead eigen states are mixing together to form some new
sets. For the $\Psi^{0}_{k}$, it can see that the new eigen state of
$\Phi_{k}$ has mixed some of its adjacent eigen states of
$\Psi^{0}_{l}$ ($l \neq k$) on the first order approximation:
\begin{equation}\label{}
\Phi_{k} = \Phi_{k}^{0} + \lambda\Phi_{k}^{1} = \Psi^{0}_{k} +
\lambda\sum_{n}a^{1}_{nk}\Psi^{0}_{n}
\end{equation}
In the following content, we will further investigate such relation.

On the other hand, we note that under the perturbation of
$\hat{H}^{'}$ the new set of eigen states of $\Phi_{i}$ are indeed
orthogonal with each other on the first order of approximation:
\begin{align}\label{}
\langle\Phi_{i}|\Phi_{j}\rangle &= \delta_{ij} + \lambda(a^{1}_{ij}
+ a^{1*}_{ji}) + \lambda^{2}(\sum_{m}\sum_{n}a^{1*}_{mi}a^{1}_{nj}
\langle\Psi^{0}_{m}|\Psi^{0}_{n}\rangle) \nonumber \\
&=\lambda(a^{1}_{ij} + a^{1*}_{ji}) + O(\lambda^{2})
\end{align}
Here we can see some interesting conclusions. On the first order
approximation, if $\Phi_{i}$ and $\Phi_{j}$ are orthogonal with each
other, then it requires that:
\begin{equation}\label{}
a^{1}_{ij} + a^{1*}_{ji} = 0
\end{equation}
Here the higher order(related to the $\lambda^{2}$) is omitted. In
the latter content, we will give the concrete expression for the
$a^{1}_{ij}$, then it can see that they satisfy such relation above.

Now let's bring the (\ref{PTIQMeq:9}) and (\ref{PTIQMeq:8}) into the
first order of perturbation equation defined in (\ref{PTIQMeq:7}):
\begin{align}\label{}
(\hat{H}_{0} - E^{0}_{k})\sum_{n}a^{1}_{nk}\Psi^{0}_{n} &=
(E^{1}_{k} - \hat{H}^{'})\Phi^{0}_{k} \nonumber \\
\sum_{n}(E^{0}_{n} - E^{0}_{k})a^{1}_{nk}\Psi^{0}_{n} &= (E^{1}_{k}
- \hat{H}^{'})\Phi^{0}_{k}
\end{align}
This equation holds true for a given energy level of $k$. If we
multiply both side of the above equation with $\bra{\Psi^{0}_{m}}$
and make integration, then it gives:
\begin{align}\label{PTIQMeq:10}
\sum_{n}(E^{0}_{n} - E^{0}_{k})a^{1}_{nk}\delta_{mn} &=
E^{1}_{k}\delta_{mk} -
\bra{\Psi^{0}_{m}}\hat{H}^{'}\ket{\Psi^{0}_{k}} \nonumber \\
(E^{0}_{m} - E^{0}_{k})a^{1}_{mk} &= E^{1}_{k}\delta_{mk} -
H^{'}_{mk}
\end{align}

Given by the (\ref{PTIQMeq:10}), if $m=k$, thus we can get
\begin{equation}\label{}
E^{1}_{m} = H^{'}_{mm} =
\bra{\Psi^{0}_{m}}\hat{H}^{'}\ket{\Psi^{0}_{m}}
\end{equation}

If $m\neq k$, then we can get the $a^{1}_{mk}$ as:
\begin{equation}\label{}
a^{1}_{mk} = \frac{H^{'}_{mk}}{E^{0}_{k} - E^{0}_{m}}
\end{equation}
The non-degenerate condition has guaranteed that $E^{0}_{k} -
E^{0}_{m} \neq 0$.

However, from the (\ref{PTIQMeq:10}) the $a^{1}_{kk}$ is unable to
be determined. Here we use the normalized condition for the
$\Phi_{k}$:
\begin{align}\label{}
\bra{\Phi_{k}}\hat{H}^{'}\ket{\Phi_{k}} &= 1 \nonumber \\
a^{1}_{kk} + a^{1*}_{kk} &= 0
\end{align}
Therefore, the $a^{1}_{kk}$ is some imaginary number. It can be
proved that this number can be set to $0$ without hurting the
validity of the whole derivation.

All in all, for the first order approximation we can give the wave
function as well as the corresponding energy as:
\begin{align}\label{PTIQMeq:11}
\Phi_{k} &= \Psi^{0}_{k} + \sum_{m\neq k}\frac{H^{'}_{mk}}{E^{0}_{k}
- E^{0}_{m}}\Psi^{0}_{m} \nonumber \\
E_{k} &= E_{k}^{0} + \bra{\Psi^{0}_{k}}\hat{H}^{'}\ket{\Psi^{0}_{k}}
\end{align}

Now let's back to the physical discussion of the first order
approximation. From the result in (\ref{PTIQMeq:11}), it can see
that energy modification is only
$\bra{\Psi^{0}_{k}}\hat{H}^{'}\ket{\Psi^{0}_{k}}$, which is the
expectation value of $\hat{H}^{'}$ for the zero order wave
functions. The first order wave function of $\Phi^{1}$ doest not
appear in the energy expression.

On the other hand, for the $\Phi_{k}$ corrected at first order,
there's really some adjacent eigen states of $\Psi^{0}_{m}$ mixing
into the $\Psi^{0}_{k}$. The magnitude of the mixing state is
determined by the coefficient of $a_{mk}$, which is in turn largely
affected by the energy difference of $E^{0}_{k} - E^{0}_{m}$.
Therefore, it's the adjacent eigen states for the $\Psi^{0}_{k}$
that is able to mix ``strongly'' into the first order wave function,
and if the energy difference is larger, the mixing will get smaller.

\subsubsection{Second Order Approximation}
Analogically, we can expand the $\Phi^{2}_{k}$ over the
$\Psi^{0}_{n}$:
\begin{equation}\label{PTIQMeq:12}
\Phi^{2}_{k} = \sum_{n}a^{2}_{n}\Psi^{0}_{n}
\end{equation}
By taking the (\ref{PTIQMeq:9}), (\ref{PTIQMeq:11}) and
(\ref{PTIQMeq:12}) into the second order perturbation equation in
(\ref{PTIQMeq:7}), we can get:
\begin{equation}\label{PTIQMeq:13}
\sum_{n}(E^{0}_{n} - E^{0}_{k})a^{2}_{nk}\Psi^{0}_{n} = (E^{1}_{k} -
\hat{H}^{'})\Phi^{1}_{k} + E^{2}_{k}\Phi^{0}_{k}
\end{equation}

Similarly by multiplying with $\bra{\Psi^{0}_{m}}$ and make
integration, the (\ref{PTIQMeq:13}) gives:
\begin{align}\label{PTIQMeq:14}
\sum_{n}(E^{0}_{n} - E^{0}_{k})a^{2}_{nk}\delta_{mn} &=
E^{1}_{k}\langle\Psi^{0}_{m}|\Psi^{1}_{k}\rangle -
\bra{\Psi^{0}_{m}}\hat{H}^{'}\ket{\Psi^{1}_{k}} +
E^{2}_{k}\delta_{mk} \nonumber \\
(E^{0}_{m} - E^{0}_{k})a^{2}_{mk} &=
E^{1}_{k}\langle\Psi^{0}_{m}|\Psi^{1}_{k}\rangle -
\bra{\Psi^{0}_{m}}\hat{H}^{'}\ket{\Psi^{1}_{k}} +
E^{2}_{k}\delta_{mk}
\end{align}

Next let's expand the $\langle\Psi^{0}_{m}|\Psi^{1}_{k}\rangle$ and
$\bra{\Psi^{0}_{m}}\hat{H}^{'}\ket{\Psi^{1}_{k}}$ in details. For
the $\langle\Psi^{0}_{m}|\Psi^{1}_{k}\rangle$, it gives:
\begin{align}\label{PTIQMeq:15}
\langle\Psi^{0}_{m}|\Psi^{1}_{k}\rangle &= \sum_{n\neq
k}\frac{H^{'}_{nk}}{E^{0}_{k} - E^{0}_{n}}
\langle\Psi^{0}_{m}|\Psi^{0}_{n}\rangle \nonumber \\
&=\sum_{n\neq k}\frac{H^{'}_{nk}}{E^{0}_{k} - E^{0}_{n}}\delta_{mn}
\end{align}

For the $\bra{\Psi^{0}_{m}}\hat{H}^{'}\ket{\Psi^{1}_{k}}$, it gives:
\begin{align}\label{PTIQMeq:16}
\bra{\Psi^{0}_{m}}\hat{H}^{'}\ket{\Psi^{1}_{k}} &= \sum_{n\neq
k}\frac{H^{'}_{nk}}{E^{0}_{k} -
E^{0}_{n}}\bra{\Psi^{0}_{m}}\hat{H}^{'}\ket{\Psi^{0}_{n}} \nonumber
\\
&=\sum_{n\neq k}\frac{H^{'}_{nk}}{E^{0}_{k} - E^{0}_{n}}H^{'}_{mn}
\end{align}

Now by the (\ref{PTIQMeq:15}) and (\ref{PTIQMeq:16}), let's explore
the (\ref{PTIQMeq:14}) in details. First, if $m=k$, we can get:
\begin{equation}\label{}
E^{2}_{k} = \sum_{n\neq k}\frac{|H^{'}_{nk}|^{2}}{E^{0}_{k} -
E^{0}_{n}}
\end{equation}

Else if $m \neq k$, then it gives:
\begin{equation}\label{}
(E^{0}_{m} - E^{0}_{k})a^{2}_{mk} =
H^{'}_{kk}\frac{H^{'}_{mk}}{E^{0}_{k} - E^{0}_{m}} - \sum_{n\neq
k}\frac{H^{'}_{nk}}{E^{0}_{k} - E^{0}_{n}}H^{'}_{mn}
\end{equation}

That gives:
\begin{equation}\label{}
a^{2}_{mk} = -\frac{H^{'}_{mk}H^{'}_{kk}}{(E^{0}_{k} -
E^{0}_{m})^{2}} + \sum_{n\neq
k}\frac{H^{'}_{mn}H^{'}_{nk}}{(E^{0}_{k} - E^{0}_{n})(E^{0}_{k} -
E^{0}_{m})}
\end{equation}

Now it's only that $a^{2}_{kk}$ has not been determined yet. Like
what has been done in the first order approximation, the
normalization condition is used to work out the $a^{2}_{kk}$. First,
the wave function corrected at second order is:
\begin{equation}\label{}
\Psi_{k} = \Psi_{k}^{0} + \lambda\Psi_{k}^{1} +
\lambda^{2}\Psi_{k}^{2}
\end{equation}
Then by employing the normalization condition of
$\langle\Psi_{k}|\Psi_{k}\rangle = 0$, it gives:
\begin{multline}\label{PTIQMeq:17}
\langle\Psi_{k}^{0}|\Psi_{k}^{0}\rangle +
\lambda\Big(\langle\Psi_{k}^{1}|\Psi_{k}^{0}\rangle +
\langle\Psi_{k}^{0}|\Psi_{k}^{1}\rangle\Big) + \\
\lambda^{2}\Big(\langle\Psi_{k}^{2}|\Psi_{k}^{0}\rangle +
\langle\Psi_{k}^{0}|\Psi_{k}^{2}\rangle +
\langle\Psi_{k}^{1}|\Psi_{k}^{1}\rangle\Big) = 0
\end{multline}

Since that for the terms related to the $\lambda^{1}$, they has been
shown to be zero; so that in the (\ref{PTIQMeq:17}) the terms
involved with $\lambda^{2}$ should be zero to maintain the
normalization condition:
\begin{equation}\label{}
\langle\Psi_{k}^{2}|\Psi_{k}^{0}\rangle +
\langle\Psi_{k}^{0}|\Psi_{k}^{2}\rangle +
\langle\Psi_{k}^{1}|\Psi_{k}^{1}\rangle = 0
\end{equation}
That finally gives:
\begin{equation}\label{}
a^{2}_{kk} = -\frac{1}{2}\sum_{m\neq
k}\frac{|H^{'}_{mk}|^{2}}{(E^{0}_{k} - E^{0}_{m})^{2}}
\end{equation}

All in all, we can express the wave function which correlated at
second order as:
\begin{multline}\label{PTIQMeq:18}
\Phi_{k} = \Psi^{0}_{k} + \sum_{m\neq k}\frac{H^{'}_{mk}}{E^{0}_{k}
- E^{0}_{m}}\Psi^{0}_{m} + \\
\sum_{m\neq k}\Bigg\{-\frac{H^{'}_{mk}H^{'}_{kk}}{(E^{0}_{k} -
E^{0}_{m})^{2}}\Psi^{0}_{m} + \sum_{n\neq
k}\frac{H^{'}_{mn}H^{'}_{nk}}{(E^{0}_{k} - E^{0}_{n})(E^{0}_{k} -
E^{0}_{m})}\Psi^{0}_{m}\Bigg\} - \\
\frac{1}{2}\sum_{m\neq k}\frac{|H^{'}_{mk}|^{2}}{(E^{0}_{k} -
E^{0}_{m})^{2}}\Psi^{0}_{k}
\end{multline}

It's corresponding energy is:
\begin{equation}\label{PTIQMeq:19}
E_{k} = E_{k}^{0} + \bra{\Psi^{0}_{k}}\hat{H}^{'}\ket{\Psi^{0}_{k}}
+ \sum_{n\neq k}\frac{|H^{'}_{nk}|^{2}}{E^{0}_{k} - E^{0}_{n}}
\end{equation}

Finally, let's make some analysis to the $E^{2}$. Interestingly it
can see that $E^{2}$ is the expectation value of $\hat{H}^{'}$
between the $\Phi^{1}$ and $\Phi^{0}$:
\begin{equation}\label{}
E^{2}_{k} = \bra{\Phi^{0}_{k}}\hat{H}^{'}\ket{\Phi^{1}_{k}} =
\sum_{n\neq k}\frac{|H^{'}_{nk}|^{2}}{E^{0}_{k} - E^{0}_{n}}
\end{equation}
On the other hand, if the $E^{0}_{k}$ is selected to be the ground
state of energy; then for any $n \neq k$ we have $E^{0}_{k} -
E^{0}_{n}$. That means, for the ground state the second order
correlation is always less than zero.

%%%%%%%%%%%%%%%%%%%%%%%%%%%%%%%%%%%%%%%%%%%%%%%%%%
\subsection{Limitation for Non-Degenerate Perturbation
Treatment}\label{PTIQM:1}
%
% 1  the second order form of wave function is complicated
% 2  as the energy is near degenerated, the perturbation is wrong
%
%
%
As we can see, the corrected wave function for the second order has
been made very complicated, therefore it's seldom used. Usually in
the Rayleigh perturbation treatment, the wave functions is corrected
at first order, and the energy is corrected at second order.
Actually, the first order corrected wave function is suitable enough
for most of the qualitative discussions. On the other hand, there's
also some other form of perturbation treatment; but the Rayleigh
perturbation method is most widely used.

There's some strong restriction to the above method, which is
attributed from the energy degeneracy phenomenon. As shown in the
(\ref{PTIQMeq:11}) and (\ref{PTIQMeq:19}), if there's near energy
degeneracy for the energy level of $k$, that means energy level $k$
and its adjacent energy levels are very close; it will lead to
abnormal increasing for the terms of
$\frac{|H^{'}_{nk}|^{2}}{E^{0}_{k} - E^{0}_{n}}$ in the second order
corrected energy, and there's similar problem in the expression of
wave functions. Such problems are left for the perturbation
treatment in the degenerate case.

%%%%%%%%%%%%%%%%%%%%%%%%%%%%%%%%%%%%%%%%%%%%%%%%%%%%%%%%%%%%%%%%%%%%%%%%%%%%%%%%%%%%%%%%%%%%%%
\section{Some Examples for Non-Degenerate Perturbation
Treatment}
%
% to give some examples for the non-degenerate perturbation
%
%
\subsection{Perturbation of An Oscillator}
%
%
%
For a simple case\footnote{In this section the example are all taken
from the book \cite{ZengJinYan}}, let's consider the perturbation to
a linear harmonic oscillator. Physically, such linear harmonic
oscillator can be interpreted as an ion in some ``perfect'' crystal
who is oscillating around its equilibrium position (for simplicity
we suggest that the motion of ion is all along the x axis), and the
perturbation is some constant electric filed which can be expressed
as $\hat{H}^{'} = -q\varepsilon x$.

All in all, the Hamiltonian for this system is:
\begin{align}\label{}
\hat{H} &= -\frac{\hbar^{2}}{2m}\frac{d^{2}}{dx^{2}} +
\frac{1}{2}m\omega_{0}^{2}x^{2} - q\varepsilon x \nonumber \\
\hat{H}_{0} &= -\frac{\hbar^{2}}{2m}\frac{d^{2}}{dx^{2}} +
\frac{1}{2}m\omega_{0}^{2}x^{2} \nonumber \\
\hat{H}^{'} &= -q\varepsilon x
\end{align}
For the harmonic oscillator whose Hamiltonian is $\hat{H}_{0}$, it's
eigen states can be easily gotten (please see the chapter discussing
the harmonic oscillator):
\begin{align}\label{}
\Psi_{n}^{(0)}(x) &= N_{n}e^{-\frac{1}{2}a^{2}x^{2}}H_{n}(ax)
\nonumber \\
a &= \sqrt{\frac{m\omega_{0}}{\hbar}} \nonumber \\
N_{n} &= \left(\frac{a}{\sqrt{\pi}2^{n}n!}\right)^{\frac{1}{2}}
\end{align}
Here the $H_{n}(ax)$ is the hermite polynomial, and $N_{n}$ is the
normalized factor.

The energy level corresponds to the unperturbed states is:
\begin{equation}\label{}
E_{n}^{(0)} = (n+\frac{1}{2})\hbar\omega_{0} \quad n=0,1,2\cdots
\end{equation}
It's easily seen that if $n\neq m$ $E_{n}^{(0)}\neq E_{m}^{(0)}$,
thus it's some non-degenerate states.

Now let's use the formula of perturbation treatment to calculate the
perturbed correlation to the energy as well as the wave functions.
Before the derivation, it's convenient to introduce equation to
calculate the perturbation matrix element for the
$\Psi_{n}^{(0)}(x)$:
\begin{align}\label{PTIQMeq:20}
\hat{x}\Psi_{n}^{(0)}(x) &=
\frac{1}{a}\left[\sqrt{\frac{n}{2}}\Psi_{n-1}^{(0)}(x) +
\sqrt{\frac{n+1}{2}}\Psi_{n+1}^{(0)}(x) \right] \Rightarrow
\nonumber \\
x_{kn} = \bra{\Psi_{k}^{(0)}}\hat{x}\ket{\Psi_{n}^{(0)}} &=
\frac{1}{a}\left[\sqrt{\frac{n}{2}}\delta_{k,n-1} +
\sqrt{\frac{n+1}{2}}\delta_{k,n+1} \right]
\end{align}
This equation can be easily gotten from the character of Hermite
polynomial (also see the chapter discussing the harmonic
oscillator).

According to the (\ref{PTIQMeq:19}) the second order correlation to
the energy level of $n$ is:
\begin{align}\label{}
E_{n} &= E_{n}^{(0)} + H^{'}_{nn} + \sum_{k\neq
n}\frac{|H^{'}_{kn}|^{2}}{E_{n}^{(0)} - E_{k}^{(0)}} \nonumber \\
&= (n+\frac{1}{2})\hbar\omega_{0} +
\frac{q^{2}\varepsilon^{2}}{\hbar\omega_{0}} ( |x_{n-1,n}|^{2} -
|x_{n+1,n}|^{2}) \nonumber \\
&= (n+\frac{1}{2})\hbar\omega_{0} -
\frac{q^{2}\varepsilon^{2}}{2m\omega_{0}^{2}}
\end{align}
We note that from the (\ref{PTIQMeq:20}) the $H^{'}_{nn}$ is easily
known to be zero. Here we can see that all the energy level shift
down for $\frac{q^{2}\varepsilon^{2}}{2m\omega_{0}^{2}}$ due to the
perturbation.

For the wave function, the first order correlation is:
\begin{align}\label{}
\Psi_{n}(x) &= \Psi_{n}^{(0)}(x) + \sum_{k\neq
n}\frac{H^{'}_{kn}}{E^{(0)}_{n}
- E^{(0)}_{k}}\Psi^{(0)}_{k}(x) \nonumber \\
&= \Psi_{n}^{(0)}(x) + \frac{q\varepsilon}{\hbar\omega_{0}}
\frac{1}{a}\left[\sqrt{\frac{n}{2}}\Psi_{n-1}^{(0)}(x) -
\sqrt{\frac{n+1}{2}}\Psi_{n+1}^{(0)}(x)\right]
\end{align}
Now after the perturbation in the $\Psi_{n}^{(0)}(x)$ it has been
mixed with two adjacent wave functions; namely the
$\Psi_{n-1}^{(0)}(x)$ and $\Psi_{n+1}^{(0)}(x)$.

In the case without the perturbation, from the (\ref{PTIQMeq:19}) we
can see that the average coordinate value for the oscillator is
zero: $x_{nn} = 0$, it means that the average position for the
oscillator is the equilibrium position. That's coincided with the
natural understanding with the harmonic oscillator. However, In the
perturbation field, it turns out that there's some offset for its
average position:
\begin{equation}\label{}
x_{nn}^{'} = \bra{\Psi_{n}(x)}\hat{x}\ket{\Psi_{n}(x)} =
\frac{q\varepsilon}{m\omega_{0}^{2}}
\end{equation}
However, it can see that the electric filed does not alter the
character of oscillator.

%%%%%%%%%%%%%%%%%%%%%%%%%%%%%%%%%%%%%%%%%%%%%%%%%%%%%%%%%%%%%%%%%%%%%%%%%%%%%%%%%%%%%%%%%%%%%%
\subsection{Van der Waals Interaction}
%
%
%
Van der Waals force plays crucial rule in chemistry. From the view
of quantum mechanics, it can be attributed to the dipole-dipole
interaction between two neutralized molecules and atoms. In this
section let's go to see how to use the approximated dipole-dipole
interaction operator to derive the expression of Van der Waals
energy.

The most simplest system for evaluating the Van der Waals
interaction is two hydrogen atoms system, where both of the two
atoms are far away from each other compared with the atom radius.
For example, the average atom radius $r$ for H atom is $1$ Bohr
radius, which is approximated at $0.5${\AA}.; then if the two
hydrogen atoms separate each other at $R \approx 4\sim 5${\AA}.; we
have $r/R \approx \frac{1}{10}$. In this case the interactions
between the two hydrogen atoms can be fully considered as the Van
der Waals interactions.

For this two hydrogen system, we denote that the distance between
the two nucleus are $R$, and the electron $1$ is near the nuclear
$a$ with average radius of $r_{1}$ while the electron $2$ is
centered around nuclear $b$ with average radius of $r_{2}$.
Therefore the Hamiltonian for this system is:
\begin{align}\label{}
\hat{H} &= \hat{H}_{0} + \hat{H}^{'} \nonumber \\
\hat{H}_{0} &= -\frac{\hbar^{2}}{2m}
(\nabla^{2}_{1} + \nabla^{2}_{2}) - \frac{e^{2}}{r_{1}}
- \frac{e^{2}}{r_{2}} \nonumber \\
\hat{H}^{'} &= \frac{e^{2}}{R^{2}} + \frac{e^{2}}{r_{12}^{2}} -
\frac{e^{2}}{r_{a2}} - \frac{e^{2}}{r_{b1}}
\end{align}

The $\hat{H}^{'}$ is hard for evaluation. However, if $r/R \ll 1$
then we can use the dipole-dipole interactions to replace the old
form:
\begin{equation}\label{PTIQMeq:21}
\hat{H}^{'} = \frac{1}{R^{3}}\Big\{\hei{D}_{1}\cdot\hei{D}_{2}
-3(\hei{D}_{1}\cdot\heiti{e}_{R})(\hei{D}_{2}\cdot\heiti{e}_{R})\Big\}
\end{equation}
Here $\hei{D}_{1} = -e\hei{r}_{1}$, $\hei{D}_{2} = -e\hei{r}_{2}$
and $\hei{e}_{R} = \heiti{R}/R$. The form for the $\hat{H}^{'}$ in
(\ref{PTIQMeq:21}) can be gotten by compared with the electrostatic
field in electrodynamics (This actually corresponds to the induced
dipole in the classical mechanics, and the operator expression could
be similar gotten by comparing with the classical mechanics).

Assume that the hydrogen atom is in ground state. Because that both
of the two atoms are far away from each other, thus we can omit
their overlap to write the whole wave function simply as their
product:
\begin{equation}\label{}
\Psi = \Psi_{1} (r_{1})\Psi_{2} (r_{2})
\end{equation}
Here we can prove that $\bra{\Psi}\hat{H}^{'}\ket{\Psi} = 0$(see the
page of 366 in \cite{ZengJinYan} Vol. one for more details).
Therefore the first order correlation for the $\hat{H}^{'}$ is zero.
Thus the second order correlation for the $\hat{H}^{'}$ is:
\begin{equation}\label{}
E^{(2)} = \sum_{n\neq 0}\frac{|H^{'}_{n0}|^{2}}{E^{0}_{0} -
E^{0}_{n}} \approx -\frac{A}{R^{6}}
\end{equation}
Here $|H^{'}_{n0}| \infty 1/R^{3}$, thus we got the energy scales
with $1/R^{6}$. This result is qualitatively coincide with the
empirical formula which is used to evaluate the Van der Waals
energy.


%%%%%%%%%%%%%%%%%%%%%%%%%%%%%%%%%%%%%%%%%%%%%%%%%%%%%%%%%%%%%%%%%%%%%%%%%%%%%%%%%%%%%%%%%%%%%%
\section{Perturbation Treatment for The Degenerate Case}
%
% 1  clue that how to tackle down the perturbation treatment
%    for this case
% 2  how to understand the perturbation will remove the degeneracy
%    for the degenerate states
%
In the above content, we have discussed the limitations for the
perturbation treatment applied on the non-degenerate situation. In
this section, we are going to see how to tackle down such
difficulties.

Suggest that we have two states, namely the $\Psi_{m}^{(0)}$ and
$\Psi_{n}^{(0)}$; they are the eigen states for the $\hat{H}_{0}$
and we assume their energy ($E_{m}^{(0)}$ and $E_{n}^{(0)}$
separately) level are very close or just in degenerate states.
According to the discussion in \ref{PTIQM:1}, we can not directly
use the perturbation formula which is attributed to the fact that
the $E_{m}^{(0)} - E_{n}^{(0)}$ will cause infinite number (for the
degenerate states) or abnormal high component in these expressions.
However, if the corresponding integral of
$\bra{\Psi_{m}^{(0)}}\hat{H}^{'}\ket{\Psi_{n}^{(0)}}$ equals to
zero, then for the near degenerate case it looks like that the
perturbation equations can be used again. therefore, the condition
that $\bra{\Psi_{m}^{(0)}}\hat{H}^{'}\ket{\Psi_{n}^{(0)}} = 0$ is
some kind of clue which help us to further investigate how to
facilitate the perturbation treatment for the degenerate case.

Usually, the perturbed operator of $\hat{H}^{'}$ tends to break the
original degenerate states situation for the $\hat{H}_{0}$. In
section \ref{SE:3}, we have proved such fact that if two operator
$\hat{F}$ and $\hat{G}$ all commute with Hamiltonian, that is
$[\hat{F}, \hat{H}] = [\hat{G}, \hat{H}] = 0$; however, we have
$[\hat{F}, \hat{G}] \neq 0$ then there must have energy degeneracy
phenomenon for the specific system. For example, we know that for
the angular momentum of $\hei{l}$ we have $[\hei{l}, \hat{H}] = 0$
However, for the $\hat{l}_{x}$, $\hat{l}_{y}$ and $\hat{l}_{z}$ they
can not commute with each other: $[\hat{l}_{i}, \hat{l}_{j}] \neq 0$
for $i, j \in x, y, z$. therefore, we can expected that in the
system, where it possesses the isotropy quality (all the single atom
system possesses the isotropy quality); then there must have energy
degeneracy phenomenon. The most simplest case is the hydrogen atom,
where its three eigen states corresponding to $l=1$ (namely the
three states that $m= \pm 1, 0$) are degenerate.

However, if the perturbed operator of $\hat{H}^{'}$ is added into
the whole system, such degenerate situation usually will break down.
This is because the perturbed operator of $\hat{H}^{'}$ usually has
lower symmetry character compared with the $\hat{H}_{0}$ (see the
reference book \cite{ZengJinYan} in Vol. two. PP $427$). Therefore,
we will concentrate on the situation that the adding of
$\hat{H}^{'}$ will partially or wholly remove the degeneracy for the
original states.

For the degenerate states, they form some subspace for a given
energy level so that there's some uncertainty in the expression of
wave functions. For example, if $\Psi_{m}^{(0)}$ and
$\Psi_{n}^{(0)}$ are two eigen states for the $\hat{H}$ and give the
same energy of $E^{(0)}_{k}$, and we assume that they are forming
some subspace; then the vector building from this subspace:
\begin{equation}\label{}
\Psi_{k}^{(0)} = c_{1}\Psi_{m}^{(0)} + c_{2}\Psi_{n}^{(0)}
\end{equation}
also corresponding to the eigen state for the some arbitrary $c_{1}$
and $c_{2}$. However, the perturbation of $\hat{H}^{'}$ will rule
out such uncertainty in the subspace.

In the perturbation treatment, we use some arbitrary real number of
$\lambda$ to signal the perturbation states for the wave function (
see the expression of \ref{PTIQMeq:5}) as well as the energy. Here
we require that they should be the continuous function for the
$\lambda$ so that as $\lambda\rightarrow 0$; then the wave function
approach to some definite linear combination between the eigen
states in the subspace. Later in the content, we can see that this
linear combination is just diagonalizing the perturbed operator
$\hat{H}^{'}$.

%%%%%%%%%%%%%%%%%%%%%%%%%%%%%%%%%%%%%%%%%%%%%%%%%%%%%%%%%%%%%%%%%%%%%%%%%%%%%%%%%%%%%%%%%%%%%%
\subsection{Mathematical Expression}
%
%  1  mathematical expression
%
%
Suppose that for the $\hat{H}_{0}$ the system is in some degenerate
state of $E_{k}^{(0)}$:
\begin{equation}\label{}
E = E_{k}^{(0)}
\end{equation}
Assume that there are $f_{k}$ eigen states giving the energy of
$E_{k}^{(0)}$, then the wave function is some linear combination of
them:
\begin{equation}\label{PTIQMeq:22}
\Phi^{(0)}_{k} = \sum_{\mu}^{f_{k}}a_{\mu}\Psi_{k\mu}^{(0)}
\end{equation}
Here we assume that such $f_{k}$ eigen states have been
orthogonalized so that we have:
\begin{equation}\label{}
\langle\Psi_{k\mu}^{(0)}|\Psi_{k\mu^{'}}^{(0)}\rangle =
\delta_{\mu\mu^{'}}
\end{equation}
Let's take the (\ref{PTIQMeq:22}) into the first order perturbation
treatment in (\ref{PTIQMeq:7}):
\begin{equation}\label{}
(\hat{H}_{0} - E^{0}_{k})\Phi^{1}_{k} = \sum_{\mu}^{f_{k}}(E^{1}_{k}
- \hat{H}^{'})a_{\mu}\Psi_{k\mu}^{(0)}
\end{equation}
If we multiply the above equation with
$\bra{\Psi_{k\mu^{'}}^{(0)}}$, then we can get:
\begin{equation}\label{PTIQMeq:23}
\sum_{\mu}^{f_{k}}(\hat{H}^{'}_{\mu\mu^{'}} -
E^{1}_{k}\delta_{\mu\mu^{'}})a_{\mu} = 0
\end{equation}
Here in the derivation we have to remember that:
\begin{equation}\label{}
\bra{\Psi_{k\mu^{'}}^{(0)}}(\hat{H}_{0} -
E^{0}_{k})\ket{\Phi^{1}_{k}} = (E^{0}_{k} -
E^{0}_{k})\langle\Psi_{k\mu^{'}}^{(0)}|\Phi^{1}_{k}\rangle = 0
\end{equation}

The equation in the (\ref{PTIQMeq:23}) is some secular equation,
where the non-zero solution demands that:
\begin{equation}\label{PTIQMeq:27}
det|\hat{H}^{'}_{\mu\mu^{'}} - E^{1}_{k}\delta_{\mu\mu^{'}}| = 0
\end{equation}
this $n$ fold determinant will produce $f_{k}$ eigenvalue, which can
be labeled as $E^{1}_{k1}, E^{1}_{k2}, \cdots, E^{1}_{kf_{k}}$. For
each certain $E^{1}_{ki}$, there corresponds some eigen vector, it's
the definite linear combination of the zero order wave function.
Hence, we can write the new zero order wave function as:
\begin{equation}\label{PTIQMeq:24}
\Phi^{(0)}_{\alpha k} =
\sum_{\mu}^{f_{k}}a_{\alpha\mu}\Psi_{k\mu}^{(0)} \quad \alpha =
1,2,\cdots, f_{k}
\end{equation}
It's a bit of complicated form in (\ref{PTIQMeq:24}). We note that
the label of $\alpha$ represents the dimension for the new zero
order wave function in the given subspace, the $k$ indicates that it
corresponds to the energy level of $E_{k}^{(0)}$; and the $\mu$
loops over all the old wave function gotten before the perturbation.
Here it's worthy to note that since the subspace is some
``complete'' space - which means that all the eigen states give the
energy of $E_{k}^{(0)}$ are all produced from the linear combination
of such $f_{k}$ original eigen states. Hence the treatment has
nothing to do with the eigen vectors outside the subspace.

If the new root of $E^{1}_{k1}, E^{1}_{k2}, \cdots, E^{1}_{kf_{k}}$
are all different with each other, then it can say that the
perturbation of $\hat{H}^{'}$ has removed the energy degeneracy.
However, it may have possibility that the $E^{1}_{ki}$ are still
degenerate, or partially degenerate. If this is the case, the higher
correlation for the degenerate states is required because the wave
function is still not fixed.

Finally, there's some question that how to do this perturbation
treatment in reality? In practice, This perturbation treatment for
the degenerate states will be executed in the first step - in an
attempt to relieve the degeneracy situation for all the eigen
states. If there's still some degenerate states after the above
procedure then some higher correlation is expected to bring up so as
to fully remove the degenerate situation for the specific subspace.
Then on the base of the new eigen states, the perturbation treatment
for the non-degenerate case is carried on to generate the final
result.

%%%%%%%%%%%%%%%%%%%%%%%%%%%%%%%%%%%%%%%%%%%%%%%%%%%%%%%%%%%%%%%%%%%%%%%%%%%%%%%%%%%%%%%%%%%%%%
\subsection{Characters for The Zero Order Wave Functions}
%
% 1 different eigen vectors are orthogonalized
% 2 the \hat{H}^{'} is diagonalized
%
Now let's consider the characters for the new sets of the zero order
wave function of $\Phi^{(0)}_{\alpha k}$, $\alpha = 1,2,\cdots,
f_{k}$. In the following content we assume that the degeneracy has
been fully relieved.

Firstly, we can prove that $\langle\Phi^{(0)}_{\alpha
k}|\Phi^{(0)}_{\beta k}\rangle = \delta_{\alpha\beta}$.
\begin{align}\label{}
\langle\Phi^{(0)}_{\alpha k}|\Phi^{(0)}_{\beta k}\rangle &=
\sum_{\mu}\sum_{\mu^{'}}a_{\alpha\mu}^{*}a_{\beta\mu^{'}}
\int\Psi_{k\mu^{'}}^{(0)}\Psi_{k\mu}^{(0)}d\tau \nonumber \\
&=\sum_{\mu}\sum_{\mu^{'}}a_{\alpha\mu}^{*}a_{\beta\mu^{'}}
\delta_{\mu\mu^{'}} \nonumber \\
&=\sum_{\mu}a_{\alpha\mu}^{*}a_{\beta\mu}
\end{align}

However, for different eigen value; their eigen vectors are
orthogonal with each other (see the mathematical book related to the
linear algebra, section of eigen value and eigen vectors). Now we
can easily prove this point by starting from the equation
(\ref{PTIQMeq:23}).
\begin{align}\label{PTIQMeq:25}
\sum_{\mu}^{f_{k}}(\hat{H}^{'}_{\mu\mu^{'}} -
E^{(1)}_{k\alpha}\delta_{\mu\mu^{'}})a_{\alpha\mu}^{*} &= 0 \nonumber \\
\sum_{\mu}^{f_{k}}\sum_{\mu^{'}}^{f_{k}}(\hat{H}^{'}_{\mu\mu^{'}} -
E^{(1)}_{k\alpha}\delta_{\mu\mu^{'}})a_{\alpha\mu}^{*}
a_{\beta\mu^{'}} &=
0 \Rightarrow \nonumber \\
E^{(1)}_{k\alpha}\sum_{\mu}a_{\alpha\mu}^{*}a_{\beta\mu} &=
\sum_{\mu}^{f_{k}}\sum_{\mu^{'}}^{f_{k}}a_{\alpha\mu}^{*}
a_{\beta\mu^{'}}\hat{H}^{'}_{\mu\mu^{'}}
\end{align}
We note that in the (\ref{PTIQMeq:25}) we have multiply the
$a_{\beta\mu^{'}}$ and loop over all the $\mu^{'}$ label.

Similarly, if we exchange the label $\alpha\leftrightarrow\beta$,
$\mu\leftrightarrow\mu^{'}$, we can get:
\begin{align}\label{}
\sum_{\mu}^{f_{k}}\sum_{\mu^{'}}^{f_{k}}a_{\beta\mu}^{*}
a_{\alpha\mu^{'}}\hat{H}^{'}_{\mu\mu^{'}} &=
\Bigg(\sum_{\mu}^{f_{k}}\sum_{\mu^{'}}^{f_{k}}a_{\alpha\mu}^{*}
a_{\beta\mu^{'}}\hat{H}^{'}_{\mu\mu^{'}}\Bigg)^{*} \nonumber \\
&=\sum_{\mu}^{f_{k}}\sum_{\mu^{'}}^{f_{k}}a_{\alpha\mu}^{*}
a_{\beta\mu^{'}}\hat{H}^{'}_{\mu\mu^{'}}
\end{align}
We can get this because the coefficients should be real, and the
perturbed operator of $\hat{H}^{'}$ is hermite. That's some very
natural restrictions.

Therefore for the (\ref{PTIQMeq:25}) it can be transformed into:
\begin{equation}\label{PTIQMeq:26}
E^{(1)}_{k\beta}\sum_{\mu}a_{\beta\mu}^{*}a_{\alpha\mu} =
\sum_{\mu}^{f_{k}}\sum_{\mu^{'}}^{f_{k}}a_{\alpha\mu}^{*}
a_{\beta\mu^{'}}\hat{H}^{'}_{\mu\mu^{'}}
\end{equation}

If we make the (\ref{PTIQMeq:26}) subtract the (\ref{PTIQMeq:25}),
it gives:
\begin{align}\label{}
(E^{(1)}_{k\beta} -
E^{(1)}_{k\alpha})\sum_{\mu}a_{\alpha\mu}^{*}a_{\beta\mu} &= 0
\Rightarrow\nonumber \\
\sum_{\mu}a_{\alpha\mu}^{*}a_{\beta\mu} &= 0
\end{align}
Here we have assumed that the degeneracy has been removed by the
$\hat{H}^{'}$. Therefore we finally have:
\begin{equation}\label{}
\sum_{\mu}a_{\alpha\mu}^{*}a_{\beta\mu} = \delta_{\alpha\beta}
\end{equation}

Based on this fact, we can now finally prove that the new sets of
wave functions diagonalize the $\hat{H}^{'}$ matrix.
\begin{align}\label{}
\bra{\Phi_{\alpha k}^{(0)}}\hat{H}^{'}\ket{\Phi_{\beta k}^{(0)}} &=
\sum_{\mu}\sum_{\mu^{'}}a_{\alpha\mu}^{*}
a_{\beta\mu^{'}}\bra{\Psi_{k\mu}^{(0)}}\hat{H}^{'}
\ket{\Phi_{k\mu^{'}}^{(0)}} \nonumber \\
&= \sum_{\mu}\sum_{\mu^{'}}a_{\alpha\mu}^{*}
a_{\beta\mu^{'}}\hat{H}^{'}_{\mu\mu^{'}} \nonumber \\
&=E^{(1)}_{k\alpha}\sum_{\mu}a_{\alpha\mu}^{*}a_{\beta\mu} \nonumber \\
&=E^{(1)}_{k\alpha}\delta_{\alpha\beta}
\end{align}
This fact coincides with our suspicion made in the head of this
section: the new wave functions after the perturbation should be
made the $\hat{H}^{'}$ matrix diagonalized. In other words, the
solution we are seeking for is the eigen states in the degenerate
subspace which makes the $\hat{H}^{'}$ matrix diagonalized.

%%%%%%%%%%%%%%%%%%%%%%%%%%%%%%%%%%%%%%%%%%%%%%%%%%%%%%%%%%%%%%%%%%%%%%%%%%%%%%%%%%%%%%%%%%%%%%
\subsection{A Simple Example: Two Near Degenerate States}
%
%
%
Suggest that we have $\hat{H} = \hat{H}_{0} + \hat{H}^{'}$. For the
$\hat{H}_{0}$ there are two eigen states of $\Psi^{(0)}_{1}$ with
energy $E^{(0)}_{1}$ and $\Psi^{(0)}_{2}$ with energy $E^{(0)}_{2}$.
Both of the two energy level are in near degeneracy. Then by the
formula of (\ref{PTIQMeq:27}), we can get:
\begin{equation}\label{}
\begin{bmatrix}
  H_{11}^{'} - E & H_{12}^{'} \\
  H_{12}^{'} & H_{22}^{'} - E \\
\end{bmatrix} = 0
\end{equation}
This gives two solutions:
\begin{equation}\label{}
E^{(1)}_{\pm} = \frac{(H_{11}^{'} + H_{22}^{'}) \pm
\sqrt{(H_{11}^{'} - H_{22}^{'})^{2} + 4(H_{12}^{'})^{2}}}{2}
\end{equation}

Therefore, we can have the energy correlated to the first order as:
\begin{align}\label{}
E^{(1)}_{1} &= E^{(0)}_{1} + E^{(1)}_{-} \nonumber \\
E^{(1)}_{2} &= E^{(0)}_{2} + E^{(1)}_{+}
\end{align}
Hence the two energy levels will be separate from each other as long
as the perturbation is strong enough. For the new sets of the wave
functions of $\Psi^{'(0)}_{1}$ and $\Psi^{'(0)}_{2}$, they can be
expressed as the linear combination of the $\Psi^{(0)}_{1}$ and
$\Psi^{(0)}_{2}$. Through the $E^{(1)}_{\pm}$, by solving the eigen
equation we can get the coefficients, thus the new sets wave
functions can be got.


%%%%%%%%%%%%%%%%%%%%%%%%%%%%%%%%%%%%%%%%%%%%%%%%%%%%%%%%%%%%%%%%%%%%%%%%%%%%%%%%%%%%%%%%%%%%%%


%%% Local Variables: 
%%% mode: latex
%%% TeX-master: "../../main"
%%% End: 
