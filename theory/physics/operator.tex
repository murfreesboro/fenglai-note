%
% revised at Jan. 30th; 2009
% we have added two parts: the unitary operator and the
% project operator
%
% revised on Mar. 22th, 2009
% to use the environment of theorem and law to improve it
%
% revised at July 27th, 2009
% finish the part of project operator and the unitary operator
%
% revised at Aug. 10th, 2009
% nearly rewrite all the part in the operator.tex
%
%
%%%%%%%%%%%%%%%%%%%%%%%%%%%%%%%%%%%%%%%%%%%%%
%  general discussion for the operator:
%  definition and Algorithms
%  linear operator
%  hermite operator
%
%%%%%%%%%%%%%%%%%%%%%%%%%%%%%%%%%%%%%%%%%%%%%%%%%%%%%%%%%%%%%%%%%%%%%%

\chapter{General discussion to operator}
%
%
%
%
In the above content, we have discussed the Hilbert space. However,
how to define the Hilbert space which is correspond to a given
physical system? How to extract physical information from the
Hilbert space? In this chapter, we will give some general discussion
over this topic.


%%%%%%%%%%%%%%%%%%%%%%%%%%%%%%%%%%%%%%%%%%%%%%%%%%%%%%%%%%%%%%%%%%%%%%
\section{Definition of operators}
%
% 1   definition
% 2   linear operators,required by the superposition principle
%
%
%
Generally to say, mathematically the operator can be seen as the
correspondence between two different vectors in a Hilbert
space\footnote{In the following content, the Greek letters are used
to indicate the vectors in the Hilbert space.}:
\begin{equation}\label{OPERATOReq:4}
\hat{O}\ket{\Psi} = \ket{\Phi}
\end{equation}
If there's no specific instruction, the operator in the following
content will all done on the vectors from ket space.

Mathematically to say, operator can be seen as some kind of ``function
transformation machine'', it transforms the function from one form to
another. For example, the $\frac{d^{2}}{dx^{2}}$ is a kind of
operator. For the function of $e^{ikx}$, we have:
\begin{equation}\label{}
\frac{d^{2}}{dx^{2}} (e^{ikx}) = -k^{2}e^{ikx}
\end{equation}
So it converts the $e^{ikx}$ to $-k^{2}e^{ikx}$.

In quantum mechanics, all the physical quantity are corresponding to
some form of operators. For example, the kinetic operator is expressed
as $-\frac{\hbar^{2}}{2m}\nabla^{2}$, the momentum operator is
expressed as $-i\hbar\nabla$.

However, it's not all the operators that are physically meaningful
in quantum mechanics. Most of the operators used in quantum
mechanics, are the linear operators; which satisfy the conditions
below:
\begin{eqnarray}
% \nonumber to remove numbering (before each equation)
  \hat{A}(c_{1}\ket{\Psi} + c_{2}\ket{\Phi}) &=&
  c_{1}\hat{A}\ket{\Psi} + c_{2}\hat{A}\ket{\Phi} \nonumber \\
  \hat{A}(\ket{\Psi}a) &=& (\hat{A}\ket{\Psi})a
\end{eqnarray}
Here it can be seen that the linear operator actually is required by
the superposition principle. Since that an arbitrary wave function
can be composed by some linear combination, thus the action of an
operator on some arbitrary wave function should also be decomposed
into the actions on its linear components.

%%%%%%%%%%%%%%%%%%%%%%%%%%%%%%%%%%%%%%%%%%%%%%%%%%%%%%%%%%%%%%%%%%%%%%%
\section{Axioms for quantum mechanics in terms of operator}
\label{AFQMITOO_in_operator}
%  1  physical quantity, and should be hermitian
%  2  observable
%  3  operator to an arbitrary wave function; how to express it
%
%
Now let's state the axioms related to operator in quantum
mechanics. The first one is the physical meaning of operator:
\begin{axiom}\label{axiom2}
\textbf{In quantum mechanics, all the physical quantities are
  corresponding to hermitian operator working on the Hilbert space.}
\end{axiom}
As what we have demonstrated, here such operators corresponding to the
physical quantity should be also the linear operator. In the following
content, we will discuss what's the hermitian character of the
operator.

Furthermore, the second axiom is related to the ``observable''
physical quantity in the quantum mechanics:
\begin{axiom}\label{axiom3}
  \textbf{ In quantum mechanics, the observable physical quantity
has the eigen functions below to determine the observed value:
\begin{equation}\label{OPERATOReq:8}
  \hat{A}\ket{\Psi_{i}} = a_{i}\ket{\Psi_{i}} \quad i=1, 2, \cdots
\end{equation}
Here the $\ket{\Psi_{i}}$ is called the eigen states for the
observable physical quantity of $\hat{A}$, and the real number of
$a_{i}$ is just corresponding to the observed value (on the other
words, the measured result of the physical quantity) for the physical
quantity.}
\end{axiom}

Now there comes very important question, that is what's the
``observable'' physical quantity in the quantum mechanics?

Here the meaning of ``observable'' is that such physical quantity can
be measured and given certain measured result. As we know, in classic
mechanics, all the physical quantities are ``observable'', the
position, the momentum, the energy etc. are all measurable and can be
given some certain observed value.

However, things in quantum mechanics is totally different. For
example, as what has been demonstrated in the chapter
\ref{basic_chapter}, the position of the quantum particles can not be
measured to give some definite value, only the probability of the
position can be obtained through the $|\Psi(\bm{r},
t)|^{2}d^{3}\bm{r}$ of wave function of $\Psi(\bm{r}, t)$; so there's
some great difference between the physical operators: some can be
observable, the other are not.

The observable physical quantity has very important meaning in quantum
mechanics, that through the (\ref{OPERATOReq:8}) we can derive the eigen
states for the physical quantity of $\hat{A}$. From the contents
below, we can see that such group of eigen states are forming some
complete sets, that means, we are able to get the Hilbert space from
solving the eigen function. Hence, we have answered the question
prompted at the beginning: \textbf{the observable physical quantities
  are used to define the Hilbert space corresponding the system.}

In quantum mechanics, the Schrodinger equation is holding true for all
system; then if the Hamiltonian does not contain the time, we can
split the time part from the Schrodinger equation (see the following
chapter discussing the Schrodinger equation), hence we have the eigen
function for the Hamiltonian operator; that is to say, the energy is
some observable value:
\begin{equation}
\label{OPERATOReq:20}
  \hat{H}\ket{\Psi_{i}} = E_{i} \ket{\Psi_{i}}
\end{equation}
$E_{i}$ is the energy of the system. In quantum mechanics, nearly all
the circumstances are belonging to this type of situation. To some
extent, we can say that the equation in (\ref{OPERATOReq:20}) is used
to define the Hilbert space for the given quantum system.

In such circumstance, we can prove that if the physical quantity is
commuted with the Hamiltonian operator, then such physical quantity
may have eigen function so that to give observable value. The further
information will be detailed analyzed in the following content.

On the other hand, for the physical quantity which is not
observable, we can define the expectation value over the complete
sets of $\Psi_{i}$ (in other words, the average measurable value)
for the physical quantity:
\begin{equation}\label{OPERATOReq:7}
\bra{\Psi_{i}}\hat{A}\ket{\Psi_{i}} = \langle\hat{A}\rangle
\end{equation}

For example, the mean coordinate for the position operator of
$\hat{x}$:
\begin{equation}\label{}
\langle\hat{x}\rangle = \bra{\Psi_{i}}\hat{x}\ket{\Psi_{i}}
\end{equation}
This value depicts the average position for the $\ket{\Psi_{i}}$ over
the x axis.

Now there's still a question left that for some arbitrary quantum
state of $\ket{\Phi}$, how to evaluate the physical information for
this state? This is the second question prompted in the beginning, and
we have the third axiom below to answer the question:
\begin{axiom}\label{axiom4}
  \textbf{When some physical quantity of $\hat{A}$ is measured on an
    arbitrary quantum state of $\ket{\Psi}$, if the $\ket{\Psi}$ is
    not the eigen state for $\hat{A}$ then the measurable result can
    not achieve definite value. Instead, there are a range of
    measurable results can be potentially achieved, the probability
    for each of measurable result is determined as:
    \begin{equation}
\label{OPERATOReq:21}
      \text{probability of measurable result} = \langle \Psi|\Psi_{i} \rangle^{2}
    \end{equation}
    The $\ket{\Psi_{i}}$ is some complete sets on which the
    $\ket{\Psi}$ is expanding over:
    \begin{equation}
      \ket{\Psi} = \sum_{i}c_{i}\ket{\Psi_{i}}
    \end{equation}
    Hence, the measurable result for the $\hat{A}$ is just the
    collection of measurements on each $\ket{\Psi_{i}}$, each of them
    has some probability to appear in the final result; and the
    probability is given by (\ref{OPERATOReq:21}). Since $c_{i} =
    \langle \Psi|\Psi_{i} \rangle$, then we can also express the above
    result as:
    \begin{equation}
      \text{probability of measurable result} = |c_{i}|^{2}
    \end{equation} }
\end{axiom}

Usually in quantum mechanics, the physical quantity of $\hat{A}$ is
chosen to be observable kind (obviously it meaningless to pursue the
measurable result for some unobservable physical quantity over an
arbitrary quantum state! ), therefore, we can have that:
\begin{eqnarray}\label{OPERATOReq:2}
% \nonumber to remove numbering (before each equation)
  \hat{A}\ket{\Psi}
  &=& \sum_{i}c_{i}\hat{A}\Psi_{i} \nonumber \\
  &=& \sum_{i}c_{i}a_{i}\Psi_{i}  \Rightarrow \nonumber \\
  \bra{\Psi}\hat{A}\ket{\Psi}
  &=&
  \sum_{i}\sum_{j}c^{*}_{j}c_{i}a_{i}\langle\Psi_{j}|\Psi_{i}\rangle
  \nonumber \\
  &=& \sum_{i}|c_{i}|^{2}a_{i}
\end{eqnarray}
This is the average measurable result for the $\hat{A}$ over the
arbitrary state of $\ket{\Psi}$.

Now we have established all the axioms related to the operators, and
answered the questions that how to define the Hilbert space for a
system, and how to extract physical information from given arbitrary
quantum state. In the following content, we are going to present a
detailed analysis in term of the items above.



%%%%%%%%%%%%%%%%%%%%%%%%%%%%%%%%%%%%%%%%%%%%%%%%%%%%%%%%%%%%%%%%%%%%%%%%%%%%%
\section{Algorithm of operators}
% 1  some special operators, unit operator etc.
% 2  addition, substraction, multiplication, division
% 3  commutation relationship
% 4  transposing and conjugating operators
% 5  hermite operation, related to the transposing and
%    conjugating operations
%
Now we concentrate on the algorithms of the linear operators. It can
see that the algorithms are similar to the rules in matrix.

Unit operator:
\begin{equation}\label{}
\hat{I}\ket{\Psi} = \ket{\Psi}
\end{equation}

Zero operator:
\begin{equation}\label{}
\hat{O}\ket{\Psi} = 0
\end{equation}

Two identical operators which leads to $\hat{A}=\hat{B}$:
\begin{equation}\label{}
  \hat{A}\ket{\Psi} = \hat{B} \ket{\Psi} \qquad \text{for any $\ket{\Psi}$}
\end{equation}

Addition and substraction of operators:
\begin{equation}\label{}
(\hat{A} \pm \hat{B})\ket{\Psi} = \hat{A}\ket{\Psi} \pm \hat{B} \ket{\Psi}
\end{equation}

Such algorithms introduced above are simple and straightforward, yet
the multiplication for the operators is some kind of complex: if
$\hat{A}$ and $\hat{B}$ are two operators, their multiplication is
determined by their sequence:
\begin{equation}\label{}
\hat{A}\hat{B}\ket{\Psi} \quad \text{is not same to} \quad
\hat{B}\hat{A}\ket{\Psi} \nonumber
\end{equation}
This is similar to the matrix multiplication. Two operators are
commuted if $\hat{A}\hat{B} = \hat{B}\hat{A}$, but most of time they
are not commuted with each other. Commutation is some significant
relation, therefore in the following content, we use $[\hat{A},
\hat{B}]$ to represent $\hat{A}\hat{B} - \hat{B}\hat{A}$; so if
$[\hat{A}, \hat{B}] = 0$, the two operators are commuting with each
other.

The commutation relationship of the operators is vital in quantum
mechanics, so this is going to leave for a detailed discussion in
the following content.

Like the matrix, where we have inverse matrix; in quantum mechanics,
we also have the inverse operation. For any $\ket{\Psi}$ and
$\ket{\Phi}$ , if we have:
\begin{eqnarray}
% \nonumber to remove numbering (before each equation)
  \hat{A}\ket{\Psi} &=& \ket{\Phi}  \nonumber \\
  \hat{B}\ket{\Phi} &=& \ket{\Psi}
\end{eqnarray}
We say that $\hat{B}$ is the inverse operator for the $\hat{A}$,
which is labeled as $\hat{A}^{-1}$. It's easy to see that
$\hat{A}\hat{A}^{-1} = I$. However, not every operator has its own
inverse operator.

Besides the algorithms above, the operator also has conjugating
operations and transposing operations. The conjugating operation is
defined as: to change the operator of $\hat{O}$ into its conjugated
form. For example, the conjugating operator for the operator of
$-i\hbar \frac{d}{dx}$ is $i\hbar \frac{d}{dx}$.

Transposing operation is a bit more complex, for $\langle\psi|
\hat{O}|\phi\rangle$ it's transposing defined as:
\begin{equation}\label{OPERATOReq:5}
\langle\psi|\widetilde{\hat{O}}|\phi\rangle = \langle\phi|
\hat{O}|\psi\rangle
\end{equation}
In transposing operation, it's clear to see in the integrals the bra
and ket exchanged their position.

It's convenient to take $-i\hbar\frac{d}{dx}$ operator as an example
to show how the transposing operation works. Since we have:
\begin{align}\label{OPERATOReq:3}
\int^{+\infty}_{-\infty}\phi^{*}\frac{d}{dx}\psi dx &=
\phi^{*}\psi|^{+\infty}_{-\infty} -
\int^{+\infty}_{-\infty}\psi^{*}\frac{d}{dx}\phi dx \nonumber \\
&= - \int^{+\infty}_{-\infty}\psi^{*}\frac{d}{dx}\phi dx
\end{align}
We have $\widetilde{\hat{p_{x}}} = -\hat{p_{x}}$.

Here the transposing operation seems to be a bit of obscure. Actually
we can see that the transposing operation and the conjugating
operation are closely related to the adjacent operation, which is
going to be demonstrated right now.  first let's introduce the
operator on the \brat{\Psi}.

Since bra space is conjugated to the ket space, we can define the
operator working on the bra space. Following the same condition
defined in (\ref{OPERATOReq:4}), we have:
\begin{equation}\label{}
\bra{\Psi}\hat{P} = \bra{\Phi}
\end{equation}
The operator $\hat{P}$ working on the \brat{\Psi} is called
$\hat{A}$'s adjacent operator, labeled as $\hat{A}^{+}$.

It's easy to see that we have:
\begin{equation}\label{}
\hat{A} \ket{\Psi} = a\ket{\Psi}, \quad  \bra{\Psi}\hat{A}^{+} =
a^{*}\bra{\Psi}
\end{equation}

To see the association between the operator of $\hat{A}$ and
$\hat{A}^{+}$, we can strick up an example here. Suppose that
$\ket{\Psi_{1}}$ and $\ket{\Psi_{2}}$ constitute some complete sets
for $\hat{A}$, they are orthogonal with each other; where any other
vectors in this space can be expressed as $\ket{\Psi} =
\lambda_{1}\ket{\Psi_{1}} + \lambda_{2}\ket{\Psi_{2}}$, the
$\lambda$ is some complex number. Therefore we have:
\begin{align}\label{}
(\bra{\Psi_{1}} + \bra{\Psi_{2}})\hat{A}(\lambda_{1}\ket{\Psi_{1}} +
\lambda_{2}\ket{\Psi_{2}}) &=
\lambda_{1}\bra{\Psi_{1}}\hat{A}\ket{\Psi_{1}} +
\lambda_{2}\bra{\Psi_{2}}\hat{A}\ket{\Psi_{2}}  \nonumber \\
&= \lambda_{1}a_{1} + \lambda_{2}a_{2}
\end{align}

For the $\ket{\Psi} = \lambda_{1}\ket{\Psi_{1}} +
\lambda_{2}\ket{\Psi_{2}}$, it's conjugated vector is: $\bra{\Psi} =
\lambda_{1}^{*}\bra{\Psi_{1}} + \lambda_{2}^{*}\bra{\Psi_{2}}$, thus
for the $\hat{A}^{+}$; we have:
\begin{align}\label{}
(\lambda_{1}^{*}\bra{\Psi_{1}} +
\lambda_{2}^{*}\bra{\Psi_{2}})\hat{A}^{+}(\ket{\Psi_{1}} +
\ket{\Psi_{2}}) &=
\lambda_{1}^{*}\bra{\Psi_{1}}\hat{A}^{+}\ket{\Psi_{1}} +
\lambda_{2}^{*}\bra{\Psi_{2}}\hat{A}^{+}\ket{\Psi_{2}}  \nonumber \\
&= \lambda_{1}^{*}a_{1}^{*} + \lambda_{2}^{*}a_{2}^{*}
\end{align}

From this example, we can see that the operation of $\hat{A}
\rightarrow \hat{A}^{+}$ is equivalent to the transposing operation
plus the conjugating operation. This specific operation is called
adjacent operation.

The adjacent operation can be finally defined as:
\begin{equation}\label{}
\bra{\Psi}\hat{A}^{+}\ket{\Phi} = \bra{\Phi}\hat{A}\ket{\Psi}^{*}
\end{equation}

%%%%%%%%%%%%%%%%%%%%%%%%%%%%%%%%%%%%%%%%%%%%%%%%%%%%%%%%%%%%%%%%%%%%%%%%%%%%%%%
\section{Hermite operators}
%
% why the hermite operator is important?
% to prove the hermite operator
%
%
In the section related to the axioms of operators, we have shown that
the operators which correspond to physical quantity should be
hermitian, so in this part let's see why it should be.

The hermite operator is defined as: $\hat{O} = \hat{O}^{+}$; that
is, such operator equals to its adjacent operator. What's the
important meaning behind hermite operators? That is, only the
hermite operators can produce the real eigen value, which is
demanded by all the physical dynamic values. Since the result of
measurement should be physically meaningful, so it must be a real
number.

Thus we prove the judgement below:
\begin{theorem}
if and only if the operator of $\hat{O}$ is hermite, the
$\langle\Psi|\hat{O}|\Psi\rangle$ equals to a real number.
\end{theorem}

\begin{proof}
If $\hat{O} = \hat{O}^{+}$, then for an arbitrary \kett{\Psi}, we
have:
\begin{equation}\label{}
\bra{\Psi}\hat{O}\ket{\Psi} = \langle\Psi|\hat{O}^{+}\ket{\Psi}^{*}
= \langle\Psi|\hat{O}\ket{\Psi}^{*}
\end{equation}
Thus the $\bra{\Psi}\hat{O}\ket{\Psi}$ is a real number.

On the other hand, if for any \kett{\Psi}, the
$\bra{\Psi}\hat{O}\ket{\Psi}$ is a real number (here the operator
can be both type, on bra or ket); we have:
\begin{equation}\label{}
\bra{\Psi}\hat{O}\ket{\Psi} =\bra{\Psi}\hat{O}\ket{\Psi}^{*} =
\langle\Psi|\hat{O}^{+}\ket{\Psi}
\end{equation}
Thus for any \kett{\Psi}, $\bra{\Psi}(\hat{O}-\hat{O}^{+})\ket{\Psi}
= 0$. Then $\hat{O} = \hat{O}^{+}$. \qedhere
\end{proof}

Finally the hermite operator can be written as:
\begin{equation}\label{OPERATOReq:6}
\bra{\psi}\hat{O}\ket{\phi} = \bra{\psi}\hat{O}^{+}\ket{\phi} =
\bra{\phi}\hat{O}\ket{\psi}^{*}
\end{equation}

Formally this theorem implies one thing: that the hermite operator
can both working the bra space and the ket space (they give the same
result); thus for the hermite operator we need not to distinguish
the hermite operator and its adjacent operator:
\begin{equation}\label{}
\langle\Psi|\hat{O}\Phi\rangle = \langle\Psi\hat{O}|\Phi\rangle =
\langle\hat{O}\Psi|\Phi\rangle
\end{equation}

%%%%%%%%%%%%%%%%%%%%%%%%%%%%%%%%%%%%%%%%%%%%%%%%%%%%%%%%%%%%%%%%%%%%%%%%%%%%%%
\subsection{What kind of operator is hermitian?}
\label{hermitian_in_operator}
%
% What kind of operator is hermitian?
%
%
Now we come to some question that what kind of operator is
hermitian? The hermitian operator is that $\hat{A} = \hat{A}^{+}$,
potentially; to judge that whether a given operator is hermitian
needs its concrete expression. However, just as what we are going to
demonstrate, there are some general rules to determine whether a
operator is hermitian or not.

Firstly, let's prove some character related to the adjacent
operation:
\begin{theorem}
For any two arbitrary operators, it satisfy that:
\begin{equation}\label{OPERATOReq:28}
(\hat{A}\hat{B})^{+} = \hat{B}^{+}\hat{A}^{+}
\end{equation}
\end{theorem}

\begin{proof}
Suggest that $\hat{B}\ket{\Psi} = \ket{\Phi}$, $\hat{A}\ket{\Phi} =
\ket{\Omega}$. So we have:
\begin{equation}\label{}
\hat{A}\hat{B}\ket{\Psi} =\ket{\Omega}
\end{equation}
On the other hand, we have:
\begin{equation}\label{}
\bra{\Psi}\hat{B}^{+}\hat{A}^{+} = \bra{\Phi}\hat{A}^{+} =
\bra{\Omega} =\bra{\Psi}(\hat{A}\hat{B})^{+}
\end{equation}
Furthermore, it's obvious that for $\hat{A}^{+}\hat{B}^{+}$ on
$\bra{\Psi}$ we can not get the $\bra{\Omega}$. Hence according to
the one to one correspondence between the bra and ket, the
(\ref{OPERATOReq:28}) is true.
 \qedhere
\end{proof}

For the adjacent operation, we have some other theorems which are
easily proved (so the proof is omitted here):
\begin{theorem}
For any arbitrary operators, it satisfy that:
\begin{equation}\label{OPERATOReq:29}
\begin{split}
  (\hat{A}^{+})^{+} &=  \hat{A} \\
   (\lambda\hat{A})^{+} &=  \lambda^{*}\hat{A}^{+} \\
(\hat{A} + \hat{B})^{+} &= \hat{A}^{+} + \hat{B}^{+}
\end{split}
\end{equation}
\end{theorem}

then let's go to prove some general characters related to the
hermitian operator. Suggest that $\hat{A}$ and $\hat{B}$ are two
hermitian operators, now we can prove that the operators below are
also hermitian.
\begin{theorem}
\begin{align}\label{OPERATOReq:30}
\hat{C} &= \hat{A} \pm \hat{B} \nonumber \\
\hat{C} &= c\hat{A} \quad \text{c is a real number} \nonumber \\
\hat{C} &= \hat{A}\hat{B} \quad \text{as long as $[\hat{A}, \hat{B}] = 0$} \nonumber \\
\hat{C} &= c(\hat{A}\hat{B} + \hat{B}\hat{A}) \nonumber \\
\hat{C} &= \frac{c}{i}(\hat{A}\hat{B} - \hat{B}\hat{A})
\end{align}
\end{theorem}

\begin{proof}
The first two expressions for $\hat{C}$ are straightforward. For
$\hat{C} = \hat{A}\hat{B}$, we have that:
\begin{equation}\label{}
\begin{split}
  \hat{C}^{+} &= (\hat{A}\hat{B})^{+}  \\
    &= \hat{B}\hat{A} \\
    &= \hat{A}\hat{B} \quad
    \text{$\hat{A}$ and $\hat{B}$ are commutative} \\
    &= \hat{C}
\end{split}
\end{equation}
The fourth expression for $\hat{C}$ can be also proved in the
similar way. For the fifth expression, we have:
\begin{equation}\label{OPERATOReq:24}
\begin{split}
  \hat{C}^{+} &= \left(\frac{1}{i}\right)^{*}
  [\hat{A}\hat{B} - \hat{B}\hat{A}]^{+}
  \\
    &= i[\hat{B}\hat{A} - \hat{A}\hat{B}] \\
    &=-i[\hat{A},\hat{B}] \\
    &=\frac{1}{i}[\hat{A},\hat{B}] \\
    &=\hat{C}
\end{split}
\end{equation}
 \qedhere
\end{proof}

%%%%%%%%%%%%%%%%%%%%%%%%%%%%%%%%%%%%%%%%%%%%%%%%%%%%%%%%%%%%%%%%%%%%%%%%%%%%%%
\section{Observable operator}
%
% 1  what's the physical meaning of observable operator?
% 2  the theorems related to the observable
%
%
%
In axiom \ref{axiom3}, we have introduced the observable operator;
which has some eigen function to correspond it:
\begin{equation}\label{OPERATOReq:1}
\hat{A}\ket{\Psi_{i}} = a_{i}\ket{\Psi_{i}} \quad i=1, 2, \cdots
\end{equation}
$\ket{\Psi_{i}}$ are $\hat{A}$'s eigen states, and the $a_{i}$ are
the corresponding measurable value for $\hat{A}$.

Firstly, how to physically understand the observable operator? Now
let's start from axiom \ref{axiom4}.

Suggest that $\hat{A}$ is some observable operator, from
(\ref{OPERATOReq:2}) we know that for some arbitrary state of
$\ket{\Psi}$:
\begin{equation}\label{OPERATOReq:9}
\bra{\Psi}\hat{A}\ket{\Psi} = \sum_{i}c_{i}^{2}a_{i} \quad
\sum_{i}c_{i}^{2} = 1
\end{equation}
Here each $a_{i}$ has some possibility to appear in the final
measurable result, hence the measurement for $\hat{A}$ on
$\ket{\Psi}$ is not certain; maybe the result is $a_{1}$ in this
time measurement, but change to be $a_{2}$ in the next.

Hence, the mean value of $\sum_{i}c_{i}^{2}a_{i}$ depicts some
``average'' situation for the measurable result, The real
measurement is fluctuating around this average value. However, if
the $\ket{\Psi}$ is the eigen state for $\hat{A}$, then we have
$\bra{\Psi}\hat{A}\ket{\Psi} = a\langle\Psi|\Psi\rangle =a$; there's
no fluctuation anymore, all the measurements will give only one
certain value, which is the $a$. That's the physical meaning behind
the (\ref{OPERATOReq:9}).

For the observable operator, since the measurable result is
meaningful so that it's demanded to be the hermitian operator first.
secondly, we can have such theorem below for the observable:

\begin{theorem}\label{OPERATOR:3}
For a given observable operator, its eigen states which give
different eigen value are orthogonal with each other.
\end{theorem}

\begin{proof}
  Suggest that the operator is $\hat{A}$, then it has two different
  eigen states; namely $\ket{\Psi_{1}}$ and $\ket{\Psi_{2}}$, they
  give different eigen value of $a_{1}$ and $a_{2}$ ($a_{1} \neq
  a_{2}$).  Therefore we have:
\begin{align}\label{}
\bra{\Psi_{1}}\hat{A}\ket{\Psi_{2}} &=
a_{2}\langle\Psi_{1}|\Psi_{2}\rangle \nonumber \\
\bra{\Psi_{1}}\hat{A}\ket{\Psi_{2}} &=
a_{1}\langle\Psi_{1}|\Psi_{2}\rangle \Rightarrow \nonumber \\
(a_{1}-a_{2})\langle\Psi_{1}|\Psi_{2}\rangle &= 0 \Rightarrow \nonumber \\
\langle\Psi_{1}|\Psi_{2}\rangle &= 0
\end{align}
\qedhere
\end{proof}
As a result, according to the analysis in section
\ref{LIV_in_Hilbert}, the eigen states for the $\hat{A}$ is able to
constitute into some complete sets to represent the Hilbert space.
However, there's some unproved proposition; that whether the eigen
states for a given observable operator $\hat{A}$ is complete to
describe the corresponding Hilbert space?

This question is very difficult to answer, however, in physics such
proposition is considered to hold true for any observable operator
$\hat{A}$, that the whole eigen states for the $\hat{A}$ are really
constitute some complete sets where the corresponding Hilbert space
can be expressed over this group of basis functions\footnote{Here
such detailed analysis can read the book by Xinlin
Ke\cite{XingLinKe}, PP 33-36}.


%%%%%%%%%%%%%%%%%%%%%%%%%%%%%%%%%%%%%%%%%%%%%%%%%%%%%%%%%%%%%%%%%%%%%%%%%%%%%%
\subsection{Subspace for eigen states}
\label{subspace_in_operator}
%
% 1  what's the subspace for eigen states
% 2  physical meaning of subspace
% 3  consider the completeness for the Hilbert space plus the
%    subspace
%
%
So far for the observable $hat{A}$ and its eigen states of
$\ket{\Psi_{i}}$ in (\ref{OPERATOReq:1}), we have only considered
it's non-degenerate cases; where it's eigen values of $a_{i}$ are
different from each other. However, there has some circumstance that
the eigen values are degenerate:
\begin{equation}\label{OPERATOReq:32}
\hat{A}\ket{\Psi_{ij}} = a_{i}\ket{\Psi_{ij}} \quad j=1, 2, \cdots,
m
\end{equation}
It's easy to see that for such $m$ $\ket{\Psi_{ij}}$, their
arbitrary linear combination $\ket{\Psi}$ is also the eigen state
for $\hat{A}$ and gives the eigen value of $a_{i}$:
\begin{equation}\label{OPERATOReq:33}
\begin{split}
  \hat{A}\ket{\Psi} &= \hat{A}(\sum_{j=1}^{m}c_{j}\ket{\Psi_{ij}}) \\
    &= \sum_{j=1}^{m}c_{j}(\hat{A}\ket{\Psi_{ij}}) \\
    &= \sum_{j=1}^{m}c_{j}a_{i}\ket{\Psi_{ij}} \\
    &= a_{i}\sum_{j=1}^{m}c_{j}\ket{\Psi_{ij}} \\
    &= a_{i}\ket{\Psi}
\end{split}
\end{equation}
Hence it turns out that such $m$ eigen states has been configured
into some small Hilbert space where all the other vector who gives
the eigen value of $a_{i}$ should be constructed from its linear
combination:
\begin{equation}\label{}
\ket{\Psi} = \sum_{j=1}^{m}c_{j}\ket{\Psi_{ij}} \quad \text{$c_{j}$
is some real number}
\end{equation}

In quantum mechanics, the concept of subspace is very important.
Here we wish to present an example to demonstrate its importance
that in symmetry theory of quantum mechanics, the subspace for
Hamiltonian operator is considered to carry an irreducible
representation for the symmetry group. Later, we will discuss the
origin for the subspace.

It's only after considering the degeneracy of the eigen states then
the completeness for these eigen states to form the complete sets is
strictly established. According to the theorem (\ref{OPERATOR:3}),
it's easy to see that for any vector in the subspace it's orthogonal
with the eigen states outside this subspace. Furthermore, for such
$m$ eigen states in the subspace, we can always form some orthogonal
basis functions according to section \ref{LIV_in_Hilbert}. Hence, by
counting all the normalized basis functions in subspace, plus the
eigen states that are non-degenerate; we can form the complete sets
to represent the corresponding Hilbert space.

%%%%%%%%%%%%%%%%%%%%%%%%%%%%%%%%%%%%%%%%%%%%%%%%%%%%%%%%%%%%%%%%%%%%%%%%%%%%%%
\subsection{Origin of the degeneracy in eigen states}
\label{origin_degeneracy_in_operator}
%
% the origin for the degeneracy for the eigen states
%
%
Generally to say, the degeneracy of the eigen states is usually due
to the fact that only one physical quantity can not fix the whole
system; in other words, we need other independent physical freedom
to identify the corresponding system.

Now let's give an example. The plane wave function of $\Psi_{p}(x) =
\frac{1}{2\pi\hbar}e^{\frac{ipx}{\hbar}}$ is the eigen state for the
momentum operator $\hat{p}$(it's expressed as $\hat{p} = -i\hbar
\frac{\partial }{\partial x}$, in the following content; we will
prove this fact):
\begin{equation}\label{}
-i\hbar \frac{\partial }{\partial
x}\left(\frac{1}{2\pi\hbar}e^{\frac{ipx}{\hbar}}\right)
=p\Psi_{p}(x)
\end{equation}
Now let's consider the Hamiltonian for the system, it's $\hat{H} =
\frac{p^{2}}{2m}$; it's easily seen that both of the $p$ and $-p$
will give the same energy level so it's double degenerate. Hence for
the free particle state which in energy level $\frac{p^{2}}{2m}$, it
can be $\Psi_{p}(x)$ or $\Psi_{-p}(x)$ or any linear combination
between the two components. However, these wave functions are
physically distinguished with each other. Hence, for such state with
certain energy level, what kind of wave function I can choose to
portray it?

Now let's prove that the origin in degeneracy in the eigen states,
which is the most potential reason for the degeneracy.

\begin{theorem}
Suggest we have some observable $hat{A}$ and its eigen states
$\ket{\Psi_{i}}$. On the other hand, we have two operators $\hat{F}$
and $\hat{G}$ that they both commute with $\hat{A}$, but they do not
commute with each other ($[\hat{F}, \hat{G}] \neq 0$). then there
must have degeneracy in the eigen states of $\ket{\Psi_{i}}$.
\end{theorem}

\begin{proof}
Now we prove it. For the $\hat{F}$:
\begin{equation}\label{}
\hat{A}\hat{F}\ket{\Psi_{i}} = \hat{F}\hat{A}\ket{\Psi_{i}} =
\hat{F}a_{i}\ket{\Psi_{i}} = a_{i}\hat{F}\ket{\Psi_{i}}
\end{equation}

For $\hat{G}$:
\begin{equation}\label{}
\hat{A}\hat{G}\ket{\Psi_{i}} = \hat{G}\hat{A}\ket{\Psi_{i}} =
\hat{G}a_{i}\ket{\Psi_{i}} = a_{i}\hat{G}\ket{\Psi_{i}}
\end{equation}
However, the two vectors of $\hat{F}\ket{\Psi_{i}}$ and
$\hat{G}\ket{\Psi_{i}}$ should differentiate by more than the
constant, that is : $\hat{F}\ket{\Psi_{i}} \neq C\times
\hat{G}\ket{\Psi_{i}}$. Since $\hat{F}$ and $\hat{G}$ do not commute
with each other, so they can not share the same eigen states
simultaneously (this point will be proved in the next section). On
the other hand, it's clear that they are all the eigen states for
the operator $\hat{A}$, so both of the two eigen states take up the
same eigen value. Thus the eigen states has degeneracy. \qedhere
\end{proof}

Finally, let's give some example to illustrate the degenerate case
above. For hydrogen atom, we have its angular momentum operator of
$\hei{l}$ ($\hei{l}$ is some vector operator) commutes with the
Hamiltonian, however; the three components for the $\hei{l}$ can not
commute with each other:
\begin{equation}\label{}
[\hat{l}_{i}, \hat{l}_{j}] \neq 0 \quad i, j \in x, y, z
\end{equation}
Therefore, it can be well expected that the energy level
corresponding to different angular momentum $l=1, l=2, \cdots$ ($l$
is the eigen states for the angular momentum operator) are all
degenerate; and the calculation result for the hydrogen atom just
confirms such guess.



%%%%%%%%%%%%%%%%%%%%%%%%%%%%%%%%%%%%%%%%%%%%%%%%%%%%%%%%%%%%%%%%%%%%%%%%%%%%%%
\subsection{The essence of commutation}
\label{essence_in_operator}
%
% derive the Heisenberg relationships
%
%
%
%
In the above section, we have just demonstrated that the commutation
relationship is the origin for the eigen states degeneracy. Thus
what's the essence of the commutation ($\hat{A}\hat{B}=
\hat{A}\hat{B}$)? Here below it can proved that two operators are
commutative as long as they share the same eigen states.

Suppose we have two operators which correspond to dynamic quality,
so they are hermitian. Consider the $|\Phi\rangle$ below:
\begin{equation}\label{OPERATOReq:19}
|\Phi\rangle = \xi\hat{A}|\Psi\rangle+i\hat{B}|\Psi\rangle
\end{equation}
Here $|\Psi\rangle$ is an arbitrary vector in Hilbert space, and
$\xi$ is an arbitrary real number. What's more, we note that the
corresponding vector in bra is:
\begin{equation}\label{}
\langle\Phi| = \langle\Psi|\hat{A}\xi-i\langle\Psi|\hat{B}
\end{equation}
We have made use of the hermitian character of $\hat{A}$ and
$\hat{B}$.

Now consider the expression below:
\begin{align}\label{OPERATOReq:22}
I(\xi) &= \langle\Phi|\Phi\rangle  \nonumber \\
&= \Big\langle \langle\Psi|\hat{A}\xi-i\langle\Psi|\hat{B}\Big|
\xi\hat{A}|\Psi\rangle+i\hat{B}|\Psi\rangle\Big\rangle \geq 0
\end{align}
To expand this expression, then we have:
\begin{multline}\label{OPERATOReq:23}
\Big\langle \bra{\Psi}\hat{A}\xi-i\bra{\Psi}\hat{B} \Big|
\xi\hat{A}\ket{\Psi}+i\hat{B}\ket{\Psi} \Big\rangle =
\\
\xi^{2}\bra{\Psi}\hat{A}^{2}\ket{\Psi} +
i\xi\bra{\Psi}\hat{A}\hat{B}\ket{\Psi} -
i\xi\bra{\Psi}\hat{B}\hat{A}\ket{\Psi}+
\bra{\Psi}\hat{B}^{2}\ket{\Psi} \\
=\xi^{2}\bra{\Psi}\hat{A}^{2}\ket{\Psi} +
\bra{\Psi}\hat{B}^{2}\ket{\Psi}  +
i\xi\bra{\Psi}\hat{A}\hat{B}-\hat{B}\hat{A}\ket{\Psi} \\
=\xi^{2}\bra{\Psi}\hat{A}^{2}\ket{\Psi} +
\bra{\Psi}\hat{B}^{2}\ket{\Psi}  + i\xi\bra{\Psi}[\hat{A},
\hat{B}]\ket{\Psi}
\end{multline}

Now suggest that $[\hat{A},\hat{B}] = i\hat{C}$. From
(\ref{OPERATOReq:24}) We have known that $\hat{C}$ is also some
hermitian operator.

By using the $\hat{C}$, we can drop the complex term in
(\ref{OPERATOReq:23}):
\begin{equation}\label{OPERATOReq:25}
i\xi\langle\Psi|[\hat{A},\hat{B}]|\Psi\rangle =
i\xi\langle\Psi|i\hat{C}|\Psi\rangle =
i^{2}\xi\langle\Psi|\hat{C}|\Psi\rangle =
-\xi\langle\Psi|\hat{C}|\Psi\rangle
\end{equation}

Let's express the expectation value of
$\langle\Psi|\hat{A}|\Psi\rangle = \overline{A}$, then the
(\ref{OPERATOReq:23}) can be expressed as:
\begin{eqnarray}\label{EIGENCTONSeq:4}
% \nonumber to remove numbering (before each equation)
  I(\xi) &=& \xi^{2}\overline{A^{2}} - \xi\overline{C} +
  \overline{B^{2}} \nonumber \\
  &=& \overline{A^{2}}
  \left ( \xi - \frac{\overline{C}}{2\overline{A^{2}}}\right)^{2} +
  \left ( \overline{B^{2}} - \frac{\overline{C}^{2}}{4\overline{A^{2}}}
  \right) \nonumber \\
  &\geq& 0
\end{eqnarray}
Additionally, we mention that since $\hat{A}$ and $\hat{B}$ are
hermitian operators, then $\hat{A}^{2}$ and $\hat{B}^{2}$ is also
hermitian. So their expectation value is real.

Since that $\xi$ is an arbitrary number, we can make $\xi =
\frac{\overline{C}}{2\overline{A^{2}}}$. Thus we have
$\overline{B^{2}} - \frac{\overline{C}^{2}}{4\overline{A^{2}}} \geq
0$. This equation finally leads to:
\begin{eqnarray}\label{OPERATOReq:26}
% \nonumber to remove numbering (before each equation)
  \overline{A^{2}}\cdot
\overline{B^{2}} &\geq&  \frac{1}{4}\overline{C}^{2} \rightarrow \nonumber \\
 \sqrt{ \overline{A^{2}}\cdot
\overline{B^{2}}} &\geq&  \frac{1}{2} |\overline{C}|
\end{eqnarray}
Here $|\overline{C}|$ indicates the absolute value for the
$\overline{C}$.

Now let's make some transformation:
\begin{equation}\label{}
\begin{split}
  \hat{C} &= \frac{1}{i}
  [\hat{A}\hat{B} - \hat{B}\hat{A}] \Rightarrow \\
   |\overline{C}| &= |\frac{1}{i}\overline{[\hat{A},\hat{B}]}|
   \\
   &= |\frac{1}{i}| |\overline{[\hat{A},\hat{B}]}| \\
   &= |\overline{[\hat{A},\hat{B}]}|
\end{split}
\end{equation}

So the result in the (\ref{OPERATOReq:26}) finally can be:
\begin{equation}\label{OPERATOReq:27}
\sqrt{\overline{A^{2}}\cdot \overline{B^{2}}}   \geq \frac{1}{2}
\left|\overline{[\hat{A}, \hat{B}]} \right|
\end{equation}

Next let's make some modification to the result to get some
conclusion. Since $\overline{A}$ is the average value of the
operator A; now in the (\ref{OPERATOReq:27}) we make that the
$\hat{A}$ and $\hat{B}$ are replaced by $\Delta\hat{A}$ and $\Delta
\hat{B}$, respectively:
\begin{equation}\label{}
\Delta \hat{A} = \hat{A} - \langle\Phi|\hat{A}|\Phi\rangle
\end{equation}
Since $\langle\Phi|\hat{A}|\Phi\rangle$ is some real number, we can
prove that $\Delta\hat{A}$ is also some hermitian operator.
Furthermore, we can show that:
\begin{equation}\label{}
\begin{split}
  [\Delta \hat{A}, \Delta \hat{B}] &=
  [\hat{A} - \overline{A}, \hat{B} - \overline{B}] \\
    &= [\hat{A}, \hat{B}]
\end{split}
\end{equation}

After such replacement, the (\ref{OPERATOReq:27}) becomes:
\begin{equation}\label{OPERATOReq:28}
\sqrt{\overline{\Delta \hat{A}^{2}}\cdot \overline{\Delta
\hat{B}^{2}}} \geq \frac{1}{2} \left|\overline{[\Delta\hat{A},
\Delta\hat{B}]} \right| \Rightarrow \sqrt{\overline{\Delta
\hat{A}^{2}}\cdot \overline{\Delta \hat{B}^{2}}} \geq \frac{1}{2}
\left|\overline{[\hat{A}, \hat{B}]} \right|
\end{equation}

For the $\overline{\Delta \hat{A}^{2}}$, actually we can prove that
$\overline{\Delta \hat{A}^{2}}= \overline{\Delta \hat{A}}^{2}$.
However, since this result is trivial and not be used in other
place, we do not intend to give the proof; instead just to accept
this result. so the result can be finally transformed as:
\begin{equation}\label{OPERATOReq:31}
\overline{\Delta A}\cdot \overline{\Delta B}   \geq \frac{1}{2}
\left|\overline{[\hat{A}, \hat{B}]} \right|
\end{equation}

This result holds true for any two hermite operators. What's more,
in this universal inequation of (\ref{OPERATOReq:31}) the
implication within it is some very important and physical
meaningful.

For two arbitrary hermite operators of $\hat{A}$ and $\hat{B}$; if
they are not commutative; $\overline{[\hat{A}, \hat{B}]} > 0$. Thus
$\Delta A$ and $\Delta B$ can not equal to $0$ simultaneously. As a
result, the measurement of $\hat{A}$ and the measurement of
$\hat{B}$ can not have definite measurement result at the same time;
in other words, $\hat{A}$ and $\hat{B}$ can not share the same eigen
states. By the way, what we have deducted in the above process is
the universal uncertainty principle. If we have $\hat{A} = \hat{x}$
and $\hat{B} = \hat{p}$, according to the (\ref{OPERATOReq:31}), it
leads to: $\Delta x\Delta p \geq \frac{\hbar}{2}$, here we have used
the relation that $[\hat{x}, \hat{p}] = i\hbar$ (this relation will
be revealed in the following chapter). That's the final conclusion
behind the commutation.

%%%%%%%%%%%%%%%%%%%%%%%%%%%%%%%%%%%%%%%%%%%%%%%%%%%%%%%%%%%%%%%%%%%%%%%%%%%%%%
\subsection{Complete sets of commuting observables}
\label{CSCO_in_operator}
%
% What's the complete sets of commuting observables?
% How to do that?
%
In section \ref{origin_degeneracy_in_operator}, we have put forward
plane wave function example to show that there's some uncertainty in
fixing the wave functions if there's degenerate situation. Thus, how
to eliminate such uncertainty in the description for Hilbert space?

Generally to say, while there's some degeneracy in the the space of
eigen states, we can use a group of observable operators rather than
one observable operator to determine the corresponding Hilbert
space. In this group of operators any two of them commute with each
other (physically to say, each physical quantity provides a freedom
to specify the investigated system) so that to ensure they can share
the same eigen states. It can be proved that this group of
observable operators determine some ``definite'' Hilbert space to
describe the quantum system. Therefore, this group is called
``complete set of commutating observables''; in the future, they
will be abbreviated as ``CSCO''.

Now let's going to demonstrate the validity of the above words. The
validity can be confirmed by the theorem below:

\begin{theorem}
For two commutative observable operators of $\hat{A}$ and $\hat{B}$,
there must have some space of eigen states for both of them.
\end{theorem}

Here for simplicity we only consider the discrete case. However, for
the continuous Hilbert space such as plane wave functions, we can
get the same result; but the mathematical demonstration is much more
complicated.

\begin{proof}
Suggest that \kett{\Psi_{i}} ($i=1, 2, \cdots$) are the complete set
of normalized eigen states for the $\hat{A}$. We can know that
$\hat{B}\ket{\Psi_{i}}$ is also $\hat{A}$'s eigen states:
\begin{equation}\label{}
\hat{A}(\hat{B}\ket{\Psi_{i}}) = \hat{B}\hat{A}\ket{\Psi_{i}}
=a_{i}(\hat{B}\ket{\Psi_{i}})
\end{equation}

The process will be classified into two parts in terms of the
presence of degeneracy in the space of $\ket{\Psi_{i}}$.

First, there's no degeneracy in the eigen states of $\hat{A}$.

Thereby, if $\hat{A}\ket{\Psi_{i}} = a_{i}\ket{\Psi_{i}}$, and
$\hat{A}(\hat{B}\ket{\Psi_{i}}) = a_{i}(\hat{B}\ket{\Psi_{i}})$; the
$\ket{\Psi_{i}}$ and $\hat{B}\ket{\Psi_{i}}$ should distinguish by
only an constant: $\hat{B}\ket{\Psi_{i}} = b_{i}\ket{\Psi_{i}}$.
Hence each of $\ket{\Psi_{i}}$ is also the eigen function for
$\hat{B}$, and the eigen value is $b_{i}$.

Second, there's degeneracy in the eigen states of $\hat{A}$.

suggest that there's a $s$ dimensional orthogonal subspace ($s$ can
be some infinite number, but the proof below is same) of
$\ket{\Psi_{mi}}$ ($i=1, 2, \cdots, s$); they all give the same
eigen value of $l$ by $\hat{A}\ket{\Psi_{mi}} = l\ket{\Psi_{mi}}$ .
Now base on this subspace we are going to construct a new set of
$\ket{\Psi^{'}_{m1}}, \cdots, \ket{\Psi^{'}_{ms}}$; each of them are
the linear combination of the original vectors:
\begin{equation}\label{}
\ket{\Psi^{'}_{mi}} = \sum^{s}_{j=1}c_{ij}\ket{\Psi_{mj}}
\end{equation}
each of the $\ket{\Psi^{'}_{mi}}$ is the eigen function for both
$\hat{A}$ and $\hat{B}$.

So we have (we use $\ket{\Psi^{'}_{m}}$ to represent an arbitrary
$\ket{\psi^{'}_{mi}}$):
\begin{eqnarray}
% \nonumber to remove numbering (before each equation)
  \hat{B} \ket{\Psi^{'}_{m}} &=& b^{'}\ket{\Psi^{'}_{m}} \nonumber \\
                             &=&b^{'} \sum^{s}_{j=1}c_{j}\ket{\Psi_{mj}}
\end{eqnarray}
By multiplying $\bra{\Psi_{m1}}, \bra{\Psi_{m2}}, \cdots,
\bra{\Psi_{ms}}$ to the above equation, we can get $s$ equations:
\begin{eqnarray}
% \nonumber to remove numbering (before each equation)
  \sum^{s}_{j=1}c_{j} \bra{\Psi_{m1}}\hat{B}\ket{\Psi_{mj}} &=& b^{'}c_{1} \nonumber \\
  \sum^{s}_{j=1}c_{j} \bra{\Psi_{m2}}\hat{B}\ket{\Psi_{mj}} &=& b^{'}c_{2} \nonumber \\
  \cdots &\cdots& \cdots  \nonumber \\
  \sum^{s}_{j=1}c_{j} \bra{\Psi_{ms}}\hat{B}\ket{\Psi_{mj}} &=& b^{'}c_{s} \nonumber \\
\end{eqnarray}
Here we implicitly use the orthogonality of the subspace:
$\langle\Psi_{mi}\ket{\Psi_{mj}} = \delta _{ij}$.

If the $\bra{\Psi_{mi}}\hat{B}\ket{\Psi_{mj}}$ is abbreviated as
$B_{ij}$, it can see that the above equations can be transformed in
to a matrix:
\begin{equation}\label{SEeq:4}
\begin{bmatrix}
  B_{11}-b^{'} & B_{12} & \cdots & B_{1s} \\
  B_{21} & B_{22}-b^{'} & \cdots & B_{2s} \\
  \cdots & \cdots & \cdots & \cdots \\
  B_{s1} & B_{s2} & \cdots & B_{ss}-b^{'} \\
\end{bmatrix}
\begin{bmatrix}
  c_{1} \\
  c_{2} \\
  \cdots \\
  c_{s} \\
\end{bmatrix}
 = 0
\end{equation}
Here in this matrix, $B_{ij}$ is some number we can calculate out,
the coefficient of $c_{i}$ is the solution we are pursuing, the
$b^{'}$ is the corresponding eigen values, which is also need to
know.

The presence of the solution of $c_{i}$ requires that:
\begin{equation}\label{}
\begin{vmatrix}
  B_{11}-b^{'} & B_{12} & \cdots & B_{1s} \\
  B_{21} & B_{22}-b^{'} & \cdots & B_{2s} \\
  \cdots & \cdots & \cdots & \cdots \\
  B_{s1} & B_{s2} & \cdots & B_{ss}-b^{'} \\
\end{vmatrix} = 0
\end{equation}

For this determinant, it has roots of $b^{'}_{i}$ ($i=1, 2, \cdots,
s$), some of them may get to be same. For each of $b^{'}_{i}$, the
linear equation of (\ref{SEeq:4}) can yield a group of coefficients
of $c_{ij}$ ($j=1, 2, \cdots, s$); thus the $\ket{\Psi^{'}_{mi}}$
has been fixed out.

If different $b^{'}_{i}$ takes different values, the set of Hilbert
space therefore has been determined. If there's still degeneracy in
the new space, we will take another operator of $\hat{C}$ which
commuted with the $\hat{A}$ and $\hat{B}$ to work out some new space
until the complete and definite Hilbert space has been found out.
\qedhere
\end{proof}

For the complete set of observable operators, we can use their eigen
values to label each vector in the Hilbert space. For example, for
the $\ket{\Psi_{i}}$, the $\hat{A}$, $\hat{B}$ and $\hat{C}$
correspond to the eigen values as $a_{i}$, $b_{i}$ and $c_{i}$
respectively; so this vector can be abbreviated as
$\ket{a_{i}b_{i}c_{i}}$. In the H atom schrodinger equation, we will
see how can we do this.


%%%%%%%%%%%%%%%%%%%%%%%%%%%%%%%%%%%%%%%%%%%%%%%%%%%%%%%%%%%%%%%%%%%%%%%%%%%%
\section{Further discussion to the operator}
\label{further_in_operator}

%%%%%%%%%%%%%%%%%%%%%%%%%%%%%%%%%%%%%%%%%%%%%%%%%%%%%%%%%%%%%%%%%%%%%%%%%%%%
\subsection{How to bridge the physical quantities in quantum mechanics
with the ones in classic mechanics}
\label{vector_schalar_in_operator}
%
%  1  how to understand the correspondence between the physical
%     quantities in quantum mechanics and the ones in classic
%     1.1  how to build the operators? from r and p
%     1.2  they are correspond to each other
%  2  axiom for the r and p
%  3  representations for the r and p
%  4  vector operator and scalar operator
%
So far we have made many discussions to the general properties of
the operator, which is concentrating on the mathematical properties
of the operators. However, how can be understand the physical
essence of the operator? How can we bridge the physical quantities
in quantum mechanics with the ones in classic mechanics?

The understanding for the physical quantity in quantum mechanics
should be based on the physical quantities in classical mechanics.
In most cases, we have physical quantities one to one mapping
between the microworld to the macrocosm, that is to say; both of
them share the same expressions. The only difference is, in classic
mechanics such physical quantities are expressed by variables who
are defining on the real number field; and in quantum mechanics the
corresponding physical quantities are expressed by operators, which
is restrict to be hermitian type. However, the expression for them
are same. For example, the kinetic energy are both expressed as
$\frac{\bm{p}^{2}}{2m}$, the angular momentum are both expressed as
$\bm{r}\times\bm{p}$ etc. In essence, perhaps such one to one
correspondence depicts some kind of ``continuity'' and
``consistency'' for our real world.

However, there's not all the physical quantities that can be found
one to one correspondence between the the microworld and the
macrocosm. For example, the spin is contained in quantum particles
but diminished in classical objects. The spin phenomenon is some
relativistic effects, so we can only ``phenomenologically''
introduce the spin operator by analogous with the angular momentum
operator; the detailed will be discussed in the following contents.

Nevertheless, for most of the physical quantities in quantum
mechanics, such one to one mapping relation exists. Hence we can
construct the operators in quantum mechanics by analogous with the
corresponding physical quantities in classic mechanics.

Additionally, there are two important illustrations for the one to
one corresponding relationship:
\begin{itemize}
  \item  the vector quantities in classical mechanics is still
  corresponding to vector quantities, so as the scalar physical
  quantities.
  \item  in classical mechanics, all the physical quantities can be
  expressed based on some function of
  position $\bm{r}$ and momentum $\bm{p}$, hence in quantum
  mechanics all the physical operators can be constructed
  by position operator and momentum operator.
\end{itemize}
In the following content, the vector operators will be labeled as
$\hei{A}$, and the scalar operator is labeled as $\hat{A}$.

So far it can see that similar to the classical mechanics, the
position and the momentum also constitutes the foundation for
quantum mechanics. However, since that the methodology in classical
mechanics and the quantum mechanics are distinguished greatly, so we
can expect that there must have some ``distinctness'' between the
description for the position and the momentum, and actually we
indeed have an axiom:
\begin{axiom}\label{axiom5}
\textbf{In quantum mechanics, each particle's position operator
under Cartesian coordinate, namely the $\hat{x}_{i}$ ($\hat{x}_{1} =
\hat{x}$, $\hat{x}_{2} = \hat{y}$ and $\hat{x}_{3} = \hat{z}$); and
it's corresponding momentum operator $\hat{p}_{i}$ satisfy the
commutation rules below:
\begin{equation}\label{PRAMReq:37}
[\hat{x}_{i}, \hat{x}_{j}] = 0 \quad [\hat{p}_{i}, \hat{p}_{j}] = 0
\quad [\hat{x}_{i}, \hat{p}_{j}] = i\hbar\delta_{ij}
\end{equation}
Furthermore, the operators for different particle are all commuting
with each other.}
\end{axiom}

Now let's give some physical analysis to the axiom above. According
to the previous discussion in (\ref{CSCO_in_operator}), if two
physical operators are commuting with each other; physically such
two operators stand for two independent freedoms to describe the
corresponding system. therefore, the relation $[\hat{x}_{i},
\hat{p}_{j}] = i\hbar\delta_{ij}$ indicates that the position
information and momentum information for the system are not
independent with each other. On the other hand, if we can obtain the
expression for $\hei{r}$, then the (\ref{PRAMReq:37}) indicates that
we can get the expression for $\hei{p}$ via $\hei{r}$. Hence from
this aspect, they are also not independent.

Here such knowledge has been already obtained through the
commutation relationship discussion in (\ref{essence_in_operator}),
there the analysis on the commutation shows a coherent interaction
between the position information and the momentum information.
What's more, in the discussion related to the basis functions (see
the \ref{sec:PWF_in_Hilbert}), we can see something interesting that
for any square-integrable wave function, from Fourier transformation
it can be decompressed into the $\Phi(x)$ (the eigen state for
$\hat{x}$) or the $\Psi(p)$ (the eigen state for $\hat{p}$). Such
relation indicates that we can express the quantum state either by
position or by momentum; they are identical with each other. This is
another character to understand the dependence between the position
information and momentum information.

Furthermore, in the following chapter we will prove such relation
that if we express the momentum operator via position operator, then
the correspondent quantum state can be expressed as some function of
position; on the other hand, if the position operator is expressed
via the momentum operator, then the quantum state can be expressed
as some function of momentum; such relation can be depicted as:
\begin{equation}\label{OPERATOReq:34}
\begin{split}
  \hei{r} = \hei{r}(\bm{r}), \hei{p} = \hei{p}(\bm{r})
  &\Rightarrow
\ket{\Psi} \Leftrightarrow \Psi(\bm{r}) \\
  \hei{r} = \hei{r}(\bm{p}), \hei{p} =
\hei{p}(\bm{p})  &\Rightarrow \ket{\Psi} \Leftrightarrow
\Psi(\bm{p})
\end{split}
\end{equation}
The corresponding $\Psi(\bm{r})$ is the position representation for
vector of $\ket{\Psi}$ in given Hilbert space, and similarly
$\Psi(\bm{p})$ is called momentum representation.

In the derivation for such relation, we firstly introduce the eigen
states for the position and the momentum operator (From the
discussion in (\ref{ASCWFFFP_in_basic}) it can see that their eigen
states are only for free particles. The wave function in
(\ref{BASICeq:14}) is the eigen states for momentum operator, and by
Fourier transformation in (\ref{BASICeq:19}) and (\ref{BASICeq:20})
we can get the corresponding eigen state for position operator).
Here the details for the eigen functions will be omitted and they
are only expressed as:
\begin{align}
\label{PRAMReq:1}
\hat{x}\ket{x} &= x\ket{x} \nonumber \\
\hat{p}\ket{p} &= p\ket{p}
\end{align}
By investigating the changing of eigen states and eigen values, and
through the transformation between the $\ket{x}$ and $\ket{p}$; we
can finally get the expression that how to express $\hat{p}$ via
$\hat{x}$, or express $\hat{x}$ via $\hat{p}$. The details will be
given in chapter \ref{position_momentum representation}.


%%%%%%%%%%%%%%%%%%%%%%%%%%%%%%%%%%%%%%%%%%%%%%%%%%%%%%%%%%%%%%%%%%%%%%%%%%%%
\subsection{Operator functions}
\label{operator_functions_in_operator}
%
%
%
the operators can be further organized as some function form. In the
following contents, we can see that such form is vital in some
discussions (see \ref{eigen_states_in_position_momentum}). For
example, we can have such expression for $\hat{A}$:
\begin{equation}\label{}
\hat{B}_{\xi} = e^{\frac{i}{\hbar}\xi\hat{A}}
\end{equation}
According to the results in mathematical analysis, such expression
can be expanded into series that:
\begin{equation}
  \hat{B}_{\xi} = 1 + \left(\frac{i}{\hbar}\xi\hat{A}\right) +
 \frac{\left(\frac{i}{\hbar}\xi\hat{A}\right)^{2}}{2!} +
\frac{\left(\frac{i}{\hbar}\xi\hat{A}\right)^{3}}{3!} + \cdots
\end{equation}

Similarly, we can also have the differential etc. operations to the
operator. The details for this part discussion can be referred to
the book by XingLin Ke\cite{XingLinKe}.


%%%%%%%%%%%%%%%%%%%%%%%%%%%%%%%%%%%%%%%%%%%
\subsection{Unitary operator}\label{OPERATOR:2}
%
% the definition of the unitary operator
%
%
Now we introduce another important operator in the quantum
mechanics, which is called unitary operator. It can be
mathematically defined as:
\begin{align}\label{OPERATOReq:10}
\hat{U}^{+}\hat{U} &= \hat{U}\hat{U}^{+}= I \nonumber \\
U^{+}  &= U^{-1}
\end{align}
Such unitary operator has some important feature that it's able to
commute with all the other operators which represents some physical
quantity:
\begin{align}\label{OPERATOReq:14}
\hat{A}\hat{U} &= \hat{U}\hat{A} \nonumber \\
\hat{A}\hat{U}^{+} &= \hat{U}^{+}\hat{A}
\end{align}

Physically to say, the unitary operator is usually corresponding to
some transformation between two sets of basis functions in the same
Hilbert space. Such transformation is also called unitary
transformation. For example, in quantum chemistry we can get a group
of orbitals after Hatree-Fock calculation, then we can convert such
orbitals which dispersed among the whole molecule into some
localized ones, such transformation is just done by the unitary
operator (see the \ref{HFT:3} for more contents).

According to the analysis in the (\ref{LIV_in_Hilbert}), the basis
functions which depicts the same Hilbert space are actually
identical with each other. Hence here we usher in some important
question: How to describe the identity between two sets of basis
functions in the same Hilbert space?

Here we have two general rules:
\begin{itemize}\label{OPERATOReq:13}
  \item Two sets of basis functions should retain the same
  space structure, that is to say:
  \begin{equation}\label{}
|\Psi\rangle = \hat{U}|\Phi\rangle \quad
\langle\Phi_{i}|\Phi_{j}\rangle = \delta_{ij} \Rightarrow
\langle\Psi_{i}|\Psi_{j}\rangle = \delta_{ij}
  \end{equation}
  \item For any arbitrary operator of $\hat{A}$ the
  expectation value should be same:
\begin{equation}
\langle\Phi_{i}|\hat{A}|\Phi_{j}\rangle =
\langle\Psi_{i}|\hat{A}|\Psi_{j}\rangle
\end{equation}
\end{itemize}

Now let's prove that the (\ref{OPERATOReq:10}) will stratify the
identity requirements. Suggest that we have some discrete basis
functions of $\Phi_{i}$ ($i = 1, 2, \cdots, n, \cdots$) to depict
the Hilbert space, then some unitary transformation is used to
convert it into some new sets:
\begin{equation}\label{}
|\Psi_{i}\rangle = \hat{U}|\Phi_{i}\rangle
\end{equation}
then we can have:
\begin{align}\label{OPERATOReq:11}
\langle \Psi_{i}|\Psi_{j}\rangle & = \langle
\hat{U}\Phi_{i}|\hat{U}\Phi_{j}\rangle \nonumber \\
&=\langle
\Phi_{i}|\hat{U}^{+}\hat{U}|\Phi_{j}\rangle \nonumber \\
&=\langle \Phi_{i}|\Phi_{j}\rangle
\end{align}
Besides, since the unitary operator make one to one correspondence
tween $\Phi_{i}$ and $\Psi_{i}$, so the old sets of $\Phi_{i}$ and
the new sets of $\Psi_{i}$ share the same space structure. The
(\ref{OPERATOReq:13}) has been proved.

Now let's prove the (\ref{OPERATOReq:13}):
\begin{align}\label{}
\langle\Psi_{i}|\hat{A}|\Psi_{j}\rangle &=
\langle\Phi_{i}|\hat{U}^{+}\hat{A}\hat{U}|\Phi_{j}\rangle \nonumber \\
&=\langle\Phi_{i}|\hat{A}\hat{U}^{+}\hat{U}|\Phi_{j}\rangle
\underrightarrow{
\text{ from the definition in \ref{OPERATOReq:14}}}\nonumber \\
&=\langle\Phi_{i}|\hat{A}|\Phi_{j}\rangle
\end{align}

Hence we can see that the unitary transformation does not alter the
characters of Hilbert space.

%%%%%%%%%%%%%%%%%%%%%%%%%%%%%%%%%%%%%%%%%%%%
\subsection{Density operator}\label{OPERATOR:1}
%
%
%
%
In the above content, we have define the inner product between two
vectors in Hilbert space as $\langle\Phi|\Psi\rangle$. Now if we
rewrite its sequence as $|\Psi\rangle\langle\Phi|$, so what is it?

Actually such type of ``product'' between $|\Psi\rangle$ and
$\langle\Phi|$ is a kind of operator. For example, suggest
$|\Omega\rangle$ is some arbitrary state, we have:
\begin{equation}\label{}
|\Psi\rangle\langle\Phi|\Omega\rangle =
(\langle\Phi|\Omega\rangle)|\Psi\rangle
\end{equation}
So it just converts the $|\Omega\rangle$ into some new state of
$(\langle\Phi|\Omega\rangle)|\Psi\rangle$.

Usually in quantum mechanics we use the operator in the form of
$\ket{\Psi}\bra{\Psi}$, here $\ket{\Psi}$ is some arbitrary vector
in the Hilbert space. For this type of operator, we call it as
``density operator''. Now let's discuss the characters related to
this operator. Now we define this operator as $\hat{\gamma}$.

\begin{theorem}\label{}
$\hat{\gamma}^{+} = \hat{\gamma}$. So $\hat{\gamma}$ is hermitian.
\end{theorem}

\begin{proof}
\begin{equation}\label{}
\hat{\gamma}^{+} = (\ket{\Psi}\bra{\Psi})^{+} = \ket{\Psi}\bra{\Psi}
= \hat{\gamma}
\end{equation}
Hence the operator of $\hat{\gamma}$ is hermitian. \qedhere
\end{proof}

\begin{theorem}\label{}
For any $\Phi$, $\langle\Phi|\hat{\gamma}|\Phi\rangle \geq 0$.
\end{theorem}

\begin{proof}
\begin{equation}\label{}
\langle\Phi|\hat{\gamma}|\Phi\rangle = \langle\Phi|\Psi\rangle^{2}
\geq 0
\end{equation}
\qedhere
\end{proof}

\begin{theorem}\label{}
$\hat{\gamma}^{2} = \hat{\gamma}$.
\end{theorem}

\begin{proof}
\begin{equation}\label{}
\hat{\gamma}^{2} = \ket{\Psi}\bra{\Psi}\Psi\rangle\bra{\Psi} =
\ket{\Psi}\bra{\Psi} = \hat{\gamma}
\end{equation}
So $\hat{\gamma}$ is idempotent. \qedhere
\end{proof}

\begin{theorem}\label{}
$\hat{\gamma}\hat{B} = \hat{B}\hat{\gamma}$. If $\hat{B}$ is
hermitian.
\end{theorem}

\begin{proof}
Suggest that we have some arbitrary wave function of $\Phi$:
\begin{equation}\label{}
\langle\Phi|\hat{\gamma}\hat{B}|\Phi\rangle =
\langle\Phi\ket{\Psi}\bra{\Psi}\hat{B}|\Phi\rangle =
\bra{\Psi}\hat{B}|\Phi\rangle\langle\Phi\ket{\Psi} =
\bra{\Phi}\hat{B}|\Psi\rangle\langle\Psi\ket{\Phi} =
\langle\Phi|\hat{B}\hat{\gamma}|\Phi\rangle
\end{equation}
So $\hat{\gamma}$ is able to exchange with any hermitian operators.
\qedhere
\end{proof}

Finally, let's prove some very important equation in quantum
mechanics:
\begin{theorem}\label{}
For any arbitrary  $\hat{\gamma}$, we have:
\begin{equation}\label{}
i \hbar\frac{\partial \hat{\gamma}}{\partial t} = [\hat{H},
\hat{\gamma}]
\end{equation}
\end{theorem}

\begin{proof}
Suggest that $\hat{\gamma} = \ket{\Psi}\bra{\Psi}$, now let's start
from the Schrodinger equation:
\begin{align}\label{}
\hat{H}\ket{\Psi} &= i \hbar \frac{\partial \ket{\Psi}}{\partial t}
\nonumber \\
\bra{\Psi}\hat{H} &= -i \hbar \frac{\partial \bra{\Psi}}{\partial t}
\end{align}
Hence we have:
\begin{equation}\label{}
\begin{split}
  i \hbar\frac{\partial \hat{\gamma}}{\partial t} &=
   i \hbar \frac{\partial \ket{\Psi}\bra{\Psi}}{\partial t}\\
    &=
    i \hbar \frac{\partial \ket{\Psi}}{\partial t}\bra{\Psi} +
    \ket{\Psi}i \hbar \frac{\partial \bra{\Psi}}{\partial t} \\
    &= (\hat{H}\ket{\Psi})\bra{\Psi} - \ket{\Psi}(\bra{\Psi}\hat{H})
    \\
    &= [\hat{H}, \hat{\gamma}]
\end{split}
\end{equation}
 \qedhere
\end{proof}

Now let's go to see some specific kind of $\hat{\gamma}$:
\begin{equation}\label{OPERATOReq:15}
\hat{\gamma}_{i} =|\Psi_{i}\rangle\langle\Psi_{i}|
\end{equation}
Here $|\Psi_{i}\rangle$ designates the $i$th component in the basis
functions of $\Psi_{i}$ ($i = 1, 2, \cdots$). It can see that for
some arbitrary state of $|\Psi\rangle$, we have:
\begin{equation}\label{}
\hat{\gamma}_{i}|\Psi\rangle =
|\Psi_{i}\rangle\langle\Psi_{i}|\Psi\rangle = c_{i}|\Psi_{i}\rangle
\end{equation}
$c_{i}$ characterizes the weight for the component of $i$ in the
$|\Psi\rangle$, so the effects of the operator is just to projecting
out the $i$th component in the $|\Psi\rangle$. Hence, this kind of
operator is also called ``project operator''.

Finally, we note that if we add all the $\hat{\gamma}_{i}$ across
the whole Hilbert space:
\begin{equation}\label{}
\hat{\Gamma} = \sum_{i} \hat{\gamma}_{i}
\end{equation}
We can see that for some arbitrary state of $|\Psi\rangle$:
\begin{align}\label{OPERATOReq:18}
\hat{\Gamma}|\Psi\rangle &=
\sum_{i}|\Psi_{i}\rangle\langle\Psi_{i}|\Psi\rangle \nonumber \\
&=\sum_{i}|(\langle\Psi_{i}|\Psi\rangle)\Psi_{i}\rangle
\underrightarrow{
\text{ according to \ref{Hilbert:1}}}\nonumber \\
&=|\Psi\rangle
\end{align}
Hence we have $\hat{\Gamma} \equiv I$. If we compare the
(\ref{OPERATOReq:18}) with the (\ref{Hilberteq:15}), we can see that
they are the two faces on the same coin. So the
(\ref{OPERATOReq:18}) is another expression for the closure relation
(more details see \ref{sec:CR_in_Hilbert}).

%%%%%%%%%%%%%%%%%%%%%%%%%%%%%%%%%%%%%%%%%%%%%


%%% Local Variables:
%%% mode: latex
%%% TeX-master: "../../main"
%%% End:
