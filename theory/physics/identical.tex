%
%  revised at Aug 10th, 2009
%
%
%

%%%%%%%%%%%%%%%%%%%%%%%%%%%%%%%%%%%%%%%%%%%%%%%%%%%%%%%%%%%%%%%%%%%%%%%%%%%%%%%%%%%%%%%%%
\chapter{Identical particles}\label{identical_particles}
%
%  1  it's an independent hypothesis, but coherent with the other
%     properties such as the uncertainty principle
%
%
%
%
%
%
%
%%%%%%%%%%%%%%%%%%%%%%%%%%%%%%%%%%%%%%%%%%%%%%%%%%%%%%%%%%%%%%%%%%%%%%%%%%

\section{Identity of quantum particle}
%
% 1 the identity of the quantum particles should be some hypothesis
%   of the quantum mechanics, but it's coherent with the other
%   hypothesis in the quantum mechanics
%
The particles, which fall into the range of quantum type; can be
classified into different types; such as electrons, different atoms,
photons and so on. It's found that the same type of particles are
identical with each other; that is to say, it's impossible to
distinguish any particles which is in the same type. For example,
two electrons are same with each other.

The property of identity is correlated with the quantum effects.
Since the uncertainty principle limits our understanding to the
quantum states, it can say that there's no enough information to
distinguish two identical particles. Although this is some kind of
"subjective" interpretation, and should not be taken seriously.
However, from such understanding we can see that the description of
the quantum mechanics is coherent, the character of identity is not
independent with the other properties such as uncertainty principle;
they are strongly correlated with each other to form a whole system
to portray the phenomenon in the quantum mechanics.


%%%%%%%%%%%%%%%%%%%%%%%%%%%%%%%%%%%%%%%%%%%%%%%%%%%%%%%%%%%%%%%%%%%%%%%%
\section{Exchange operator}
%
% 1  introduce the exchange operator
% 2  how to construct composite wave function (HF)
%
%
Because of the identity in the quantum particles, for any $\Psi$
describe quantum system, to exchange two arbitrary particles (they
are the same type), and reverse the process; the system should keep
to be same.

Therefore we introduce the exchange operator of $\hat{P_{ij}}$,
which exchange i particle and j particle in the $\Psi$, in respect
that i and j are the identical particles, the $\hat{P_{ij}}\Psi$
should be different with $\psi$ by no more than a constant. That is:
\begin{equation}\label{IDENTICALeq:1}
\hat{P_{ij}}\Psi(1, 2, \cdots i, \cdots j, \cdots ) = \Psi(1, 2,
\cdots j, \cdots i, \cdots ) = \lambda\Psi(1, 2, \cdots i, \cdots j,
\cdots )
\end{equation}

If $\hat{P_{ij}}$ acts on $\Psi$ twice, then the $\Psi$ must turn
back to be the same:
\begin{equation}\label{IDENTICALeq:2}
\hat{P^{2}_{ij}}\Psi = \Psi
\end{equation}
Therefore we have:
\begin{equation}\label{IDENTICALeq:3}
\lambda^{2} = 1
\end{equation}
If the $\Psi$ gives the $\lambda=1$, such wave function is called to
be "symmetric", its associated quantum particle is called "bosons",
and if $\Psi$ gives the $\lambda=-1$, the wave function is called to
be "antisymmetric", and the corresponding quantum particle is called
"fermions".

How can we know what kind of particles is fermions or bosons? So far
we can not get any judgement based on the discussion made before. We
have to usher into some axiom to understand such phenomenon, which
is also the last one in quantum mechanics:
\begin{axiom}\label{axiom7}
\textbf{For the quantum state which describe identical particles, if
the exchanging of a pair of identical particles lead to symmetric
wave function, then such particles is called ``bosons'', else it's
called ``fermions''. }
\end{axiom}

In nature, such exchange property is correlated with the spin
property of the quantum particles. If its spin state is one, two,
three $\cdots$ etc. fold of $\hbar$ ($s=0, \hbar, 2\hbar, 3\hbar,
\cdots$); the corresponding wave function is symmetric; one the
other hand, if its spin state is half-integral of $\hbar$ ($s=0,
\frac{\hbar}{2}, \frac{3\hbar}{2}, \cdots$); the associated wave
function is antisymmetric.

Since in quantum chemistry only the electron in involved into
consideration, so we consider a system composed of n identical
fermions (such as electrons); to further show the essence of its
exchange property. Here the mutual interactions between the
identical fermions is neglected, so that they are moving in an
irrelevant way. The question is: How can we construct the total wave
function of $\Psi$ from the $\varphi_{1}, \varphi_{2}, \cdots,
\varphi_{n}$; where individual particle resides on?

The answer is direct and clear. We can use a determinant to
represent the $\Psi$:
\begin{equation}\label{IDENTICALeq:4}
\Psi = \frac{1}{\sqrt[2]{n!}}\begin{vmatrix}
  \varphi_{1}(1) & \varphi_{1}(2) & \cdots & \varphi_{1}(n) \\
  \varphi_{2}(1) & \varphi_{2}(2) & \cdots & \varphi_{2}(n) \\
  \cdots & \cdots & \cdots & \cdots \\
  \varphi_{n}(1) & \varphi_{n}(2) & \cdots & \varphi_{n}(n) \\
\end{vmatrix}
\end{equation}
It's clear to see that if two fermions are exchanged (two columns
exchanged), the $\Psi$ will reverse its sign so that to meet the
requirement. The coefficient of $\frac{1}{\sqrt[2]{n!}}$ is the
normalized factor so as to make $\int \Psi^{2} d\tau =1$.

Behind the determinant expression, what really underlies it? Here
below we are going to give a in-depth discussion.

Because of the identity property of fermions, they can not
distinguish with each other; so it's impossible to assign which
fermion particle to reside on some specific $\psi_{i}$. Thus, all
the possible arrangements of "match" between fermions and different
$\psi$ should appear in the total wave function of $\Psi$.

We can use the permutation process to achieve this. For two fermions
on $\psi_{1}$ and $\psi_{2}$; there's only two possible matches:
$\psi_{1}(1)\psi_{2}(2)$ and $\psi_{1}(2)\psi_{2}(1)$. We can use an
exchange operator to change the $\psi_{1}(1)\psi_{2}(2)$ into
$\psi_{1}(2)\psi_{2}(1)$:
\begin{equation}\label{IDENTICALeq:5}
\hat{P}_{12}\psi_{1}(1)\psi_{2}(2) = \psi_{1}(2)\psi_{2}(1)
\end{equation}
Since the permutation will change its sign, we can write the total
wave function as:
\begin{eqnarray}
% \nonumber to remove numbering (before each equation)
  \Psi &=& \frac{1}{\sqrt{2}}(1-\hat{P_{12}})(\psi_{1}(1)\psi_{2}(2)) \nonumber\\
       &=& \frac{1}{\sqrt{2}}(\psi_{1}(1)\psi_{2}(2)-\psi_{1}(2)\psi_{2}(1))
\end{eqnarray}

Generally this way can be expanded to n fermions system. For the
original arrangement of $\psi_{1}(1)\psi_{2}(2)\cdots\psi_{n}(n)$
(it corresponds to some possible composition of the total wave
function), we can make $n!$ permutations to make fermion of i on any
arbitrary $\psi_{j}$. Each of the permutation (for example,
$\psi_{1}(2)\psi_{2}(1)\cdots\psi_{n}(n)$) is some possible
composition of the total wave function of $\Psi$, and these
permutations are only different with each other by an constant, and
it's clear that they all orthogonal to each other. Therefore we can
say that each of this permutation is some possible state for whole
wave function. the By principle of superposition, their linear
combination finally forms the total wave function.

Since that each permutation of $\hat{P}$ can be expressed as some
multiplication between $\hat{P}_{ij}$:
\begin{equation}\label{}
\hat{P} = \prod_{1\leq i<j\leq N}\hat{P}_{ij}
\end{equation}
For each $\hat{P}_{ij}$ the sign will change once; so if there's odd
number of $\hat{P}_{ij}$ in $\hat{P}$, the sign of $\hat{P}$ is
$-1$, otherwise it's $+1$.

Thus the total wave function is:
\begin{equation}\label{}
\Psi =
\frac{1}{\sqrt[2]{n!}}\sum_{P}(-1)^{P}P(\psi_{1}(1)\psi_{2}(2)\cdots\psi_{n}(n))
\end{equation}
It's same with the determinant expression.

%%%%%%%%%%%%%%%%%%%%%%%%%%%%%%%%%%%%%%%%%%%%%%%%%%%%%%%%%%%%%%%%%%%%%%%%%%%%%%%%%%%%%%%%%%%%%%%%%%%%%%%
\section{Pauli principle}


Based on the discussion above, we can naturally arrive at the Pauli
Principle:\textbf{There's no two fermions occupy the same single
particle state.}

From the expression of (\ref{IDENTICALeq:4}), we can see that if two
fermions of i and j take the same quantum state of $\psi$, the
determinant mast be $0$. So that leads to the Pauli Principle.

In quantum mechanics, the difference lies in the $\psi$ is labeled
by spin state. More discussion here can see the HF chapter.




%%% Local Variables:
%%% mode: latex
%%% TeX-master: "../../main"
%%% End:
