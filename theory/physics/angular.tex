%
% revised at Jan 21th, 2009
% does not change too much on the content, only get
% mild modification on the multi-system discussion. However,
% this part is still not very satisfied. I do not think it
% has been clearly discussed.
%

%%%%%%%%%%%%%%%%%%%%%%%%%%%%%%%%%%%%%%%%%%%%%%%%%%%%%%%%%%%%%%%%%%%%%%%%%%%%%%%%%%%%%%%%%%%%%%%%%%%%%%%%%%%%%%%%%%%%%
% outline for this chapter:
%
%  1  the angular operator; expression, some concrete characters and relation with H
%  2  the eigen value and eigen function of angular operator
%  3  angular operator in multi-particle system
%
%%%%%%%%%%%%%%%%%%%%%%%%%%%%%%%%%%%%%%%%%%%%%%%%%%%%%%%%%%%%%%%%%%%%%%%%%%%%%%%%%%%%%%%%%%%%%%%%%%%%%%%%%%%%%%%%%%%%%
\chapter{Angular Momentum}

%%%%%%%%%%%%%%%%%%%%%%%%%%%%%%%%%%%%%%%%%%%%%%%%%%%%%%%%%%%%%%%%%%%%%%%%%%%%%%%%%%%%%%%%%%%%%%%%%%%%%%%%%%%%%%%%%%%%%
\section{Definition of angular momentum operator}
%
%  1  the expression of  l, l^{2} and l_{x},l_{y},l_{z},l_{+},l_{z}
%  2  hermite character
%  3  commute relationship
%  4  expression under sphere coordinates


In classical mechanics, the angular momentum is defined as
$\overrightarrow{l} = \overrightarrow{r}\times\overrightarrow{p}$,
in quantum mechanics; the angular momentum operator is defined as
$\hei{l} = \hei{r}\times\hei{p}$. So we have:
\begin{equation}\label{}
\begin{aligned}
         \hat{l}_{x} &= y\hat{p}_{z} -  z\hat{p}_{y} \nonumber \\
         \hat{l}_{y} &= z\hat{p}_{x} -  x\hat{p}_{z} \nonumber \\
         \hat{l}_{z} &= x\hat{p}_{y} -  y\hat{p}_{x}
 \end{aligned}
\end{equation}

First we can prove that \heit{l} is hermite. Actually if the
$\hat{l}_{x}, \hat{l}_{y}$ and $\hat{l}_{z}$ are hermite, the
\heit{l} is hermite.

For $\hat{l}_{\alpha} = x_{\beta}\hat{p}_{\gamma} -
x_{\gamma}\hat{p}_{\beta}$; we have:
\begin{align}\label{}
\hat{l}_{\alpha}^{+} &= \{x_{\beta}\hat{p}_{\gamma}\}^{+} -
\{x_{\gamma}\hat{p}_{\beta}\}^{+} \nonumber \\
&=\hat{p}_{\gamma}x_{\beta} - \hat{p}_{\beta}x_{\gamma} \nonumber \\
&=x_{\beta}\hat{p}_{\gamma} - x_{\gamma}\hat{p}_{\beta} \nonumber \\
&=\hat{l}_{\alpha}
\end{align}
Here we note that $[x_{\alpha}, p_{\beta}] =
-i\hbar\delta_{\alpha\beta}$. Therefore, the \heit{l} is hermite.

Then we concentrate on the commutative relationship between
$\hat{l}_{\alpha}$ and $x_{\beta}$:
\begin{align}\label{}
[\hat{l}_{x}, x] &= 0  &  [\hat{l}_{x}, y] &= i\hbar z  &
[\hat{l}_{x}, z] &= -i\hbar y  \nonumber \\
[\hat{l}_{y}, x] &= -i\hbar z  &  [\hat{l}_{y}, y] &= 0  &
[\hat{l}_{y}, z] &=  i\hbar x  \nonumber \\
[\hat{l}_{z}, x] &= i\hbar y  &  [\hat{l}_{z}, y] &= -i\hbar x  &
[\hat{l}_{z}, z] &= 0  \nonumber \\
\end{align}
In short, we use the formula below to totally express the above
equations:
\begin{equation}\label{}
[\hat{l}_{\alpha}, x_{\beta}] =
i\hbar\varepsilon_{\alpha\beta\gamma} x_{\gamma}
\end{equation}
$\varepsilon_{\alpha\beta\gamma}$ is a three rank antisymmetric unit
tensor, within it if two index number are changed from the order of
$\alpha, \beta, \gamma$ it equals to -1; if changed twice it equals
to $+1$. Thus this is similar to the sign changing in the
determinant.

For example, for the tensor we have:
\begin{equation}\label{}
\varepsilon_{\alpha\beta\gamma} = 1, \quad
\varepsilon_{\alpha\gamma\beta} =\varepsilon_{\beta\alpha\gamma}
=\varepsilon_{\gamma\beta\alpha}= -1, \quad
\varepsilon_{\gamma\alpha\beta}= 1
\end{equation}

For the commutation between $\hat{l}_{\alpha}$ and
$\hat{p}_{\beta}$; we also have similar expressions:
\begin{align}\label{}
[\hat{l}_{\alpha}, \hat{p}_{\beta}] &=
i\hbar\varepsilon_{\alpha\beta\gamma} \hat{p}_{\gamma} \nonumber \\
[\hat{l}_{\alpha}, \hat{l}_{\beta}] &=
i\hbar\varepsilon_{\alpha\beta\gamma} \hat{l}_{\gamma}
\end{align}

For the expression above, we can see that $\hat{l}_{x}, \hat{l}_{y},
\hat{l}_{z}$ can not commuted with each other, thus they can not
share the same eigen states. This is an important extinguishment
between angular momentum and momentum, where the $\hat{p}_{x},
\hat{p}_{y}, \hat{p}_{z}$ could commute with each other.

On the other hand, we can see that for some specific systems (which
includes all the quantum system in quantum chemistry, so that's
enough) the \heit{l} commutes with the $\hat{H}$. Since that we can
write the $\hat{H}$ as the sum of $\hat{T}$ and $\hat{V}(r)$; they
can be taken into account respectively.
\begin{align}\label{}
[\hat{V}(r), \hei{l}] &= [\hat{V}(r), \hei{r}\times\hei{p}] \nonumber \\
&=[\hat{V}(r), \hei{r}]\times\hei{p} + \hei{r}\times[\hat{V}(r), \hei{p}] \nonumber \\
&=\hei{r}\times[\hat{V}(r), \hei{p}]
\end{align}
If $\hat{V}(r)$ is some polynomial function of $\hat{R}$ (such as in
molecule Hamiltonian, the $\hat{V}(r) = \hat{R}^{-1}$), the
$[\hat{V}(r), \hei{p}]$ is some function of $\hat{R}$ and $\hei{r}$
according to the (\ref{OPERATORMOREeq:8}); thus this term is $0$.

On the other hand, for the kinetic operator of $\hat{T}$:
\begin{align}\label{}
[\hat{T}, \hei{l}] &= [\hat{T}, \hei{r}\times\hei{p}] \nonumber \\
&=[\hat{T}, \hei{r}]\times\hei{p} + \hei{r}\times[\hat{T}, \hei{p}] \nonumber \\
&=[\hat{T}, \hei{r}]\times\hei{p}
\end{align}
From the (\ref{OPERATORMOREeq:9}) we have known that $[\hei{r},
\hei{p}\cdot\hei{p}] =-2i\hbar\hei{p}$, thus this term is $0$.

Consequently, the angular momentum is able to commute with the
$\hat{H}$.

On the other hand, it's easy to get this only from physical guess.
Since the angular momentum describes the rotation states of the
quantum particle, while if the rotation does not affect the overall
energy state, it can be expected that the Hamiltonian operator
commutes with the angular momentum operator.

 Similar to the kinetic operator, we can define the operator
of the square of the modulus of the angular momentum vector, that
is:
\begin{equation}\label{}
\hat{l}^{2} = \hei{l}\cdot\hei{l} = \hat{l}^{2}_{x} +
\hat{l}^{2}_{y} + \hat{l}^{2}_{z}
\end{equation}

It's easy to see that $\hat{l}^{2}$ is a hermite operator. For the
$\hat{l}^{2}$, it can prove that:
\begin{equation}\label{}
[\hat{l}^{2}, \hat{l}_{\alpha}] = 0 \quad \text{$\alpha = x, y, z$}
\end{equation}
Since for any arbitrary $\alpha$ and $\beta$, we have:
\begin{eqnarray}
% \nonumber to remove numbering (before each equation)
  [\hat{l}_{\alpha}, \hat{l}^{2}_{\beta}] &=&  \hat{l}_{\beta}[\hat{l}_{\alpha}, \hat{l}_{\beta}] + [\hat{l}_{\alpha},
  \hat{l}_{\beta}]\hat{l}_{\beta} \nonumber \\
   &=& i\hbar\varepsilon_{\alpha\beta\gamma}\hat{l}_{\beta} \hat{l}_{\gamma}  + i\hbar
   \varepsilon_{\alpha\beta\gamma}\hat{l}_{\gamma}\hat{l}_{\beta} \nonumber \\
\end{eqnarray}
If we have $\alpha = x$, as the $\beta = y, z$, it has:
\begin{align}\label{}
  [\hat{l}_{x}, \hat{l}^{2}_{y}] &= i\hbar\hat{l}_{y} \hat{l}_{z}  + i\hbar
   \hat{l}_{z}\hat{l}_{y} \nonumber \\
  [\hat{l}_{x}, \hat{l}^{2}_{z}] &= -i\hbar\hat{l}_{z} \hat{l}_{y}  - i\hbar
   \hat{l}_{y}\hat{l}_{z}
\end{align}

Therefore, we can know that:
\begin{equation}\label{}
[\hat{l}_{x}, \hat{l}^{2}] = [\hat{l}_{x}, \hat{l}^{2}_{y}] +
[\hat{l}_{x}, \hat{l}^{2}_{z}] + [\hat{l}_{x}, \hat{l}^{2}_{x}] = 0
\end{equation}

Similarly, we know that $[\hat{l}^{2}, \hat{l}_{\alpha}] = 0$.

on the other hand, it can see that if $[\hei{l}, \hat{H}] = 0$, the
$[\hat{l}^{2}, \hat{H}] = 0$.
\begin{align}\label{}
[\hat{H}, \hat{l}^{2}] &= [\hat{H}, \hei{l}^{2}] \nonumber \\
&=[\hat{H}, \hei{l}]\cdot\hei{l} + \hei{l}\cdot[\hat{H}, \hei{l}]
\end{align}

Thus, within the quantum chemistry; we can arrive two important
conclusions:
\begin{align}\label{}
[\hat{H}, \hei{l}] &= 0 \nonumber \\
[\hat{H}, \hat{l}^{2}] &= 0
\end{align}

Next, let's introduce another two operators based on the $\hei{l}$:
\begin{align}\label{}
\hat{l}_{+} &= \hat{l}_{x} + i\hat{l}_{y}  \nonumber \\
\hat{l}_{-} &= \hat{l}_{x} - i\hat{l}_{y}
\end{align}
For both of the new operators, it's easy to prove the commutative
relationship for them:
\begin{equation}\label{}
[\hat{l}_{+}, \hat{l}_{-}] = 2\hbar\hat{l}_{z}, \quad [\hat{l}_{z},
\hat{l}_{+}] = \hbar\hat{l}_{+}, \quad [\hat{l}_{z}, \hat{l}_{-}] =
-\hbar\hat{l}_{-}
\end{equation}

Furthermore, we can show that:
\begin{equation}\label{ANGULAReq:3}
\hat{l}^{2} = \hat{l}_{+}\hat{l}_{-} + \hat{l}^{2}_{z} - \hat{l}_{z}
=\hat{l}_{-}\hat{l}_{+} + \hat{l}^{2}_{z} + \hat{l}_{z}
\end{equation}

In the following content, we can see that why we introduce both of
the two new operators.

Lastly, since that we always use the spheral coordinates to
represent the angular momentum operator, its form is listed below:
\begin{align}\label{ANGULAReq:1}
\hat{l}_{x} &=i\hbar\left\{\sin\varphi \frac{\partial}{\partial
\theta} + \cot\theta\cos\varphi \frac{\partial}{\partial\varphi} \right\}       \nonumber \\
\hat{l}_{y} &=i\hbar\left\{-\cos\varphi \frac{\partial}{\partial
\theta} + \cot\theta\sin\varphi \frac{\partial}{\partial\varphi}
\right\} \nonumber \\
\hat{l}_{z} &= -i\hbar
\frac{\partial}{\partial\varphi}
\end{align}

\begin{equation}\label{ANGULAReq:2}
\hat{l}^{2}
=-\hbar^{2}\left\{\frac{1}{\sin\theta}\frac{\partial}{\partial\theta}
\left(\sin\theta\frac{\partial}{\partial\theta}\right)
+\frac{1}{\sin^{2}\theta}\frac{\partial^{2}}{\partial\varphi^{2}}
\right\}
\end{equation}

To get the result has to change from the Cartesian coordinate to the
spheral coordinates. Thus we have to use some basic differential
relationships. For the $\frac{\partial}{\partial x}$, we have:
\begin{equation}\label{}
\frac{\partial}{\partial x} = \frac{\partial}{\partial
r}\frac{\partial r}{\partial x} +
\frac{\partial}{\partial\theta}\frac{\partial\theta}{\partial x} +
\frac{\partial}{\partial\varphi}\frac{\partial\varphi}{\partial x}
\end{equation}
The $\frac{\partial}{\partial y}$ and $\frac{\partial}{\partial z}$
are similar to the expression above. By inserting the above
transformation into the $\hat{l}_{\alpha}$, we can immediately get
the (\ref{ANGULAReq:1}).

However, if we adopt the same procedure to deduce the
$\frac{\partial^{2}}{\partial x^{2}}$, it will be very complicated.
However, we can use the $\hat{l}_{+}$ and $\hat{l}_{+}$ to avoid
such trouble.

Directly from the (\ref{ANGULAReq:1}), we can get the expression for
the $\hat{l}_{+}$ and $\hat{l}_{+}$:
\begin{equation}\label{}
\hat{l}_{\pm} = e^{\pm i\varphi}\left\{\pm \frac{\partial}{\partial
\theta} + i\cot\theta \frac{\partial}{\partial\varphi} \right\}
\end{equation}
Thus from the (\ref{ANGULAReq:3}) we can easily get the expression
for the $\hat{l}^{2}$ as shown in (\ref{ANGULAReq:2}).

%%%%%%%%%%%%%%%%%%%%%%%%%%%%%%%%%%%%%%%%%%%%%%%%%%%%%%%%%%%%%%%%%%%%%%%%%%%%%%%%%%%%%%%%%%%%%%%%%%%%%%%%%%%%%%%%%%%%%
\section{Eigen value of angular momentum operator}
%
% eigen value for l^{2} and l_{z}
% since the CSCCO has pickup the l^{2} and l_{z}, thus it's
%  impossible to discuss the the eigen value for l_{x} and l_{y}.
%  but we can talk about their matrix elements
%
Here in this section we begin to concentrate on the eigen value of
the angular momentum operator. There has some classical methods
tackle down this problem, so we will follow these methods.

First, let's concentrate on the eigen value of $\hat{l}_{z}$. Since
that $\hat{l}_{z}$ has a simple expression which is shown in
(\ref{ANGULAReq:1}), also the $\hat{l}_{z}$ can commute with
$\hat{H}$; it's picked up into the CSCCO for $\hat{H}$. Thus its
eigen function is:
\begin{align}\label{}
-i\hbar\frac{\partial \Psi}{\partial\varphi} &= l_{z}\Psi
\Rightarrow
\nonumber \\
\frac{\partial \ln\Psi}{\partial\varphi} &= \frac{il_{z}}{\hbar}
\Rightarrow
\nonumber \\
\Psi &= C exp(il_{z}\varphi/\hbar)
\end{align}

Now we have to determine the $l_{z}$ and $C$. $C$ can be fixed by
normalization condition; yet for the $l_{z}$, to ensure that
$\hat{l}_{z}$ is a hermite operator (see the book by JinYan
Zeng(Volume I)\cite{ZengJinYan} to get more details), the eigen
state of $\Psi$ has to satisfy: $\Psi(\varphi + 2\pi) =
\Psi(\varphi)$. Therefore the $l_{z}$ has to be:
\begin{equation}\label{}
\frac{l_{z}}{\hbar} = m \quad (m=0, \pm 1, \pm 2, \cdots)
\end{equation}
Thus the eigen state for the $\hat{l}_{z}$ is:
\begin{align}\label{ANGULAReq:6}
\Psi &= C exp(im\varphi) \nonumber \\
&= \frac{1}{\sqrt{2 \pi}} exp(im\varphi)
\end{align}

Here there's something so interesting that need to elucidate.
Although judging from the normal experience that there should has no
difference between $\hat{l}_{x}$, $\hat{l}_{y}$ and $\hat{l}_{z}$;
therefore the $\hat{l}_{y}$ and $\hat{l}_{x}$ should be able to get
their corresponding eigen states. However, there's an important
point we have previously made; that the $\hat{l}_{x}$, $\hat{l}_{y}$
and $\hat{l}_{z}$ can not commute with each other. Thus, if the
$\hat{l}_{z}$ is picked out into the complete set of dynamic
operators for $\hat{H}$, we can not use this Hilbert space to
compose the eigen states for $\hat{l}_{y}$ and $\hat{l}_{z}$.

Secondly, we are going to turn to the $\hat{l}^{2}$. Since that
$\hat{l}^{2}$ also commutes with $\hat{H}$, so it's selected into
the complete set of dynamic operators for $\hat{H}$, just like
$\hat{l}_{z}$.

However, because the expression of the $\hat{l}^{2}$ (see the
\ref{ANGULAReq:2}) is very complicated, it's hard to get the eigen
values directly from solving the eigen function like the procedure
done to the $\hat{l}_{z}$. Nevertheless, from the expression of
(\ref{ANGULAReq:3}), we can see that $\hat{l}^{2}$ can de directly
related to the $\hat{l}_{z}$ and $\hat{l}_{z}^{2}$, while their
eigen values are easily gained. Therefore, can we find out a way to
achieve the eigen value but avoid to solve the eigen function of
$\hat{l}^{2}$?

Here, to get through the answer we must use the operator of
$\hat{l}_{+}$ and $\hat{l}_{-}$, and that's the reason why we
introduce both of the two operators.

Now let's go to see more details for $\hat{l}_{+}$ and
$\hat{l}_{-}$. First we suggest some wave functions of $\psi_{m}$
whose eigen values for $\hat{l}_{z}$ are labeled as $m$:
\begin{align}\label{}
\hat{l}_{z}\hat{l}_{+} &= \hat{l}_{+}\hat{l}_{z} + \hbar\hat{l}_{+}
\Rightarrow \nonumber \\
\hat{l}_{z}\hat{l}_{+}\psi_{m} &= (\hat{l}_{+}\hat{l}_{z} +
\hbar\hat{l}_{+})\psi_{m} \Rightarrow \nonumber \\
&=m\hbar\hat{l}_{+}\psi_{m} + \hbar\hat{l}_{+}\psi_{m} \nonumber \\
&=(m+1)\hbar\hat{l}_{+}\psi_{m}
\end{align}
Thus we can see that $\hat{l}_{+}\psi_{m}$ is also some eigen
function for the $\hat{l}_{z}$, while whose eigen value is
$(m+1)\hbar$. That means, after the operation on the $\hat{l}_{+}$,
the $\hat{l}_{+}\psi_{m}$ increase its $m$ value by $1$.

the $\hat{l}_{-}$ has likewise the same procedure to reach the
result:
\begin{equation}\label{}
\hat{l}_{z}\hat{l}_{-}\psi_{m} = (m-1)\hbar\hat{l}_{-}\psi_{m}
\end{equation}
Thus the $\hat{l}_{+}\psi_{m}$ decrease its $m$ value by $1$.

For this reason, the $\hat{l}_{+}$ is called ``up operator'', and
the $\hat{l}_{-}$ is called ``down operator''.

Now let's go to see the eigen values for the $\hat{l}^{2}$. Since
that $\hat{l}_{z}$ possesses no speciality among $\hat{l}_{x}$,
$\hat{l}_{y}$ and $\hat{l}_{z}$; thus for a common system, different
$l_{z}$ value for $\hat{l}_{z}$ may correspond to the same eigen
value for $\hat{l}^{2}$. This is similar to the case of the three
dimensional wave function in the cubic box (see the previous
chapter), where whose wave function is:
\begin{align}\label{}
\Psi (x,y,z) &=
\frac{\sqrt{8}}{l^{3}}\sin\left(\frac{n_{1}\pi}{l}x\right)
\sin\left(\frac{n_{2}\pi}{l}y\right)
\sin\left(\frac{n_{3}\pi}{l}z\right) \nonumber \\
E (n_{1},n_{2},n_{3}) &=
\frac{\pi^{2}\hbar^{2}}{2ml^{2}}(n_{1}^{2}+n_{2}^{2}+n_{3}^{2})
\end{align}
In this case there's also no speciality between x, y and z, so for
the $n^{2} = n_{1}^{2}+n_{2}^{2}+n_{3}^{2}$; different $n_{i}$ may
correspond to the same $n^{2}$.

Now let's consider a system, whose wave function are the eigen
functions both for the $\hat{l}_{z}$ and $\hat{l}^{2}$.  Now we can
see that if the eigen value for the $\hat{l}^{2}$ is fixed up, there
must have maximum number of $\overline{m}$: $l_{z} =
\overline{m}\hbar$. Because that $\hat{l}^{2}-\hat{l}^{2}_{z} =
\hat{l}^{2}_{y} + \hat{l}^{2}_{x}$, the eigen values for the
$\hat{l}^{2}$ should always larger than (or at least equal to)
$\overline{m}^{2}$. Similarly there must have a minimum number of
$\underline{m}$, $l_{z} = \underline{m}\hbar$. Here, we do not know
whether they equal to each other, but actually their absolute value
should equal to each other; since that both of the two directions
for the $\hat{l}_{z}$ are identical.

Suggest that as the eigen value of $\hat{l}^{2}$ is fixed up to
$l^{2}$, and the eigen state of $\psi$ gives $\overline{m}$
($\hat{l}_{z}\psi = \overline{m}\hbar\psi$); here we have:
\begin{align}\label{}
\hat{l}_{+}\psi &= 0 \rightarrow \nonumber \\
\hat{l}_{-}\hat{l}_{+}\psi
&=(\hat{l}^{2}-\hat{l}_{z}^{2}-\hat{l}_{z})\psi= 0 \rightarrow \nonumber \\
l^{2} &= \overline{m}(\overline{m}+1)
\end{align}

Similarly for the $\psi^{'}$ which gives $\underline{m}$
($\hat{l}_{z}\psi^{'} = \underline{m}\hbar\psi^{'}$); we have:
\begin{align}\label{}
\hat{l}_{-}\psi^{'} &= 0 \rightarrow \nonumber \\
\hat{l}_{+}\hat{l}_{-}\psi^{'}
&=(\hat{l}^{2}-\hat{l}_{z}^{2}+\hat{l}_{z})\psi^{'}= 0 \rightarrow \nonumber \\
l^{2} &= \underline{m}(\underline{m}-1)
\end{align}

Thus we have $\overline{m}(\overline{m}+1) =
\underline{m}(\underline{m}-1)$. This yields that $\overline{m} =-
\underline{m}$, another solution is abandoned.

Anyway, for a specific eigen value of $l$, $l^{2} =
\overline{m}(\overline{m}+1)$; the eigen function corresponding to
the $\hat{l}_{z}$ are $2\overline{m}+1$ folds of degenerate; their
$m$ can adopt $m=-\overline{m}, -\overline{m}+1, \cdots, 0, 1,
\cdots, \overline{m}$.

Next let's consider the matrix elements for the $\hat{l}_{x}$ and
$\hat{l}_{y}$ for the same system, where we choose the
$(\hat{H},\hat{l}_{z})$ as the CSCCO. Therefore, the wave functions
$\psi_{1}, \psi_{2}, \cdots, \psi_{n}$ are the eigen vectors for
them $(\hat{H},\hat{l}_{z})$.

First for the different $\psi$ and $\psi^{'}$, which hold different
energy level, since $\hat{l}_{x}$ commuted with the Hamiltonian
operator, it's easy to see that $\bra{\psi^{'}}\hat{l}_{x}\ket{\psi}
= 0$ ($\hat{l}_{y}$ is same).
\begin{align}\label{}
\bra{\psi^{'}}\hat{H}\hat{l}_{x}\ket{\psi} &=
\bra{\psi^{'}}\hat{l}_{x}\hat{H}\ket{\psi} \nonumber \\
&=E\bra{\psi^{'}}\hat{l}_{x}\ket{\psi}
\end{align}
While on the other hand, we have:
\begin{align}\label{}
\bra{\psi^{'}}\hat{H}\hat{l}_{x}\ket{\psi} &=
\bra{\psi}\hat{H}\hat{l}_{x}\ket{\psi^{'}} \nonumber \\
&=\bra{\psi}\hat{l}_{x}\hat{H}\ket{\psi^{'}} \nonumber \\
&=E^{'}\bra{\psi}\hat{l}_{x}\ket{\psi^{'}} \nonumber \\
&=E^{'}\bra{\psi^{'}}\hat{l}_{x}\ket{\psi}
\end{align}
Thus we can have that :
\begin{equation}\label{}
(E-E^{'})\bra{\psi^{'}}\hat{l}_{x}\ket{\psi} = 0 \Rightarrow
\bra{\psi^{'}}\hat{l}_{x}\ket{\psi} = 0
\end{equation}

Hence, we will consider the wave functions which occupy the same
energy level and have different z directional angular momentum
($l_{z}(m)$ is different).

Now we will achieve this through $\hat{l}_{+}$ and $\hat{l}_{-}$.
Since $\hat{l}_{+}\ket{\psi_{m}} = c\ket{\psi_{m+1}}$, and
$\hat{l}_{-}\ket{\psi_{m}} = c\ket{\psi_{m-1}}$; only these
$\bra{\psi_{m}}\hat{l}_{+}\ket{\psi_{m-1}}$ and
$\bra{\psi_{m}}\hat{l}_{-}\ket{\psi_{m+1}}$ do not equal to $0$.

So far we have:
\begin{equation}\label{}
\hat{l}^{2} = \hat{l}_{+}\hat{l}_{-} + \hat{l}^{2}_{z} - \hat{l}_{z}
\end{equation}
we can multiply the operator with $\ket{\psi_{m}}$, whose eigen
value for $\hat{l}^{2}$ is $l(l+1)$, and $l_{z}$ is $m$; we
integrate it:
\begin{align}\label{}
\bra{\psi_{m}}\hat{l}^{2}\ket{\psi_{m}} &= \sum_{i}^{n}
\bra{\psi_{m}}\hat{l}_{+}\ket{\psi_{i}}
\bra{\psi_{i}}\hat{l}_{-}\ket{\psi_{m}} +
\bra{\psi_{m}}\hat{l}^{2}_{z}\ket{\psi_{m}} -
\bra{\psi_{m}}\hat{l}_{z}\ket{\psi_{m}} \nonumber \\
l(l+1) &= \bra{\psi_{m}}\hat{l}_{+}\ket{\psi_{m-1}}
\bra{\psi_{m-1}}\hat{l}_{-}\ket{\psi_{m}} + m^{2} -m
\end{align}
Since we have $\hat{l}_{+} = \hat{l}_{-}^{+}$, we have:
\begin{equation}\label{}
\bra{\psi_{m}}\hat{l}_{+}\ket{\psi_{m-1}} =
\bra{\psi_{m-1}}\hat{l}_{-}\ket{\psi_{m}}^{*}
\end{equation}
Hence we have:
\begin{equation}\label{}
|\bra{\psi_{m}}\hat{l}_{+}\ket{\psi_{m-1}}|^{2} = l(l+1)-m^{2}+m
\end{equation}
$\bra{\psi_{m}}\hat{l}_{+}\ket{\psi_{m-1}}= \sqrt{(l+m)(l-m+1)}$.

Since that $\hat{l}_{+} = \hat{l}_{x} + i\hat{l}_{y}$, and
$\hat{l}_{-} = \hat{l}_{x} - i\hat{l}_{y}$; now we can get:
\begin{align}\label{}
\bra{\psi_{m}}\hat{l}_{x}\ket{\psi_{m-1}} &=
\frac{1}{2}\sqrt{(l+m)(l-m+1)} \nonumber \\
\bra{\psi_{m}}\hat{l}_{y}\ket{\psi_{m-1}} &=
-\frac{1}{2}i\sqrt{(l+m)(l-m+1)}
\end{align}





%%%%%%%%%%%%%%%%%%%%%%%%%%%%%%%%%%%%%%%%%%%%%%%%%%%%%%%%%%%%%%%%%%%%%%%%%%%%%%%%%%%%%%%%%%%%%%%%%%%%%%%%%%%%%%%%%%%%%
\section{Eigen functions for the $\hat{l}_{z}$ and $\hat{l}^{2}$}
%
% Eigen functions for the $\hat{l}_{z}$ and $\hat{l}^{2}$
%
%
%

In the last section, the eigen functions for the $\hat{l}_{z}$ has
been easily got:
\begin{align}\label{}
-i\hbar\frac{\partial \Psi}{\partial\varphi} &= l_{z}\Psi
\Rightarrow \nonumber
\\
\Psi &= C exp(il_{z}\varphi/\hbar)
\end{align}
Now we begin to derive the eigen function for the $\hat{l}^{2}$.

For the $\hat{l}^{2}$(whose expression under sphere coordinate can
see \ref{ANGULAReq:2}), we can safely presume that it's eigen
function is $Y(\theta, \varphi)$($\hat{l}^{2}$ is only function of
$\theta$ and  $\varphi$); furthermore, the $Y(\theta, \varphi)$ can
be split into:
\begin{equation}\label{ANGULAReq:5}
Y(\theta, \varphi) = \Theta (\theta)\Phi_{m}(\varphi)
\end{equation}
We can have this because judging from (\ref{ANGULAReq:2}) the
variables of $\theta$ and  $\varphi$ can be separated from each
other. Here the $\Phi_{m}(\varphi)$ adopts the form in
(\ref{ANGULAReq:6}).

Moreover, suggest we have such the eigen functions for the
$\hat{l}^{2}$:
\begin{equation}\label{ANGULAReq:4}
\hat{l}^{2}Y(\theta, \varphi) = \lambda \hbar^{2}Y(\theta, \varphi)
\end{equation}
Now by bring the sphere coordinate expression of the $\hat{l}^{2}$
and (\ref{ANGULAReq:5}) into (\ref{ANGULAReq:4}), we can have:
\begin{align}\label{}
-\Phi_{m}\times\frac{\hbar^{2}}{\sin\theta}\frac{d}{d
\theta}\left(\sin\theta\frac{d\Theta}{d \theta}\right) +
\frac{\Theta}{\sin^{2}\theta}(\hat{l}^{2}_{z}\Phi_{m}) &=
\lambda \hbar^{2}\Theta \Phi_{m} \nonumber \\
-\Phi_{m}\times\frac{\hbar^{2}}{\sin\theta}\frac{d}{d
\theta}\left(\sin\theta\frac{d\Theta}{d \theta}\right) +
\frac{\Theta}{\sin^{2}\theta}m^{2}\hbar^{2}\Phi_{m} &=
\lambda \hbar^{2}\Theta \Phi_{m} \nonumber \\
\frac{1}{\sin\theta}\frac{d}{d
\theta}\left(\sin\theta\frac{d\Theta}{d \theta}\right) +
\left(\lambda - \frac{m^{2}}{\sin^{2}\theta}\right)\Theta &= 0
\end{align}

However, this function is not easy to gain its solution. So we make
$\xi = \cos \theta$ (here we have $0\leq\theta\leq\pi$). Thus:
\begin{align}\label{}
\frac{d\Theta}{d \xi} = -\frac{1}{\sin \theta}\frac{d\Theta}{d
\theta}
\end{align}
Then we have:
\begin{align}\label{}
\frac{1}{\sin\theta}\frac{d}{d
\theta}\left(-\sin^{2}\theta\frac{d\Theta}{d \xi}\right) +
\left(\lambda - \frac{m^{2}}{(1-\xi^{2})}\right)\Theta &= 0
\nonumber \\
\frac{d}{d \xi}\left((1-\xi^{2})\frac{d\Theta}{d \xi}\right) +
\left(\lambda -
\frac{m^{2}}{(1-\xi^{2})}\right)\Theta &= 0 \nonumber \\
(1-\xi^{2})\frac{d^{2}\Theta}{d \xi^{2}} - 2\xi\frac{d\Theta}{d \xi}
+ \left(\lambda - \frac{m^{2}}{(1-\xi^{2})}\right)\Theta &= 0
\end{align}

This is the associated Legendre function. Under the condition of
$0\leq\theta\leq\pi$, the function has only one meaningful physical
solution, and this solution requires that $\lambda = l(l+1)$ and
$|m| \leq l$; here the $\lambda$ is the eigen value for the
$\hat{l}^{2}$, which is in agreement with the answer we found out in
the last section.

In this case, the solution is a polynomial:
\begin{equation}\label{}
P^{m}_{l}(\cos\theta) \quad |m| \leq l
\end{equation}
By normalized condition, we can finally get the expression for the
$\Theta$:
\begin{equation}\label{}
\Theta(\theta) =
(-1)^{m}\sqrt{\frac{2l+1}{2}\frac{(l-m)!}{(l+m)!}}P^{m}_{l}(\cos\theta)
\end{equation}

Therefore, the eigen function for the $\hat{l}^{2}$ is:
\begin{equation}\label{}
Y^{m}_{l}(\theta, \varphi) =
(-1)^{m}\sqrt{\frac{2l+1}{4\pi}\frac{(l-m)!}{(l+m)!}}P^{m}_{l}(\cos\theta)e^{im\pi}
\end{equation}

Here we note that different $Y^{m}_{l}$ are orthogonal with each
other:
\begin{equation}\label{}
\int_{0}^{2\pi}\int_{0}^{\pi}Y^{m^{'}}_{l^{'}}Y^{m}_{l}\sin\theta
d\theta d\varphi = \delta_{l^{'}l}\delta_{m^{'}m}
\end{equation}
This is important in understanding the orthogonal characters of
concrete wave functions.

%%%%%%%%%%%%%%%%%%%%%%%%%%%%%%%%%%%%%%%%%%%%%%%%%%%%%%%%%%%%%%%%%%%%%%%%%%%%%%%%%%%%%%%%%%%%%%%%%%%%%%%%%%%%%%%%%%%%%
\section{ Coupling between two angular momentums}
%
% 1  to choose the L^{2} and L_z into the CSCCO
%    because they can commute with the \hat{H}
%
%
Now let's consider a system, which is composed by two subsystems. If
there's no interactions between the two subsystems, the total
angular momentum $\hat{L}$ for the whole system will simply be the
addition between the two subsystems; each subsystem is a fully
conserved system:
\begin{equation}\label{ANGULAReq:7}
\hat{\mathbf{L}} = \hat{\mathbf{L}}_{1} + \hat{\mathbf{L}}_{2}
\end{equation}

On the other hand, if there exists some interactions between the two
subsystems, the (\ref{ANGULAReq:7}) can not be applied anymore;
that's because the angular momentum for each subsystem is not
conserved anymore.

However, the weight of $\hat{\mathbf{L}}$ on each specific
direction, namely the $\hat{\mathbf{L}}_{x}$, $\hat{\mathbf{L}}_{y}$
and $\hat{\mathbf{L}}_{z}$ can be still made certain:
\begin{align}\label{}
\hat{\mathbf{L}}_{x} &= \hat{\mathbf{L}}_{1x} +
\hat{\mathbf{L}}_{2x} \nonumber \\
\hat{\mathbf{L}}_{y} &= \hat{\mathbf{L}}_{1y} +
\hat{\mathbf{L}}_{2y} \nonumber \\
\hat{\mathbf{L}}_{z} &= \hat{\mathbf{L}}_{1z} +
\hat{\mathbf{L}}_{2z}
\end{align}
This relation is obvious because the angular momentum operator is
the vector operator, so it obeys the rules for the vectors. For this
reason, we can select the $\hat{\mathbf{L}}_{z}$ into the CSCCO.

On the other hand, even though the total angular momentum is not the
conserved physical quantity anymore, the $\hat{L}^{2}$ can still
commute with the Hamiltonian, thus we can select the $\hat{L}^{2}$
into the CSCCO, too:
\begin{equation}\label{}
\hat{L}^{2} = \hat{\mathbf{L}}_{x}^{2} + \hat{\mathbf{L}}_{y}^{2} +
\hat{\mathbf{L}}_{z}^{2}
\end{equation}

For the $\hat{\mathbf{L}}_{z}$, it's eigen value and eigen states
are straightforward. Here we remember that different subsystem has
its own $l_{z}$, and they do not affect the others:
\begin{equation}\label{}
[\hat{\mathbf{L}}_{1z}, \hat{\mathbf{L}}_{2z}] = 0 \Rightarrow L_{z}
= L_{z1} + L_{z2}
\end{equation}
Nevertheless, for the $\hat{L}^{2}$ there does not have such clear
relations. Therefore, we should solve this question in another way.

If we select the $\hat{L}_{1}^{2}$, $\hat{L}_{2}^{2}$,
$\hat{\mathbf{L}}_{1z}$ and $\hat{\mathbf{L}}_{2z}$ into the CSCCO,
(this can be done for that all of the four operators are commuting
with the Hamiltonian, and also they are commuting with each other),
any quantum state related to the angular momentum will be determined
by $L^{2}_{1},L^{2}_{2},L_{1z},L_{2z}$. Therefore for the fixed
value of $L_{1}, L_{2}$, they will be $(2L_{1}+1)(2L_{2}+1)$
different states.

On the other hand, we can express the quantum states by choosing
another representation, which is derived from the operators of
$\hat{L}^{2}_{1}$, $\hat{L}^{2}_{2}$, $\hat{L}^{2}$ and
$\hat{\mathbf{L}}_{z}$. Here the quantum states can be labeled by
$L^{2}_{1},L^{2}_{2},L^{2},L_{z}$. Similarly, for the fixed value of
$L_{1}, L_{2}$, they will be $(2L_{1}+1)(2L_{2}+1)$ different
states.

We can determine these states by the way below. By addition between
different $L_{1z}$ and $L_{2z}$, we can get the $L_{z}$:
\begin{center}\label{ANGULARTABLE:1}
\begin{tabular}{c c c}
  \hline
  % after \\: \hline or \cline{col1-col2} \cline{col3-col4} ...
  $L_{1z}$ & $L_{2z}$ & $L_{z}$ \\
  \hline
  $L_{1}$     & $L_{2 }$     & $L_{1}+L_{2}$ \\
  $L_{1}-1$   & $L_{2 }$     & $L_{1}+L_{2}-1$ \\
  $L_{1}$     & $L_{2 }-1$   & $L_{1}+L_{2}-1$ \\
  $L_{1}-2$   & $L_{2 }$     & $L_{1}+L_{2}-2$ \\
  $L_{1}-1$   & $L_{2 }-1$   & $L_{1}+L_{2}-2$ \\
  $L_{1}$     & $L_{2 }-2$   & $L_{1}+L_{2}-2$ \\
  $\cdots$    & $\cdots$     & $\cdots$ \\
  \hline
\end{tabular}
\end{center}

Here we can see that for same $L_{z}$, it's always corresponding to
many states. These states must have different $L$ value (since that
for a fixed $L$, its angular momentum on z axis must be different.
Hence there's no two states can share the same $L_{z}$ if they have
shared the same $L$), thus we can count the number of states as the
change of $L_{z}$. For example, as $L_{z} = L_{1}+L_{2}-2$, there
must have three states which possess different $L$, and because the
$L \geq L_{z}$; thus the three states must be $L = L_{1}+L_{2}$, $L
= L_{1}+L_{2}-1$ and $L = L_{1}+L_{2}-2$. Therefore, if the number
of states for the $L_{z}$ continues to grow, there implies it has
new states of different $L$.

Such continuing will be stopped at $L_{z} = |L_{1} - L_{2}|$. As the
$L_{z}$ decreases, the number of states do not grow. This implies
that the eigen value of $\hat{L}$ for the whole system will be:
\begin{equation}\label{}
L = L_{1}+L_{2}, L_{1}+L_{2}-1, L_{1}+L_{2}-2, \cdots, |L_{1} -
L_{2}|
\end{equation}
Here the $L_{1}$ and $L_{2}$ has to give to be certain.

This phenomenon can be well understood by the "vector model" of
angular momentum. Since that the angular momentum is some vector, as
the angular momentum in the subsystems take same direction, they
will give the maximum angular momentum addition for the whole; and
if they take opposite direction, they will give the minimum angular
momentum for the whole. This coincides with the conclusion above.



%%% Local Variables: 
%%% mode: latex
%%% TeX-master: "../../main"
%%% End: 
