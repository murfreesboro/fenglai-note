%%
%% symmetry.tex
%%
%% Made by liufenglai
%% Login   <liufenglai@phoenix>
%%
%% Started on  Wed Aug 19 20:37:30 2009 liufenglai
%% Last update Wed Aug 19 20:40:53 2009 liufenglai
%%
%% modified at Aug, 21th; 2009
%% add the content of Spatial transformation as operator
%% on quantum state
%%
%%
%%
%%
%%
%%
%%
%%
%%

%%%%%%%%%%%%%%%%%%%%%%%%%%%%%%%%%%%%%%%%%%%%%%%%%%%%%%%%%%%%%%%%%%%%%%%%%%%%%%%%%%%%%%
\chapter{The application of symmetry theory in quantum mechanics}
%
% scheme:
% 1  general discussion to the symmetry theory in quantum mechanics
% 2  what kinds of symmetry operation have in quantum mechanics?
%    mainly concentrating on the spatial transformation
% 3  consider the characters for the operator corresponding the
%    symmetry operation.
% 4  concentrating on the conserved property, several examples;
%    the translation, and the rotation
% 5  symmetry group for Hamiltonian
%
%%%%%%%%%%%%%%%%%%%%%%%%%%%%%%%%%%%%%%%%%%%%%%%%%%%%%%%%%%%%%%%%%%%%%%%%%%%%%%%%%%%%%%
\section{Introduction}
%
% why we have this subject here?
% concept of symmetry operation
%
Here in this chapter we are going to give some general discussions
for the application of symmetry theory in quantum mechanics.

This part of knowledge is very important. In quantum chemistry, as
we investigate the symmetrical molecules (for example, the water
molecule, and the carbon dioxide molecule etc.); it's symmetry
character is able to give us great help in calculation as well as
understanding its properties. For instance, it's symmetry character
determines the excitation energy profile, the transition states for
dipole moment etc.

On the other hand, the symmetry theory in quantum mechanics is also
very important. Interestingly to see, the application of symmetry
theory in quantum mechanics is much more important than it's in
classic mechanics. Why? In quantum mechanics, the particles belongs
to same type are unable to distinguish with each other so that we
can them as ``identical particles'' (see chapter
\ref{identical_particles}), hence; the symmetry operation such as
rotation etc. will make the quantum particle exactly same with the
one before rotation. Intuitively to say, such physical phenomenon
will correspond to some very important physical rules. In the
following content, we will give some detailed analysis on this
issue.

Here we can define a concept of ``symmetry operation''. The symmetry
operation is considered to be some action on some physical objects,
if the physical objects after the action retain to be same with its
state before the action, or in other words; the physical objects are
distinguishable with its initial states. In this circumstance, we
call the operation is ``symmetry operation''.

%%%%%%%%%%%%%%%%%%%%%%%%%%%%%%%%%%%%%%%%%%%%%%%%%%%%%%%%%%%%%%%%%%%%%%%%%%%%%%%%%%%%%%
\section{Types for symmetry operation}
%
%
%
%
%%%%%%%%%%%%%%%%%%%%%%%%%%%%%%%%%%%%%%%%%%%%%%%%%%%%%%%%%%%%%%%%%%%%%%%%%%%%%%%%%%%%%%
\subsection{General types of symmetry operation}
%
%   general types of symmetry operations
%
%
Fundamentally there are several symmetry operations in quantum
mechanics, they are depending on the basic variables in the quantum
states. Hence, the symmetry operation can be divided into three
types; namely the transformation on spatial variables and time
variables in the quantum state, and the transformation on the spin
variables in the quantum state.

However, the symmetry transformation for the time variables and spin
variables are rarely referred in quantum chemistry so that we do not
intend to give discussion to them (the reader can refer to the
material in \cite{XingLinKe}). Hence, the discussion will only
concentrate on the spatial transformation.


%%%%%%%%%%%%%%%%%%%%%%%%%%%%%%%%%%%%%%%%%%%%%%%%%%%%%%%%%%%%%%%%%%%%%%%%%%%%%%%%%%%%%%
\subsection{Spatial transformation}
%
% 1  general character for spatial transformation
% 2  it composes into the group
%
The symmetry spatial transformation on one position of $\bm{r}$ can
be expressed as:
\begin{equation}\label{SYMMETRY_IN_QMeq:1}
\bm{r^{'}} = Q\bm{r}
\end{equation}
Here the particle in position of $\bm{r}$ is transformed to the
location of $\bm{r^{'}}$. The ``symmetry'' character requires that
such transformation should not change the relative position for
particles, else the physical objects can not be same with the one
before the transformation.

Such inherent requirement leads to some important result, that the
spatial symmetry operations are enclosed into some ``group'' for the
given system.

Mathematically, the forming of group demands several conditions;
they are:
\begin{itemize}
  \item The elements in the group should have enclosed property.
  That is to say, the operation between the elements within the
  group should be equal to some element within the same group. Such
  relation can be express that if $A$, $B$ $\in$ group $\Omega$,
  then $AB$ must equal to some element of $C$ $\in$ $\Omega$.
  \item The operation satisfies the associative law. That is to say,
  $A(BC) = (AB)C$.
  \item In the group, the unit element of $I$ should always exist.
  \item Each element has an inverse element: $\forall$ $A$
  $\in$ $\Omega$, we have $A^{-1}A = AA^{-1} = I$.
\end{itemize}

Now let's go to analyze each item. Firstly, for any system, there
always has the ``identical transformation'' which makes the system
unchanged; it amounts to the unit element of $I$ in the group.

Second, since the symmetry operation does not change the relative
position that there always has inverse element of the symmetry
operation to restore the system. Hence the fourth item is satisfied.

third, the consecutive symmetry operations is still some symmetry
operation. So the second item is satisfied. Furthermore, such
operations is naturally satisfy the associative law; hence for any
system the spatial symmetry operations are composed into some group.


%%%%%%%%%%%%%%%%%%%%%%%%%%%%%%%%%%%%%%%%%%%%%%%%%%%%%%%%%%%%%%%%%%%%%%%%%%%%%%%%%%%%%%
\subsection{Spatial transformation as operator on quantum state}
%
% 1  the spatial transformation as operator on quantum state
%    it's unitary operator, and commute with any physical quantity
% 2  they compose into the group, homomorphism with the group
%    of symmetry operation
%
Mow let's consider some single particle wave function of
$\Psi(\bm{r})$, according to the (\ref{SYMMETRY_IN_QMeq:1}) and the
definition of symmetry operation we have:
\begin{equation}\label{}
\Psi^{'}(\bm{r^{'}}) = \Psi^{'}(Q\bm{r}) = \Psi(\bm{r})
\end{equation}
Here the transformation of $Q\bm{r}$ lead the wave function of
$\Psi$ to some new value so that we label it as $\Psi^{'}$.

Such transformation can be also expressed via the operator:
\begin{equation}\label{SYMMETRY_IN_QMeq:2}
\Psi^{'}(\bm{r^{'}}) = \hat{D}(Q)\Psi(\bm{r}) =
\Psi(Q^{-1}\bm{r^{'}})
\end{equation}
Now let's give some general discussion to the property of the
operator of $\hat{D}(Q)$.

First, because of the superposition principle requirement the
operator of $\hat{D}(Q)$ should be some linear operator:
\begin{equation}\label{}
\begin{split}
  \Psi &= \sum_{i}\Psi_{i}\Rightarrow \\
  \hat{D}(Q)\Psi &= \sum_{i}\hat{D}(Q)\Psi_{i} \\
\end{split}
\end{equation}

Second, Since the symmetry operation $Q$ makes the system exactly
the same with the one before the symmetry operation, thus for an
arbitrary quantum state of $\ket{\Psi}$. and $\ket{\Psi^{'}} =
\hat{D}(Q)\ket{\Psi}$; then we have:
\begin{equation}\label{SYMMETRY_IN_QMeq:3}
\begin{split}
\langle\Psi^{'}|\Psi^{'}\rangle &= \langle\Psi|\Psi\rangle \Rightarrow \\
\bra{\Psi}\hat{D}^{+}(Q)\hat{D}(Q)\ket{\Psi} &=
\langle\Psi|\Psi\rangle \\
\hat{D}^{+}(Q)\hat{D}(Q) &= I
\end{split}
\end{equation}
Here the $\hat{D}^{+}(Q)$ is the adjacent operator of $\hat{D}(Q)$
which is working on the bra of $\bra{\Psi}$. The relation in the
(\ref{SYMMETRY_IN_QMeq:3}) indicates that the operator $\hat{D}(Q)$
is unitary operator.

Furthermore, since the quantum state is unchanged after the symmetry
operation, hence for any physical quantity of $\hat{A}$, we have:
\begin{equation}\label{SYMMETRY_IN_QMeq:4}
\begin{split}
\langle\Psi^{'}|\hat{A}|\Psi^{'}\rangle &=
\langle\Psi|\hat{A}|\Psi\rangle \Rightarrow \\
\bra{\Psi}\hat{D}^{+}(Q)\hat{A}\hat{D}(Q)\ket{\Psi} &=
\langle\Psi|\hat{A}|\Psi\rangle \\
\hat{D}^{+}(Q)\hat{A}\hat{D}(Q) &= \hat{A} \\
\hat{A}\hat{D}(Q) &= \hat{D}(Q)\hat{A} \\
[\hat{A}, \hat{D}(Q)] &= 0
\end{split}
\end{equation}
Therefore, the $\hat{D}(Q)$ is able to commute with any physical
quantity; such character is very similar to the unitary
transformation between the representations ( see
\ref{transformation_in_representation} for more details).

Importantly, if the $\hat{A}$ is the Hamiltonian operator, then we
can see that the $\hat{D}(Q)$ can be included into the ``CSCCO'' so
that it's used to be identified as some ``freedom'' to determine the
quantum state for the system. In a sense, the symmetry character for
the quantum system is just one part of it's physical properties.

Finally, we can see that the operator of $\hat{D}(Q)$ also compose
some group, such group can be express as $\hat{D}(Q), \hat{D}(R),
\hat{D}(S), \hat{D}(T) \cdots$; where it changes as the symmetry
operation changes.

Mathematically, we can prove that such group composed by
$\hat{D}(Q)$ is homomorphism with the group of symmetry operation.
Such relation can be expressed as:
\begin{equation}\label{}
\begin{split}
  \hat{D}(Q)\hat{D}(R) &= \hat{D}(QR)  \\
  \hat{D}^{+}(Q) &= \hat{D}^{-1}(Q) = \hat{D}(Q^{-1})
\end{split}
\end{equation}
Hence, we can safely discuss the symmetry character via the group of
$\hat{D}(Q)$.


%%%%%%%%%%%%%%%%%%%%%%%%%%%%%%%%%%%%%%%%%%%%%%%%%%%%%%%%%%%%%%%%%%%%%%%%%%%%%%%%%%%%%%
\section{Conserved property for symmetry operation}
%
%
%
In quantum mechanics, if an operator is commuting with Hamiltonian
operator, then the physical quantity represented by this operator
does not vary as time alters. Such character is called ``conserved''
property.

In the above section, we have demonstrated that the operators
represented the symmetry operation is able to commute with
$\hat{H}$, so what's the conserved physical property behind them?
Next we are going to show several cases around this issue.


%%%%%%%%%%%%%%%%%%%%%%%%%%%%%%%%%%%%%%%%%%%%%%%%%%%%%%%%%%%%%%%%%%%%%%%%%%%%%%%%%%%%%%
\subsection{Translation invariance with the momentum conservation}
%
%
%
Suggest that we have some one dimensional system. It makes an
infinitesimal displacement:
\begin{equation}\label{}
x^{'} = x + \delta x
\end{equation}

Assume that the wave function has the translation invariance. Hence
according to the (\ref{SYMMETRY_IN_QMeq:2}) we can express such
invariance relation as:
\begin{equation}\label{}
\begin{split}
   \hat{D}\psi(x + \delta x) &= \psi(x) \Rightarrow \\
   \hat{D}\psi(x) &= \psi(x - \delta x)
\end{split}
\end{equation}

Now let's go to see under such translation invariance what kind of
physical property is conserved?

By using Taylor expansion, we can expand the $\psi(x - \delta x)$
as:
\begin{equation}\label{}
\begin{split}
  \psi(x - \delta x) &= \psi(x) -
\delta x\frac{\partial \psi}{\partial x} + \cdots \\
    &= e^{-\delta x \frac{\partial}{\partial x}} \psi(x) \\
    &= exp(\frac{-i \delta x \hat{p}_{x}}{\hbar}) \psi(x)
\end{split}
\end{equation}
Where in position representation, the momentum operator of
$\hat{p}_{x}$ is expressed as:
\begin{equation}\label{}
\hat{p}_{x} = -i\hbar\frac{\partial}{\partial x}
\end{equation}
Finally, we can express the operator of $\hat{D}$ as:
\begin{equation}\label{}
\hat{D} = exp(\frac{-i \delta x \hat{p}_{x}}{\hbar})
\end{equation}

It's very interesting to see that the expression for the $\hat{D}$
is same with the operator $\hat{q}(\xi)$ in the \ref{PRAMReq:2}.
Therefore, it's easy to know it's unitary character. Now since
$[\hat{D}, \hat{H}] = 0$, it finally turns out that we have:
\begin{equation}\label{}
[\hat{p}_{x}, \hat{H}] = 0
\end{equation}
It means the momentum is the conserved physical quantity for such
translation invariant system.



%%%%%%%%%%%%%%%%%%%%%%%%%%%%%%%%%%%%%%%%%%%%%%%%%%%%%%%%%%%%%%%%%%%%%%%%%%%%%%%%%%%%%%


%%%%%%%%%%%%%%%%%%%%%%%%%%%%%%%%%%%%%%%%%%%%%%%%%%%%%%%%%%%%%%%%%%%%%%%%%%%%%%%%%%%%%%



%%% Local Variables:
%%% mode: latex
%%% TeX-master: "../../main"
%%% End:
