%
% revise the whole content at Aug 10th, 2009
%
%
%

\chapter{More discussion on operator}
%
% this chapter used to describe more on the
% momentum operator and coordinate Operator; they are the foundations
% to build more complicated operators, so we also give some
% mathematical discussion to them. so that in the future, we can
% use them to derive more properties for the given operator
%
\section{Momentum operator and position operator}
%
%
%
In the above content, we have rigorously introduced the momentum
operator and position operator. Now we are going to give further
discussions for them.

As a kind of vector operator, the momentum operator and position
operator can be defined as:
\begin{eqnarray}
% \nonumber to remove numbering (before each equation)
  \mathbf{\hat{r}} &=& \hat{x}\overrightarrow{i} + \hat{y}\overrightarrow{j} + \hat{z}\overrightarrow{k} \nonumber \\
  \mathbf{\hat{p}} &=& \hat{p}_{x}\overrightarrow{i} + \hat{p}_{y}\overrightarrow{j} + \hat{p}_{z}\overrightarrow{k}
\end{eqnarray}

For the $\hat{x}, \hat{y}, \hat{z}$; in the position representation
according to the analysis in
(\ref{sec:PRAMR_in_position_representation}) we can simply express
them as:
\begin{align}\label{OPERATORMOREeq:14}
\hat{x} &= x \nonumber \\
\hat{y} &= y \nonumber \\
\hat{z} &= z
\end{align}

Accordingly, the x, y, z directions of the component for the
momentum operator $\hei{p}$ is defined as:
\begin{align}\label{OPERATORMOREeq:13}
\hat{P}_{x} &= -i\hbar\frac{\partial}{\partial x}   \nonumber \\
\hat{P}_{y} &= -i\hbar\frac{\partial}{\partial y}   \nonumber \\
\hat{P}_{z} &= -i\hbar\frac{\partial}{\partial z}
\end{align}

As for the commutation relationship, it's defined in the axiom
\ref{axiom5} and read as:
\begin{equation}\label{OPERATORMOREeq:12}
[\hat{x}_{i}, \hat{x}_{j}] = 0 \quad [\hat{P}_{i}, \hat{P}_{j}] = 0
\quad [\hat{x}_{i}, \hat{P}_{j}] = i\hbar\delta_{ij}
\end{equation}

However, we can also get this expression from
(\ref{OPERATORMOREeq:13}) and (\ref{OPERATORMOREeq:14}). For an
arbitrary $\Psi(x)$ we can see that:
\begin{eqnarray}
% \nonumber to remove numbering (before each equation)
  x\hat{P}_{x}\Psi(x) &=& -i\hbar x\frac{\partial\Psi(x)}{\partial x} \nonumber \\
  \hat{P}_{x}x\Psi(x) &=& -i\hbar\frac{\partial}{\partial x}(\Psi(x) x) \nonumber \\
                   &=& -i\hbar x\frac{\partial\Psi(x)}{\partial x}
                   -i\hbar\Psi(x)
\end{eqnarray}
Therefore we have:
\begin{equation}\label{}
[\hat{x}, \hat{P}_{x}]\Psi(x) = i\hbar\Psi(x) \rightarrow [\hat{x},
\hat{P}_{x}] = i\hbar
\end{equation}

Now let's introduce an operator $\hat{R}$, which is defined as:
\begin{equation}\label{}
\hat{R}^{2} = \hei{r}^{2} = x^{2} + y^{2} + z^{2} \Rightarrow
\hat{R} = \sqrt{x^{2} + y^{2} + z^{2}}
\end{equation}
The real Hamiltonian always refers to this scalar operator of
$\hat{R}$ rather than the $\hei{r}$.

On the other hand, we also use the inverse operator of $\hat{R}$
(represented by $\hat{R}^{-1}$):
\begin{equation}\label{}
\hat{R}^{-1} = \frac{1}{\sqrt{x^{2} + y^{2} + z^{2}}}
\end{equation}

In quantum chemistry, for an ordinary molecule system its
Hamiltonian only contains the kinetic operator of $\hat{T}$ and
$\hat{R}^{-1}$ (for the electrons and nuclear-electron terms); this
is clearly shown in (\ref{BASICeq:4}).

By the way, both of $\hei{p}$ and $\hei{r}$ are hermitian operators.
Specifically, in the position representation we have:
\begin{align}\label{OPERATORMOREeq:1}
\bra{\psi}\hei{r}\ket{\phi} &= \int\psi^{*}\bm{r}\phi d\tau   \nonumber \\
&= \int\bm{r}\psi^{*}\phi d\tau   \nonumber \\
&= \left\{\int\bm{r}\phi^{*}\psi d\tau \right \}^{*}   \nonumber \\
&= \bra{\phi}\hei{r}\ket{\psi}^{*}
\end{align}

What's more, it's natural to see that any function that only related
to the $\hei{r}$ or ($\hat{x}, \hat{y}, \hat{z}$) is the hermite
operator. Consequently, the potential operator of $\hat{V}(r)$ (for
example, the $\hat{R}$ and $\hat{R}^{-1}$)in the Hamiltonian
operator is hermite, since the momentum operator is hermitian, then
according to (\ref{hermitian_in_operator}) the kinetic operator
$\hat{T} = \frac{\hei{p}^{2}}{2m}$ is also hermitian; so the
Hamiltonian operator ($\hat{H} = \hat{T} + \hat{V}(r)$) itself is
hermitian.

%%%%%%%%%%%%%%%%%%%%%%%%%%%%%%%%%%%%%%%%%%%%%%%%%%%%%%%%%%%%%%%%%%%%%%%%%%%%%
\section{More discussion on the algorithm of operator}
%
% only directly give more relations to the algorithm of operator
% see below
%
%
%
The introduction to the momentum operator and position operator here
can be wound up for a while, now let's go to present more algorithms
on operators.

The dot product and the cross product for the vector operators are
defined as:
\begin{equation}\label{OPERATORMOREeq:2}
\hei{A}\cdot\hei{B} = \hat{A}_{x}\hat{B}_{x} +
\hat{A}_{y}\hat{B}_{y} + \hat{A}_{z}\hat{B}_{z}
\end{equation}
\begin{equation} \label{OPERATORMOREeq:3}
\hei{A}\times\hei{B} = \left\{ \begin{aligned}
         (\hei{A}\times\hei{B})_{x} &= \hat{A}_{y}\hat{B}_{z} -  \hat{A}_{z}\hat{B}_{y} \nonumber \\
         (\hei{A}\times\hei{B})_{y} &= \hat{A}_{z}\hat{B}_{x} -  \hat{A}_{x}\hat{B}_{z} \nonumber \\
         (\hei{A}\times\hei{B})_{z} &= \hat{A}_{x}\hat{B}_{y} -  \hat{A}_{y}\hat{B}_{x}
                          \end{aligned} \right.
\end{equation}

Here we can see that the cross product between operators are similar
to the case in classical mechanics.

The power for the operators is defined as:
\begin{equation}\label{}
\hat{A}^{n} = \underbrace{\hat{A}\cdot\hat{A}\cdots\hat{A}}_{n}
\end{equation}
Here we note that the definition of $\hat{R}^{2} = \hei{r}^{2}$ is
the dot product between $\hei{r}$, also the kinetic operator is
related to $\hat{P}^{2}$:
\begin{equation}\label{}
\frac{1}{2}\frac{\hat{P}^{2}}{2m} = \frac{\hbar^{2}}{2m}\left(
\frac{\partial^{2}}{\partial x^{2}} + \frac{\partial^{2}}{\partial
y^{2}} + \frac{\partial^{2}}{\partial z^{2}}  \right)
\end{equation}

Then we are going to focus on the commutative relationship, it's
easy to prove that for any of the arbitrary operators, the
expression below are correct(only use the definition of the
commutation):
\begin{align}\label{OPERATORMOREeq:4}
[\hat{A}, \hat{B}] &= -[\hat{B}, \hat{A}] \nonumber \\
[\hat{A}, \hat{A}] &= 0  \nonumber \\
[\hat{A}, \hat{B} \pm \hat{C}] &= [\hat{A}, \hat{B}] \pm  [\hat{A},
  \hat{C}] \nonumber \\
[\hat{A}, \hat{B}\hat{C}] &= \hat{B}[\hat{A}, \hat{C}] + [\hat{A},
  \hat{B}]\hat{C} \nonumber \\
[\hat{A}\hat{B}, \hat{C}] &= \hat{A}[\hat{B}, \hat{C}] + [\hat{A},
  \hat{C}]\hat{B} \nonumber \\
[\hat{A}, \hei{B}\cdot\hei{C}] &= [\hat{A}, \hei{B}]\cdot\hei{C} +
\hei{B}\cdot
  [\hat{A}, \hei{C}] \nonumber \\
[\hat{A}, \hei{B}\times\hei{C}] &= [\hat{A}, \hei{B}]\times\hei{C} +
\hei{B}\times
  [\hat{A}, \hei{C}]
\end{align}
These rules may be very useful when evaluating the characters of
some particular operator.

%%%%%%%%%%%%%%%%%%%%%%%%%%%%%%%%%%%%%%%%%%%%%%%%%%%%%%%%%%%%%%%%%%%%%%%%%%%%%%%
\section{Commutation relationship between $\hei{r}$ and $\hei{p}$}
%
% derive some important commutation relationship for the
% p and r so that in the future we can use them to prove
% more properties for the composite operators
%
%
Commutation is a kind of important subject in quantum mechanics.
Here in this section we focus on some commutation relationship
between $\hei{r}$ and $\hei{p}$, which is very useful in proving the
commutation relationship for composite operators. Here below for
convenience, the vector operator is expressed as:
\begin{equation}\label{}
\hei{A} = \sum_{i=1}^{3}\overrightarrow{e}_{i}\hat{A}_{i} =
\overrightarrow{e}_{1}\hat{A}_{1} +
\overrightarrow{e}_{2}\hat{A}_{2} +
\overrightarrow{e}_{3}\hat{A}_{3}
\end{equation}
For convenience, the $\overrightarrow{e}$ refers to $e$.

First let's consider the $[\hei{p}, \hat{R}^{2}]$.
\begin{align}\label{OPERATORMOREeq:6}
[\hei{p}, \hei{r}^{2}] &= [\sum e_{i}\hat{p}_{i}, \sum
e_{j}\hat{r}_{j} \cdot \sum e_{j}\hat{r}_{j}] \nonumber \\
&=\sum_{ij} e_{i}e_{j}[\hat{p}_{i}, \hat{r}_{j}^{2}] \nonumber \\
&=\sum_{ij} e_{i}e_{j}\left\{\hat{r}_{j}[\hat{p}_{i}, \hat{r}_{j}] +
[\hat{p}_{i}, \hat{r}_{j}]\hat{r}_{j} \right\} \nonumber \\
&=-\sum_{ij} e_{i}e_{j}2i\hbar\delta_{ij}r_{j} \nonumber \\
&=-\sum_{j} e_{j}2i\hbar r_{j} \nonumber \\
&=-2i\hbar\hei{r}
\end{align}

So far we can deduce the $[\hei{p}, \hat{R}^{2n}]$:
\begin{equation}\label{}
[\hei{p}, \hat{R}^{2n}] = \hat{R}^{2}[\hei{p}, \hat{R}^{2n-2}] +
[\hei{p}, \hat{R}^{2n-2}]\hat{R}^{2}
\end{equation}
This is a clear recursion, so we can get that $[\hei{p},
\hat{R}^{2n}] = -ni\hbar\hat{R}^{2n-2}\hei{r}$.

Next we are going to derive the $[\hei{p}, \hat{R}]$ from $[\hei{p},
\hat{R}^{2}]$.
\begin{align}\label{OPERATORMOREeq:7}
[\hei{p}, \hat{R}^{2}] &= \hat{R}[\hei{p}, \hat{R}]+ [\hei{p},
\hat{R}]\hat{R} \nonumber \\
-2i\hbar\hei{r} &= 2\hat{R}[\hei{p}, \hat{R}]
\end{align}
Thus the $[\hei{p}, \hat{R}] =i\hbar\frac{\hei{r}}{\hat{R}}$.

Here in (\ref{OPERATORMOREeq:7}), why $\hat{R}$ and $[\hei{p},
\hat{R}]$ commute with each other? Here we prove it. Since that we
have $[\hei{p},\hat{R}^{2}] = \hat{R}[\hei{p}, \hat{R}]+ [\hei{p},
\hat{R}]\hat{R}$, so the $[\hei{p},\hat{R}^{2}]$ directly related to
the $[\hei{p}, \hat{R}]$; thus we start from
$[\hei{p},\hat{R}^{2}]$:
\begin{align}\label{}
\left[\hat{R}, [\hei{p},\hat{R}^{2}]\right] &= \left[\hat{R},
\hat{R}[\hei{p},
\hat{R}]+ [\hei{p}, \hat{R}]\hat{R}\right] \nonumber \\
&=\left[\hat{R}, \hat{R}[\hei{p}, \hat{R}]\right]+ \left[\hat{R},
[\hei{p}, \hat{R}]\hat{R}\right]
\end{align}
By using the formula in (\ref{OPERATORMOREeq:4}) related to
$[\hat{A}, \hat{B}\hat{C}]$, we have the expression above
transforming into:
\begin{align}\label{}
\left[\hat{R}, [\hei{p},\hat{R}^{2}]\right] &=
\hat{R}\left[\hat{R},[\hei{p}, \hat{R}] \right] + [\hat{R},
\hat{R}][\hei{p}, \hat{R}] + [\hei{p}, \hat{R}][\hat{R}, \hat{R}] +
\left[\hat{R},[\hei{p}, \hat{R}] \right]\hat{R} \nonumber \\
&=\hat{R}\left[\hat{R},[\hei{p}, \hat{R}] \right]+
\left[\hat{R},[\hei{p}, \hat{R}] \right]\hat{R}
\end{align}
Here we can see the relationship that the $\hat{R}$ commutes with
$[\hei{p}, \hat{R}]$ is wholly determined by the $\left[\hat{R},
[\hei{p},\hat{R}^{2}]\right]$. However, since we have proved that
$[\hei{p},\hat{R}^{2}]$ equals to $-2i\hbar\hei{r}$, it commutes
with $\hat{R}$; thus $\hat{R}$ commutes with $[\hei{p}, \hat{R}]$.
The demonstration finished.

By induction, we can prove the general expression between
$\hat{R}^{n}$ and $\hei{p}$:
\begin{equation}\label{OPERATORMOREeq:8}
[\hei{p}, \hat{R}^{n}] = -ni\hbar\hat{R}^{n-2}\hei{r} \quad
(n=1,2,3, \cdots)
\end{equation}

For the $[\hei{p}, \hat{R}^{-n}]$, we can derive it from the
(\ref{OPERATORMOREeq:8}):
\begin{align}\label{}
[\hei{p}, 1] &= \left[\hei{p},
\hat{R}^{n}\frac{1}{\hat{R}^{n}}\right] \Rightarrow \nonumber \\
           0 &=\hat{R}^{n}\left[\hei{p}, \frac{1}{\hat{R}^{n}}\right] +
\left[\hei{p}, \hat{R}^{n}\right]\frac{1}{\hat{R}^{n}}
\end{align}
Therefore $[\hei{p}, \hat{R}^{-n}] = -ni\hbar\hat{R}^{n-2}\hei{r}$;
the $n$ ranges over all the negative number.

Here, we note that by the same procedure in the
(\ref{OPERATORMOREeq:6}), we can prove that:
\begin{equation}\label{OPERATORMOREeq:9}
[\hei{r}, \hei{p}\cdot\hei{p}] =-2i\hbar\hei{p}
\end{equation}
This is useful while considering the commutative relationship
between angular momentum operator and the Hamiltonian operator.

By the way, we will derive two useful relations which is needed in
the later content. Suggest that $f(\hei{r})$ is the function for the
$\hei{r}$, in the position representation we can see that:
\begin{align}\label{OPERATORMOREeq:10}
(\hei{p}f(\hei{r}) - f(\hei{r})\hei{p})\ket{\Psi} &=
-i\hbar\nabla(f(\hei{r})\ket{\Psi}) + i\hbar
f(\hei{r})(\nabla\ket{\Psi}) \quad \hei{p} = -i\hbar\nabla
\nonumber \\
&=-i\hbar\ket{\Psi}\nabla(f(\hei{r})) =
-i\hbar\ket{\Psi}\frac{\partial f(\hei{r})}{\partial \hei{r}}
\end{align}

thus we have:
\begin{equation}\label{}
[\hei{p}, f(\hei{r})] = -i\hbar \nabla(f(\hei{r})) =
-i\hbar\frac{\partial f(\hei{r})}{\partial \hei{r}}
\end{equation}

Similarly, if we express the $\hei{r} =
i\hbar\frac{\partial}{\partial \hei{p}}$ in momentum representation,
then by exchanging the position between \heit{p} and \heit{r} in the
expression of (\ref{OPERATORMOREeq:10}), we can get similar
expression:
\begin{equation}\label{OPERATORMOREeq:11}
[\hei{r}, f(\hei{p})] = i\hbar \frac{\partial f(\hei{p})}{\partial
\hei{p}}
\end{equation}





%%%%%%%%%%%%%%%%%%%%%%%%%%%%%%%%%%%%%%%%%%%%%%%%%%%%%%%%%%%%%%%%%%%%%%%%%

%%% Local Variables:
%%% mode: latex
%%% TeX-master: "../../main"
%%% End:
