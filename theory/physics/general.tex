%
% OK
%
%
% revised on Jan. 29th, 2009
% it seems good enough. the expression related to the English
% may be need to be polished, but the general idea is OK.
%
% revised on Mar. 22th, 2009
% to use the environment of theorem and law to improve it
%
% revised at Aug. 1st, 2009
% slightly modifed the words
%

\chapter{General outline for the quantum mechanics}
%
% set up the general outlines for understanding the quantum mechanics
%
%
\section{Introduction}
%
% why we need to firstly introduce the mathematical framework
% but not gradually have an introduction?
%
To some extent, the quantum mechanics can be seen as a set of rules;
on which the quantum phenomenon can be therefore explained.

How can we set up these rules? On the one hand, they can be
gradually introduced and finally weaved into an entire system. For
example, in classical mechanics; the concept of velocity and
acceleration etc. can be initially introduced, then several examples
can be provided to discuss the basis relation between them(the
universal Newton function has been implicitly contained); through
these discussion the reader may get some impress between the
``force'' and the acceleration. Finally, the fundamental Newton
equation will be unveiled.

In classical mechanics, we can do this because that the concept of
velocity, acceleration etc., and the relationship between them are
something that we can feel, they are some part of our common sense
to this world. However, in the quantum mechanics the phenomenon is
so strange that it's totally impossible for us to refer ourself to
the common sense.

Therefore, we can only rely on the mathematics and logic to solve
the problems we met in the quantum mechanics. On the other hand,
this implies that firstly we need a self-contained and rigorous
system to provide description to the rules of quantum mechanics,
then we can use these rules to analyze the phenomenon we met in this
range.

As a result, firstly we will establish all the fundamental concepts
and relations in the quantum mechanics to form the ``base''; then we
will go to the discussion to each concrete problems we met.

In this chapter, we will briefly introduced the conceptual framework
of quantum mechanics. In the following part, the concepts presented
here will be further discussed.

%%%%%%%%%%%%%%%%%%%%%%%%%%%%%%%%%%%%%%%%%%%%%%%%%%%%%%%%%%%%%%%%%%%%%%%%%%%%%%
\section{General outline}
%
% general outline
%
After some time on the study of quantum mechanics, I think that the
quantum mechanics can be firstly viewed as some ``algebra system''.
In this system, we provide the concepts, and the relations(or more
appropriate to call them as ``algorithm'') between them. Here below
is the framework for them.

Firstly, how to define the physical quantity for this system? In
classical mechanics, the physical quantity must be presented as some
kind of mathematical variables; just like
$\vec{v}=\frac{d\vec{x}}{dt}$, etc. Therefore, what's the
correspondence in the quantum mechanics?

\begin{law}
\textbf{In quantum mechanics, the operator is corresponding to the
physical quantity. Moreover, only the hermite operators are
corresponding to the common dynamic variables.}
\end{law}

Operator can be seen as some kind of ``function transformation
machine'', it transforms the function from one form to another. For
example, the $\frac{d^{2}}{dx^{2}}$ is a kind of operator. For the
function of $e^{ikx}$, we have $\frac{d^{2}}{dx^{2}} (e^{ikx}) =
-k^{2}e^{ikx}$.

In quantum mechanics, all the physical quantity are expressed as
some form of operators. For example, the kinetic operator is
expressed as $-\frac{\hbar^{2}}{2m}\nabla^{2}$, the momentum
operator is expressed as $-i\hbar\nabla$.

After the setting up of physical quantity, the next question is how
to get the information related to the system state? Just like in the
classical mechanics, for the simple falling movement, $s =
\frac{1}{2}gt^{2}$ may tell anything about the state in any time;
what's the correspondence in the quantum mechanics?

\begin{law}
\textbf{In quantum mechanics, the state is determined by the
schrodinger equation:
\begin{equation}
\hat{H}\Psi(r,t) = i \hbar \frac{\partial \Psi(r,t)}{\partial t}
\end{equation}
In which the $\Hat{H}$ is the Hamiltonian operator. The $\Psi(r,t)$
is the solution of this equation, called ``wave functions''.}
\end{law}

However, the schrodinger equation is not easy to solve, especially
for the multi-particle system. Commonly in the quantum chemistry, we
use the variational principle to approach to the answer rather than
directly solve the schrodinger equation. In the following content,
this important theorem will be stated and proved.

After setting up the schrodinger equation, we may also get the
concept of ``wave functions''. The wave functions can be seen as the
solution of the schrodinger equation, it totally describe the
quantum states. As what we have seen in the first chapter, wave
function describe the probability distribution of the system.

There's an important feature which needed to emphasized in the
quantum mechanics. In classical mechanics, for a specific system we
can only get one function to describe its state, for example, the $s
= \frac{1}{2}gt^{2}$ for the simple falling movement; however, in
solving the schrodinger equation we may get a group of
solutions(it's number can be finite or infinite); which is weaved
into a function space (That can be abstracted as Hilbert space).

\begin{law}
\textbf{In quantum mechanics, the quantum state is totally described
by the wave functions. The character of the wave functions, is
abstracted by the Hilbert space.}
\end{law}

For the Hilbert space, it's character can be generalized below:
\begin{quote}
  \begin{center}
    \begin{itemize}
    \item the description of the quantum system is got by some
      function space composed by the wave functions.
    \item this function space can have many different form of
      representation, that is; we can represent it by choosing a set
      of basis functions (and the probability character of the wave
      functions require that only the normalized basis functions is
      meaningful) which composed into some complete
      sets. Furthermore,any two set of basis functions are equivalent
      with each other.
    \item any quantum states which describe the state of the system,
      can be achieved by the linear combination of the basis functions
      in the Hilbert space. Thus the complete sets gives a complete
      description of the quantum system.
    \end{itemize}
  \end{center}
\end{quote}

Thus we can know that how to describe an arbitrary state based on
the base vectors we got. That's what we call the superposition
principle:
\begin{law}
\textbf{ If some measurement done to the wave function of $\Psi_{1}$
get the result of 1, and this measurement done to the wave function
of $\Psi_{2}$ get the result of 2; and the measurement only has
these two results; thus we can make a linear combination of $\Psi =
C_{1}\Psi_{1} + C_{2}\Psi_{2}$($C_{i}$ is some coefficient), which
can be used to express an arbitrary quantum states.}
\end{law}
We can see that this important feature can be generalized into the
concept of Hilbert space.

So far we have got the description of physical quantity, and got the
wave functions to describe the quantum states; however, if we want
to derive some specific dynamic values of this system, what can we
do? For example, the angular momentum of the system, the energy, and
others?

To answer this question, we may distinguish the physical quantity
into two groups: for a specific set of wave functions, some of them
can give a definite measurement of dynamic values (for example,
through the schrodinger equation the Hamiltonian operator can give
the system energy over its eigen states); while others can not (for
example, the coordinate operator; only the average location we can
get).

For the first group, it has some function (called eigen function)
connecting the operator the wave functions and the measurement of
dynamic values together:
\begin{equation}\label{}
\hat{A}\Psi = a\Psi
\end{equation}
Here the $\Psi$ is the wave function, it's also called eigen states
for the physical quantity of $\hat{A}$; the real value of $a$
corresponds to the the measurement of the physical quantity, called
eigen value. It's very important to note that the operator belong to
the first group is that they can commute with Hamiltonian operator;
those can not commuted with Hamiltonian operator are in the second
group.

For the second group, although we do not have a clear measurement
value over it, we can still define the average value (expectation
value) for the given operator $\hat{A}$:
\begin{equation}\label{}
\langle\hat{A}\rangle = \int \Psi^{*}\hat{A}\Psi d\tau
\end{equation}

Here, we provide an example of coordinate operator of $\hat{x}$:
\begin{equation}\label{}
\langle\hat{x}\rangle = \int \Psi^{*}\hat{x}\Psi d\tau
\end{equation}
This value depicts the average coordinate movement for the $\Psi$
along the x axis.

All in all, these answers can be sum up into some principle:
\begin{law}
\textbf{In quantum mechanics, the measurement of some physical
quantity is derived from the corresponding eigen functions; which
connect the operator and wave functions together:
\begin{equation}\label{}
\hat{A}\Psi = a\Psi
\end{equation}
here, to ensure the measurement of a is a real number, the operator
of $\hat{A}$ must be hermitian.}
\end{law}

However, the solution of the Schrodinger equation may have the
degenerate states, from which it leads to an uncertainty to
determine the eigen states to build the whole Hilbert space. For
example, the Schrodinger equation for the one dimensional free
particle movement is (in the first chapter it has been discussed):
\begin{equation}\label{}
-\frac{\hbar^{2}}{2m}\frac{d^{2}}{dx^{2}}\Psi(x) = E\Psi(x)
\end{equation}
the solution for this equation can be $\Psi_{1} = exp (
iP_{x}x/\hbar )$ or $\Psi_{2} = exp ( -iP_{x}x/\hbar )$. Both of the
two solutions occupy the same energy level.

It's clear that the linear combination of the two functions is still
the solution of the above schrodinger equation ($\Psi =
c_{1}\Psi_{1} + c_{2}\Psi_{2}$, the $c_{1}$ and $c_{2}$ are some
arbitrary real number ), this is coincident with the above
discussion. To eliminate such uncertainty in the solution, we will
introduce the ``CSCCO'', that is:
\begin{law}
  \textbf{``CSCCO'' is a group of operators rather than one
    Hamiltonian operator used to determine the corresponding Hilbert
    space. This group of operators contain the $\hat{H}$, and the
    other operators which all commuted with $\hat{H}$, any two of them
    also commute with each other (physically to say, each physical
    quantity provides a freedom to specify the investigated
    system). It can be proved that this group of operators determine a
    definite Hilbert space to describe the quantum state. Therefore,
    this group is called ``complete set Of commutating conserved
    operators''; in the future, they will be abbreviated as
    ``CSCCO''.}
\end{law}

Finally, the particles, which fall into the range of quantum type; can
be classified into different types; such as electrons, different
atoms, photons and so on. It's found that the same type of particles
are identical with each other; that is to say, it's impossible to
distinguish any particles which is in the same type. For example, two
electrons are same with each other.

Therefore, the identical quantum particles lead us to another
fundamental hypothesis in the quantum mechanics:
\begin{law}
\textbf{for the wave functions describe the quantum states, if we
alter the position of two identical particles (such as two
electrons), the wave functions may retain its sign or change its
sign. The former is call boson particle, and the later is called
fermion.}
\end{law}

%%%%%%%%%%%%%%%%%%%%%%%%%%%%%%%%%%%%%%%%%%%%%%%%%%%%%%%%%%%%%%%%%%%%
\section{Scheme for the following chapters}
%
%
%
%
Firstly, we will generally introduce the Hilbert space and the
operators, and discuss their properties in general. Here at this
point we will set up the skeleton of the framework.

Then, the discussion will shift to the position representation and
the momentum representation. Through this part, we wish to set up
the transformation from the abstract Hilbert space to the concrete
wave function which is the function of coordinate of $\bm{r}$.

After that, we will give more algebra analysis on the coordinate
operator and the momentum operator. Since they are the base for all
the other operators.

Next the Schrodinger equation will be formally introduced and given
an complete and rigorous discussion. the most important thing is the
concept of ``CSCCO''.

Finally we should focus on the identical properties for the quantum
system. So that to introduce the Pauli principle.

Here at this point, the general framework for the quantum mechanics
has been finished. In the following content, we will focus on each
of the subjects in the quantum mechanics.

the subjects should containing:
\begin{itemize}
  \item the angular
momentum and the spin. the discussion will focus on the form of
operators, eigen function, eigen value; commutation relations
between the operators. Besides, most importantly, the angular
momentum and spin in many particle system.
  \item  some important one dimension model, including the
harmonic oscillator function discussion.
  \item  central field model.
  \item  the perturbation theory for the time independent and time
  dependent case.
  \item  radiation theory.
  \item  uniform electron gas model.
  \item  second quantization.
\end{itemize}


%%%%%%%%%%%%%%%%%%%%%%%%%%%%%%%%%%%%%%%%%%%%%%%%%%%%%%%%%%%%%%%%%%%%
\section{Tmp list for the work will be done since July 29th, 2009}
%
%
%
%
All has been concentrated on the physics part:

operator:
\begin{itemize}
\item one dimension model. including the harmonic oscillator.
\item Heisenberg uncertainty relation prove in basic.tex
\item perhaps have to revise the Schroedinger equation discussion. So
  far we do not have clear idea about it.
\item central field model discussion.
\item etc......
\end{itemize}


So far, we only concentrate on the position representation and
momentum representation. This part should be done first.


%%%%%%%%%%%%%%%%%%%%%%%%%%%%%%%%%%%%%%%%%%%%%%%%%%%%%%%%%%%%%%%%%%%%


%%% Local Variables:
%%% mode: latex
%%% TeX-master: "../../main"
%%% End:
