%
% OK
%
% revised on Jan. 29th; 2009
% problems behind:
%   this part is not fully finished. we wish to evaluate simple
% Heisenberg relationship through free wave packets.
%
% revised at July, 26th, 2009
% make a whole revision to the chapter, free particle part rewrite
% and correct the bug related to the h
% and do many corrections
%
% revised at Aug. 1st, 2009
% only small modifications made, so far all has been satisfied, leaving
% only the Heisenberg relation unsolved
%


%%%%%%%%%%%%%%%%%%%%%%%%%%%%%%%%%%%%%%%%%%%%%%%%%%%%%%%%%%%%%%%%%%%%%%%%%%%%%%%%%%
% Basic Framework In Quantum Mechanics:
% the main target of this chapter is to set up the general framework
% to understand the QM in a very coarse way and conceptual way. we
% decide that only provide the concept below :
% 1 wave function and its general meaning
%   1.1 it contains everything about the quantum state
%   1.2 probability waves
%   1.3 all the wave functions composed into a function space
% 2 physical quantity:
%   2.1 represented as operator
%   2.2 eigen value and eigen vector
% 3 schrodinger equation
% 4 Principle of Superposition the discussing examples we are  involved:
%   1 the free particle wave function
%   2 infinite well case
%   3 simple heisenberg relationship through free wave packets
%%%%%%%%%%%%%%%%%%%%%%%%%%%%%%%%%%%%%%%%%%%%%%%%%%%%%%%%%%%%%%%%%%%%%%%%%%%%%%%%%%%


\chapter{Basic framework in quantum mechanics}
\label{basic_chapter}
%
% only provide the conceptual framework to understand
% the quantum mechanics
%
%
This chapter here is providing a general way to understand the
quantum mechanics, here only the physical concepts are introduced;
the rigorous expressions for them will be given in the following
chapters.
%%%%%%%%%%%%%%%%%%%%%%%%%%%%%%%%%%%%%%%%%%%%%%%%%%%%%%%%%%%%%%%%%%%%%%%%%%%%%%%%%%%%
\section{Some basic idea}
%
% quantum mechanics is concentrated on small particles
%
Quantum mechanics is created around the first decades in 20th century,
while the physicist had found that some experiments could not get
explanation within classical mechanics, these experiments were all
related to the small particles, such as atom, electron, light,
etc. After a long time effort, a new framework was established to
explain the phenomenon occurring in such particles of very small mass
at very small distance, this is the fundamental modification to the
foundamental physical concepts and law; called quantum mechanics.

How can we understand the difference between the classical mechanics
and the quantum mechanics? What's the general feature of the quantum
mechanics? Here in the following content, we will concentrate on the
phenomenon related to the electromagnetic radiation to see the
difference and the characteristic of quantum mechanics.

%%%%%%%%%%%%%%%%%%%%%%%%%%%%%%%%%%%%%%%%%%%%%%%%%%%%%%%%%%%%%%%%%%%%%%%%%%%%%%%%%%%%%%
\subsection{Planck-Einstein relations}
%
% introduce the photons
%
%
In 19th century, light is considered as a kind of electromagnetic
wave; and its motion complies with the Maxwell equation. However, in
the beginning of 20th century, in the study of the blackbody
radiation, where the electron magnetic theory failed to give a
satisfied explanation; Planck suggested that to use the quantization
of energy to solve this problem: $E = h\nu$. That is, for an
electromagnetic wave of frequency of $\nu$, the only possible energy
are integral multiples of quantum $h\nu$; where the $h$ is a
new physical constant.

Basically this result implies that the ``light'' is composed by some
elementary indivisible components(called photon); so the character
of light depends on the summation of such individual components.

Later, Einstein used this idea to successfully explain the
photo-electron effects. It has been found that for the photons, the
equation below is satisfied:
\begin{eqnarray}\label{BASICeq:1}
% \nonumber to remove numbering (before each equation)
  E &=& h\nu \nonumber \\
  p &=& \frac{h}{\lambda}
\end{eqnarray}
Here the $h$ is the same new constant, as a very small number; it
approximates to $10^{-34}Joule\cdot second$. We call this constant as
Planck constant to memorize the physicist of Planck. Additionally, we
note that in quantum mechanics, we will frequently meet another
physical constant of $\hbar$, which is equal to $\frac{h}{2\pi}$.

%%%%%%%%%%%%%%%%%%%%%%%%%%%%%%%%%%%%%%%%%%%%%%%%%%%%%%%%%%%%%%%%%%%%%%%%%%%%%%%%%%%%
\subsection{Wave-particle duality: analysis on Young's
experiment}\label{sec:WPDAOYE_in_basic}
%
% how can we understand the phenomenon based on the photons
% concept? Introduce the feature of the quantum particles
%
However, as a kind of wave; light behaves just like waves has the
phenomenon of interference; where such character does not retain in
particles, such as bullets.  Therefore, the existence of the photon,
and the wave character of the lights; both combined into the
"wave-particle paradox".

The famous Young's experiments clearly demonstrates the interference
character of the light. Herein this experiment, the light emits from
the source, falls onto an opaque screen which pierced by two split
of $A_{1}$ and $A_{2}$; the light penetrates through the $A_{1}$ and
$A_{2}$ interacted with each other and forming an interference
pattern on the final project screen; where used for display the
result.

If $E_{1}(x)$ and $E_{2}(x)$ represents the electric field produced
by split $A_{1}$ and $A_{2}$; we have:
\begin{equation}\label{}
E(x)= E_{1}(x) + E_{2}(x)
\end{equation}
and the intensity is:
\begin{equation}\label{}
|E(x)|^{2}= |E_{1}(x) + E_{2}(x)|^{2}
\end{equation}
How can we understand such phenomenon based on the photons?

There's an experiments developed for further understanding this
situation. If the project screen is a kind of photographic plate,
firstly to make the light as ``weak'' as possible that they are
passing through the split nearly one by one; then the project screen
receives a kind of pattern on which some dots are irregularly
arranged, these dots are the final position of the photons on the
project screen. Next, while increasing the intensity of the light;
more dots are on the project screen; and finally composed into the
interference pattern.

There are many efforts had been made to try to understand this
circumstance; for example, such interference pattern is considered
as a result of interaction between photons. However, after many
attempts to understand such phenomenon, the people recognized that
there's only one way to understand such circumstance; that the
motion of the photons in this experiment should be only considered
as some ``waves'', such wave portrays the probability of the
particle's motion; finally the most important property for the wave
is that it should satisfy the superposition principle so that the
interference phenomenon can occur.

More specifically, we can arrive such conclusion:
\begin{center}
  \begin{quote}
    \begin{itemize}
    \item the motion of quantum particle is described by some wave
      function of $\Psi(\bm{r},t)$, which is related to the coordinate
      vector of $\bm{r}$ and time $t$. Thus as a kind of wave, the
      classical concept of trajectory has to be abandoned. Sometimes
      in the wave function we can separate the time part and the
      position part from each other, so in this case the wave function
      can be shorten as $\Psi(\bm{r})$.
    \item the quantum character in the motion of small particles is
      its inherent property, so even the single quantum particle will
      have the quantum pattern.
    \item wave function of $\Psi(\bm{r},t)$ contains all the
      information related to the motion of the corresponding quantum
      particle.
    \item wave function of $\Psi(\bm{r},t)$ is interpreted as a
      probability amplitudes of the particle presence.
    \item the principle of superposition is satisfied by the wave
      function.
\end{itemize}
    \end{quote}
\end{center}

The declaration above are all fundamental concepts; they are only
derived from facts; so such concepts is considered as the foundation
of the whole quantum mechanics.

As for the wave-particle paradox, the wave function is some kind of
integration between wave and particle. In classical mechanics, a
particle means that such object has properties of mass, charge etc.,
which is its ``corpuscularity'' or ``atomicity''; also such classical
particle possess definite velocity and coordinate while in moving. In
quantum mechanics, the property of ``atomicity'' is conserved and the
concept of the trajectory is swept out so as to adapt to the
experiment facts. That's what we mean the ``particle'' in quantum
mechanics.

Similarly, the wave we discussed in classical mechanics is always
referred to periodic change on some real physical objects, just like
the waves in water, sound, or the electromagnetic field. However, such
understanding has to be abandoned in quantum mechanics and only leaves
the inherent superposition property of the wave, which is the essence
behind the interference and the diffraction phenomenon.

%%%%%%%%%%%%%%%%%%%%%%%%%%%%%%%%%%%%%%%%%%%%%%%%%%%%%%%%%%%%%%%%%%%%%%%%%%%%%%%%%%%
\subsection{Probability waves character}
\label{sec:PWC_in_basic}
%
% how to understand the probability wave?
% 1  definition
% 2  property
%
As a probability wave, the wave function states that $
|\Psi(\bm{r})|^{2} dxdydz$ represents the probability to find such
particle in a little volume of $dxdydz$ around $\bm{r}$; if the
integration is over all space, that is the probability of the particle
over the whole space; this should  equal to $1$.

For the multi-particle system, $ |\Psi(\bm{r}_{1}, \bm{r}_{2}, \cdots,
\bm{r}_{N})|^{2} d\tau_{1}d\tau_{2}\cdots d\tau_{N}$ represents the
probability of finding:
\begin{center}
particle 1 existed in the $(\bm{r}_{1}, \bm{r}_{1}+d\tau_{1})$; \\
particle 2 existed in the $(\bm{r}_{2}, \bm{r}_{2}+d\tau_{2})$; \\
$\cdots$ \\
particle $N$ existed in the $(\bm{r}_{N}, \bm{r}_{N}+d\tau_{N})$.
\end{center}

There are some distinct requirements by the probability waves.
First, the wave function should be normalized; that:
\begin{equation}\label{}
\int |\Psi(\bm{r})|^{2} d\tau = 1
\end{equation}
This property does not exist in classical mechanics.

Secondly, the amplitude of wave function is not uniquely determined.
If the wave function is multiplied by a constant, after
normalization this new function is same with the old one. Thus only
the relative probability distribution is meaningful.

Thirdly, the phase of the wave function is also not uniquely
determined. the $\Psi$ is same with the $e^{i\alpha}\Psi$. Actually,
the phase is the most difficult concept in understanding the wave
functions.

%%%%%%%%%%%%%%%%%%%%%%%%%%%%%%%%%%%%%%%%%%%%%%%%%%%%%%%%%%%%%%%%%%%%%%%%%%%%%%%%%%%
\subsection{Experiment Of spectral decomposition}
%
% introduce the concept of eigen value and eigen states
%
%
%
This experiment consists of directing a polarized plane
monochromatic light wave onto an analyzer of A. This light is
directly through Z axis, and the unit vector of $\bm{e_{p}}$ is used
to indicate its polarization. The analyzer A transmits light
polarized parallel to X axis, and absorbs the light polarized
parallel to Y axis.

Suggest that the wave for the light is:
\begin{equation}\label{}
E(\bm{r},t) = E_{0}\bm{e_{p}}e^{i(kz-\omega t )}
\end{equation}
After passing the analyzer of A, the wave becomes:
\begin{equation}\label{}
E^{'}(\bm{r},t) = E_{0}\bm{e_{x}}e^{i(kz-\omega t )}
\end{equation}

It's known that after passing the A the intensity ($I^{'}$) has some
relation with the original intensity of $I$:
\begin{equation}\label{}
I^{'} = I \cos^{2}\theta
\end{equation}

So what happens on the quantum level, while the intensity is as weak
as possible for the photons to pass the analyzer one by one?

First, only the entire photons are found to pass the A; and it's
impossible to predict whether a given incident photon will pass or
absorbed by the analyzer, only the corresponding probability is
known. Finally, if a larger number of photons are sent out, there
are $N\cos^{2}\theta$ photons will be detected by the analyzer, this
is coordinated with the intensity: $I \sim NE^{2}_{0}$.

Compared with the Young's experiment, it has something new in this
one. The analyzer of A gives a determined result while the wave
function for each photon is passing through: whether it entirely
passed or not. In quantum mechanics, the A works as a measurement
device; and the measuring result is called eigen value of the
measurement.

Each of the eigen value corresponds an eigen state. Here the two
eigen states character as: $\bm{e_{x}}, \bm{e_{y}}$. While the light
is polarized in $\bm{e_{x}}$ direction, all the light are passing
through; but for the light in the $\bm{e_{y}}$ direction, all the
light are absorbed. Here the measurement to the eigen state leads to
definite result of eigen value.

However, for an arbitrary light, only the passing probability can be
predicted. In fact, any arbitrary light can be decomposed as:
\begin{equation}\label{}
\bm{e_{p}} = \bm{e_{x}}\cos\theta+ \bm{e_{y}}\sin\theta
\end{equation}
Thus for the light, in which composing of $N$ photons ($N$ is a huge
number); for each photon we can predict its function (it's wave
function):
\begin{equation}\label{}
E(\bm{r},t) = E_{0}\bm{e_{x}}e^{i(kz-\omega t )}\cos\theta +
E_{0}\bm{e_{y}}e^{i(kz-\omega t )}\sin\theta
\end{equation}
For this wave function, its measurement probability is:
\begin{equation}\label{}
|\hat{A}E(\bm{r},t)|^{2} = |E_{0}|^{2}\cos^{2}\theta
\end{equation}
While for the large number of $N$, the probability is
$N|E_{0}|^{2}\cos^{2}\theta$; coincident with the result. Here the
$\hat{A}$ is the representation of the measurement, it cause the
wave function to collapse on to single direction of $\bm{e_{x}}$; so
this phenomenon is called spectral decomposition.

Through this experiment, something more fundamental is revealed;
actually the rules we concluded below can be extended to all the
quantum systems:
\begin{center}
\begin{quote}
\begin{itemize}
  \item compared with the wave function,  the measurement is used to
  extract useful information from it. Mathematically it is
  represented by operator, and corresponds to an arbitrary physical
  quality.
  \item For the specific measurement, it has a set of eigen state
  function corresponding to it; each one gives an definite value
  after the measurement; these values are the eigen value (result of the
  measurement).
  \item For any $\Psi(\bm{r},t)$, the probability to find the eigen value
  of $a_{i}$ for a measurement is found by decomposing $\Psi(\bm{r},t)$
  in  terms of eigen function of $\Psi_{i}(\bm{r},t)$:
  \begin{equation}\label{BASICeq:2}
  \Psi(\bm{r},t) = \sum_{i}a_{i}\Psi_{i}(\bm{r},t)
  \end{equation}
  the probability to find $a_{i}$ is: $\frac{|a_{i}|^{2}}{\sum|a_{i}|^{2}}$
  \item For such mixing state of $\Psi$ defined in
  (\ref{BASICeq:2}); the measurement will cause the wave function to
  collapse:
  \begin{equation}\label{}
  \hat{A}\Psi(\bm{r},t) = a \Psi_{a}(\bm{r},t)
  \end{equation}
\end{itemize}
\end{quote}
\end{center}


%%%%%%%%%%%%%%%%%%%%%%%%%%%%%%%%%%%%%%%%%%%%%%%%%%%%%%%%%%%%%%%%%%%%%%%%%%%%%%%
\subsection{Matter wave function}
%
% matter also obeys the quantum relation
%
%
\label{MWF_in_basic}
%
%
It's strikingly to find out that the unusual behavior for the light
is not unique but universal to all the small particles which fall
into the arrange of quantum mechanics.

Inspiring by the discovery on light, De Brogile put forward his
matter wave hypothesis:
\begin{eqnarray}
\label{BASICeq:16}
% \nonumber to remove numbering (before each equation)
  E &=& h\nu = \frac{h\omega}{2\pi} = \hbar\omega \nonumber \\
  \bm{p} &=& \frac{h}{\lambda} = \frac{h\bm{k}}{2\pi} = \hbar \bm{k}
\end{eqnarray}
Here the $\nu$ is the frequency of the wave, and $\omega$ is the
corresponding angular frequency which is $\nu =
\frac{\omega}{2\pi}$. Similarly, the $\bm{k}$ is the wave vector so we
also have $\bm{k}\lambda = 2\pi$.

This equation is same with (\ref{BASICeq:1}); it implies that motion
of electrons and photons are controlled by the same equation with
light.  After discussing so much over the light experiment, we can see
the characteristic of quantum mechanics; however, it still need a
function to determine the wave function. This function is the famous
Schrodinger function.


%%%%%%%%%%%%%%%%%%%%%%%%%%%%%%%%%%%%%%%%%%%%%%%%%%%%%%%%%%%%%%%%%%%%%%%%%%%%%%%%%%%
\section{Schrodinger equation}
\label{SE_in_basic}
%
% Schrodinger equation
%
%
Schrodinger equation is the equivalent function of Newton function
in quantum mechanics. It's some kind of fundamental law in quantum
mechanics.

The Schrodinger equation states:
\begin{equation}\label{BASICeq:3}
  \left( -\frac{\hbar^{2}}{2m}\nabla^{2} + \hat{V}\right)\Psi(\bm{r},t)
= i \hbar \frac{\partial \Psi(\bm{r},t)}{\partial t}
\end{equation}
The $\hat{V}$ is the potential energy operator. Here we can
understand the operator as some ``function generating machine'' which
is used to transform one function (the waves) to another form, they
are all related to the details of the given quantum system.  By
solving this equation, we can get the wave function of $\Psi(\bm{r},t)$.

Here we can write the $\hat{V}$ and kinetic part together as
$\hat{H}$, which is called Hamiltonian operator. Actually the
expression of this operator can be compared with the Hamiltonian
function in the classic mechanics (see the detailed
discussion\cite{Landau}); in the following content, we can see that
this operator possesses central position.

If the potential energy operator does not contain the time factor, the
wave function can be separated into $\Psi(\bm{r},t) =
\Psi(\bm{r})f(t)$.  While inserting this form into (\ref{BASICeq:1}),
we can finally get (more detailed derivation see the chapter
discussing the Schrodinger equation):
\begin{equation}\label{BASICeq:5}
\hat{H}\Psi(\bm{r}) = E\Psi(\bm{r})
\end{equation}
This expression can be abbreviated as $\hat{H}\Psi = E\Psi$. So
finally the time dependent wave function of $\Psi(\bm{r},t) =
\Psi(\bm{r})exp(-iEt/\hbar)$. Where in this expression, the $E$ is the
energy of this system.

For an arbitrary molecule system, the Schrodinger equation under
Born-Oppenheimer condition is:
\begin{multline}\label{BASICeq:4}
  \left\{\sum_{i=1}^{n}\left(\frac{-\hbar^{2}}{2m}\times\nabla_{i}^2\right)+
    \sum_{i=1}^{n}\sum_{\alpha=1}^{N}
    \left(\frac{-Z_{\alpha}e}{4\pi\epsilon_{0}}\frac{1}{\bm{r_{i\alpha}}}\right)+
  \right. \\
\left. \sum_{i<j}\left(\frac{e^{2}}{4\pi\epsilon_{0} \bm{r_{ij}}}\right)
    \right\}\,\Psi=E\Psi
\end{multline}
In this equation, $n$ is the number of electrons, $N$ is the number of
nucleus; $\alpha$ refers to indicate the nuclear, and index of $i,j$
is used for the electrons. $Z_{\alpha}$ is the charge for the
corresponding nuclear.In this equation, only the electrons are
considered as the quantum particles existing in the wave function. In
quantum chemistry, most of the problems are directly or indirectly
related to the solving of this equation.


%%%%%%%%%%%%%%%%%%%%%%%%%%%%%%%%%%%%%%%%%%%%%%%%%%%%%%%%%%%%%%%%%%%%%%%%%%%%%%%%%%%
\section{Atomic units}
%
%
%
%
Now it's time to introduce the atom unit which is frequently used
across the quantum mechanics. In this case, the physical constants
below are appointed to $1$:
\begin{equation}
  \label{BASICeq:7}
  \frac{h}{2\pi} = m_{electron} = \frac{e^{2}}{4\pi\epsilon_{0}} = 1
\end{equation}

Then the above Schrodinger equation in (\ref{BASICeq:3}) is shorten
as:
\begin{equation}\label{BASICeq:8}
  \left( -\frac{1}{2}\nabla^{2} + \hat{V}\right)\Psi(\bm{r},t)
= i  \frac{\partial \Psi(\bm{r},t)}{\partial t}
\end{equation}

The Schrodinger equation shown in the (\ref{BASICeq:4}) can be written
as:
\begin{equation}
  \label{BASICeq:9}
  \left\{\sum_{i=1}^{n}(\frac{-1}{2}\times\nabla_{i}^2)+
\sum_{i=1}^{n}\sum_{\alpha=1}^{N}(\frac{-Z}{\bm{r_{i\alpha}}})+
\sum\sum_{i<j}(\frac{1}{\bm{r_{ij}}})
\right\}\,\Psi=E\Psi
\end{equation}
It can see this equation is much more simple by dropping the
constants.

In the new form of Schrodinger equations, the energy gotten by solving
the equation is named with ``Hatree'', where we have such
transformation relations:
\begin{equation}
  \label{BASICeq:10}
  1Hatree = 27.211 eV = 627.51 kcal/mol = 2625.5 kj/mol
\end{equation}


%%%%%%%%%%%%%%%%%%%%%%%%%%%%%%%%%%%%%%%%%%%%%%%%%%%%%%%%%%%%%%%%%%%%%%%%%%%%%%%%%
\section{ Superposition principle}
\label{SP_in_basic}
%
% this part of contents may contains two folders of meanings:
%  1  the description of the quantum mechanics may be different mathematically
%  2  the superposition character deeply portray the wave character of the
%     quantum particles
%

The wave functions are the result of the corresponding Schrodinger
equation. Just like the Maxwell equations, if $\Psi_{1}$ and
$\Psi_{2}$ are two solutions of the Schrodinger equation in
(\ref{BASICeq:3}), it can see that
$\lambda_{1}\Psi_{1}+\lambda_{2}\Psi_{1}$($\lambda_{i}$ is some
coefficient) is also the solution of this equation. Thus the
superposition character of the wave functions is satisfied in the
quantum mechanics.

This is a very important character to understanding the essence of the
quantum mechanics. Unlike the classical mechanics, in which we use
only one function to describe the state of the objects(this function
may be defined by the Newton equation). In the macroscopic word,
generally to say, to solve the function we only need information
related to coordinates and velocities (or momentum), then all the
physical quantity can be derived.

However, in the quantum mechanics; we need a function space to
describe the state of the quantum particle. Such different way of
description implies that we may potentially have different choice to
represent the same system (later in the content, we will discuss the
representation; that is what we are talking here); thus we may use
some very different mathematical way to tackle the problems in the
quantum mechanics.

On the other hand, the character of superposition is some essential
property to understand the behavior of the quantum particles. Since
the wave functions is some dispersed function among all the space,
the superposition character implies that two wave functions may have
the chance to mix together to form a more stable quantum state (we
consider the two system together to form a one); that's the origin
of the interference phenomenon in the quantum mechanics.

%%%%%%%%%%%%%%%%%%%%%%%%%%%%%%%%%%%%%%%%%%%%%%%%%%%%%%%%%%%%%%%%%%%%%%%%%%%%%%%%%%
\section{A simple case: wave function for free particle}
\label{ASCWFFFP_in_basic}
%
%
%
%
Here we are going to present the most simplest example in solving
the Schrodinger equation: the wave function for the free particles.

What is the free particle? Imagine that in the space only a quantum
particle exists, so that there's nothing affecting it's motion.
Under this state, such quantum particle is called free particle. In
the above content, the matter wave function actually describe the
motion of free particles.

Thus, in the Schrodinger equation we have only the kinetic operator,
and the potential operator always equals to zero. So the corresponding
Schrodinger equation is (in atom units):
\begin{equation}
  \label{BASICeq:6}
  -\frac{1}{2}\nabla^{2}\Psi(\bm{r},t) = i \frac{\partial
\Psi(\bm{r},t)}{\partial t}
\end{equation}

Since that the left part in the (\ref{BASICeq:6}) is irrelevant to the
time, so according to the (\ref{BASICeq:5}) we can rewrite the
(\ref{BASICeq:6}) as:
\begin{equation}
  \label{BASICeq:11}
    -\frac{1}{2}\nabla^{2}\Psi(\bm{r},t) = E \Psi(\bm{r},t)
\end{equation}
Here $E$ characterizes the energy of the free particle.

Since that it's kinetic energy contained in this system, so it's
appropriate to express the total energy via the momentum of $\bm{p}$:
\begin{equation}
  E = \frac{\bm{p} \cdot \bm{p}}{2m} = \frac{\bm{p} \cdot \bm{p}}{2} =
  \frac{p^{2}}{2}
\end{equation}
Therefore we can have the (\ref{BASICeq:11}) as:
\begin{equation}
  \label{BASICeq:12}
      -\nabla^{2}\Psi(\bm{r},t) = p^{2} \Psi(\bm{r},t)
\end{equation}

The equation in the (\ref{BASICeq:12}) is simple to solve, actually
it's solution is $A exp(i\bm{p}\cdot\bm{r})$ ($A$ is some
constant). Combining with the time part, we can express the whole
solution as:
\begin{equation}
  \label{BASICeq:13}
  \Psi(\bm{r},t) = Ae^{i(\bm{p}\cdot\bm{r} - Et)}
\end{equation}

Now we can prove that the solution of (\ref{BASICeq:13}) will lead to
the character of matter wave function in the (\ref{MWF_in_basic}). For
simplicity we only consider the one dimension case, then the
(\ref{BASICeq:13}) is (without considering the time part):
\begin{equation}
  \label{BASICeq:14}
    \Psi(x,t) = Ae^{ip_{x}x}
\end{equation}
If $x^{'} = x + \lambda$, then we should have $\Psi(x^{'},t) =
\Psi(x,t)$; so this implies that:
\begin{equation}
  \label{BASICeq:15}
  p_{x}\lambda = 2\pi \rightarrow  p_{x} = \frac{2\pi}{\lambda} = k_{x}
\end{equation}
Which is identical to the (\ref{BASICeq:16}).

For the energy, similarly we have $\Psi(x,t) = \Psi(x,t+T)$ while $T$
is the period of the wave. So by the same derivation we can have:
\begin{equation}
  \label{BASICeq:17}
  ET = 2\pi \rightarrow E = \frac{2\pi}{T} = 2\pi\nu =
  \frac{h}{\hbar}\nu = h\nu
\end{equation}
So we finally arrive at the matter wave functions!

Now let's concentrate on the time independent state, where the wave
function is $\Psi(\bm{r}) = Aexp ( i\bm{p}\cdot\bm{r})$. It's amusing
to see that this function indicates that the probability distribution
in every location of $\bm{r}$ is constant: $|\Psi(\bm{r})|^{2} =
A^{2}$. The same relative probability distribution implies that
particle's position can not fixed anywhere, while this particle has a
fixed momentum everywhere.

Furthermore, the energy of this free particle is changing in
continuously way, that the E changes from $0$ to $\infty$. For each
definite energy, we get a corresponding wave function of
$\Psi(\bm{r})$. For different wave function (corresponding to
different energy), it's easy to prove that
$\int\Psi_{\bm{p}}(\bm{r})\Psi_{\bm{p}^{'}}(\bm{r})dp = 0$; so this
implies that all the wave functions corresponding to the solution of
(\ref{BASICeq:3}) are composed into a full space, this space is in
$\infty$ dimension.


%%%%%%%%%%%%%%%%%%%%%%%%%%%%%%%%%%%%%%%%%%%%%%%%%%%%%%%%%%%%%%%%%%%%%%%%%%%%%%%%%

\subsection{How to express an arbitrary wave function for free particle }
\label{sec:HTEAAWFFFP_in_basic}

Now let's go to see how to express an arbitrary state for free
particles. First, let's rewrite the plane wave function in
(\ref{BASICeq:13}) into another form (for simplicity the constant of
$A$ is dropped):
\begin{equation}
  \label{BASICeq:18}
  \psi(\bm{r},t) = e^{i(\bm{k}\cdot\bm{r} - \omega t)}
\end{equation}
Since that $E$ and $\bm{p}$ only differ with $\omega$ and $\bm{k}$
with constant of $\hbar$.

From the principle of superposition shown in the (\ref{SP_in_basic}),
we can express such arbitrary wave function through the Fourier
integration as:
\begin{equation}
  \label{BASICeq:19}
  \Phi (\bm{r}, t) = \frac{1}{(2\pi)^{\frac{3}{2}}}\int \Psi(\bm{k}, t)
e^{i(\bm{k}\cdot\bm{r} - \omega t)}d^{3}\bm{k}
\end{equation}
Here the $t$ is considered to be fixed and the $\psi$ characterizes
each ``component'' within the composite wave function of $\Phi$, and
the $\Psi$ characterizes the weight of the component(equivalent to the
coefficient of $\lambda$ in the \ref{SP_in_basic}). Mathematically to
say, any wave function which is square-integrable across the whole
space can be expressed through the (\ref{BASICeq:19}).

The wave function in the (\ref{BASICeq:19}), as a form of
superposition of plane waves; is called ``wave packet''. It's
interesting to see that through the Fourier transformation, we can get
another expression similar to the (\ref{BASICeq:19}):
\begin{equation}
  \label{BASICeq:20}
    \Psi (\bm{k}, t) = \frac{1}{(2\pi)^{\frac{3}{2}}}\int \Phi(\bm{r}, t)
e^{-i(\bm{k}\cdot\bm{r} - \omega t)}d^{3}\bm{r}
\end{equation}

It's amazing to see that both of the $\Phi$ and $\Psi$ are the ``dual
functions'', one is expressed in the momentum space (related to wave
vector of $\bm{k}$) and the other is expressed in coordinate space of
$\bm{r}$. Such relation will be further investigated in the following
content.

Finally we note that the transformation in the (\ref{BASICeq:19}) and
(\ref{BASICeq:20}) are not only limited to the free particle system
wave function, but can be applied to any square integrable wave
functions; this is because the generality of the Fourier integration.

%%%%%%%%%%%%%%%%%%%%%%%%%%%%%%%%%%%%%%%%%%%%%%%%%%%%%%%%%%%%%%%%%%%%%%%%%%%%%%%%%

\subsection{Heisenberg Uncertainty Principle}
\label{sec:HUP_in_basic}
%
% to derive this through the wave packets
%



%%%%%%%%%%%%%%%%%%%%%%%%%%%%%%%%%%%%%%%%%%%%%%%%%%%%%%%%%%%%%%%%%%%%%%%%%%%%%%%%%
\section{Another case: particle in infinite deep well}
\label{ASPIIDW_in_basic}
%
%
%
%
After make some small modification on the free particle, we can get
another case: particle in the infinite deep well; in which we have
the potential operator as:
\begin{equation}\label{}
 V(x) = \left\{
  \begin{array}{ll}
    0,      & \hbox{ $0 < x < a$} \\
    \infty, & \hbox{ $x<0, x>a$}
  \end{array}
\right.
\end{equation}
Here in this case for simplicity the Hamiltonian is restricted into X
axis.

It's easy to see that the Schrodinger function have got two
independent solutions, one is $exp(ipx)$, the other is $exp(-ipx)$ ($p
$ is the momentum on the X axis); so the solution can be written as:
\begin{equation}\label{}
  \Psi(x) = a \times exp(ipx) + b
\times exp(-ipx)
\end{equation}
Now we use the boundary condition to determine the real solution.  The
physical condition requires that $\Psi(0)=0$ and $\Psi(a)=0$; so from
$\Psi(0)=0$ we know that $\Psi = 2ia \sin px$; it can be further
written as $\Psi = A \sin px$ (the imaginary number does not affect the
wave function, so simply drop it). From the $\Psi(a)=0$, the energy is
achieved: $E = (n\pi)^{2}/2ma^{2}$ ($n=1,2,\cdots$).

Finally, we can get the specific wave function: $\Psi (x) =
\sqrt{\frac{2}{a}} \sin (\frac{n\pi}{a})$.

It's very interesting to make some analysis to this system. First,
compared with the free particle, the system has a discrete energy
level; and at the lowest energy level, the energy of the particle is
not zero (this can be demonstrated by the uncertainty law).  For
each energy, it corresponds a wave function. For different wave
function, they are also orthogonal with each other. So here the
Hamiltonian defined in this system also forms a space.



%%%%%%%%%%%%%%%%%%%%%%%%%%%%%%%%%%%%%%%%%%%%%%%%%%%%%%%%%%%%%%%%%%%%%%%%%%%%%%%%%


%%% Local Variables:
%%% mode: latex
%%% TeX-master: "../../main"
%%% End:
