%
% revised at Aug. 1st, 2009
% to add some citing materials
%

\chapter{Introduction}
At first, I hope to introduce this part of knowledge as an
independent chapter, soon this idea is abandoned. As the base of the
quantum chemistry, the quantum mechanics provides the fundamental
physical concept, algorithms and framework for dealing with the
problems in quantum chemistry, so it deserves more attention to
build in a more subtle and elaborate way.

However, the quantum chemistry has its own language to describe and
solve the problem encountered in the chemistry, which behaves
differently with the traditional way in quantum mechanics. I hope,
that a bridge can be set up to connect the two different system, so
that's the ultimate goal of this individual part. Herein the content
we recommend the material by Dirac\cite{Dirac}, XingLin
Ke\cite{XingLinKe}, Coden\cite{Coden}, Landau\cite{Landau},
Greiner\cite{Greiner_Quantum_Mechanics} and
Schiff\cite{Schiff_Quantum_Mechanics}; I think that they are providing
the most clearest idea to understand the quantum mechanics. Some of
them are clear and detailed in physical concepts, such as the book by
Landau and Coden etc., some are clear in mathematical derivation such
as the book by Greiner. On the other hand, the book written by
Feynman\cite{Feynman-Lecture} is very interesting in a lot of detailed
analysis, so worthy of reading. As for the book by JinYan Zeng(Volume
I and II)\cite{ZengJinYan}, it's a good reference book; but not a easy
book to read. There are some other books for reading choice, but the
author has not tried to read them; but they may be good materials.
They are the books by E. H. Wichmann\cite{Wichmann}, and by
Messiah\cite{Messiah}.


%%%%%%%%%%%%%%%%%%%%%%%%%%%%%%%%%%%%%%%%%%%%%%%%%%%%%%%%%%%%%%%%%%%%%%%%%%%%%%%




%%% Local Variables: 
%%% mode: latex
%%% TeX-master: "../../main"
%%% End: 
