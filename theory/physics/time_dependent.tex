%
% set up at March 31th, 2009
%
%
%
%
%%%%%%%%%%%%%%%%%%%%%%%%%%%%%%%%%%%%%%%%%%%%%%%%%%%%%%%%%%%%%%%%%%%%%%%%%%%%%%%%%%%%%%%%%%%%%%
\chapter{Perturbation treatment for the time-dependent systems}
\label{perturbation_in_time}
%
%  what's the perturbation treatment for the time-dependent system?
%  quantum transition is mostly important
%  zero order and first order expression
%
In quantum mechanics, the study related to the quantum states can be
divided into two groups:
\begin{itemize}
  \item the possible states for describing the system
  \item the quantum system evolves with time
\end{itemize}

In the first case, all the study concentrate on the eigen states and
eigen values. For the specific system, we choose its CSCCO and
evaluate the corresponding eigen states. By using such eigen states,
we can figure out any physical quantity, obviously such physical
quantity should be time independent; such as the dipole moments for
the molecule.

For the second case, nearly all the study concentrate on the
perturbation treatment for the time dependent system. In practice,
it's usually impossible to derive the general form for the evolution
of wave functions as time varies. Instead, to determine the quantum
transition probabilities is much more important in practical
application. For example, in the photochemistry to determine the
excitation energy level due to some perturbed radiation, or in the
UV spectrum to determine the ultraviolet spectrum intensity. they
are all related to the transition probability between two quantum
states. therefore, we usually express the time dependent part as
some ``perturbed'' operator:
\begin{equation}\label{}
\hat{H} = \hat{H}_{0} + \hat{H}^{'}(t)
\end{equation}
So we calculate the response of the wave functions to this perturbed
operator by progressively expanding it as:
\begin{align}\label{}
\hat{H} &= \hat{H}_{0} + \lambda\hat{H}^{'}(t) \nonumber \\
\Psi(r,t) &= \Psi^{(0)}(r) + \lambda\Psi^{(1)}(r,t) +
\lambda^{2}\Psi^{(2)}(r,t) + \cdots
\end{align}

For the $\hat{H}_{0}$, we can calculate its eigen states. Suggest
that such eigen states for the $\hat{H}_{0}$ are labeled as
$\psi_{i}(r)$ ($i=1,2,\cdots$) so that for the zero order
approximation, an arbitrary state of $\Psi^{(0)}(r)$ can be
expressed as:
\begin{equation}\label{}
\Psi^{(0)}(r) = \sum_{i}a_{i}^{(0)}\psi_{i}(r)
\end{equation}
Here the $\Psi^{(0)}(r)$ can be any states for the $\hat{H}_{0}$.
However, in practical application the system usually resides in the
ground state for the $\hat{H}_{0}$; so here we can extent this
condition by requiring that:
\begin{equation}\label{}
\Psi^{(0)}(r) = \psi_{k}(r)
\end{equation}
So the initial state for the system is assumed in $\ket{k}$ state.
Therefore, for the zero order approximation we have:
\begin{equation}\label{PTFTDSeq:4}
a_{i}^{(0)} = \delta_{ik}
\end{equation}

As the perturbation of $\hat{H}^{'}(t)$ is switched on at $t > 0$,
the original eigen states will be broken down; instead the new
states will be the linear combination of the original states (see
the discussion in \ref{SE:5}):
\begin{equation}\label{PTFTDSeq:1}
\Psi(r,t) = \sum_{i}a_{i}(t)\psi_{i}(r)e^{-iE_{i}t/\hbar}
\end{equation}
Here the coefficients of $a_{i}$ should be time dependent.
Furthermore, in (\ref{PTFTDSeq:1}) the wave function is no longer
the eigen states for Hamiltonian anymore, the new state will be
partially in state $\psi_{i}$ or partially in state $\psi_{j}$,
which is decided by the coefficient of $a_{i}$, it is in turn
influenced by the time of $t$.

In perturbation treatment, now we can expand the $\Psi(r,t)$ as well
as the coefficients of $a_{i}(t)$ in order of $\lambda$. however,
let's firstly bring the (\ref{PTFTDSeq:1}) into the Schrodinger
equation to see the general form:
\begin{equation}\label{}
i \hbar \frac{\partial \Psi(r,t)}{\partial t} = (\hat{H}_{0} +
\hat{H}^{'}(t))\Psi(r,t)
\end{equation}
Which it yields:
\begin{multline}\label{}
\sum_{i}i\hbar \psi_{i}(r)e^{-iE_{i}t/\hbar}\frac{\partial
a_{i}(t)}{\partial t} +
\sum_{i}a_{i}(t)E_{i}\psi_{i}(r)e^{-iE_{i}t/\hbar} = \\
\sum_{i}a_{i}(t)E_{i}\psi_{i}(r)e^{-iE_{i}t/\hbar} +
\sum_{i}a_{i}(t)e^{-iE_{i}t/\hbar}\hat{H}^{'}(t)\psi_{i}(r)
\end{multline}

Then we can drop the some terms in the above equation to make it to
be:
\begin{equation}\label{}
\sum_{i}i\hbar \psi_{i}(r)e^{-iE_{i}t/\hbar}\dot{a}_{i}(t) =
\sum_{i}a_{i}(t)e^{-iE_{i}t/\hbar}\hat{H}^{'}(t)\psi_{i}(r)
\end{equation}
By using the orthogonal condition (whatever we can always get some
orthogonal sets) for the $\psi_{i}(r)$, we can multiply both side of
the equation with $\bra{\psi_{j}(r)}$:
\begin{align}\label{PTFTDSeq:2}
i\hbar e^{-iE_{j}t/\hbar}\dot{a}_{j}(t) &=
\sum_{i}a_{i}(t)e^{-iE_{i}t/\hbar}\hat{H}^{'}_{ji}
 \quad \underrightarrow{e^{iE_{j}t/\hbar}}\nonumber \\
i\hbar \dot{a}_{j}(t) &= \sum_{i}a_{i}(t)e^{-i\omega_{ij}
t}\hat{H}^{'}_{ji}
\end{align}
Here we have $\omega_{ij}$ as:
\begin{equation}\label{}
\omega_{ij} = \frac{E_{i} - E_{j}}{\hbar}
\end{equation}

For the first order perturbation, where we can express the first
order correlated wave function as:
\begin{equation}\label{}
\Psi^{(1)}(r,t) =
\sum_{i}a^{(1)}_{i}(t)\psi_{i}(r)e^{-iE_{i}t/\hbar}
\end{equation}
So now it's easy to transform the (\ref{PTFTDSeq:2}) into the first
order equation:
\begin{equation}\label{PTFTDSeq:3}
i\hbar \dot{a}^{(1)}_{j}(t) = \sum_{i}a^{(0)}_{i}(t)e^{-i\omega_{ij}
t}\hat{H}^{'}_{ji}
\end{equation}
Here it's worthy to note that the $\hat{H}^{'}$ already has the
order of $\lambda$ so that the coefficients of $a_{i}(t)$ should be
in zero order on the right side of (\ref{PTFTDSeq:3}). Furthermore,
by using the relation in the (\ref{PTFTDSeq:4}) where $a_{i}^{(0)} =
\delta_{ik}$; we can finally get:
\begin{equation}\label{PTFTDSeq:5}
i\hbar \dot{a}^{(1)}_{j}(t) = e^{-i\omega_{kj} t}\hat{H}^{'}_{jk}
\end{equation}
Here the label of $k$ indicates that the system is originally in the
$\ket{k}$ state. by integrating the (\ref{PTFTDSeq:5}), it gives:
\begin{equation}\label{}
a^{(1)}_{j}(t) = -\frac{i}{\hbar}\int_{0}^{t}
\hat{H}^{'}_{jk}e^{-i\omega_{kj} t}dt
\end{equation}

For the (\ref{PTFTDSeq:6}), now in this equation we have two index:
$j$ and $k$. $k$ indicates the original state, while $j$ stands for
the component for $\ket{j}$ in the first order correlated wave
functions. Hence it's better to express the $a^{(1)}_{j}(t)$ as two
indices $a^{(1)}_{kj}(t)$:
\begin{equation}\label{PTFTDSeq:6}
a^{(1)}_{kj}(t) = -\frac{i}{\hbar}\int_{0}^{t}
\hat{H}^{'}_{jk}e^{-i\omega_{kj} t}dt
\end{equation}
This equation fixes the first order approximated wave function.

%%%%%%%%%%%%%%%%%%%%%%%%%%%%%%%%%%%%%%%%%%%%%%%%%%%%%%%%%%%%%%%%%%%%%%%%%%%%%%%%%%%%%%%%%%%%%%
\section{Periodic situation}
%
%
%
Now let's analyze some important situation where the perturbation
operator can be expressed as periodic function of time\cite{Landau}.

In this case, $\hat{H}^{'}$ is:
\begin{equation}\label{}
\hat{H}^{'}(t) = \hat{F}e^{-i\omega t} + \hat{G}e^{i\omega t}
\end{equation}
therefore we have $\hat{H}^{'}(t) = \hat{H}^{'}(t +
\frac{2k\pi}{\omega})$.

Here the operator of $\hat{F}$ and $\hat{G}$ are time independent.
Since that the $\hat{H}^{'}$ is required to be some physical
quantity, thus it's some hermite operator:
\begin{equation}\label{}
\hat{F}e^{-i\omega t} + \hat{G}e^{i\omega t} = \hat{F}^{+}e^{i\omega
t} + \hat{G}^{+}e^{-i\omega t} \Rightarrow \hat{F} = \hat{G}^{+}
\end{equation}
Then for the $H^{'}_{jk}$, we have:
\begin{align}\label{}
a^{(1)}_{kj}(t) &= -\frac{i}{\hbar}\int_{0}^{t}
\left\{\hat{F}_{jk}e^{-i(\omega_{kj}+\omega) t} +
\hat{F}^{*}_{kj}e^{-i(\omega_{kj}-\omega) t}\right\}dt \nonumber
\\
&=-\frac{1}{\hbar}\bigg\{
\frac{\hat{F}_{jk}e^{-i(\omega_{kj}+\omega)t}} {\omega_{kj}+\omega}
+ \frac{\hat{F}^{*}_{kj}e^{-i(\omega_{kj}-\omega)t}}
{\omega_{kj}-\omega}\bigg\}
\end{align}


%%%%%%%%%%%%%%%%%%%%%%%%%%%%%%%%%%%%%%%%%%%%%%%%%%%%%%%%%%%%%%%%%%%%%%%%%%%%%%%%%%%%%%%%%%%%%%


%%% Local Variables: 
%%% mode: latex
%%% TeX-master: "../../main"
%%% End: 
