%
% finished at Aug. 5th, 2009
%
%
\chapter{Position representation and momentum representation}
\label{position_momentum representation}
% introduce the coordinates
%
%%%%%%%%%%%%%%%%%%%%%%%%%%%%%%%%%%%%%%%%%%%%%%%%%%%%%%%%%%%%%%%%%%%%
\section{Introduction}
%
% why we have this chapter?
%
%
In the previous sections, we have brought into the concept of
Hilbert space and operator, and have discussed their features in
general. However, all the discussions have been made are abstract.
For example, the wave functions are only considered as some
``vectors'' in the Hilbert space and the discussion are mainly on
the ``vector'' property. Samely, in the discussion about the
operators, no analysis has bee made on the concrete expression of
the operators; for instance; does the Hamiltonian operator commute
with the momentum operator? such subjects has been put down so far.

Well, since this chapter we will go into such discussions. We will
derive the concrete expression for both wave functions and operators
from the above general discussion. Here we will focus on the subject
that how to associate the concrete wave function of $\Psi(\bm{r},
t)$ with the corresponding vector of $\Psi$ in the Hilbert
space?\footnote{Here the analysis within in this chapter is taken
from the Xinlin Ke's book\cite{XingLinKe}.}

The general thread for derivation within this chapter has been given
in section \ref{vector_schalar_in_operator}. Here we only follow the
clues there.

%%%%%%%%%%%%%%%%%%%%%%%%%%%%%%%%%%%%%%%%%%%%%%%%%%%%%%%%%%%%%%%%%%%%
\section{Eigen states and eigen values for position
and momentum operator}
\label{eigen_states_in_position_momentum}
%
% the changing of eigen states are continuous for x and p
%
%
Now let's concentrate on one dimension motion of a single particle for
simplicity. Suggest that $\ket{x}$ and $\ket{p}$ are some normalized
eigen states for position operator $\hat{x}$ and momentum operator
$\hat{p}$, the corresponding eigen values are $x$ and $p$. hence we
have:
\begin{align}
\label{PRAMReq:1}
\hat{x}\ket{x} &= x\ket{x} \nonumber \\
\hat{p}\ket{p} &= p\ket{p}
\end{align}
Here the $\ket{x}$ and $\ket{p}$ are just some generalization for
the wave functions we have shown in the (\ref{sec:PWF_in_Hilbert}).
There the $\nu_{p}(x)$ is just some concrete function to stand for
the $\ket{p}$ in the position representation.

Now let's go to see what kind of value can be achieved by $x$ and
$p$.

From the commutation relation of $[\hat{x}, \hat{p}] = i\hbar$, we can
use $\hat{p}$ to construct some unitary operator of $\hat{q}$:
\begin{align}
  \label{PRAMReq:2}
\hat{q}(\xi) &= e^{\frac{i}{\hbar}\xi\hat{p}} \Rightarrow \nonumber \\
\hat{q}^{+}(\xi) &= e^{-\frac{i}{\hbar}\xi\hat{p}} \Rightarrow
\nonumber \\
\hat{q}(\xi)\hat{q}^{+}(\xi) &= I
\end{align}
Here $\xi$ is some arbitrary real number. From the mathematical
analysis, it's known that we can expand the $\hat{q}(\xi)$ into series
expansion (it holds true for all $\xi$ value):
\begin{equation}
  \label{PRAMReq:3}
  \hat{q}(\xi) = 1 + \left(\frac{i}{\hbar}\xi\hat{p}\right) +
 \frac{\left(\frac{i}{\hbar}\xi\hat{p}\right)^{2}}{2!} +
\frac{\left(\frac{i}{\hbar}\xi\hat{p}\right)^{3}}{3!} + \cdots
\end{equation}

Now we have to resort to an equation which will be introduced in the
following content (\ref{OPERATORMOREeq:11}):
\begin{equation}
  \label{PRAMReq:4}
[\hei{r}, f(\hei{p})] = i\hbar \frac{\partial f(\hei{p})}{\partial
\hei{p}}
\end{equation}

Here the operation is for the vector operator, but it can be directly
applied to the scalar operator; so by the (\ref{PRAMReq:4}) we can
get:
\begin{align}
  \label{PRAMReq:5}
    [\hat{x}, \hat{q}(\xi)] &=
i\hbar \frac{\partial\hat{q}(\xi)}{\partial \hat{p}} \nonumber \\
&=i\hbar \frac{i}{\hbar}\xi\hat{q}(\xi) \nonumber \\
&=-\xi\hat{q}(\xi)
\end{align}

So we have:
\begin{align}
  \label{PRAMReq:6}
\hat{x}\hat{q}(\xi) - \hat{q}(\xi)\hat{x} &= -\xi\hat{q}(\xi)
\Rightarrow \nonumber \\
\hat{x}\hat{q}(\xi) &= \hat{q}(\xi)\hat{x} - \xi\hat{q}(\xi)
\end{align}

Hence for the $\ket{x}$, we have:
\begin{align}
  \label{PRAMReq:7}
\hat{x}\hat{q}(\xi)\ket{x} &= (\hat{q}(\xi)\hat{x} -
\xi\hat{q}(\xi))\ket{x} \nonumber \\
&=\hat{q}(\xi) (x - \xi)\ket{x} \nonumber \\
&=(x - \xi)\hat{q}(\xi)\ket{x}
\end{align}

From (\ref{PRAMReq:7}) it turns out that if $\ket{x}$ is the eigen
state for the $\hat{x}$, then $\hat{q}(\xi)\ket{x}$ is also the
$\hat{x}$ eigen state which gives the eigen value of $x - \xi$. This
holds true for any real value of $\xi$. Hence, this conclusion implies
that the position can adopt any real value for the wave functions.

What's more, the (\ref{PRAMReq:7}) indicates that through
$\hat{q}(\xi)$ the eigen state of $\ket{x}$ is transformed into the
eigen state of $\ket{x - \xi}$, that is:
\begin{equation}
  \label{PRAMReq:8}
  \hat{q}(\xi)\ket{x} = \ket{x - \xi}
\end{equation}
Thus the $\hat{q}(\xi)$ is also called ``down operator'' for the
$\ket{x}$.

Similarly, by the same procedure we can construct the ``up operator''
for the $\ket{x}$, which is $\hat{q}^{+}(\xi) =
e^{-\frac{i}{\hbar}\xi\hat{p}}$. It transforms the $\ket{x}$ into the
$\ket{x + \xi}$:
\begin{equation}
  \label{PRAMReq:9}
    \hat{q}^{+}(\xi)\ket{x} = \ket{x + \xi}
\end{equation}

Hence, from the $\ket{x}$, and the corresponding up and down operators
we can get all the eigen states for the $\ket{x}$.

As for the momentum, we can build the similar up and down operators of
$\hat{t}(\xi)$ and $\hat{t}^{+}(\xi)$:
\begin{align}
  \label{PRAMReq:10}
  \hat{t}(\xi) = e^{\frac{i}{\hbar}\xi\hat{x}} &\quad \hat{t}^{+}(\xi) =
  e^{-\frac{i}{\hbar}\xi\hat{x}} \nonumber \\
 \hat{t}(\xi)\ket{p} = \ket{p + \xi} &\quad \hat{t}^{+}(\xi)\ket{p} =
 \ket{p - \xi}
\end{align}
through same procedure, we can know that $\ket{p}$ also adopts all the
real values.

%%%%%%%%%%%%%%%%%%%%%%%%%%%%%%%%%%%%%%%%%%%%%%%%%%%%%%%%%%%%%%%%%%%%

\section{Position representation and momentum representation}
\label{sec:PRAMR_in_position_representation}
%
% 1 how to express the \ket{x} and \ket{p} in matrix form
% 2 the physical meaning of matrix element
% 3 obtain the expression of \hat{x} for \ket{x}
%
From the above section, we have generally set up some important
feature for the $\ket{x}$ and $\ket{p}$, that both of their eigen
values are continuously changed. Here in this section, we will further
investigate their details by concentrating on one question: how to
associate the abstract Hilbert vector of $\ket{x}$ with the concrete
wave function of $\Psi(x)$?

Now let's consider the complete sets of $\ket{x}$ and $\ket{p}$. Since
the $\ket{x}$ is the eigen states for the $\hat{x}$, then it's the
representation for the $\hat{x}$, we can call it ``position
representation''; on the other hand, the $\ket{p}$ is the
representation for the $\hat{p}$ so that we can call it as ``momentum
representation''.

According to the discussion in the above paragraph, the eigen states
and eigen values for the $\ket{x}$ and $\ket{p}$ are continuously
changed. So we can express the closure relation as:
\begin{align}
\label{PRAMReq:11}
  \int^{+\infty}_{-\infty}\ket{x}\bra{x}dx &= 1 \nonumber \\
  \int^{+\infty}_{-\infty}\ket{p}\bra{p}dp &= 1
\end{align}

Firstly let's consider the position representation. Let's multiply
$\ket{x^{'}}$ to the (\ref{PRAMReq:11}), it leads to:
\begin{equation}
  \label{PRAMReq:12}
    \int^{+\infty}_{-\infty}\ket{x}\bra{x} x^{'}\rangle dx =
    \ket{x^{'}}
\end{equation}

However, since the position is continuously changed, we can use the
delta function to express it:
\begin{equation}
  \label{PRAMReq:13}
  \ket{x^{'}}  = \int^{+\infty}_{-\infty}
  \ket{x}\delta(x - x^{'}) dx
\end{equation}

Hence we have:
\begin{equation}
  \label{PRAMReq:14}
 \langle x|x^{'} \rangle = \delta(x - x^{'})
\end{equation}

From (\ref{PRAMReq:14}) we have gotten the expression for the inner
product. Now let's express some arbitrary quantum state of
$\ket{\Psi}$ via $\ket{x}$:
\begin{align}
  \label{PRAMReq:15}
\ket{\Psi} &= \left\{\int^{+\infty}_{-\infty}  \ket{x}\bra{x}\,
dx\right\}\ket{\Psi}\nonumber \\
&=\int^{+\infty}_{-\infty}  \langle x|\Psi \rangle
\ket{x} dx \nonumber \\
&= \int^{+\infty}_{-\infty} \Psi_{x}\ket{x} dx
\end{align}

According to the discussion in the section of \ref{REPRESENTATION:2},
the $\Psi_{x}$ is just the representation for the $\ket{\Psi}$; so the
$\ket{\Psi}$ can be expressed as:
\begin{equation}
  \label{PRAMReq:16}
  \ket{\Psi} \Leftrightarrow \begin{pmatrix}
                               \vdots \\
                               \Psi_{x} \\
                               \vdots \\
                               \Psi_{x^{'}} \\
                               \vdots \\
                             \end{pmatrix}
\end{equation}
Here an important feature is, because the index of $x$ is continuously
changed so the the matrix element of the $\Psi_{x}$ is also continuously
changed.

The $\bra{\Psi}$ can be similarly expressed as:
\begin{equation}
  \label{PRAMReq:17}
  \bra{\Psi} \Leftrightarrow
  \begin{pmatrix}
  \cdots & \Psi^{*}_{x} & \cdots & \Psi^{*}_{x^{'}} & \cdots \\
  \end{pmatrix}
\end{equation}
The inner product of $\langle\Psi|\Psi\rangle$ is:
\begin{equation}
  \label{PRAMReq:18}
  \langle\Psi|\Psi\rangle \Leftrightarrow
  \begin{pmatrix}
  \cdots & \Psi^{*}_{x} & \cdots & \Psi^{*}_{x^{'}} & \cdots \\
  \end{pmatrix}
  \begin{pmatrix}
  \vdots \\
  \Psi_{x} \\
  \vdots \\
  \Psi_{x^{'}} \\
  \vdots \\
  \end{pmatrix}
\end{equation}
For some arbitrary operator of $\hat{A}$, we have:
\begin{equation}
  \label{PRAMReq:19}
  \begin{split}
    \langle\Psi|\hat{A}|\Psi\rangle &=
\int^{+\infty}_{-\infty}dx \int^{+\infty}_{-\infty}dx^{'}
   \Psi^{*}_{x}\Psi_{x^{'}} \langle x|\hat{A}|x^{'}\rangle
\quad
\underrightarrow{A_{xx^{'}} = \langle x|\hat{A}|x^{'}\rangle} \\
&\Leftrightarrow
\int^{+\infty}_{-\infty}dx \int^{+\infty}_{-\infty}dx^{'}
  \begin{pmatrix}
  \cdots & \Psi^{*}_{x} & \cdots & \Psi^{*}_{x^{'}} & \cdots \\
  \end{pmatrix} \\
&  \begin{pmatrix}
    \cdots & \cdots & \cdots & \cdots & \cdots \\
    \cdots & \cdots & \cdots & \cdots & \cdots \\
    \cdots & \cdots & A_{xx^{'}} & \cdots & \cdots \\
    \cdots & \cdots & \cdots & \cdots & \cdots \\
    \cdots & \cdots & \cdots & \cdots & \cdots \\
  \end{pmatrix}
  \begin{pmatrix}
  \vdots \\
  \Psi_{x} \\
  \vdots \\
  \Psi_{x^{'}} \\
  \vdots \\
  \end{pmatrix}
  \end{split}
\end{equation}

Suggest that the $\hat{A}$ in the (\ref{PRAMReq:19}) is $\hat{x}$,
then we can have:
\begin{align}
  \label{PRAMReq:26}
\langle x^{'}|\hat{x}|x \rangle &= x \langle x^{'}|x \rangle \nonumber
\\
&= x \delta(x - x^{'})
\end{align}
Hence the matrix for $\hat{x}$ is some continuous diagonal matrix with
the its elements all equal to infinity.

Now let's try to associate the $\ket{x}$ with the wave function form
of $\Psi(x)$. Firstly, let's give some physical interpretation for the
$\Psi_{x}$.  Since the $x$ inside the $\Psi_{x}$ are continuously
varied, and $\Psi_{x}$ physically characterizes the weight the
$\ket{x}$ in the $\ket{\Psi}$; so it turns out that we can set up such
one to one correspondence:
\begin{equation}
\label{PRAMReq:20} \Psi_{x} \Leftrightarrow \Psi(x)
\end{equation}
This means, the abstract vector of $\ket{\Psi}$ in Hilbert space can
be considered as some function based on the variable of $x$. The two
expressions are identical to each other. Such one to one
correspondence generally holds true for an arbitrary vector of
$\ket{\Psi}$ in any Hilbert space.

This final conclusion also give an explanation that why in the
(\ref{sec:WPDAOYE_in_basic}) and (\ref{sec:PWC_in_basic}) we express
the wave function in terms of $\bm{r}$. This can be naturally
derived from the the more strict framework of quantum mechanics.

Generally such conclusion can be extended to three dimensional
space, that is:
\begin{equation}
\label{PRAMReq:21} \Psi_{\bm{r}} \Leftrightarrow \Psi(\bm{r})
\end{equation}
For multi-particles system, we can also get:
\begin{equation}\label{PRAMReq:22}
\Psi_{\bm{r_{1}}, \bm{r_{2}}, \cdots, \bm{r_{n}}} \Leftrightarrow
\Psi(\bm{r_{1}}, \bm{r_{2}}, \cdots, \bm{r_{n}})
\end{equation}

From now on we can just write the $\Psi_{x}$ as $\Psi(x)$. However,
here there's one thing still needed to be solved, that is how to
express the operator of $\hat{x}$ and $\hat{p}$ in the $\Psi_{x}$? Now
let's go to see how to express the $\hat{x}$.

Suggest that $\hat{A}\ket{\Psi} = \ket{\Phi}$, hence we have:
\begin{equation}\label{PRAMReq:25}
\begin{split}
  \ket{\Phi} &= \hat{A}\ket{\Psi} \quad
  \underrightarrow{\text{From closure relation}} \\
  \langle x|\Phi \rangle &= \int^{+\infty}_{-\infty} \langle
  x|\hat{A}|x^{'} \rangle \langle x^{'}|\Psi
  \rangle dx^{'}\\
  \Phi_{x} &= \int^{+\infty}_{-\infty} A_{xx^{'}}\Psi_{x^{'}}dx^{'}
\end{split}
\end{equation}
Now suggest the $\hat{A}$ is just the $\hat{x}$, then according to the
(\ref{PRAMReq:26}) we can have:
\begin{equation}\label{PRAMReq:23}
  \begin{split}
      \Phi_{x} &= \int^{+\infty}_{-\infty} x \delta(x -
      x^{'})\Psi_{x^{'}}dx^{'} \\
&= x \Psi_{x}
  \end{split}
\end{equation}
Therefore, if we have $\hat{x}\ket{\Psi} = \ket{\Phi}$, then in the
position representation we can conveniently express the position
operator as $\hat{x}\Psi(x) = x\Phi(x)$. Generally, for the position
operator of $\hei{r}\ket{\Psi} = \ket{\Phi}$, we have
$\hei{r}\Psi(\bm{r}) = \bm{r}\Phi(\bm{r})$ to correspond to it.

%%%%%%%%%%%%%%%%%%%%%%%%%%%%%%%%%%%%%%%%%%%%%%%%%%%%%%%%%%%%%%%%%%%%
\subsection{How to express momentum operator in the position
  representation}
\label{sec:momentum_operator_in_position_momentum}
%
%
% how to express the momentum operator in position representation?
%
%
%
Now in this section, we are going to seek the concrete expression for
the $\hat{p}$ in the representation of $\ket{x}$. Firstly, let's
concentrate how to express $\langle x|p \rangle$.

By using the up and down operators in the
(\ref{eigen_states_in_position_momentum}), we can have:
\begin{equation}
  \label{PRAMReq:24}
  \langle x|p \rangle = \langle x|e^{\frac{i}{\hbar}p\hat{x}}|0_{p}
  \rangle
\end{equation}
Here $\ket{0_{p}}$ is just to express the eigen state gives the $0$
eigen value: $\hat{p}\ket{0_{p}} = 0\ket{0_{p}}$.

On the other hand, the operator of $e^{\frac{i}{\hbar}p\hat{x}}$ is
also some operator for the $\bra{x}$, and by using that
$\bra{x}\hat{x} = \bra{x}x$, we can have:
\begin{equation}
  \label{PRAMReq:27}
  \langle x|p \rangle = \langle x|e^{\frac{i}{\hbar}p\hat{x}}|0_{p}
  \rangle  = e^{\frac{i}{\hbar}px}\langle x|0_{p} \rangle
\end{equation}

For the $\langle x|0_{p} \rangle$, by using the up operator for
$\bra{x}$, we can have:
\begin{equation}
  \label{PRAMReq:28}
  \langle x|0_{p} \rangle = \langle 0_{x}|e^{\frac{i}{\hbar}x\hat{p}}|0_{p}
  \rangle
\end{equation}
Similarly the $e^{\frac{i}{\hbar}x\hat{p}}$ is also the operator on
the $\ket{p}$, so we have $e^{\frac{i}{\hbar}x\hat{p}}\ket{0_{p}} =
e^{\frac{i}{\hbar}x\times 0}\ket{0_{p}} = \ket{0_{p}}$; therefore we have:
\begin{equation}
  \label{PRAMReq:29}
  \langle x|p \rangle = e^{\frac{i}{\hbar}px}\langle 0_{x}|0_{p} \rangle
\end{equation}

Now let's go to see how to express the $\langle 0_{x}|0_{p} \rangle$:
\begin{equation}
  \label{PRAMReq:30}
  \begin{split}
    \langle p^{'}|p \rangle = \delta (p^{'} - p) &=
    \int^{+\infty}_{-\infty} \langle p^{'}|x \rangle \langle x|p
    \rangle dx \\
    &= \int^{+\infty}_{-\infty}
    e^{-\frac{i}{\hbar}p^{'}x}e^{\frac{i}{\hbar}px} |\langle
    0_{x}|0_{p}
    \rangle|^{2} dx \\
    &= \int^{+\infty}_{-\infty} e^{\frac{i}{\hbar}(p-p^{'})x}|\langle
    0_{x}|0_{p}
    \rangle|^{2} dx \quad
    \underrightarrow{\text{From delta function}} \\
    &= |\langle 0_{x}|0_{p} \rangle|^{2} 2\pi\hbar\delta(p^{'} - p)
    \quad \Rightarrow \\
    \langle 0_{x}|0_{p} \rangle &= \frac{1}{\sqrt[2]{2\pi\hbar}}
  \end{split}
\end{equation}

Based on the (\ref{PRAMReq:30}), let's go to see how to express the
$\langle x|\hat{p}|x^{'}\rangle$:
\begin{equation}
  \label{PRAMReq:31}
  \begin{split}
    \langle x|\hat{p}|x^{'}\rangle &= \int^{+\infty}_{-\infty}dp
    \int^{+\infty}_{-\infty}dp^{'} \langle x|p\rangle \langle
    p|\hat{p}|p^{'}\rangle \langle p^{'}|x^{'}\rangle \quad
    \text{by closure relation in \ref{PRAMReq:1}}  \\
    &= \frac{1}{2\pi\hbar}\int^{+\infty}_{-\infty}dp
    \int^{+\infty}_{-\infty}dp^{'}
    e^{\frac{i}{\hbar}px}e^{-\frac{i}{\hbar}p^{'}x^{'}}
    p^{'}\delta(p-p^{'}) \\
    &= \frac{1}{2\pi\hbar}\int^{+\infty}_{-\infty}pdp
    e^{\frac{i}{\hbar}p(x-x^{'})}  \\
    &= \frac{1}{2\pi\hbar}\int^{+\infty}_{-\infty}dp
    \frac{\hbar}{i}\frac{\partial e^{\frac{i}{\hbar}p(x-x^{'})} }
    {\partial x} \\
    &= \left(\frac{\hbar}{i}\frac{\partial}{\partial x}\right)
    \frac{1}{2\pi\hbar}\int^{+\infty}_{-\infty}dp
    e^{\frac{i}{\hbar}p(x-x^{'})}  \\
    &=\frac{1}{2\pi\hbar}\left(-i\hbar \frac{\partial}{\partial
        x}\right)
    2\pi\hbar \delta(x - x^{'}) \\
    &=-i\hbar \frac{\partial}{\partial x}\delta(x - x^{'})
  \end{split}
\end{equation}

Now let's use the (\ref{PRAMReq:31}) to get the concrete expression
for the momentum operator. Similar to the (\ref{PRAMReq:25}), we
suggest that we have $\hat{p}\ket{\Psi} = \ket{\Phi}$, so we have:
\begin{equation}\label{PRAMReq:32}
\begin{split}
  \ket{\Phi} &= \hat{p}\ket{\Psi} \quad
  \underrightarrow{\text{From closure relation}} \\
  \langle x|\Phi \rangle &= \int^{+\infty}_{-\infty} \langle
  x|\hat{p}|x^{'} \rangle \langle x^{'}|\Psi
  \rangle dx^{'}\\
  \Phi_{x} &= \int^{+\infty}_{-\infty} p_{xx^{'}}\Psi_{x^{'}}dx^{'}
  \quad
  \underrightarrow{\text{From \ref{PRAMReq:31}}} \\
  &= \int^{+\infty}_{-\infty} \left(-i\hbar \frac{\partial}{\partial
      x}\delta(x - x^{'})\right) \Psi_{x^{'}}dx^{'} \\
  &=  -i\hbar \frac{\partial}{\partial x}\Psi_{x}
\end{split}
\end{equation}
Hence we get the expression of momentum operator for wave function
of $\Psi(x)$:
\begin{equation}\label{PRAMReq:33}
\hat{p}\Psi(x) = -i\hbar \frac{\partial}{\partial x}\Psi(x)
\end{equation}

For three-dimensional space, such conclusion can be generally
extended as:
\begin{equation}
  \label{PRAMReq:34}
  \hei{p} =  -i\hbar \nabla
\end{equation}

In quantum mechanics, since all the other operators are constructed
by the position operator and the momentum operator, then from the
$\hei{r}$ and $\hei{p}$ we are able to construct the expression for
all the other operators in the position representation.

%%%%%%%%%%%%%%%%%%%%%%%%%%%%%%%%%%%%%%%%%%%%%%%%%%%%%%%%%%%%%%%%%%%%
\subsection{How to express the wave function and operators
in momentum expression}
\label{sec:momentum_operator_in_position_momentum}
%
%
% How to express the wave function and operators
% in momentum expression
%
%
%
In the above content, we have fully solved the question that how to
associate the $\ket{\Psi}$ with wave function of $\Psi(\bm{r})$, and
how to express the $\hei{r}$ and $\hei{p}$ in terms of the position
representation.

However, according to the axiom \ref{axiom5}, the position and
momentum in quantum mechanics are ``paired'' together. That means,
if we can express the $\hei{r}$ and $\hei{p}$ in terms of $\bm{r}$,
then all the other operators will be some function of $\bm{r}$; and
the corresponding wave functions will be in the form of
$\Psi(\bm{r})$; which is one to one correspondent to the
$\ket{\Psi}$. To be contrary, the judgment holds true for the
momentum $\bm{p}$. We can express the relation as:
\begin{align}\label{PRAMReq:35}
 \Psi(\bm{r}) \Leftrightarrow &\ket{\Psi}
 \Leftrightarrow \Psi(\bm{p}) \nonumber \\
\hei{r} = \hei{r}(\bm{r}), \hei{p} = \hei{p}(\bm{r})
&\Leftrightarrow \hei{r} = \hei{r}(\bm{p}), \hei{p} =
\hei{p}(\bm{p})
\end{align}

The derivation process for getting the expression for the $\hei{r}$
and $\hei{p}$ in terms of the momentum representation is nearly the
same with the above derivation for position representation. By
starting from the (\ref{PRAMReq:11}), and each step we exchange the
position representation $\ket{x}$ with the momentum representation
$\ket{p}$, we are indeed getting the conclusions shown in
(\ref{PRAMReq:35}). More specifically, we point out that in the
momentum representation, the operator of $\hei{r}$ and $\hei{p}$ are
expressed as:
\begin{align}\label{PRAMReq:36}
\hei{p} &= p \nonumber \\
\hei{r} &= -i\hbar \frac{\partial}{\partial \bm{p}}
\end{align}



%%%%%%%%%%%%%%%%%%%%%%%%%%%%%%%%%%%%%%%%%%%%%%%%%%%%%%%%%%%%%%%%%%%%

%%% Local Variables:
%%% mode: latex
%%% TeX-master: "../../main"
%%% End:
