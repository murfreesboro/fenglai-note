%
% revised at Jan 21th, 2009
% however, the discussion about the spin operator is not
% satisfied. may be it's too short, and the discussion
% about the multi-system is not good. need to further revised.
%


\chapter{Spin}
%%%%%%%%%%%%%%%%%%%%%%%%%%%%%%%%%%%%%%%%%%%%%%%%%%%%%%%%%%%%%%%%%%%%%%%%%%%%%%%%%%%%%%%%%%
% contents:
%  1  introduction:
%  2  operators in the spin, express them in the matrix
%  3  two electrons spin states
%
%
%
%
%%%%%%%%%%%%%%%%%%%%%%%%%%%%%%%%%%%%%%%%%%%%%%%%%%%%%%%%%%%%%%%%%%%%%%%%%%%%%%%%%%%%%%%%%%
\section{Introduction}
%
% 1  spin is the inherent character, and vectorial variable
% 2  in all directions, s_x, s_y, s_z; the eigen value is (+-)1/2 \hbar
% 3  spin is a kind of angular momentum, so its operator and the relation
%   between operators can be set up by analogy with the angular momentum
%
%
So far what have been dealt with is the non-relativistic quantum
theory. In this part, the spin is introduced as some ``assumption'',
and the physical essence behind it can only be unveiled in the
relativistic theory.

At first, the existence of the spin is observed in the experiments,
then according to the result of the experiments, the theoretical
treatment related to the spin has been set up. Now in the following
content we are going to give the base of the theoretical treatment
which is derived from the experiments.

First, the spin is some kind of inherent character of quantum
particles, so besides the coordinates of x, y and z the wave
functions has to add another variable $s$ to express the spin state.
this can be expressed as:
\begin{equation}\label{}
\Psi \Leftrightarrow \Psi (x,y,z,s)
\end{equation}
Here we note that the spin variable of $s$ is also some
``vectorial'' variables which is similar to the coordinate or the
momentum. Thus the operator for the spin is also vector operators.

Second, the experiments show that the spin on a specific direction
(such as z or x direction) can only take two eigen values,
$1\hbar/2$ or $-1\hbar/2$:
\begin{equation} \label{SPINeq:1}
\hat{s}_{z}\Psi = \left\{ \begin{aligned}
           & \frac{1}{2}\hbar \\
           & -\frac{1}{2}\hbar
           \end{aligned} \right.
\end{equation}
This is different with the angular momentum; where the $l_{z}$ can
adopt $0, \hbar, 2\hbar, \cdots$ at the given limit of $l$ (take
$2l+1$ values, see the previous chapter).

Usually we always use the z direction of the spin projection to
express the spin, thus it's appropriate to use the wave functions
below to express the spin states:
\begin{equation}\label{}
\Psi(\mathbf{r},s_{z}) = \begin{bmatrix}
                        \Psi(\mathbf{r},\frac{1}{2}\hbar) \\
                        \Psi(\mathbf{r},-\frac{1}{2}\hbar) \\
                      \end{bmatrix}
\end{equation}
The $\Psi(\mathbf{r},\frac{1}{2}\hbar)$ express the probability of
holding the ``u'' spin state in the position of $\mathbf{r}$, and
$\Psi(\mathbf{r},-\frac{1}{2}\hbar)$ express the probability of
holding the ``down'' spin state in the position of $\mathbf{r}$.

The orthogonal condition for the wave function is:
\begin{align}\label{}
\int \Psi^{*}(\mathbf{r},s_{z})\Psi(\mathbf{r},s_{z})d\tau &= \int
\begin{bmatrix}
  \Psi^{*}(\mathbf{r},\frac{1}{2}\hbar) & \Psi^{*}(\mathbf{r},-\frac{1}{2}\hbar) \\
\end{bmatrix}
\begin{bmatrix}
\Psi(\mathbf{r},\frac{1}{2}\hbar) \\
\Psi(\mathbf{r},-\frac{1}{2}\hbar) \\
\end{bmatrix} \nonumber \\
&= \int
\Psi^{*}(\mathbf{r},\frac{1}{2}\hbar)\Psi(\mathbf{r},\frac{1}{2}\hbar)d\tau
+ \int
\Psi^{*}(\mathbf{r},-\frac{1}{2}\hbar)\Psi(\mathbf{r},-\frac{1}{2}\hbar)d\tau
\nonumber \\
&= 1
\end{align}

If the Hamiltonian does not contain the spin operator (for example,
the molecules which do not include the heavy atoms), the wave
functions can be expressed as the product between the spatial part
and the spin part:
\begin{equation}\label{}
\Psi(\mathbf{r},s_{z}) = \psi(\mathbf{r})\chi(s_{z})
\end{equation}
$\chi(s_{z})$ is used to describe the spin state, it's common
expression is:
\begin{equation}\label{SPINeq:8}
\chi(s_{z}) = \begin{bmatrix}
                a \\
                b \\
              \end{bmatrix}
\end{equation}
Here $|a|^{2}$ indicates the probability for $s_{z} =
\frac{1}{2}\hbar$, and $|b|^{2}$ indicates the probability for
$s_{z} = -\frac{1}{2}\hbar$.

For the $\hat{s}_{z}$, its eigen states can be expressed as:
\begin{equation}\label{}
\chi_{\frac{1}{2}\hbar}(s_{z}) = \begin{bmatrix}
                1 \\
                0 \\
              \end{bmatrix}
\end{equation}
\begin{equation}\label{}
\chi_{-\frac{1}{2}\hbar}(s_{z}) = \begin{bmatrix}
                0 \\
                1 \\
              \end{bmatrix}
\end{equation}
We can abbreviate the $\chi_{\frac{1}{2}\hbar}(s_{z})$ as $\alpha$,
and $\chi_{-\frac{1}{2}\hbar}(s_{z})$ as $\beta$. Thus the $\alpha$
and $\beta$ make up a complete basis for the spin state, any spin
state which described by (\ref{SPINeq:8}) can be expressed as linear
combination between $\alpha$ and $\beta$:
\begin{equation}\label{}
\chi(s_{z}) = a\alpha + b\beta
\end{equation}

Third, the spin is a kind of special angular momentum so that its
operator and their relationship can be introduced by analogy with
the angular momentum operator. However, since that the spin has no
corresponding phenomenon in classical mechanics, we can not give the
concrete expression of spin operator just like $\hei{l},
\hat{l}^{2}$ etc. in the angular momentum part.

%%%%%%%%%%%%%%%%%%%%%%%%%%%%%%%%%%%%%%%%%%%%%%%%%%%%%%%%%%%%%%%%%%%%%%%%%%%%%%%%%%%%%%%%%%
\section{Spin operators}
%
% 1 hermite operator, supposed to be
% 2 import the operator of s, s_x, s_y, s_z, s^{2} s+, s- etc.
% 3 pauli matrix expression
%  3.1  relations for the spin operator
%  3.2  choose s_z as CSCCO, express them into the Pauli matrix
%  3.3  some simple use of the Pauli matrix
%
Similar to the angular momentum operator, we can also construct the
operators for the spin below:
\begin{align}\label{SPINeq:2}
\hei{s} &= \hat{s}_{x}\overrightarrow{i} +
\hat{s}_{y}\overrightarrow{j} + \hat{s}_{z}\overrightarrow{k}
\nonumber \\
\hat{s}^{2} &= \hei{s}\cdot\hei{s} \nonumber \\
\hat{s}_{+} &= \hat{s}_{x} + i\hat{s}_{y}  \nonumber \\
\hat{s}_{-} &= \hat{s}_{x} - i\hat{s}_{y}
\end{align}
For the operator of $\hei{s}$, $\hat{s}^{2}$ etc., they are supposed
to be hermite operators so that they can have eigen values. However,
this assumption is in accordance with the experiments.

The relations between $\hat{s}_{x}$, $\hat{s}_{y}$ and $\hat{s}_{z}$
can also follow as:
\begin{align}\label{SPINeq:3}
\hat{s}_{x}\hat{s}_{y} - \hat{s}_{y}\hat{s}_{x} &= i\hbar\hat{s}_{z}
\nonumber \\
\hat{s}_{y}\hat{s}_{z} - \hat{s}_{z}\hat{s}_{y} &= i\hbar\hat{s}_{x}
\nonumber \\
\hat{s}_{z}\hat{s}_{x} - \hat{s}_{x}\hat{s}_{z} &= i\hbar\hat{s}_{y}
\end{align}

Because that we do not have the concrete expression for the spin
operator, so it's impossible for us to discuss the relation between
spin operator and other operators (such as $\hei{p}, \hei{r},
\hat{l}^{2} etc.$); that's something we want to emphasize.

Usually in the system which contains the spin freedom, we always
choose the $\hat{s}_{z}$ and $\hat{s}^{2}$ into the CSCCO. Compared
with the angular momentum operator, we can set up the similar
relationship for eigen states and eigen values in terms of the spin
operators.

For the $\hat{s}^{2}$, we also have such eigen functions:
\begin{equation}\label{}
\hat{s}^{2}\chi = s(s+1)
\end{equation}
Where the $s = \max(s_{z})$, which is same with the result got in
the angular momentum discussion. Since that for single particle
there's only two permitted values for the $\hat{s}_{z}$, thus for
single particle $s = 1/2$.

For a multi-particle system in which the spin is involved, the total
spin operator of $\hat{S}$ on each direction, namely the
$\hat{S}_{x}$, $\hat{s}_{y}$ and $\hat{s}_{z}$; they are always
simply the addition of each particle's operator:
\begin{equation}
  \hat{S}_{x} = \sum_{i=1}^{n}\hat{s}_{x}(i) \quad
  \hat{S}_{y} = \sum_{i=1}^{n}\hat{s}_{y}(i) \quad
  \hat{S}_{z} = \sum_{i=1}^{n}\hat{s}_{z}(i)
\end{equation}
This is same with the angular momentum operators. On the other hand,
we can also define the $\hat{S}^{2}$ as:
\begin{equation}\label{}
\hat{S}^{2} = \hat{S}_{x}^{2} + \hat{S}_{y}^{2} + \hat{S}_{z}^{2}
\end{equation}
And it's eigen function is given as:
\begin{equation}\label{}
\hat{S}^{2}\chi = S(S+1)
\end{equation}
Here the $S=\max(S_{z})$.


%%%%%%%%%%%%%%%%%%%%%%%%%%%%%%%%%%%%%%%%%%%%%%%%%%%%%%%%%%%%%%%%%%%%%%%%%%%%%%%%%%%%%%%%%%
\section{Pauli matrix}
%
% how to express the spin operator in some matrix form
%
Based on the internal relations we have constructed in
(\ref{SPINeq:3}); it's possible to introduce some matrix form to
express the operator, this method was firstly suggested by Pauli.

Firstly of all, let's drop the $\hbar$ in the expression of
(\ref{SPINeq:3}) by making a new operator of $\hei{s} =
\frac{\hbar}{2}\hei{\sigma}$. Then we have:
\begin{align}\label{SPINeq:4}
\hat{\sigma}_{x}\hat{\sigma}_{y} - \hat{\sigma}_{y}\hat{\sigma}_{x}
&= 2i\hat{\sigma}_{z}
\nonumber \\
\hat{\sigma}_{y}\hat{\sigma}_{z} - \hat{\sigma}_{z}\hat{\sigma}_{y}
&= 2i\hat{\sigma}_{x}
\nonumber \\
\hat{\sigma}_{z}\hat{\sigma}_{x} - \hat{\sigma}_{x}\hat{\sigma}_{z}
&= 2i\hat{\sigma}_{y}
\end{align}

Since that for any specific direction ($x$, $y$ or $z$), the eigen
values it adopts can only be $\pm\frac{\hbar}{2}$, thus $\sigma$ has
eigen value limited to $\pm 1$; so we have:
\begin{equation}\label{SPINeq:5}
\hat{\sigma}_{x}^{2} = \hat{\sigma}_{y}^{2} = \hat{\sigma}_{z}^{2} =
1
\end{equation}
That means they are the unit operators.

By using the (\ref{SPINeq:5}) and (\ref{SPINeq:4}) together, we can
arrive the expressions below:
\begin{align}\label{SPINeq:6}
\hat{\sigma}_{x}\hat{\sigma}_{y} + \hat{\sigma}_{y}\hat{\sigma}_{x}
&= 0
\nonumber \\
\hat{\sigma}_{y}\hat{\sigma}_{z} + \hat{\sigma}_{z}\hat{\sigma}_{y}
&= 0
\nonumber \\
\hat{\sigma}_{z}\hat{\sigma}_{x} + \hat{\sigma}_{x}\hat{\sigma}_{z}
&= 0
\end{align}
That's some very interesting equations, indicating that there has
some "anti-commutation" between the $\hat{\sigma}_{x}$,
$\hat{\sigma}_{y}$ and $\hat{\sigma}_{z}$ operators.

On the other hand, from the (\ref{SPINeq:6}) and (\ref{SPINeq:4}),
we have:
\begin{align}\label{SPINeq:7}
\hat{\sigma}_{x}\hat{\sigma}_{y} &= i\hat{\sigma}_{z}
\nonumber \\
\hat{\sigma}_{y}\hat{\sigma}_{z} &= i\hat{\sigma}_{x}
\nonumber \\
\hat{\sigma}_{z}\hat{\sigma}_{x} &= i\hat{\sigma}_{y}
\end{align}
Thus the expressions above (\ref{SPINeq:4}, \ref{SPINeq:6},
\ref{SPINeq:7}) describe the characters of the spin operators.

Finally we can choose some representation to express them. Usually
we choose the $\hat{\sigma}_{z}$ into the "CSCCO" so that the wave
functions are all diagonal elements in this representation. Since
for $\hat{\sigma}_{z}$ the eigen values only equal to $\pm 1$, so we
can write it as:
\begin{equation}\label{}
\hat{\sigma}_{z} = \begin{pmatrix}
                     1 & 0 \\
                     0 & -1 \\
                   \end{pmatrix}
\end{equation}
From this expression we can see that the $\alpha$ and $\beta$ are
the eigen states for the $\hat{\sigma}_{z}$, they give the eigen
values of $1, -1$; respectively.

From the equations above, we can naturally get the matrix expression
for the $\hat{\sigma}_{y}$ and $\hat{\sigma}_{z}$:
\begin{align}\label{}
\hat{\sigma}_{y} &= \begin{pmatrix}
                     0 & -i \\
                     i & 0 \\
                   \end{pmatrix} \nonumber \\
\hat{\sigma}_{x} &= \begin{pmatrix}
                     0 & 1 \\
                     1 & 0 \\
                   \end{pmatrix}
\end{align}
That's the Pauli matrix for spin.

Here below we are going to give some simple examples to show that
how to use the Pauli matrix for the spin operator.
\begin{equation}\label{SPINeq:10}
\hat{\sigma}_{x}\alpha = \begin{pmatrix}
                     0 & 1 \\
                     1 & 0 \\
                   \end{pmatrix}
                   \begin{bmatrix}
                     1 \\
                     0 \\
                   \end{bmatrix} \nonumber \\
                      =
                   \begin{bmatrix}
                     0 \\
                     1 \\
                   \end{bmatrix}
                   =\beta
\end{equation}
Also we have $\hat{\sigma}_{x}\beta=\alpha$.
\begin{equation}\label{SPINeq:11}
\hat{\sigma}_{y}\alpha = \begin{pmatrix}
                     0 & -i \\
                     i & 0 \\
                   \end{pmatrix}
                   \begin{bmatrix}
                     1 \\
                     0 \\
                   \end{bmatrix} \nonumber \\
                      =i
                      \begin{bmatrix}
                      0 \\
                      1 \\
                   \end{bmatrix}
                   =i\beta
\end{equation}
Also we have $\hat{\sigma}_{y}\beta=-i\alpha$.

%%%%%%%%%%%%%%%%%%%%%%%%%%%%%%%%%%%%%%%%%%%%%%%%%%%%%%%%%%%%%%%%%%%%%%%%%%%%%%%%%%%%%%%%%%
\section{Spin states for two electrons}
%
% here we introduce the coupling of spin between two electrons,
% which is similar to the angular momentum (see the coupling for the
% angular momentum).
% the most important thing, is to introduce the singlet state and
% triplet state.
% also in this part, we may use the Pauli matrix.
%
Now let's consider the two electrons system, here we assume that
there's no interactions between the two electrons so that the spin
operator for the whole system can be expressed as:
\begin{equation}\label{}
\hei{s} = \hei{s}_{1} + \hei{s}_{2}
\end{equation}
$\hei{s}_{1}$ is the spin operator for electron $1$, and
$\hei{s}_{2}$ is the spin operator for electron $2$. According to
the postulation, we have:
\begin{equation}\label{}
[\hei{s}_{1}, \hei{s}_{2}] = 0
\end{equation}

It's clear that for this system, it has a degree of $2$ (here only
spin state is considered); thus we can choose $(\hat{s}_{1z},
\hat{s}_{2z})$ as CSCCO, or choose $(\hat{s}^{2}, \hat{s}_{z})$ as
CSCCO.

For the electron $1$, the eigen states for the $\hat{s}_{z}$ are
$\alpha(1)$ and $\beta(1)$, also the eigen states for the
$\hat{s}_{z}$ of electron $2$ are $\alpha(2)$ and $\beta(2)$; thus
the eigen states for the $(\hat{s}_{1z}, \hat{s}_{2z})$ are:
\begin{equation}\label{SPINeq:9}
\alpha(1)\alpha(2) \quad \alpha(1)\beta(2) \quad \alpha(2)\beta(1)
\quad \beta(1)\beta(2)
\end{equation}

According to the discussion related to the representation theory in
the quantum mechanics, different representations associated with
each other by some unitary transformation; thus it's implied that
the eigen states for the CSCCO of $(\hat{s}^{2}, \hat{s}_{z})$ can
be expressed as some linear combination of the eigen states in the
(\ref{SPINeq:9}).

For the $\hat{s}_{z}$, it's clear to see that all the states in the
(\ref{SPINeq:9}) are its eigen states, their eigen values are
$\hbar, 0, 0, -\hbar$; respectively.

For the operator of $\hat{s}^{2}$, since that $\alpha(1)\alpha(2)$
and $\beta(1)\beta(2)$ are two non-degenerate states for the
$\hat{s}_{z}$, thus both of the two states must correspond to one
eigen state for the $\hei{s}$. Hence they are the eigen states for
the operator of $\hat{s}^{2}$.

On the other hand, we can use the operator expression of
$\hat{s}^{2}$ to prove this point. for $\hat{s}^{2}$ we have:
\begin{align}\label{}
\hat{s}^{2} &= (\hei{s}_{1} + \hei{s}_{2})^{2} \nonumber \\
            &= \hat{s}_{1}^{2} + \hat{s}_{2}^{2} +
            2\hei{s}_{1}\hei{s}_{2} \nonumber \\
            &= \hat{s}_{1}^{2} + \hat{s}_{2}^{2} +
            \frac{\hbar^{2}}{2}(\sigma_{1x}\sigma_{2x}+
            \sigma_{1y}\sigma_{2y}+
            \sigma_{1z}\sigma_{2z})
\end{align}
Thus we can use this expression to testify that whether the
$\alpha(1)\alpha(2)$ and $\beta(1)\beta(2)$ are the eigen states for
the $\hat{s}^{2}$. Here we will use the eigen function in the
(\ref{SPINeq:10}) and (\ref{SPINeq:11}).

Besides the $\alpha(1)\alpha(2)$ and $\beta(1)\beta(2)$, there
should have another two eigen states for the $\hat{s}^{2}$, these
two eigen states should be the linear combination of
$\alpha(1)\beta(2)$ and $\alpha(2)\beta(1)$. So we have:
\begin{align}\label{}
\chi &= c_{1}\alpha(1)\beta(2) + c_{2}\alpha(2)\beta(1) \Rightarrow
\nonumber \\
\hat{s}^{2}\chi &= \lambda\hbar^{2}\chi
\end{align}
Through the concrete expression of $\hat{s}^{2}$, we can get:
\begin{equation}\label{}
\hat{s}^{2}\chi = \hbar^{2}\Big ( (c_{1}+c_{2})\alpha(1)\beta(2) +
(c_{1}+c_{2})\alpha(2)\beta(1)\Big)
\end{equation}

So far we can get:
\begin{equation}\label{SPINeq:12}
(c_{1}+c_{2})\alpha(1)\beta(2) + (c_{1}+c_{2})\alpha(2)\beta(1)
=c_{1}\lambda\alpha(1)\beta(2) + c_{2}\lambda\alpha(2)\beta(1)
\end{equation}

By integration to the (\ref{SPINeq:12}), first we multiply by
$\alpha^{*}(1)\beta^{*}(2)$, second by $\alpha^{*}(2)\beta^{*}(1)$;
because of the orthogonality we can get two equations:

\begin{equation} \label{SPINeq:13}
\left\{
\begin{aligned}
(1-\lambda)c_{1} + c_{2} &= 0  \\
(1-\lambda)c_{2} + c_{1} &= 0
\end{aligned} \right.
\end{equation}

The solution will be $\lambda =0, 2$, by taking them back into the
(\ref{SPINeq:13}), we will have $c_{1}/c_{2} = \pm 1$. Thus this
leads to two eigen states:
\begin{equation}\label{}
\frac{1}{\sqrt{2}}[\alpha(1)\beta(2) - \alpha(2)\beta(1)] \quad
\frac{1}{\sqrt{2}}[\alpha(1)\beta(2) + \alpha(2)\beta(1)]
\end{equation}

There are three eigen states which corresponding to ($S=1, S_{z} =0,
\pm 1$) of the $(\hat{s}^{2}, \hat{s}_{z})$, this is called triplet;
the eigen states which corresponding to ($S=0, S_{z} =0$) of the
$(\hat{s}^{2}, \hat{s}_{z})$ is called singlet.

Here these derivation are very useful in the quantum chemistry.



%%% Local Variables: 
%%% mode: latex
%%% TeX-master: "../../main"
%%% End: 
