\documentclass[a4paper,12pt]{article}
\usepackage[utf8x]{inputenc}
\usepackage{amsfonts}
\usepackage{amsmath}
\usepackage{amssymb}
\usepackage{amsthm}
\usepackage{bm}
\usepackage{extarrows}
\usepackage{graphicx}
\usepackage{xcolor}
\usepackage{indentfirst}
\usepackage{fancyhdr}
\usepackage{verbatim}
\usepackage{booktabs}
\usepackage[dvipdfm,colorlinks,linkcolor=blue,citecolor=blue]{hyperref}
\usepackage{natbib}

%%%%%%%%%%%%%%%%%%%%%%%%%%%%%%%%%%%%%%%%%%%%%%%%%%%%%%%
%%defining the bra and ket
\newcommand{\bra}[1]{\langle #1| }
\newcommand{\ket}[1]{|#1 \rangle }

%here the definition used in text
\newcommand{\brat}[1]{$\langle #1|$ }
\newcommand{\kett}[1]{$|#1 \rangle $}

%here define the math bold font for the vector operator:
\newcommand{\hei}[1]{\mathbf{\hat{#1}} }
\newcommand{\heit}[1]{$\mathbf{\hat{#1}}$}
\newcommand{\heiti}[1]{\mathbf{#1} }

%% define the mathbf type of characters
\newcommand{\vect}[1]{$\mathbf{#1}$}

%% define the theorem enviroment for the whole text
\theoremstyle{definition}\newtheorem{law}{Law}
\theoremstyle{plain}\newtheorem{theorem}{Theorem}
\theoremstyle{remark}\newtheorem{remark}{Remark}
\theoremstyle{axiom}\newtheorem{axiom}{Axiom}


%opening
\title{Matrix Element}

\begin{document}

\maketitle
Let's start by considering the equation 10 in Yihan's note, there we generally
have:
\begin{equation}
 \label{eq:1}
\langle\Psi_{I}|\Psi_{J}^{[x]}\rangle = T_{I}T_{J}^{[x]} +
\sum_{ia}\sum_{jb}T_{I,ia}T_{J,jb}\langle\Phi^{a}_{i}|\Phi^{b[x]}_{j}\rangle
\end{equation}
Here in this expression, we follow the general convention that to use $T$ to
represent the Slater determinant coefficients in CIS state, to use $C$ to
designate the MO coefficients, $\Psi$ is used to refer to CIS state, and $\Phi$
means the Slater determinants, $\varphi$ is MO.

For the index, all the capital letters such as $I,J,K,L$ etc. indicates it's
CIS states, and the lowercase letter is used to refer to MO. $\Phi^{a}_{i}$
means the electron is excited from occupied orbital $i$ to virtual orbital $a$.
For simplicity, here in this derivation we omit the spin state. 

In (\ref{eq:1}), the first term $T_{I}T_{J}^{[x]}$ generally needs to solve the
CP-CIS equation to get the response for CIS amplitude, now all the remaining
problems now concentrates on $\langle\Phi^{a}_{i}|\Phi^{b[x]}_{j}\rangle$.

\section{The first step: general consideration}
%
%
%
%
For some general Slater determinant, we have:
\begin{equation}
 \label{eq:2}
\Phi = \frac{1}{\sqrt{n!}}
\begin{vmatrix}
   \varphi_{1}(1) & \varphi_{2}(1) & \cdots & \varphi_{n}(1) \\
  \varphi_{1}(2) & \varphi_{2}(2) & \cdots & \varphi_{n}(2) \\
  \cdots & \cdots & \cdots & \cdots                                \\
  \varphi_{1}(n) & \varphi_{2}(n) & \cdots & \varphi_{n}(n) \\
\end{vmatrix}
\end{equation}
For it's derivatives, from mathematical derivation we know that:
\begin{equation}
 \label{eq:3}
\Phi^{[x]} = \sum_{p=1}^{n}\frac{1}{\sqrt{n!}}
\begin{vmatrix}
   \varphi_{1}(1) & \varphi_{2}(1) & \cdots & 
  \frac{\partial \varphi_{p}(1)}{\partial \bm{x}} &
   \cdots & \varphi_{n}(1) \\
   \varphi_{1}(2) & \varphi_{2}(2) & \cdots &
   \frac{\partial \varphi_{p}(2)}{\partial \bm{x}} &
   \cdots  & \varphi_{n}(2) \\
   \cdots & \cdots & \cdots & \cdots  & \cdots   & \cdots     \\
   \varphi_{1}(n) & \varphi_{2}(n) & \cdots & 
   \frac{\partial \varphi_{p}(n)}{\partial \bm{x}}  
   &\cdots & \varphi_{n}(n) \\
\end{vmatrix}
\end{equation}

Firstly, let's think about the second order Slater
determinants. After calculation, it turns out that
$\langle\Phi|\Phi^{[x]}\rangle$ equals to
$\sum_{p}^{2}\langle\varphi_{p}|\varphi^{[x]}_{p}\rangle$. Hence we can suggest
some hypothesis below:
\begin{theorem}
\begin{equation}
\label{eq:8}
 \langle\Phi|\Phi^{[x]}\rangle =
\sum_{p}^{n}\langle\varphi_{p}|\varphi^{[x]}_{p}
\rangle
\end{equation}
establishes for the $nth$ order Slater determinants.
\end{theorem}

Now let's go to prove it by starting from the definition of determinant:
\begin{equation}
 \label{eq:4}
\Phi = \sum_{1\leq i1<i2\cdots<in\leq n}(-1)^{P( i1, i2, \cdots,
in)}\varphi_{i1}(1)\varphi_{i2}(2)\cdots\varphi_{in}(n)
\end{equation}
$P$ is the permutation operator.

Accordingly, for $\Phi^{[x]}$, we have similar expansion:
\begin{equation}
 \label{eq:5}
\Phi^{[x]} = \sum_{p}\sum_{1\leq i1<i2\cdots<in\leq n}(-1)^{P( i1, i2, \cdots,
in)}\varphi_{i1}(1)\varphi_{i2}(2)\cdots\varphi_{p}^{'}(k)\cdots\varphi_{in}(n)
\end{equation} 
We use $\varphi_{p}^{'}(n)$ to designate the MO derivatives.

For $\langle\Phi|\Phi^{[x]}\rangle$, we can see; there's always have MO
derivatives in the ket. So generally we have have two cases:
\begin{itemize}
 \item $\ket{\varphi_{p}^{'}}$ and $\bra{\varphi_{p}}$ have same electron
residing on
 \item $\ket{\varphi_{p}^{'}}$ and $\bra{\varphi_{p}}$ have different electron
residing on
\end{itemize}
For the second situation, it's easy to see that $\bra{\varphi_{p}}$ is
impossible to find ``correct'' MO to make the integral equal to $1$. Hence the
whole expression, in the second situation will goes to zero.

For the first situation, it's easy to know that (below $p$ is fixed):
\begin{align}
 \label{eq:6}
&\frac{1}{n!}
\begin{vmatrix}
   \varphi_{1}(1) & \varphi_{2}(1) & \cdots & \varphi_{n}(1) \\
  \varphi_{1}(2) & \varphi_{2}(2) & \cdots & \varphi_{n}(2) \\
  \cdots & \cdots & \cdots & \cdots                                \\
  \varphi_{1}(n) & \varphi_{2}(n) & \cdots & \varphi_{n}(n) \\
\end{vmatrix}
\begin{vmatrix}
   \varphi_{1}(1) & \varphi_{2}(1) & \cdots & 
  \frac{\partial \varphi_{p}(1)}{\partial \bm{x}} &
   \cdots & \varphi_{n}(1) \\
   \varphi_{1}(2) & \varphi_{2}(2) & \cdots &
   \frac{\partial \varphi_{p}(2)}{\partial \bm{x}} &
   \cdots  & \varphi_{n}(2) \\
   \cdots & \cdots & \cdots & \cdots  & \cdots   & \cdots     \\
   \varphi_{1}(n) & \varphi_{2}(n) & \cdots & 
   \frac{\partial \varphi_{p}(n)}{\partial \bm{x}}  
   &\cdots & \varphi_{n}(n) \\
\end{vmatrix} \nonumber \\
&= \frac{1}{n!}\sum_{i}^{n}\langle\varphi_{p}(i)|\frac{\partial
\varphi_{p}(i)}{\partial \bm{x}}\rangle
\times\langle\bigtriangleup^{(p, i)}|\bigtriangleup^{(p,
i)}\rangle \nonumber \\
&= \frac{1}{n}\sum_{i}^{n}\langle\varphi_{p}(i)|\frac{\partial
\varphi_{p}(i)}{\partial \bm{x}}\rangle \nonumber \\
&=  \langle\varphi_{p}|\frac{\partial
\varphi_{p}}{\partial \bm{x}}\rangle
\end{align}
Here we have used the Laplace theorem to expand the Slater determinant, and 
in deriving the final conclusion we have used the relation that
electrons are indistinguishable.

Finally, combined with expression in (\ref{eq:3}), we get:
\begin{equation}
 \label{eq:7}
\langle\Phi|\Phi^{[x]}\rangle =
\sum_{p}^{n}\langle\varphi_{p}|\varphi^{[x]}_{p}\rangle
\end{equation}

\section{The second step: considering the CIS Slater determinants}
%
%
%
%
So far what we have demonstrated, is only for two identical determinants; the
only difference is that the one in the ket has MO derivatives. 
Generally, for the CIS case, the integration of Slater determinants can have
following four
choices:
\begin{itemize}
\item $\langle\Phi^{a}_{i}|\Phi^{a[x]}_{i}\rangle$, which has been proved
\item $\langle\Phi^{a}_{i}|\Phi^{b[x]}_{i}\rangle$, $a \neq b$
\item $\langle\Phi^{a}_{i}|\Phi^{a[x]}_{j}\rangle$, $i \neq j$
\item $\langle\Phi^{a}_{i}|\Phi^{b[x]}_{j}\rangle$, $i \neq j$ and $a \neq
b$
\end{itemize}
We can analyze each of the situation in the similar way.

For the second case, we may have the following possibilities:
 \begin{itemize}
 \item $\ket{\varphi_{b}^{'}}$, MO derivatives is on orbital b 
 \item $\ket{\varphi_{i}^{'}}$, MO derivatives is on orbital i 
 \item $\ket{\varphi_{p}^{'}}$, $p \neq i$ and $p \neq b$
\end{itemize} 
In the third situation, it's easy to see the integral of
$\langle\Phi^{a}_{i}|\Phi^{b[x]}_{i}\rangle$ definitely goes to
zero. Since the orbital b in the ket can not find its ``pair'' to make the
integral equal to 1. In the second situation,  while the MO derivatives is on
orbital i; for the same reason the integral will go to zero, too. Hence the MO
derivatives can be only on orbital b.

Next, let's consider the bra corresponds to $\ket{\varphi_{b}^{'}}$. It's easy
to see only the integral below is not zero:
\begin{equation}
 \label{eq:10}
\langle\varphi_{a}|\varphi_{b}^{'}\rangle
\end{equation}
Since if we have $\bra{\varphi_{p}}$ ($p \neq a$), then the orbital a in the
bra can not find its corresponding orbital in ket so that to make integral
equal to 1.Hence finally, we see:
\begin{equation}
\label{eq:11}
 \langle\Phi^{a}_{i}|\Phi^{b[x]}_{i}\rangle =
\langle\varphi_{a}|\varphi_{b}^{'}\rangle
\end{equation}
Similarly, for the $\langle\Phi^{a}_{i}|\Phi^{a[x]}_{j}\rangle$ we have:
\begin{align}
\label{eq:12}
 \langle\Phi^{a}_{i}|\Phi^{a[x]}_{j}\rangle &=
\langle\varphi_{i}|\varphi_{j}^{'}\rangle 
\end{align}
and by applying the same technique, for the
$\langle\Phi^{a}_{i}|\Phi^{b[x]}_{j}\rangle$ we can see it's always zero. 

Finally, we can express (\ref{eq:1}) as:
\begin{equation}
 \label{eq:13}
\begin{split}
 \langle\Psi_{I}|\Psi_{J}^{[x]}\rangle &= T_{I}T_{J}^{[x]} +
\sum_{ia}T_{I,ia}\sum_{p}T_{J,ia}\langle\varphi_{p}|\varphi^{[x]}_{p}\rangle
\nonumber \\
&+\sum_{iab}T_{I,ia}T_{J,ib}\langle\varphi_{a}|\varphi_{b}^{[x]}\rangle
+\sum_{ija}T_{I,ia}T_{J,ja}\langle\varphi_{i}|\varphi_{j}^{[x]}\rangle 
\end{split}
\end{equation}
 
So far we do not consider how to express the response the MO, what we do here
is just to find a way converting the Sater determinants into the MO expression.
In the next step, we will begin to consider the detailed expression of
$\varphi^{[x]}$.

\section{Detailed discussion for $\varphi^{[x]}$} 
%
%
%
Generally, the MO can be expressed as:
\begin{equation}
 \label{eq:14}
\varphi_{p} = \sum_{\mu}C_{\mu p}\phi_{\mu}
\end{equation}
Hence the MO derivatives can be expressed as:
\begin{equation}
 \label{eq:15}
\varphi_{p}^{[x]} = \sum_{\mu}C^{[x]}_{\mu p}\phi_{\mu} + \sum_{\mu}C_{\mu
p}\phi^{[x]}_{\mu}
\end{equation}
The $C^{[x]}_{\mu p}$ characterizes the MO response to the external Hamiltonian
change, and as we know; for occupied orbitals virtual part is added in, and for
virtual orbitals the occupied part added in. Hence determining from the
expression of (\ref{eq:13}), it's clear that there will be no MO response terms
appearing in the expression. Hence we can re-express the (\ref{eq:15}) as:
\begin{equation}
 \label{eq:16}
\varphi_{p}^{[x]} = \sum_{\mu}C_{\mu p}\phi^{[x]}_{\mu}
\end{equation}

Now we can bring in the (\ref{eq:16}) into the (\ref{eq:13}) to get the final
expression:
\begin{equation}
 \label{eq:17}
\begin{split}
 \langle\Psi_{I}|\Psi_{J}^{[x]}\rangle &= T_{I}T_{J}^{[x]} +
\sum_{ia}T_{I,ia}T_{J,ia}\sum_{p}^{ALL MO}
\sum_{\mu}\sum_{\nu}C_{\mu p}C_{\nu p}
\langle\phi_{\mu}|\phi^{[x]}_{\nu}\rangle
\nonumber \\
&+\sum_{iab}T_{I,ia}T_{J,ib}\sum_{\mu}\sum_{\nu}C_{\mu a}C_{\nu
b}\langle\phi_{\mu}|\varphi_{\nu}^{[x]}\rangle \nonumber \\
&+\sum_{ija}T_{I,ia}T_{J,ja}\sum_{\mu}\sum_{\nu}C_{\mu i}C_{\nu
j}\langle\phi_{\mu}|\varphi_{\nu}^{[x]}\rangle 
\end{split}
\end{equation}

\end{document}
