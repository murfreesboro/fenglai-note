%
% modified at Dec. 14th, 2009
%
%
%

%%%%%%%%%%%%%%%%%%%%%%%%%%%%%%%%%%%%%%%%%%%%%%%%%%%%%%%%%%%%%%%%%%%%%%
\chapter{Numerical Quadrature Method}


%%%%%%%%%%%%%%%%%%%%%%%%%%%%%%%%%%%%%%%%%%%%%%%%%%%%%%%%%%%%%%%%%%%%%%
\section{Introduction}
 Commonly to say, the exchange-correlation integrals we
are dealing with in the KS-DFT equation adopt such form:
\begin{equation}\label{}
    I = \int F(\rho, \bigtriangledown \rho, \cdots) d \tau
\end{equation}
Since the functional has so complicated expression, that no
analytical methods can be used to calculate its integrals. Hence the
numerical quadrature method is indispensable. The general idea of
numerical quadrature method is to transform the difficult integrals
into a summation process as approximation:
\begin{equation}\label{}
    \int^{a}_{b}y(x)dx = \sum^{n}_{i=1}A_{i}y(x_{i})
\end{equation}
Here the $x_{i}$ is the grid point where the summation is taking;
and $A_{i}$ is its integral coefficient.

According to the above thoughts, the $I = \int F(\rho,
\bigtriangledown \rho, \cdots) d \tau$ can be approximated as:
\begin{equation}\label{NQMeq:1}
\int F(\rho, \bigtriangledown \rho, \cdots) d \tau =
\sum^{n}_{i=1}A_{i}F(\rho_{i}, \bigtriangledown \rho _{i}, \cdots)
\end{equation}
For the functional F is heavily depends on the density change, so
the integral process can be well expected that two rules has to be
followed:
\begin{itemize}
  \item Near the nuclear the $\rho$ has cusp, so the integral has
  cusp near the nuclear
  \item as the $r \rightarrow \infty$, $\rho \rightarrow 0$, so 
$I \rightarrow 0$
\end{itemize}
For the reason of the first item, it's not appropriate to directly
use the grid over the whole three dimensional space. As a good
approximation method, Becke\cite{Becke} suggested an alternative way
to calculate such integrals. The general main idea will be discussed
in the following content, actually this idea has been taken in
almost all of the quantum chemistry softwares (such as \cite{g03,
QCHEM,turbomole}). As for conveniences, we will use the general
function of $F(r)$ for discussion, and take functional of
$f[\rho]=\rho ^{\frac{3}{4}}$ as an example to illustrate the whole
process.

For the common integral process, we have:
\begin{equation}\label{}
    \int \rho^{\frac{3}{4}}(r) d \tau \approx \sum_{i=1}^{n}
A_{i} \rho^{\frac{3}{4}}(r_{i})
\end{equation}
Here the $n$ denotes the number of grids, and $r_{i}$  stands for
the grid.

Since the $\rho$ has cusp near the nuclear, so a very natural idea
is to divide the multicentric integral into single center
integration over nucleus, and then sum up them together:
\begin{eqnarray}
% \nonumber to remove numbering (before each equation)
  I &=& \sum_{n} I_{n} \nonumber \\
    &=& \sum_{n} \int F_{n}(r) d \tau
\end{eqnarray}
Here the $\int F_{n}(r) d \tau$ denotes the single center
integration over nuclear. After this step, then we can use numerical
quadrature method to calculate each single center integral, then
finish the whole process. The most important question in this
process is: How to associate the $F_{n}(r)$ with the general
expression of functional of $F(r)$? In other words, how to judge
whether a single grid belongs to this atom? As for answering this
question, Becke suggested weight function method. In this method, he
express the $F_{n}(r)$ as:
\begin{equation}\label{}
F_{n}(r) = w_{n}(r)F(r)
\end{equation}
For the grid belongs to the atom n, it has $w_{n}(r) \approx 1$; and
for the grid not belongs to the atom n, $w_{n}(r) \approx 0$. For
all the atoms related to this grid, we have:
\begin{equation}\label{}
\sum_{A}w_{A}(r) = 1
\end{equation}
This process corresponds to divide the grid expression of
$A_{i}F(r_{i})$ in the (\ref{NQMeq:1}) into some pieces, each has
form of $w_{n}(r_{i})A_{i}F(r_{i})$, and if we add all the pieces
together, that is $\sum^{atoms}_{n}w_{n}(r_{i})A_{i}F(r_{i}) =
A_{i}F(r_{i}) $, we still get the summation in the (\ref{NQMeq:1}).

Relative weight function which has the form of $w_{n}(r)$,
has value unity in the vicinity of its own nucleus, but vanishes in
a continuous and well behaved manner near any other nucleus. In a
sense, the system is divided into some fuzzy, overlapping cells,
each cell contains a nuclear, and on that nuclear the numerical
quadrature method is taken on to calculating the single center
integrals; the relative weight function is used to make sure that
the grid in the calculation is belonged to the specific atom. 

Now let's use some example of $\rho^{\frac{3}{4}}$ to show the above
words, we have:
\begin{equation}
\label{NQM_general_eq}
\begin{split}
 \rho^{\frac{3}{4}}(r_{i}) &= \sum^{atoms}_{n} w_{n}(r_{i})
\rho^{\frac{3}{4}}(r_{i}) \Rightarrow  \\
I &= \int \rho^{\frac{3}{4}}(r) d^{3}r \\
  &= \sum^{grids}_{g} A_{g}\rho^{\frac{3}{4}}(r_{g}) \\
  &= \sum^{grids}_{g} A_{g} \sum^{atoms}_{n}
w_{n}(r_{g})\rho^{\frac{3}{4}}(r_{g}) \\
  &= \sum^{grids}_{g}\sum^{atoms}_{n}A_{g} 
w_{n}(r_{g})\rho^{\frac{3}{4}}(r_{g}) \\
\end{split}
\end{equation}
This is just the general method for us to evaluate the numerical
quadrature for the arbitrary functional. 


%In term of this process, we will discuss it from five aspect:
%\begin{itemize}
%  \item scheme to produce relative weight function
%  \item radical integration
%  \item numerical quadrature method to calculate the single center
%  integral
%  \item symmetry handling
%  \item linear scaling scheme
%\end{itemize}

%%%%%%%%%%%%%%%%%%%%%%%%%%%%%%%%%%%%%%%%%%%%%%%%%%%%

\section{Scheme to produce relative weight function}\label{weight_function_dft} 

The first systematic approach to introduce the relative weight
function is from Becke\cite{Becke}. where is this method we will know
how to generate the concrete $w_{n}$ for evaluating the quadrature. 

the scheme used by Becke involves the two-center coordinate
system known as confocal elliptical coordinates $(\lambda, \mu,
\phi)$. The coordinate $\phi$ denotes the angle about the
internuclear axis, and coordinate $\lambda$ and $\mu$ are defined
by:
\begin{equation}\label{}
 \lambda = \frac{r_{1}+ r_{2}}{R_{12}} \quad
\mu = \frac{r_{1} - r_{2}}{R_{12}}
\end{equation}
Where $r_{1}, r_{2}, R_{12}$ denote distance to nuclear 1, distance
to nuclear 2, and internuclear separation, respectively. Their range
is indicated below:
\begin{eqnarray}
% \nonumber to remove numbering (before each equation)
   & & 0 \leq \phi \leq 2\pi \nonumber \\
   & & 1 \leq \lambda <\infty \nonumber \\
   & & -1 \leq \mu \leq +1
\end{eqnarray}
Obviously the surface of $\mu=0$ is bisecting the internuclear axis,
so it works as sign for separating two atoms. So simply the weight
function can be:
\begin{equation}\label{NQMeq:2}
s(\mu_{ij})=\begin{cases}
1, & -1 \leq \mu_{ij} \leq 0 \\
0, & 0 \leq \mu_{ij} \leq +1
\end{cases}
\end{equation}
Hence if we use this expression, conceptually for some grid it only
belongs to only one atom so that we have the $w_{n} = 1$ for some
atom $n$. Hence the summation over the atoms will disappeared.

However, this expression is too simple so that it cause instability
in the numerical quadrature. Further Becke suggest to use some
"smooth" function to replace the expression of (\ref{NQMeq:2}), which
should obey some boundary conditions:
\begin{eqnarray}\label{NQMeq:5}
% \nonumber to remove numbering (before each equation)
  s(-1) &=& 0 \nonumber \\
  s(+1) &=& 0 \nonumber \\
  \frac{ds}{d \mu}(-1) &=& \frac{ds}{d \mu}(+1) = 0
\end{eqnarray}
Under this condition, he finally construct some polynomials
expression which satisfy the condition above. It's worthy to note
that this polynomial should vary as fast as possible so that it can
avoid the cusp near the nuclear; so the final expression is:
\begin{eqnarray}\label{NQMeq:3}
% \nonumber to remove numbering (before each equation)
  s(\mu_{ij}) &=& \frac{1}{2} \bigg[1- p 
\Big\{p \big[ p(\mu_{ij}) \big] \Big\} \bigg] \nonumber \\
  p(\mu_{ij}) &=& \frac{3}{2}\mu_{ij} - \frac{1}{2}\mu^{3}_{ij}
\end{eqnarray}
It yields a 27 degree analytical polynomial expression. Here the
$\mu_{ij}$ is just the expression of $\dfrac{r_{i} - r_{j}}{R_{ij}}$. 

So finally we have weight function defined as:
\begin{eqnarray}\label{NQM_generation_eq}
% \nonumber to remove numbering (before each equation)
  P_{i}(r) &=&  \prod_{j \neq i}s(\mu_{ij}) \nonumber \\
  w_{n}(r) &=& P_{n}(r)/\sum_{m} P_{m}(r)
\end{eqnarray}
Here i, j ,n, m all denote to the specific atoms.

Murray et al.\cite{MHL}, Gill et al.\cite{PJM},Treutler et
al.\cite{OR}, etc. all use this technic to decide the relative
weight function; however, Stratman et al.\cite{EGM} further polished
up the expression above, for speeding up the calculation efficiency.

Stratman et al. use another step function which functions similar to
the (\ref{NQMeq:3}), who has degree of 7, which is much lower than
(\ref{NQMeq:3}):
\begin{equation}\label{NQMeq:4}
z(\mu_{ij}:a) = \frac{1}{16} \Big[ 35(\frac{\mu_{ij}}{a}) - 35 (
\frac{\mu_{ij}}{a})^{3} + 21(\frac{\mu_{ij}}{a})^{5}
-5(\frac{\mu_{ij}}{a})^{7} \Big]
\end{equation}
Here they set $a$ equals to 0.64. By introducing the $z(\mu_{ij}:a)$
back into the (\ref{NQM_generation_eq}), we can get the weight for
some specific $r$.

In their paper\cite{EGM}, they presented another advantage brought
by introducing this new polynomial expression, which is "screen
weights". That means, in such scheme the weight which associated
with a certain of grid can be efficiently determined that whether it
equals to 1 or not, by the condition below:
\begin{equation}\label{}
\text{If} \quad r_{ig} < \frac{1}{2}(1-a)R_{in} \Rightarrow
w_{i}(r_{g}) = 1
\end{equation}
Where $r_{ig}$ is the distance from the grid point $g$ to its parent
atom $i$, $R_{in}$ is the distance to the nearest atomic neighbor of
$i$, and $a$ is the constant defined above. Since a significant
amount of grid has weight equals to 1, so this step avoids a lot of
calculation time, but still be able to achieve same accuracy
precision\cite{EGM}.


%%%%%%%%%%%%%%%%%%%%%%%%%%%%%%%%%%%%%%%%%%%%%%%%%%%%%%%%%%%%%%%%%%%%%%
\section{Linear scaling of the computation}
%
%
%
%
%

%%%%%%%%%%%%%%%%%%%%%%%%%%%%%%%%%%%%%%%%%%%%%%%%%%%%%%%%%%%%%%%%%%%%%%
\subsection{Batch of the grid points}
%
%
%
%
In modern quantum chemistry packages, we usually employ the Gaussian
type of atomic orbitals. However, GTO has a fast decaying nature that
for a certain grid point, only a few of basis functions (GTOs) are
making contribution to it and the others can be safely neglected
without affecting the accuracy. Based on this idea, we can introduce
the ``batch'' concept.

What does the batch mean? For an arbitrary grid point, we can
determine that what kind of GTOs ( or shells, which is a group of
GTOs who sharing the same center and angular parts) are making
contribution to this point. Such group of GTOs are called ``batch''.
On the other hand, we can also define that for a certain kind of
shell (or GTOs), what kind of grids it making contribution to. That's
two different types of definition, but should lead to the same
result. 

Now let's go to see how to put it into computation details. First, we
have to define the ``size'' of every shell (it's represented by
$\lambda$). That means, for a fixed shell, who is certering in
$\bm{R_{a}}$ and it's radical part is $\psi_{a}$; it's size is
defined as $|\psi_{a}| < \epsilon$. then for some grid point of
$\bm{r_{g}}$, we use the formula below to determine that whether a
shell contributing to the grid point or not:
\begin{equation}
 \label{linear_scale_numerical_quadrature_eq:1}
if \quad |\bm{r_{g}} - \bm{R_{a}}| \leq \lambda_{a}, then \quad
\psi_{a} \in S_{g}
\end{equation}
Finally, all the shells in this batch have the non-zero weight, and
all the others shells (GTOs) has the zero weight. Then the weight
calculation also becomes easier.  


\begin{comment}
\subsection{Batch of the grid points}
Since in the grid methods, such as Gauss-Lebedev grid; requires the
interval should be within in $-1 \leq x \leq +1$, but the radical
interval in the single center integral is to be $0 \leq r \leq
\infty$, so it's necessary to strike up a mapping between them.
Moreover, this transformation may bring some delicate effects to the
numerical integration process; which had been carefully discussed by
Treutler\cite{OR}. We will carefully consider it as we encounter
this problem.
\end{comment}

%%%%%%%%%%%%%%%%%%%%%%%%%%%%%%%%%%%%%%%%%%%%%%%%%%%%%%%%%%%%%%%%%%%%%%
\section{The single center integral}
%
%
%
Traditionally we have choices list below to achieve this process(here
only some of them are listed):
\begin{description}
  \item[Gauss-Lebedev grid]: This method is adopted by many popular
  programs.
  \item[Lobatto grid]: used in turbomole.
  \item[Euler-Maclaurin grid]: used in DGAUSS and discribled by
  Murray et al.\cite{MHL}.
\end{description}

The numerical quadrature methods related to the single center, are
very stable and the difference of result between methods of grid is
little. However, the Lebedev grid method is widely used for its high
efficiency.

Generally, the grid methods can be described in the following form:
\begin{equation}\label{}
I_{n} = \int\int\int F_{n}(r,\theta,\phi)r^{2}\sin\theta dr
d\theta d \phi \approx \sum^{N_{r}}_{i=1} A^{r}_{i}
\sum^{N_{\Omega}}_{j=1}A^{\Omega}_{j} F(r_{i}, \theta_{j}, \phi_{j})
\end{equation}
Here in this expression the $A^{r}_{i}$(related to the radical
quadrature) and $A^{\Omega}_{j}$(related to the angle quadrature)
are constant parameters, actually they are the coefficients; and
function of F has certain formation as expanding the integral into
summation.

%%%%%%%%%%%%%%%%%%%%%%%%%%%%%%%%%%%%%%%%%%%%%%%%%%%%%%%%%%%%%%%%%%%%%%
\subsection{Question With Radical Integration}
Since in the grid methods, such as Gauss-Lebedev grid; requires the
interval should be within in $-1 \leq x \leq +1$, but the radical
interval in the single center integral is to be $0 \leq r \leq
\infty$, so it's necessary to strike up a mapping between them.
Moreover, this transformation may bring some delicate effects to the
numerical integration process; which had been carefully discussed by
Treutler\cite{OR}. We will carefully consider it as we encounter
this problem.

%%%%%%%%%%%%%%%%%%%%%%%%%%%%%%%%%%%%%%%%%%%%%%%%%%%%%%%%%%%%%%%%%%%%%%

\subsection{Symmetry Handling}
The grid based methods has some limitation with the symmetry
handling, which was first pointed out by Gill et al. \cite{PGPOPLE}.
This is induced by the transformation from the the angle portion of
coordinate to the cartesian coordinate, which is closely related to
the grid scheme.

%%%%%%%%%%%%%%%%%%%%%%%%%%%%%%%%%%%%%%%%%%%%%%%%%%%%%%%%%%%%%%%%%%%%%%%%%%

\section{Linear Scaling Scheme}
So far, the most plentiful discussions about to speeding up the
numerical quadrature methods is coming from Stratmann et
al.\cite{EGM}; which is the default choice taken by Gaussian
incorporation\cite{g03}. On the other hand, Q-CHEM\cite{QCHEM}
company also published some articles related to the same
issue\cite{IncDFT}. In Stratmann's paper, they presented several
technics to speed up the calculation time, and they claimed that
they finally had achieving linear scaling in the calculations. These
technics are not complicated, just as using some new weight function
to replace the old and inefficient ones; the screen weight methods
etc.; which have been mentioned in the above content. Here in the
first manuscript we do not carefully examine these technics, it will
be carefully discussed as we step into the concrete program writing.

%%%%%%%%%%%%%%%%%%%%%%%%%%%%%%%%%%%%%%%%%%%%%%%%%%%%%%%%%%%%%%%%%%%%%%%%%%%%%%%%%%%


%%% Local Variables: 
%%% mode: latex
%%% TeX-master: "../../main"
%%% End: 
