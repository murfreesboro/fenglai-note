%%%%%%%%%%%%%%%%%%%%%%%%%%%%%%%%%%%%%%%%%%%%%%%%%%%%%%%%%%%%%%%%
%problems remained:
%
%
%#  for CPHF equation, why the MO rotation matrix of U can only
%   be solved for occ-vir block? Can we find some physical or
%   mathematical interpretation? (solved)
%
%
%
%
%%%%%%%%%%%%%%%%%%%%%%%%%%%%%%%%%%%%%%%%%%%%%%%%%%%%


\chapter{Analytical Gradient for Self-Consistent Field Equation}
%
%
%

%%%%%%%%%%%%%%%%%%%%%%%%%%%%%%%%%%%%%%%%%%%%%%%%%%%%
\section{Total Energy and Fock Matrix for Hatree-Fock-Roothaan and Kohn-Sham
Equations}
%
%
%
%
Now in this chapter, we begin to evaluate the analytical gradients for energy
and Fock matrix in Hatree-Fock and Kohn-Sham framework. According to the
conclusion given in chapter \ref{HFT}, the total energy For Hatree-Fock equation
is expressed as:
\begin{equation}
 \label{TE_FM_HF_gradient_eq:1}
E = \sum_{i}^{occ}h_{ii} + \frac{1}{2}
    \sum_{ij}^{occ}
    \left \{ ( ii|jj) - ( ij|ji)
    \right \} 
\end{equation}
This is expressed in MO orbital compared with the AO expression in
(\ref{HFTeq:11-1}), and the Fock matrix for the MO $p$ and $q$ is expressed as:
\begin{equation}
\label{TE_FM_HF_gradient_eq:2}
F_{pq} = \langle\varphi_{p}|\hat{F}|\varphi_{q}\rangle = h_{pq} + 
\sum_{k}^{occ}\left \{  (pq|kk) - ( pk|kq)\right \} 
\end{equation}
It can be directly got by integrated with $ \langle\varphi_{p}|$ to the
Hatree-Fock equation in (\ref{HFTeq:10}) (the MO orbital could be occupied or
virtual, so we use $p,q$ etc. to designate it), and because of the orthogonality
of the MO; we have:
\begin{equation}
 \label{TE_FM_HF_gradient_eq:3}
F_{pq} = 
\begin{cases}
 0  &\text{if $p \neq q$} \\
\epsilon_{p} &\text{if $p = q$}
\end{cases}
\end{equation}

For Kohn-Sham equation, the total energy is expressed by adding the
exchange-correlation functional to (\ref{TE_FM_HF_gradient_eq:1}):
\begin{equation}
 \label{TE_FM_HF_gradient_eq:4}
E = \sum_{i}^{occ}h_{ii} + \frac{1}{2}
    \sum_{ij}^{occ}
    \left \{ ( ii|jj) - a_{x}( ij|ji)
    \right \} + \sum_{i}^{occ}E_{XC}[\rho_{i}] 
\end{equation}
Where the $\rho_{i}$ is expressed through occupied orbitals:
\begin{equation}
 \begin{split}
  \rho_{i} &= |\varphi_{i}|^{2} \\
              &= \sum_{\mu\nu}c_{\mu i}c_{\nu i}\phi_{\mu}\phi_{\nu}
 \end{split}
 \label{TE_FM_HF_gradient_eq:5}
\end{equation}
$a_{x}$ is the general parameter for specifying the hybrid functional.

Similarly, for the corresponding Fock matrix, it's:
\begin{equation}
\label{TE_FM_HF_gradient_eq:6}
F_{pq} = \langle\varphi_{p}|\hat{F}|\varphi_{q}\rangle = h_{pq} + 
\sum_{k}^{occ}\left \{  (pq|kk) - a_{x}( pk|kq)\right \} + F^{XC}_{pq} 
\end{equation}
$F^{XC}$ is the corresponding exchange-correlation Fock matrix, which satisfy
the relation in (\ref{TE_FM_HF_gradient_eq:3}), too.

 
%%%%%%%%%%%%%%%%%%%%%%%%%%%%%%%%%%%%%%%%%%%%%%%
\section{The First Order Derivatives for Energy}
%
%
%
%%%%%%%%%%%%%%%%%%%%%%%%%%%%%%%%%%%%%%%%%%%%%%%
\subsection{The First Order Derivatives for MO Integrals}
%
%
%
As what we can see in the (\ref{TE_FM_HF_gradient_eq:1}), we only
have one or two indices MO integrals, so it will be help for us to study the
derivatives for MO integrals before we step into the HF derivatives.

For the core Hamiltonian integral, the $h_{ii}^{[a]}$ is:
\begin{equation}
 \label{1st_MO_energy_HF_gradient_eq:1}
h_{ii}^{[a]} = 2\sum_{r}^{ALL MO}H_{ir}U^{a}_{ri} + H^{a}_{ii}
\end{equation}
This is got by simply set the index of $p$ and $q$ to $i$ in
(\ref{core_Hamiltonian_MO_INT_gradient_eq:1}).

For the double electron integral, it has:
\begin{align}
  \label{1st_MO_energy_HF_gradient_eq:2}
(ii|jj)^{[a]} &= 2\sum_{r}^{ALL MO}\left[ U^{a}_{ri}\Pi_{rijj} +
U^{a}_{rj}\Pi_{iirj}\right] + \Pi^{a}_{iijj} \nonumber \\
(ij|ji)^{[a]} &= 2\sum_{r}^{ALL MO}\left[ U^{a}_{ri}\Pi_{rjji} +
U^{a}_{rj}\Pi_{irji}\right]  + \Pi^{a}_{ijji}
\end{align}

The first order derivatives for the exchange-correlation part is got in
(\ref{1st_XC_general_derivative}), which is:
\begin{equation}
  E_{XC}^{[x]} =  \sum\omega^{[x]}(r)F_{XC}(r) +
\sum\omega(r)F^{[x]}_{XC}(r) 
\label{1st_MO_energy_HF_gradient_eq:3}
\end{equation}
Where the $F^{[x]}_{XC}(r)$ is:
\begin{align}
F^{[x]}_{XC}(r) &= \sum_{\xi}\sum_{\sigma}\frac{\partial F}{\partial
\xi_{\sigma}} \frac{\partial \xi_{\sigma}}{\partial \bm{R}_{x}} \nonumber\\
&=  \sum_{\xi}\sum_{\sigma} V_{XC}(\xi_{\sigma})\xi_{\sigma}^{[x]}
 \label{1st_MO_energy_HF_gradient_eq:4}
\end{align}


%%%%%%%%%%%%%%%%%%%%%%%%%%%%%%%%%%%%%%%%%%%%%%%%%%
\subsection{The First Order Derivatives for Total Energy in Hatree-Fock
framework}
%
%
%
According to the (\ref{TE_FM_HF_gradient_eq:1}), we have:
\begin{equation}
 \label{FOD_energy_HF_derivatives_eq:1}
E^{[a]} = \sum_{i}^{occ}h_{ii}^{[a]} + \frac{1}{2}
    \sum_{ij}^{occ}
    \left \{ ( ii|jj)^{[a]} - ( ij|ji)^{[a]}
    \right \} 
\end{equation}
Let's bring (\ref{1st_MO_energy_HF_gradient_eq:1}) and
(\ref{1st_MO_energy_HF_gradient_eq:2}) into this equation, then it gives:
\begin{equation}
\label{FOD_energy_HF_derivatives_eq:2}
 \begin{split}
  E^{[a]} &= \sum_{i}^{occ}\left( 2\sum_{r}^{ALL MO}H_{ir}U^{a}_{ri} +
H^{a}_{ii} \right) + \frac{1}{2}\sum_{ij}^{occ}\left\lbrace 2\sum_{r}^{ALL
MO}\left[
U^{a}_{ri}\Pi_{rijj} +
U^{a}_{rj}\Pi_{iirj}\right]  \right. \\ 
&\left. + \Pi^{a}_{iijj} - 2\sum_{r}^{ALL MO}\left[
U^{a}_{ri}\Pi_{rjji} + U^{a}_{rj}\Pi_{irji}\right]  -
\Pi^{a}_{ijji}\right\rbrace \\
&= \sum_{i}^{occ}H^{a}_{ii} + \frac{1}{2}\sum_{ij}^{occ}\left( \Pi^{a}_{iijj} -
\Pi^{a}_{ijji}\right) \\
&+ \sum_{r}^{ALL MO}\left\lbrace  \sum_{i}^{occ}H_{ir}U^{a}_{ri} +
\sum_{ij}^{occ}\left[ U^{a}_{ri}\Pi_{rijj} - U^{a}_{ri}\Pi_{rjji}\right] 
\right\rbrace \\
&+ \sum_{r}^{ALL MO}\left\lbrace  \sum_{j}^{occ}H_{jr}U^{a}_{rj} +
\sum_{ij}^{occ}\left[ U^{a}_{rj}\Pi_{iirj} - U^{a}_{rj}\Pi_{irji}\right] 
\right\rbrace
 \end{split}
\end{equation}
Here in the sum of $2\sum_{i}^{occ}\sum_{r}^{ALL MO}H_{ir}U^{a}_{ri}$, since
the label $i$ is arbitrary; then we split it into two parts; one is
$\sum_{i}^{occ}\sum_{r}^{ALL MO}H_{ir}U^{a}_{ri}$ and the other is
$\sum_{j}^{occ}\sum_{r}^{ALL MO}H_{jr}U^{a}_{rj}$.

For the last two summation terms in (\ref{FOD_energy_HF_derivatives_eq:2}), in
terms of the invariance of the $\Pi$ integral ($\Pi_{irji} = \Pi_{irij} =
\Pi_{riji} = \Pi_{jiri}$ etc.) and the Fock matrix in
(\ref{TE_FM_HF_gradient_eq:2}), we can have:
\begin{align}
\label{FOD_energy_HF_derivatives_eq:3}
 &\sum_{r}^{ALL MO}\left\lbrace  \sum_{i}^{occ}H_{ir}U^{a}_{ri} +
\sum_{ij}^{occ}\left[ U^{a}_{ri}\Pi_{rijj} - U^{a}_{ri}\Pi_{rjji}\right] 
\right\rbrace \nonumber \\
&= \sum_{r}^{ALL MO}\sum_{i}^{occ}U^{a}_{ri}\left\lbrace H_{ir} + 
\sum_{j}^{occ} \left[ \Pi_{rijj} - \Pi_{rjji}\right]  \right\rbrace \nonumber \\
&= \sum_{r}^{ALL MO}\sum_{i}^{occ}U^{a}_{ri}F_{ri} 
\end{align}
\begin{align}
 \label{FOD_energy_HF_derivatives_eq:4}
&\sum_{r}^{ALL MO}\left\lbrace  \sum_{j}^{occ}H_{jr}U^{a}_{rj} +
\sum_{ij}^{occ}\left[ U^{a}_{rj}\Pi_{iirj} - U^{a}_{rj}\Pi_{irji}\right] 
\right\rbrace \nonumber \\
&= \sum_{r}^{ALL MO}\sum_{j}^{occ}U^{a}_{rj}\left\lbrace H_{jr} +
\sum_{i}^{occ}\left[\Pi_{rjii} - \Pi_{riij}\right] 
\right\rbrace \nonumber \\ 
&= \sum_{r}^{ALL MO}\sum_{j}^{occ}U^{a}_{rj}F_{rj}
\end{align}

What's more, since the $i$ and $j$ index are arbitrary, it's easy to see that
the (\ref{FOD_energy_HF_derivatives_eq:3}) and
(\ref{FOD_energy_HF_derivatives_eq:4}) are identical with each other, so the
(\ref{FOD_energy_HF_derivatives_eq:1}) finally becomes:
\begin{align}
 \label{FOD_energy_HF_derivatives_eq:5}
E^{[a]} &= \sum_{i}^{occ}H^{a}_{ii} + \frac{1}{2}\sum_{ij}^{occ}\left(
\Pi^{a}_{iijj} -
\Pi^{a}_{ijji}\right) \nonumber \\
&+ 2\sum_{r}^{ALL MO}\sum_{j}^{occ}U^{a}_{rj}F_{rj} \nonumber \\
&= \sum_{i}^{occ}H^{a}_{ii} + \frac{1}{2}\sum_{ij}^{occ}\left( \Pi^{a}_{iijj} -
\Pi^{a}_{ijji}\right) + 2\sum_{j}^{occ}U^{a}_{jj}\epsilon_{j}
\end{align}
Here we have used the relation (\ref{TE_FM_HF_gradient_eq:3}) for Fock matrix.

According the winger's theorem, we know that the first order perturbed term,
such as $U^{a}_{jj}$; should not appear in the gradient expression. Remember in
(\ref{overlap_MO_INT_gradient_eq:3}), we have:
\begin{equation}
 U^{a}_{qp} + U^{a}_{pq} + S^{a}_{pq} = 0
\end{equation}
For the $p=q=j$ case, we have $2U^{a}_{jj} = -S^{a}_{jj}$. Hence finally, the
gradient for Hatree-Fock energy becomes:
\begin{equation}
 \label{FOD_energy_HF_derivatives_eq:6}
E^{[a]} = \sum_{i}^{occ}H^{a}_{ii} + \frac{1}{2}\sum_{ij}^{occ}\left(
\Pi^{a}_{iijj} -
\Pi^{a}_{ijji}\right) - \sum_{j}^{occ}S^{a}_{jj}\epsilon_{j}
\end{equation}
This is easy to extend to the AO expression.
 
%%%%%%%%%%%%%%%%%%%%%%%%%%%%%%%%%%%%%%%%%%%%%%%
\subsection{The First Order Derivatives for Kohn-Sham equation}
%
%
%
Compared with Hatree-Fock equation, Kohn-Sham equation only adds the
exchange-correlation contribution. From the viewpoint of gradient theory, the
exchange-correlation part is same with the core Hamiltonian integral. Now let's
go to see the details.

By driving the terms related to $U$ matrix together, the
(\ref{1st_MO_energy_HF_gradient_eq:3}) can be further expressed as:
\begin{align}
E^{[x]}_{XC}(r)
&=  \sum_{\xi}\sum_{\sigma} V_{XC}(\xi_{\sigma})\xi_{\sigma}^{[x]} \nonumber \\
&=  \sum_{r}\omega^{[x]}(r)F_{XC}(r) + \sum_{r}\sum_{\xi}\sum_{\sigma}
\omega(r) V_{XC}(\xi_{\sigma})\xi_{\sigma}^{x} \nonumber \\
&+ \sum_{r}\sum_{\xi}\sum_{\sigma}
\omega(r) V_{XC}(\xi_{\sigma})(\xi^{U}_{\sigma})^{x} \nonumber \\
&= E^{x}_{XC}(r) + \sum_{r}\sum_{\xi}\sum_{\sigma}
\omega(r) V_{XC}(\xi_{\sigma})(\xi^{U}_{\sigma})^{x}
 \label{FOD_energy_KS_derivatives_eq:1}
\end{align}

Now let's consider the (\ref{FOD_energy_HF_derivatives_eq:3}) and
(\ref{FOD_energy_HF_derivatives_eq:4}) so that to add in exchange-correlation
part:
\begin{align}
\label{FOD_energy_KS_derivatives_eq:2}
 &\sum_{r}^{ALL MO}\left\lbrace  \sum_{i}^{occ}H_{ir}U^{a}_{ri} +
\sum_{ij}^{occ}\left[ U^{a}_{ri}\Pi_{rijj} - U^{a}_{ri}\Pi_{rjji}\right] 
\right\rbrace  \nonumber \\
&+ \sum_{r}\sum_{\xi}\sum_{\sigma}
\omega(r) V_{XC}(\xi_{\sigma})(\xi^{U}_{\sigma})^{x}\nonumber \\
&= \sum_{r}^{ALL MO}\sum_{i}^{occ}U^{a}_{ri}\left\lbrace H_{ir} + 
\sum_{j}^{occ} \left[ \Pi_{rijj} - \Pi_{rjji}\right]  \right\rbrace \nonumber \\
&= \sum_{r}^{ALL MO}\sum_{i}^{occ}U^{a}_{ri}F_{ri} 
\end{align}

Similarly,
\begin{align}
 \label{FOD_energy_HF_derivatives_eq:4}
&\sum_{r}^{ALL MO}\left\lbrace  \sum_{j}^{occ}H_{jr}U^{a}_{rj} +
\sum_{ij}^{occ}\left[ U^{a}_{rj}\Pi_{iirj} - U^{a}_{rj}\Pi_{irji}\right] 
\right\rbrace \nonumber \\
&= \sum_{r}^{ALL MO}\sum_{j}^{occ}U^{a}_{rj}\left\lbrace H_{jr} +
\sum_{i}^{occ}\left[\Pi_{rjii} - \Pi_{riij}\right] 
\right\rbrace \nonumber \\ 
&= \sum_{r}^{ALL MO}\sum_{j}^{occ}U^{a}_{rj}F_{rj}
\end{align}



\begin{equation}
 \label{FOD_energy_KS_derivatives_eq:1}
E^{[a]} = \sum_{i}^{occ}H^{a}_{ii} + \frac{1}{2}\sum_{ij}^{occ}\left(
\Pi^{a}_{iijj} -
\Pi^{a}_{ijji}\right) - \sum_{j}^{occ}S^{a}_{jj}\epsilon_{j} + E_{XC}^{a}
\end{equation}
The $E_{XC}^{a}$ can be got in (\ref{1st_MO_energy_HF_gradient_eq:3}).

%%%%%%%%%%%%%%%%%%%%%%%%%%%%%%%%%%%%%%%%%%%%%%%
\section{The First Order Derivatives for Fock Matrix}
%
%
%
%%%%%%%%%%%%%%%%%%%%%%%%%%%%%%%%%%%%%%%%%%%%%%%
\subsection{The First Order Derivatives of MO Integrals for Fock Matrix}
%
%
%
%
Before we step into the derivatives for Fock matrix,
it's better for us to see the corresponding derivatives for MO integrals.
\begin{equation}
\label{FOD_MO_FOCK_derivatives_eq:1}
h_{pq}^{[a]}  = \sum_{t}^{ALL MO}\left[U^{a}_{tp}H_{qt} +
  U^{a}_{tq}H_{pt}\right]
  + H^{a}_{pq}  
\end{equation}
The core Hamiltonian derivatives are same with
(\ref{core_Hamiltonian_MO_INT_gradient_eq:1}).  

For the double electron integrals, we have:
\begin{align}
  \label{FOD_MO_FOCK_derivatives_eq:2}
\Pi_{pqkk}^{[a]} &= \sum_{t}^{ALL MO}\left[ 
U^{a}_{tp}\Pi_{tqkk} +
U^{a}_{tq}\Pi_{ptkk} + 
2U^{a}_{tk}\Pi_{pqtk}   
\right] + \Pi^{a}_{pqkk} \nonumber \\
\Pi_{pkkq}^{[a]} &= \sum_{t}^{ALL MO}\left[ 
U^{a}_{tp}\Pi_{tkkq} +
U^{a}_{tq}\Pi_{pkkt} + 
U^{a}_{tk}\left(\Pi_{ptkq} + \Pi_{pktq}\right)   
\right] + \Pi^{a}_{pkkq}
\end{align}
    
%%%%%%%%%%%%%%%%%%%%%%%%%%%%%%%%%%%%%%%%%%%%%%%%%%%%%%%%%%%%%%%%%
\subsection{The First Order Derivatives of Fock Matrix}
%
%
%
%
Now let's consider the $F_{pq}$, generally it can be expressed as:
\begin{equation}
    \label{FOD_FOCK_hf_derivatives_eq:1}
F_{pq}^{[a]} = h_{pq}^{[a]} + 
\sum_{k}^{occ}\left \{  (pq|kk)^{[a]} - ( pk|kq)^{[a]}\right \} 
\end{equation}

After putting the (\ref{FOD_MO_FOCK_derivatives_eq:1}) and
(\ref{FOD_MO_FOCK_derivatives_eq:2}) into this expression, it becomes: 
\begin{align}
  \label{FOD_FOCK_hf_derivatives_eq:2}
F_{pq}^{[a]} &= \sum_{t}^{ALL MO}\left[U^{a}_{tp}H_{qt} +
  U^{a}_{tq}H_{pt}\right] + H^{a}_{pq} \nonumber \\  
  &+ 
\sum_{k}^{occ}\left \{ \sum_{t}^{ALL MO}\left[ 
U^{a}_{tp}\Pi_{tqkk} +
U^{a}_{tq}\Pi_{ptkk} + 
2U^{a}_{tk}\Pi_{pqtk}   
\right] + \Pi^{a}_{pqkk} \right. \nonumber \\
  &\left. -
\sum_{t}^{ALL MO}\left[ 
U^{a}_{tp}\Pi_{tkkq} +
U^{a}_{tq}\Pi_{pkkt} + 
U^{a}_{tk}\left(\Pi_{ptkq} + \Pi_{pktq}\right)   
\right] - \Pi^{a}_{pkkq}
  \right \} 
\end{align}

By rearrange the items in terms of the $U$ matrix, the
(\ref{FOD_FOCK_hf_derivatives_eq:2}) further becomes:
\begin{align}
  \label{FOD_FOCK_hf_derivatives_eq:3}
F_{pq}^{[a]} &= H^{a}_{pq} + 
\sum_{k}^{occ}\left \{\Pi^{a}_{pqkk} - \Pi^{a}_{pkkq} \right\}
\nonumber \\
&+  \sum_{t}^{ALL MO}U^{a}_{tp}\left\{H_{qt} + 
  \sum_{k}^{occ}\left[\Pi_{tqkk} -
   \Pi_{tkkq} \right]\right \}\nonumber \\
&+  \sum_{t}^{ALL MO}U^{a}_{tq}\left\{H_{pt} + 
  \sum_{k}^{occ}\left[\Pi_{ptkk} -
   \Pi_{pkkt} \right]\right \}\nonumber \\
&+ \sum_{t}^{ALL MO}U^{a}_{tk}\left\{ 
\sum_{k}^{occ}\left[ 
2\Pi_{pqtk} - \Pi_{ptkq} - \Pi_{pktq}\right]
\right\}
\end{align}

Again, let's apply the similar methods which we have used in
(\ref{FOD_energy_HF_derivatives_eq:3}) and
(\ref{FOD_energy_HF_derivatives_eq:4}); then the
(\ref{FOD_FOCK_hf_derivatives_eq:3}) gives:
\begin{align}
  \label{FOD_FOCK_hf_derivatives_eq:4}
F_{pq}^{[a]} &= H^{a}_{pq} + 
\sum_{k}^{occ}\left \{\Pi^{a}_{pqkk} - \Pi^{a}_{pkkq} \right\}
\nonumber \\
&+  \sum_{t}^{ALL MO}\left(U^{a}_{tp}F_{qt} + U^{a}_{tq}F_{pt}\right)
+ \sum_{k}^{occ}\sum_{t}^{ALL MO}U^{a}_{tk}A_{pq,kt} 
\end{align}
Where the $A_{pq,kt}$ is given by:
\begin{equation}
    \label{FOD_FOCK_hf_derivatives_eq:5}
A_{pq,kt}  = 2\Pi_{pqtk} - \Pi_{ptkq} - \Pi_{pktq}
\end{equation}

This is the final result for Fock matrix derivatives. Here we
preserved the $F_{pt}$ etc. and did not transform it to
$\delta_{pt}\epsilon_{p}$. In second derivatives evaluation, 
we will use the result here is derived.   

%%%%%%%%%%%%%%%%%%%%%%%%%%%%%%%%%%%%%%%%%%%%%%%%%%%%%%%%%%%%%%%%%
\section{The Second Order Derivatives for Hatree-Fock Energy}
%
%
%
Generally, the second derivatives for Hatree-Fock energy can be
taken by differentiate the first derivatives of $E^{[a]}$
\begin{comment}
\footnote{
Here we do not start from the general expression below:
\begin{equation}
E^{[ab]} = \sum_{i}^{occ}H_{ii}^{[ab]} + \frac{1}{2}
    \sum_{ij}^{occ}
    \left \{ ( ii|jj)^{[ab]} - ( ij|ji)^{[ab]}
    \right \} 
\end{equation}
Actually I doubt that this expression is wrong. I have tried to bring the
$\Pi^{[ab]}$ and $H^{[ab]}$ into this expression, but it does not give the
correct result. I think that the reason can be expressed in the following way.
Firstly, for the integral of  $\Pi$ and $H$, they are actually some multivariate
function of $\bm{R}$. For the first derivatives, they are able to keep the
invariance of differential form, so we can generally express them as $H^{[a]}$
and $\Pi^{[a]}$; but for the second derivatives, such invariance breaks; so we
can not differentiate $\Pi^{[a]}$ to get $\Pi^{[ab]}$ to obtain the result. This
is like to differentiate the function:
\begin{equation}
f(x_{1}, \cdots, x_{n}, y_{1}(x_{1},
\cdots, x_{n}), \cdots, y_{m}(x_{1}, \cdots, x_{n}), z_{1}(x_{1},
\cdots, x_{n}), \cdots, z_{l}(x_{1}, \cdots, x_{n})) 
\end{equation}
Here $y$ is similar to MO $i$, and $z$ is similar to MO $j$, and $x$ is the
variable of $\bm{R}$. 

On the other hand, we can observe that  the MO $i$ and
$j$ are symmetrical in the energy expression. Hence in the second derivatives,
The factor of $1/2$ may diminish. Based on this point, we suspect that the
section 4.7 in Fritz's book\cite{New_Dimension_for_Derivatives_Calculation} may
be wrong.}
\end{comment}
:
\begin{equation}
   \label{SOD_energy_hf_derivatives_eq:1}
\begin{split}
E^{[ab]} &= \frac{\partial E^{[a]}}{\partial
\bm{R}_{b}} \\
&=\sum_{i}^{occ}H_{ii}^{a[b]} + \frac{1}{2}
    \sum_{ij}^{occ}
    \left \{ ( ii|jj)^{a[b]} - ( ij|ji)^{a[b]}
    \right \}   \\
&+ 2\sum_{r}^{ALL
MO}\sum_{j}^{occ}\left\lbrace F_{rj}\frac{\partial U^{a}_{rj}}{\partial
\bm{R}_{b}} + U^{a}_{rj}\frac{\partial F_{rj}}{\partial
\bm{R}_{b}}\right\rbrace 
\end{split}
\end{equation}
The $M^{a[b]}$ indicates that the partial derivatives of ``a'' is only on the
AO integrals, and ``b'' is on the MO integrals. Now let's focus on each MO
integral pieces appearing in this general expression. 

For the core Hamiltonian integral, its derivatives is expressed
as:
\begin{equation}
  \begin{split}
    H_{ii}^{a[b]} &=2\sum_{r}U^{b}_{ri}H^{a}_{ir}  + H^{ab}_{ii}
  \end{split}
     \label{SOD_energy_hf_derivatives_eq:2}
\end{equation}
This is done by setting the label of $p$ and $q$ to $i$ in the
(\ref{core_Hamiltonian_MO_INT_gradient_eq:3}).  

For $(ii|jj)$, its derivatives is:
\begin{equation}
  \label{SOD_energy_hf_derivatives_eq:3}
  \begin{split}
    (ii|jj)^{a[b]} = \Pi_{iijj}^{a[b]} &= 2\sum_{r}\left[ 
U^{b}_{ri}\Pi^{a}_{rijj} +
U^{b}_{rj}\Pi^{a}_{iirj}  
\right] + \Pi^{ab}_{iijj}
  \end{split}
\end{equation}
This is achieved by setting the label of $p$, $q$ to $i$ and $r$, $s$
to $j$ in (\ref{two_electron_MO_INT_gradient_eq:9}).

Similarly, by setting the $p$ and $s$ to $i$, $q$ and $r$ to $j$; from
the (\ref{two_electron_MO_INT_gradient_eq:9}) we can obtain the
second derivatives for $(ij|ji)$:
\begin{equation}
 \label{SOD_energy_hf_derivatives_eq:4}
  \begin{split}
    (ij|ji)^{a[b]} = \Pi_{ijji}^{a[b]} &= 2\sum_{r}\left[ 
U^{b}_{ri}\Pi^{a}_{rjji} +
U^{b}_{rj}\Pi^{a}_{irji}  
\right] + \Pi^{ab}_{ijji}
  \end{split}
\end{equation}

Now let's firstly deal with these three terms:
\begin{equation}
 \label{SOD_energy_hf_derivatives_eq:5}
\begin{split}
 &\sum_{i}^{occ}H_{ii}^{a[b]} + \frac{1}{2}
    \sum_{ij}^{occ}
    \left \{ ( ii|jj)^{a[b]} - ( ij|ji)^{a[b]}
    \right \}  \\
&=\sum_{i}^{occ}H_{ii}^{ab} + \frac{1}{2}
    \sum_{ij}^{occ}
    \left \{ \Pi_{iijj}^{ab} - \Pi_{ijji}^{ab}
    \right \}  \\
&+ \sum_{r}^{ALL MO}\left\lbrace  \sum_{i}^{occ}H_{ir}^{a}U^{b}_{ri} +
\sum_{ij}^{occ}\left[ U^{b}_{ri}\Pi_{rijj}^{a} -
U^{b}_{ri}\Pi^{a}_{rjji}\right] 
\right\rbrace \\
&+ \sum_{r}^{ALL MO}\left\lbrace  \sum_{j}^{occ}H_{jr}^{a}U^{b}_{rj} +
\sum_{ij}^{occ}\left[ U^{b}_{rj}\Pi^{a}_{iirj} -
U^{b}_{rj}\Pi^{a}_{irji}\right] \right\rbrace
\end{split}
\end{equation}

By similar techniques used in (\ref{FOD_energy_HF_derivatives_eq:3}) and
(\ref{FOD_energy_HF_derivatives_eq:4}), the above equation is transformed into: 
\begin{equation}
 \label{SOD_energy_hf_derivatives_eq:6}
\begin{split}
 &\sum_{i}^{occ}H_{ii}^{a[b]} + \frac{1}{2}
    \sum_{ij}^{occ}
    \left \{ ( ii|jj)^{a[b]} - ( ij|ji)^{a[b]}
    \right \}  \\
&=\sum_{i}^{occ}H_{ii}^{ab} + \frac{1}{2}
    \sum_{ij}^{occ}
    \left \{ \Pi_{iijj}^{ab} - \Pi_{ijji}^{ab}
    \right \} + 2\sum_{r}^{ALL MO}\sum_{i}^{occ}U^{b}_{ri}F^{a}_{ir}  \\
\end{split}
\end{equation}
Here we note that for $F^{a}_{ir}$, it denotes that it has derivatives on the
AO integrals so that we do not have relation \ref{TE_FM_HF_gradient_eq:3}
established.

Next, for the last term in (\ref{SOD_energy_hf_derivatives_eq:1}), we have:
\begin{equation}
 \label{SOD_energy_hf_derivatives_eq:7}
\begin{split}
 & 2\sum_{r}^{ALL
MO}\sum_{j}^{occ}\left\lbrace F_{rj}\frac{\partial U^{a}_{rj}}{\partial
\bm{R}_{b}} + U^{a}_{rj}\frac{\partial F_{rj}}{\partial
\bm{R}_{b}}\right\rbrace \\
&= 2\sum_{r}^{ALL
MO}\sum_{j}^{occ} \left\lbrace F_{rj}\left[U^{ab}_{rj}  -
\sum_{s}^{ALL MO}U^{b}_{rs}U^{a}_{sj} \right] +  U^{a}_{rj} 
F_{rj}^{[b]}\right\rbrace 
\end{split}
\end{equation}
We have already got the $F_{rj}^{[b]}$ in the last section, and for $F_{rj}$,
according to the relation in (\ref{TE_FM_HF_gradient_eq:3}) it equals to
$\delta_{rj}\epsilon_{j}$; so the (\ref{SOD_energy_hf_derivatives_eq:7})
becomes:
\begin{equation}
  \label{SOD_energy_hf_derivatives_eq:8}
\begin{split}
 & 2\sum_{r}^{ALL
MO}\sum_{j}^{occ}\left\lbrace F_{rj}\frac{\partial U^{a}_{rj}}{\partial
\bm{R}_{b}} + U^{a}_{rj}\frac{\partial F_{rj}}{\partial
\bm{R}_{b}}\right\rbrace \\
&= 2\sum_{j}^{occ} \epsilon_{j}U^{ab}_{jj} -  2\sum_{s}^{ALL MO}\sum_{j}^{occ}
\epsilon_{j}U^{b}_{js}U^{a}_{sj} + \sum_{r}^{ALL MO}\sum_{j}^{occ}U^{a}_{rj} 
F_{rj}^{[b]}
\end{split}
\end{equation}
According to Winger's theorem, the second order perurbation term should not
appear in the $E^{[ab]}$, hence let's use the relation of
(\ref{overlap_MO_INT_gradient_eq:8}), by setting $p=q=j$ it gives:
\begin{align}
  U^{ab}_{jj}  + U^{ab}_{jj} &= 
2\sum_{r}\left[ S^{b}_{jr} S^{a}_{jr} 
-U^{b}_{jr}U^{a}_{jr}  \right]  - S^{ab}_{jj} \nonumber \\ 
&= 2\eta_{jj} - S^{ab}_{jj}\Longrightarrow \nonumber \\ 
 U^{ab}_{jj} &= \eta_{jj} - \frac{1}{2}S^{ab}_{jj}
\end{align}

finally, the second derivatives for the Hatree-Fock energy can be expressed as: 
\begin{equation}
  \label{SOD_energy_hf_derivatives_eq:9}
\begin{split}
 E^{[ab]} &= \sum_{i}^{occ}H_{ii}^{ab} + \frac{1}{2}
    \sum_{ij}^{occ}
    \left \{ \Pi_{iijj}^{ab} - \Pi_{ijji}^{ab}
    \right \} + 2\sum_{r}^{ALL MO}\sum_{i}^{occ}U^{b}_{ri}F^{a}_{ir}  \\
& +  2\sum_{j}^{occ} \epsilon_{j}\left(\eta_{jj} - \frac{1}{2}S^{ab}_{jj}
\right)  -  2\sum_{s}^{ALL MO}\sum_{j}^{occ}
\epsilon_{j}U^{b}_{js}U^{a}_{sj} + \sum_{r}^{ALL MO}\sum_{j}^{occ}U^{a}_{rj} 
F_{rj}^{[b]}
\end{split}
\end{equation}
In this expression, it explicitly contains the MO rotation $U^{a}$ and $U^{b}$
matrix. They will be evaluated in the CPHF equation.

%%%%%%%%%%%%%%%%%%%%%%%%%%%%%%%%%%%%%%%%%%%%%%%%%%%%%%%%%%%%%%%%%
\section{Comparision with General Perturbed Theory}
%
%
%
So far we have derived the first order and second order gradient for the
Hatree-Fock energy. Now it's time for us to give some interesting comparison
between the gradient results we made here and the general perturbed theory
which is presented in section \ref{Time_independent_perturbation_theory}.

The gradient expression for the total energy, similar with the expansion for
MO coefficients in the (\ref{orbital_response_gradient_eq:9}); can be also
expressed as:
\begin{align}
\label{comparision_hf_derivatives_eq:1}
E(\bm{R}_{0} + \Delta \bm{R}) &=  E(\bm{R}_{0}) + \Delta
\bm{R}_{a}\frac{\partial E}{\partial \bm{R}_{a}} + \Delta
\bm{R}_{b}\frac{\partial E}{\partial \bm{R}_{b}} \nonumber \\
& + \Delta\bm{R}_{a}\Delta\bm{R}_{a}\frac{\partial^{2} E}{\partial
\bm{R}_{a}^{2}} +  \Delta\bm{R}_{b}\Delta\bm{R}_{b}\frac{\partial^{2}
E}{\partial \bm{R}_{b}^{2}} +
\Delta\bm{R}_{a}\Delta\bm{R}_{b}\frac{\partial^{2} E }{\partial\bm{R}_{a}
\partial\bm{R}_{b}}
\end{align}
It's same that in this expression, $\Delta\bm{R}_{a}$(it can be named as
$\lambda_{a}$) and $\Delta\bm{R}_{b}$ (named as $\lambda_{b}$) are equivalent to
the first order perturbation multipliers, and
$\Delta\bm{R}_{a}\Delta\bm{R}_{a}$($\lambda_{a}\lambda_{a}$),
$\Delta\bm{R}_{a}\Delta\bm{R}_{b}$($\lambda_{a}\lambda_{b}$) and
$\Delta\bm{R}_{b}\Delta\bm{R}_{b}$($\lambda_{b}\lambda_{b}$) are equivalent to
the second order multipliers. Hence the $E^{[a]}$, $E^{[ab]}$ are the first
order perturbation and second perturbation for the total energy. Compared with
(\ref{PTIQMeq:5}), they are just the $E^{1}$ and $E^{2}$ in the general
expression.

In section \ref{Time_independent_perturbation_theory}, there's another job that
to get the perturbed coefficients of $a^{1}$ and $a^{2}$. Similarly, in the
Hatree-Fock equation it corresponds to the derivation of MO rotation of $U$
matrix. In section \ref{Time_independent_perturbation_theory}, the $a^{1}$ and
$a^{2}$ are derived through the corresponding Schrodinger equation, similarly;
we can guess that the $U$ matrix are derived through the Hatree-Fock equation.
Hence, we come to the coupled-perturbed Hatree-Fock equation.

%%%%%%%%%%%%%%%%%%%%%%%%%%%%%%%%%%%%%%%%%%%%%%%%%%%%%%%%%%%%%%%%%
\section{Couped-Perturbed Hatree-Fock Equation}\label{CPHF}
%
%
%
The primary target for the coupled pertubed Hatree-Fock equation, is
to solve the MO rotation matrix of $U$.

Let's start fom the matrix form of Hatree-Fock equation:
\begin{equation}
  \label{CPHF_hf_derivatives_eq:1}
  F_{pq} = \epsilon_{p}\delta_{pq}
\end{equation}

From the (\ref{FOD_FOCK_hf_derivatives_eq:4}), we have already get the
$F_{pq}^{[a]}$:
\begin{align}
  \label{CPHF_hf_derivatives_eq:2}
  F_{pq}^{[a]} &= H^{a}_{pq} + \sum_{k}^{occ}\left \{\Pi^{a}_{pqkk} -
    \Pi^{a}_{pkkq} \right\}
  \nonumber \\
  &+ \sum_{t}^{ALL MO}\left(U^{a}_{tp}F_{qt} + U^{a}_{tq}F_{pt}\right)
  + \sum_{k}^{occ}\sum_{t}^{ALL MO}U^{a}_{tk}A_{pq,kt}
\end{align}
Where the $A_{pq,kt}$ is given by:
\begin{equation}
  \label{CPHF_hf_derivatives_eq:3}
  A_{pq,kt}  = 2\Pi_{pqtk} - \Pi_{ptkq} - \Pi_{pktq}
\end{equation}

By using the relation in (\ref{CPHF_hf_derivatives_eq:1}), the
(\ref{CPHF_hf_derivatives_eq:2}) becomes:
\begin{align}
  \label{CPHF_hf_derivatives_eq:4}
  F_{pq}^{[a]} &= H^{a}_{pq} + \sum_{k}^{occ}\left \{\Pi^{a}_{pqkk} -
    \Pi^{a}_{pkkq} \right\}
  \nonumber \\
  &+ \left(U^{a}_{qp}\epsilon_{q} + U^{a}_{pq}\epsilon_{p}\right)
  + \sum_{k}^{occ}\sum_{t}^{ALL MO}U^{a}_{tk}A_{pq,kt} \nonumber \\
  &= H^{a}_{pq} + \sum_{k}^{occ}\left \{\Pi^{a}_{pqkk} -
    \Pi^{a}_{pkkq} \right\}
  \nonumber \\
  &+ U^{a}_{qp}\left(\epsilon_{q} - \epsilon_{p}\right) -
  S^{a}_{pq}\epsilon_{p} + \sum_{k}^{occ}\sum_{t}^{ALL
    MO}U^{a}_{tk}A_{pq,kt}
\end{align}
We note that in this derivation, we have used the relation of
(\ref{overlap_MO_INT_gradient_eq:5}).

Next let's deal with the $A_{pq.kt}$:
\begin{align}
  \label{CPHF_hf_derivatives_eq:5}
  \sum_{k}^{occ}\sum_{t}^{ALL MO}U^{a}_{tk}A_{pq,kt} &=
  \sum_{k}^{occ}\sum_{t}^{ALL MO}U^{a}_{tk}\left\{ 2\Pi_{pqtk} -
    \Pi_{ptkq} - \Pi_{pktq} \right\} \nonumber \\
  &= \sum_{k}^{occ}\sum_{t}^{occ}U^{a}_{tk}\left\{ 2\Pi_{pqtk} -
    \Pi_{ptkq} - \Pi_{pktq} \right\} \nonumber \\
  &+ \sum_{k}^{occ}\sum_{t}^{vir}U^{a}_{tk}\left\{ 2\Pi_{pqtk} -
    \Pi_{ptkq} - \Pi_{pktq} \right\} \nonumber \\
  &= \frac{1}{2}\sum_{k}^{occ}\sum_{t}^{occ}(U^{a}_{tk} +U^{a}_{kt})
  \left\{ 2\Pi_{pqtk} - \Pi_{ptkq} - \Pi_{pktq} \right\} \nonumber \\
  &+ \sum_{k}^{occ}\sum_{t}^{vir}U^{a}_{tk}\left\{ 2\Pi_{pqtk} -
    \Pi_{ptkq} - \Pi_{pktq} \right\} \nonumber \\
  &= -\frac{1}{2}\sum_{k}^{occ}\sum_{t}^{occ}S^{a}_{kt}
  \left\{ 2\Pi_{pqtk} - \Pi_{ptkq} - \Pi_{pktq} \right\} \nonumber \\
  &+ \sum_{k}^{occ}\sum_{t}^{vir}U^{a}_{tk}\left\{ 2\Pi_{pqtk} -
    \Pi_{ptkq} - \Pi_{pktq} \right\}
\end{align}
Here in the above derivation, we note that the integrals are
symmetrical for $k$ and $t$ if they are both referred to occupied
orbitals. Hence we can exchange the label of $t$ and $k$ without
changing anything. This further gives:
\begin{equation}
  \label{CPHF_hf_derivatives_eq:6}
  \sum_{k}^{occ}\sum_{t}^{occ}S^{a}_{kt}\Pi_{ptkq} =
  \sum_{k}^{occ}\sum_{t}^{occ}S^{a}_{kt}\Pi_{pktq} 
\end{equation}
Finally it leads to:
\begin{align}
  \label{CPHF_hf_derivatives_eq:7}
  \sum_{k}^{occ}\sum_{t}^{ALL MO}U^{a}_{tk}A_{pq,kt} &=
  -\sum_{k}^{occ}\sum_{t}^{occ}S^{a}_{kt}
  \left\{ \Pi_{pqtk} - \Pi_{ptkq} \right\} \nonumber \\
  &+ \sum_{k}^{occ}\sum_{t}^{vir}U^{a}_{tk}\left\{ 2\Pi_{pqtk} -
    \Pi_{ptkq} - \Pi_{pktq} \right\}
\end{align}
It turns out that only the occupied-virtual block of $U$ matrix is
left.

All in all, according to what we have got; the
(\ref{CPHF_hf_derivatives_eq:2}) finally becomes:
\begin{align}
  \label{CPHF_hf_derivatives_eq:8}
  F_{pq}^{[a]} &= H^{a}_{pq} + \sum_{k}^{occ}\left \{\Pi^{a}_{pqkk} -
    \Pi^{a}_{pkkq} \right\}
  \nonumber \\
  &+ U^{a}_{qp}\left(\epsilon_{q} - \epsilon_{p}\right) -
  S^{a}_{pq}\epsilon_{p} \nonumber \\
  &- \sum_{k}^{occ}\sum_{t}^{occ}S^{a}_{kt}
  \left\{ \Pi_{pqtk} - \Pi_{ptkq} \right\} \nonumber \\
  &+ \sum_{k}^{occ}\sum_{t}^{vir}U^{a}_{tk}\left\{ 2\Pi_{pqtk} -
    \Pi_{ptkq} - \Pi_{pktq} \right\}
\end{align}

For $F^{[a]}_{pq}$, generally we have:
\begin{equation}
  \label{CPHF_hf_derivatives_eq:9}
  \frac{\partial F_{pq}}{\partial \bm{R}_{a}} = \delta_{pq}
  \frac{\partial \epsilon_{p}}{\partial \bm{R}_{a}}
\end{equation}

We see that if $p=q$, it becomes:
\begin{align}
  \label{CPHF_hf_derivatives_eq:10}
  F_{pp}^{[a]} &= H^{a}_{pp} + \sum_{k}^{occ}\left \{\Pi^{a}_{ppkk} -
    \Pi^{a}_{pkkp} \right\}
  \nonumber \\
  &- S^{a}_{pp}\epsilon_{p} - \sum_{k}^{occ}\sum_{t}^{occ}S^{a}_{kt}
  \left\{ \Pi_{pptk} - \Pi_{ptkp} \right\} \nonumber \\
  &+ \sum_{k}^{occ}\sum_{t}^{vir}U^{a}_{tk}\left\{ 2\Pi_{pptk} -
    \Pi_{ptkp} - \Pi_{pktp} \right\}
\end{align}

In the (\ref{CPHF_hf_derivatives_eq:10}), we can find two interesting
points. The first point is that this expression contains the first
order $U$ matrix, while the $E^{[a]}$ does not. The second point is,
the expression only contains the occupied-virtual blocks of $U$
matrix. So why??

The first question is easy to answer. As what we can see from section
\ref{General_Derivatives_Expressions_hf_derivatives}, the total energy
for the system is stationary for the variation of variational
coefficients; so the gradient of energy should not contain $U$
matrix. However, the MO energy is not variational; hence it contains
the first order $U$ matrix and it's worthy to note that
$\dfrac{\partial \epsilon_{p}}{\partial \bm{R}_{a}} \neq 0$.

On the other hand, the second question is not easy to answer. In
section (\ref{overlap_int_general_derivatives}), we have found that
$U^{a}_{pp}$ can not be determined. Now let's generally consider the
four blocks of $U$ matrix:
\begin{equation}
  \label{CPHF_hf_derivatives_eq:11}
  \begin{bmatrix}
    U_{oo} & U_{ov} \\
    U_{vo} & U_{vv}
  \end{bmatrix}
\end{equation}

For the $U_{vv}$, physically, we can not determine it. The expression
of total energy under Hatree-Fock framework, only involves the
occupied orbitals, and for the other physical properties, they only
involve the electron density to describe it(see \ref{HFT:3} for
details). Therefore, we at least have one label in $U$ matrix in signaling
occupied orbitals, hence it's unable for us to determine the $U_{vv}$ block.

For the $U_{oo}$ block, mathematically it's related two
indices. Generally in the energy expression as well as physical
properties expression (such as $F^{[a]}$ and $S^{[a]}$ etc.), both of
the two indices are always symmetrical so that we can always use
$U_{oo} + U^{T}_{oo} = S^{a}_{oo}$ to eliminate it. Hence this block
can not be determined, too.

In conclusion, similar to the treatment in \ref{1st_approximation_WT}
for the term $a^{1}_{kk}$; we can safely set the two block as zero. So
(\ref{CPHF_hf_derivatives_eq:10}) becomes:
\begin{equation}
  \label{CPHF_hf_derivatives_eq:12}
  \begin{bmatrix}
    0 & U_{ov} \\
    U_{vo} & 0
  \end{bmatrix}
\end{equation} 

For the $U_{ov}$, we can see that through
(\ref{CPHF_hf_derivatives_eq:8}) we can directly calculate
it. Furthermore, since $U$ matrix is some unitary matrix $U^{+}U = I$;
we can generally have:
\begin{equation}
  \label{CPHF_hf_derivatives_eq:13}
  \begin{bmatrix}
    0 & U^{+}_{vo} \\
    U^{+}_{ov} & 0
  \end{bmatrix}
  \begin{bmatrix}
    0 & U_{ov} \\
    U_{vo} & 0
  \end{bmatrix} =
  \begin{bmatrix}
    U^{+}_{vo}U_{vo} & 0 \\
    0 & U^{+}_{ov}U_{ov}
  \end{bmatrix}
  =   \begin{bmatrix}
    I & 0 \\
    0 & I
  \end{bmatrix}
\end{equation}
hence both $U_{vo}$ and $U_{ov}$ exist, and they are forming some unitary
matrix. On the other hand, it's worthy to note that the $U_{vo}$ and $U_{ov}$
are not independent with each other. According to the relation defined in
(\ref{overlap_MO_INT_gradient_eq:3}), which is:
\begin{equation}
S_{pq}^{[a]} = \frac{\partial S_{pq}}{\partial \bm{R}_{a}} = U^{a}_{qp} +
U^{a}_{pq} + S^{a}_{pq} = 0
\end{equation}
Then the $U^{a}_{vo} + U^{a}_{ov} = S^{a}_{ov} = S^{a}_{vo}$. Hence if we get
the $U^{a}_{vo}$, we then get the $U^{a}_{ov}$. 

Finally, for the $U$ matrix in the Hatree-Fock approximation, we can
see that the expression in (\ref{orbital_response_gradient_eq:4})
\begin{equation}
  c^{per}_{\mu p} = c_{\mu p} + \lambda\sum_{q}^{ALL
    MO}c_{\mu q}U_{qp}
\end{equation}
has to be separated into two parts:
\begin{align}
  \label{CPHF_hf_derivatives_eq:14}
  c^{per}_{\mu i} &= c_{\mu i} + \lambda\sum_{a}^{vir}c_{\mu a}U_{ai}
  \nonumber \\
  c^{per}_{\mu a} &= c_{\mu a} + \lambda\sum_{i}^{occ}c_{\mu i}U_{ia}
\end{align}
and the $U$ matrix inside (\ref{CPHF_hf_derivatives_eq:14}) will be
solved through (\ref{CPHF_hf_derivatives_eq:8}).

By the way, it's worthy to note more about the $U$ matrix defined through
(\ref{CPHF_hf_derivatives_eq:14}). We know it's some unitary matrix, which
satisfy $U^{+}U = I$; however, we should note that it's not hermitian matrix
which means $U = U^{+}$. In this circumstance, we have $U_{ai} = U_{ia}$
(suppose that $U$ matrix is real). However, according to its physical
definition in (\ref{CPHF_hf_derivatives_eq:14}); it's clear that $U_{ai} =
U_{ia}$ does not hold because $U_{ai}$ is the response for occupied orbital and
$U_{ia}$ denotes the response for virtual orbitals. They should be different.  
 
Let's turn to (\ref{CPHF_hf_derivatives_eq:8}) to get the
CPHF equation. Since $F^{[a]}_{pq} = 0$ for $p \neq q$, then we can
have:
\begin{align}
  \label{CPHF_hf_derivatives_eq:15}
U^{a}_{qp}\left(\epsilon_{q} - \epsilon_{p}\right) + 
\sum_{k}^{occ}\sum_{t}^{vir}U^{a}_{tk}\left\{ 2\Pi_{pqtk} -
    \Pi_{ptkq} - \Pi_{pktq} \right\} &= B_{pq}
\end{align}
Where $B_{pq}$ is given by:
\begin{align}
  \label{CPHF_hf_derivatives_eq:16}
  B_{pq} &= H^{a}_{pq} + \sum_{k}^{occ}\left \{\Pi^{a}_{pqkk} - \Pi^{a}_{pkkq}
  \right\} \nonumber \\
  &-S^{a}_{pq}\epsilon_{p} - \sum_{k}^{occ}\sum_{t}^{occ}S^{a}_{kt}
  \left\{ \Pi_{pqtk} - \Pi_{ptkq} \right\} 
 \end{align}
Which only involves the gradient on AO integrals.

The equation in (\ref{CPHF_hf_derivatives_eq:15}) is also some
equation which need to solve self-consistently. The $U$ matrix has the
dimension of $occ\times vir$ (the another half part can be got automatically
as we get the first half part!). In the first order gradient, it's clear
that we do not need the $U$ matrix, but in the second order gradient, we need
both of the $U_{occ,vir}$ and $U_{vir, occ}$. 

Finally, let's rewrite the CPHF equation in (\ref{CPHF_hf_derivatives_eq:15})
into some matrix form. By setting $p$ to $i$, $q$ to $a$ we have:
\begin{equation}
  \label{CPHF_hf_derivatives_eq:17}
AU^{a} = B^{a}\Leftrightarrow \sum_{jb}A_{ij, ab}U^{a}_{bj} =
B^{a}_{ia}
\end{equation}
Where
\begin{equation}
\begin{split}
 A_{ij, ab} &= \left(\epsilon_{a} - \epsilon_{i}\right)\delta_{ij}\delta_{ab} + 
\left\{ 2\Pi_{iabj} -
    \Pi_{ibja} - \Pi_{ijba} \right\} \\
B^{a}_{ia} &= H^{a}_{ia} + \sum_{k}^{occ}\left \{\Pi^{a}_{iakk} - \Pi^{a}_{ikka}
  \right\}  \\
  &-S^{a}_{ia}\epsilon_{i} - \sum_{j}^{occ}\sum_{b}^{occ}S^{a}_{jb}
  \left\{ \Pi_{iabj} - \Pi_{ibja} \right\} 
\end{split}
 \label{CPHF_hf_derivatives_eq:18}
\end{equation}
For clarity, we do not use the index of $t$, $k$ etc. anymore, since
we have already demonstrated that $U$ matrix must be $occ\times vir$ dimension.
Hence again, we use $i$ to replace $p$, $a$ replace $q$, $b$ replace $t$ and $j$
replace $k$ in original (\ref{CPHF_hf_derivatives_eq:15}). Furthermore, we note
that the superscript of ``a'' in (\ref{CPHF_hf_derivatives_eq:17}) means this
term has the derivatives with respect
to $\bm{R}_{a}$.  



%%%%%%%%%%%%%%%%%%%%%%%%%%%%%%%%%%%%%%%%%%%%%%%
%%% Local Variables: 
%%% mode: latex
%%% TeX-master: "../../main"
%%% End: 
%%%%%%%%%%%%%%%%%%%%%%%%%%%%%%%%%%%%%%%%%%%%%%%