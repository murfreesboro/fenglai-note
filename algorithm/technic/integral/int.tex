%
% now OS is already added in;
% will add the MD algorithm as well as DRK too in the future
% fm will be done in the near future
% 
%
%
%%%%%%%%%%%%%%%%%%%%%%%%%%%%%%%%%%%%%%%%%%%%%%%%%%%%%%%%%%%%%%%%%%%%%%%%%%%%%%%%%%%%%%%%%%%%
\chapter{Algorithms for Integral Evaluation Over Gaussian Functions}
%
%
%
%
%

In this chapter, we will focus on the electron integrals; which generally has the form as:
\begin{equation}
 V_{ij} = \int \chi_{i}(r)\chi_{j}(r)f(r,r^{'})
\chi_{k}(r^{'})\chi_{l}(r^{'}) dr dr^{'}
\end{equation}
It is already shown in the shell pair data discussion\ref{shell_pair_eq:1}. Here $\chi$ is 
the Gaussian primitive function with arbitrary angular momentum, and $f(r,r^{'})$ is the 
electron operator for one or two electrons. In this chapter, we will discuss the integrals
and its derivatives.

%%%%%%%%%%%%%%%%%%%%%%%%%%%%%%%%%%%%%%%%%%%%%%%%%%%%%%%%%%%%%%%%%%%%%%%%%%%%%%%%%%%%%%%%%%%%
\section{History for Algorithms of Integral Evaluation}
%
%
%
%
According to the general review given by Peter Gill\cite{gill1994molecular}, the history for
the development of integral algorithm could be divided into three period of times. The first
one is its infant period, at this time only simple SCF calculation could be performed. The 
first algorithm for Gaussian functions was prompted by Boys\cite{SFBoys1950},
then his methodology was subsequently developed by
Shavitt\cite{1963_int_algorithm}, Clementi etc.
\cite{clementi1969study,clementi1972computation}.

In the second generation, the SCF algorithm was mean to be more general. The ``axis-switch''
method of Pople and Hehre(PH)\cite{PH} revolutionized the algorithms of integral in a sense 
that it enables the SCF calculations to be a standard tool for the chemists. However, this method
is constructed primarily for the S and P type of Gaussian functions. Such deficiency motivated
the development of Dupuis-Rys-King(DRK)\cite{DRK1976JCP, DRK1976JCOMP, DRK1983JCOMP} method 
and McMurchie-Davidson(MD)\cite{MD} method to give more general derivation for the integrals.

The third generation of integral algorithm started from the Obara-Saika method(OS)\cite{OS1986,
OS1988} method, it is derived from the idea to solve the derivatives of integral\cite{HB1982,HB1989}. 
Encouraged by its success, a number of new algorithms came out\cite{HGP,gill1989efficient,
gill1990efficient,PRISM,coldprism} which constitutes the fundamental
layer for modern quantum chemistry packages. 

In this chapter, we will mainly focus on the second and third generation of integral algorithms. 
Namely the MD and DRK methods, OS, HGP and PRISM methods.

%%%%%%%%%%%%%%%%%%%%%%%%%%%%%%%%%%%%%%%%%%%%%%%%%%%%%%%%%%%%%%%%%%%%%%%%%%%%%%%%
%% this is about the mathematically formulas for integral
% 
% firstly set up on May, 2011
%
% for direct calculation:
% checked the results for kinetic integral,  overlap integral, 
% and derivations for overlap.   June 2ed
% checked the derivation for nuclear attraction integral, fine.  June 13th
%
%

%%%%%%%%%%%%%%%%%%%%%%%%%%%%%%%%%%%%%%%%%%%%%%%%%%%%%%%%%%%%%%%%%%%%%%%%%%%%%%%%%%%%%%%%%%%%
\section{Direct Integrals Calculation}
%
%
%
%
In this section, we will show how to derive the one electron integrals in terms of the 
basic form of shell pair data shown in \ref{basic_shell_pair_data_form}. Generally, such
way of derivation is simple and the efficiency is low; therefore in the modern quantum 
chemistry package it's not used anymore. However, it's still listed here for archive reason.
Furthermore, from the derivation we could clearly see how to make integral from the basic
way.

%%%%%%%%%%%%%%%%%%%%%%%%%%%%%%%%%%%%%%%%%%%%%%%%%%%%%%%%%%%%%%%%%%%%%%%%%%%%%%%%%%%%%%%%%%%%
\subsection{Overlap Integrals}
%
%
%
%
Now let's step into overlap integral. Firstly let's consider the simple S type of integral 
without contraction:
\begin{equation}
 \label{overlap_direct_int_eq:1}
\begin{split}
S_{\mu\nu}  &= \int \phi_{\mu}^{*}\phi_{\nu} dr \\
&= \int e^{-\alpha r_{A}^{2} - \beta r_{B}^{2}} dr \\
&= e^{-\frac{\alpha\beta}{\alpha+\beta}\overline{AB}^{2}}\int e^{-(\alpha+\beta)r_{P}^{2}} dr \\
&= e^{-\frac{\alpha\beta}{\alpha+\beta}\overline{AB}^{2}} \int e^{-(\alpha+\beta)x_{P}^{2}} dx
\int e^{-(\alpha+\beta)y_{P}^{2}} dy \int e^{-(\alpha+\beta)z_{P}^{2}} dz \\
&= e^{-\frac{\alpha\beta}{\alpha+\beta}\overline{AB}^{2}}\left( \frac{\pi}{\alpha+\beta}\right)
^{\frac{3}{2}}   
\end{split}
\end{equation}

The more common form of overlap integral between two arbitrary Gaussian primitives without 
contraction, will be given according to the \ref{gaussian_product_rule_eq:19} and 
\ref{gaussian_product_rule_eq:20}:
\begin{equation}
 \label{overlap_direct_int_eq:2}
\begin{split}
S_{\mu\nu}  &= \int \phi_{\mu}^{*}\phi_{\nu} dr \\
&= \int x_{A}^{l_{1}}y_{A}^{m_{1}}z_{A}^{n_{1}}
        x_{B}^{l_{2}}y_{B}^{m_{2}}z_{B}^{n_{2}}
        e^{-\alpha r_{A}^{2} - \beta r_{B}^{2}} dr \\
&= \int \sum_{i=0}^{l_{1}+l_{2}}\sum_{j=0}^{m_{1}+m_{2}}\sum_{k=0}^{n_{1}+n_{2}}
   x_{P}^{i}y_{P}^{j}z_{P}^{k}
   e^{-(\alpha+\beta)r_{P}^{2}} e^{-\frac{\alpha\beta}{\alpha+\beta}\overline{AB}^{2}}
   F_{ijk} dr \\
&= e^{-\frac{\alpha\beta}{\alpha+\beta}\overline{AB}^{2}}
   \sum_{i=0}^{l_{1}+l_{2}}\sum_{j=0}^{m_{1}+m_{2}}\sum_{k=0}^{n_{1}+n_{2}}F_{ijk}
   \int x_{P}^{i}y_{P}^{j}z_{P}^{k} e^{-(\alpha+\beta)r_{P}^{2}} dr \\
&= e^{-\frac{\alpha\beta}{\alpha+\beta}\overline{AB}^{2}}
   \sum_{i=0}^{l_{1}+l_{2}}\sum_{j=0}^{m_{1}+m_{2}}\sum_{k=0}^{n_{1}+n_{2}}F_{ijk}
   \int x_{P}^{i}e^{-(\alpha+\beta)x_{P}^{2}} dx \\
&\times
   \int y_{P}^{j}e^{-(\alpha+\beta)y_{P}^{2}} dy \int z_{P}^{k}e^{-(\alpha+\beta)z_{P}^{2}} dz  
\end{split}
\end{equation}

Now the integral has been retreated into simpler form. Now let's consider the integral form
of:
\begin{equation}
 \label{overlap_direct_int_eq:3}
 \int x^{i}e^{-\gamma x^{2}} dx
\end{equation}

This is the general form of integrals appearing in the \ref{overlap_direct_int_eq:2}. 
Firstly, we can
see that this integral only exists when $i$ is even number; that is to say:
\begin{equation}
\label{overlap_direct_int_eq:even}
 \int x^{i}e^{-\gamma x^{2}} dx = 0 \quad i = 2k+1
\end{equation}
This is because the function of $x^{i}e^{-\gamma x^{2}}$ is odd function when $i = 2k+1$. Hence 
here we have to make additional treatment, that to replace all of $i$, $j$ and $k$ in 
\ref{overlap_direct_int_eq:2} into $i^{'}$, $j^{'}$ and $k^{'}$:
\begin{equation}
 \begin{split}
  i &= 2*i^{'} \\ 
  j &= 2*j^{'} \\
  k &= 2*k^{'} 
 \end{split}
 \label{overlap_direct_int_eq:4}
\end{equation}

So the result in the \ref{overlap_direct_int_eq:2} can be rewritten as:
\begin{equation}
 \label{overlap_direct_int_eq:5}
\begin{split}
S_{\mu\nu}  &= \int \phi_{\mu}^{*}\phi_{\nu} dr \\
&= e^{-\frac{\alpha\beta}{\alpha+\beta}\overline{AB}^{2}}
   \sum_{i^{'}=0}^{\left[ \frac{l_{1}+l_{2}}{2}\right] }
   \sum_{j^{'}=0}^{\left[ \frac{m_{1}+m_{2}}{2}\right] }
   \sum_{k^{'}=0}^{\left[ \frac{n_{1}+n_{2}}{2}\right] }
   F_{2i^{'},2j^{'},2k^{'}}
   \int x_{P}^{2i^{'}}e^{-(\alpha+\beta)x_{P}^{2}} dx \\
&\times
   \int y_{P}^{2j^{'}}e^{-(\alpha+\beta)y_{P}^{2}} dy 
   \int z_{P}^{2k^{'}}e^{-(\alpha+\beta)z_{P}^{2}} dz  
\end{split}
\end{equation}

This integral, actually can be decomposed into three pieces:
\begin{equation}
 \label{overlap_direct_int_eq:6}
S_{\mu\nu} = e^{-\frac{\alpha\beta}{\alpha+\beta}\overline{AB}^{2}}
S_{\mu\nu}^{x}S_{\mu\nu}^{y}S_{\mu\nu}^{z}
\end{equation}

All of the three pieces are sharing the same structure:
\begin{align}
 \label{overlap_direct_int_eq:7}
S_{\mu\nu}^{x} &= \sum_{i^{'}=0}^{\left[ \frac{l_{1}+l_{2}}{2}\right] }
                  f_{2i^{'}}(l_{1},l_{2},\overline{PA}_{x},\overline{PB}_{x})
                  \int x_{P}^{2i^{'}}e^{-(\alpha+\beta)x_{P}^{2}} dx \nonumber \\
S_{\mu\nu}^{y} &= \sum_{j^{'}=0}^{\left[ \frac{m_{1}+m_{2}}{2}\right] }
                  f_{2j^{'}}(m_{1},m_{2},\overline{PA}_{y},\overline{PB}_{y})
                  \int y_{P}^{2j^{'}}e^{-(\alpha+\beta)y_{P}^{2}} dy \nonumber \\
S_{\mu\nu}^{z} &= \sum_{k^{'}=0}^{\left[ \frac{n_{1}+n_{2}}{2}\right] }
                  f_{2k^{'}}(n_{1},n_{2},\overline{PA}_{z},\overline{PB}_{z})
                  \int z_{P}^{2k^{'}}e^{-(\alpha+\beta)z_{P}^{2}} dz
\end{align}
According to the \ref{int_sec2_eq:8}, where the integral can be expressed as:
\begin{equation}
\int x^{2k}e^{-\alpha x^{2}} dx = 
\frac{(2k-1)!!\sqrt{\pi}}{(2\alpha)^{k}\sqrt{\alpha}}
\end{equation}
we can have that:
\begin{align}
 \label{overlap_direct_int_eq:8}
S_{\mu\nu}^{x} &= \sum_{i^{'}=0}^{\left[ \frac{l_{1}+l_{2}}{2}\right] }
                  f_{2i^{'}}(l_{1},l_{2},\overline{PA}_{x},\overline{PB}_{x})
                  \frac{(2i^{'}-1)!!\sqrt{\pi}}{(2(\alpha+\beta))^{i^{'}}\sqrt{\alpha+\beta}}
                  \nonumber \\
S_{\mu\nu}^{y} &= \sum_{j^{'}=0}^{\left[ \frac{m_{1}+m_{2}}{2}\right] }
                  f_{2j^{'}}(m_{1},m_{2},\overline{PA}_{y},\overline{PB}_{y})
                  \frac{(2j^{'}-1)!!\sqrt{\pi}}{(2(\alpha+\beta))^{j^{'}}\sqrt{\alpha+\beta}}
                  \nonumber \\
S_{\mu\nu}^{z} &= \sum_{k^{'}=0}^{\left[ \frac{n_{1}+n_{2}}{2}\right] }
                  f_{2k^{'}}(n_{1},n_{2},\overline{PA}_{z},\overline{PB}_{z})
                  \frac{(2k^{'}-1)!!\sqrt{\pi}}{(2(\alpha+\beta))^{k^{'}}\sqrt{\alpha+\beta}}
\end{align}
Now this is the result for the overlap integral between two arbitrary Gaussian primitives with
different centers.

For the overlap integral on the same center, according to the result in 
\ref{gaussian_product_rule_eq:21}, we will have that ($A=B$):
\begin{equation}
 \begin{split}
&\int x_{A}^{l_{1}}y_{A}^{m_{1}}z_{A}^{n_{1}}
        x_{B}^{l_{2}}y_{B}^{m_{2}}z_{B}^{n_{2}}
        e^{-\alpha r_{A}^{2} - \beta r_{B}^{2}} dr \\
&= \int x_{A}^{l_{1}+l_{2}}y_{A}^{m_{1}+m_{2}}z_{A}^{n_{1}+n_{2}}e^{-(\alpha+\beta)r_{A}^{2}} dr\\
&= 
\int x_{A}^{l_{1}+l_{2}}e^{-(\alpha+\beta)x_{A}^{2}} dx
\int y_{A}^{m_{1}+m_{2}}e^{-(\alpha+\beta)y_{A}^{2}} dy
\int z_{A}^{n_{1}+n_{2}}e^{-(\alpha+\beta)z_{A}^{2}} dz
 \end{split}
\label{overlap_direct_int_eq:9}
\end{equation}
According to the result in \ref{overlap_direct_int_eq:even}, the result will 
be quite interesting that
this integral is not zero only if $l_{1}+l_{2} = 2i$, $m_{1}+m_{2} = 2j$, $n_{1}+n_{2} = 2k$. 
Hence it's easy to see that integrals between S and P shell etc. are all zero. However, it does 
not mean that different shell types in the same center are orthogonal. For example, the integral
between PX and FXY2 are not zero.

For the integral, according to the result in \ref{int_sec2_eq:8}; the final result 
between same center could be expressed as:
\begin{equation}
 \label{overlap_direct_int_eq:10}
\begin{split}
 & 
\int x_{A}^{l_{1}+l_{2}}e^{-(\alpha+\beta)x_{A}^{2}} dx
\int y_{A}^{m_{1}+m_{2}}e^{-(\alpha+\beta)y_{A}^{2}} dy
\int z_{A}^{n_{1}+n_{2}}e^{-(\alpha+\beta)z_{A}^{2}} dz \\
&= \left( \frac{\pi}{\alpha+\beta}\right)^{\frac{3}{2}} 
\frac{(l_{1}+l_{2}-1)!!(m_{1}+m_{2}-1)!!(n_{1}+n_{2}-1)!!}
{(2(\alpha+\beta))^{\frac{l_{1}+l_{2}+m_{1}+m_{2}+n_{1}+n_{2}}{2}}}
\end{split}
\end{equation}
 
Finally, we step into the overlap integral for an arbitrary basis function of $\phi$, which is 
a set of contracted Gaussian primitives. Firstly we can record the result in 
\ref{overlap_direct_int_eq:6}
as $\langle\chi_{\mu}|\chi_{\nu}\rangle$ (but usually we express it as 
$\langle\chi_{i}|\chi_{j}\rangle$, since the $\mu$, $\nu$ label are related to the indices of 
basis set functions). The basis set function can be expressed as:
\begin{equation}
\begin{split}
\label{overlap_direct_int_eq:11}
 \phi_{\mu} &= \sum_{i}d_{\mu i}\chi_{i} \\
 \phi_{\nu} &= \sum_{j}d_{\mu j}\chi_{j} 
\end{split}
\end{equation}

Then based on the results in \ref{overlap_direct_int_eq:6}, \ref{overlap_direct_int_eq:7} 
and \ref{overlap_direct_int_eq:8}
we have:
\begin{equation}
 \label{overlap_direct_int_eq:12}
 S_{\mu\nu} = \sum_{i}\sum_{j}d_{\mu i}d_{\mu j}\langle\chi_{i}|\chi_{j}\rangle
\end{equation}
This is the final result for overlap integral.

%%%%%%%%%%%%%%%%%%%%%%%%%%%%%%%%%%%%%%%%%%%%%%%%%%%%%%%%%%%%%%%%%%%%%%%%%%%
\subsection{Derivatives for Overlap Integral} 

There's more advantages if we use Gaussian primitive function rather than the Slater type of function
in getting the derivatives of the integral. For a common Gaussian primitive, let's derive its 
derivatives we can see that:
\begin{equation}
 \label{derivative_overlap_direct_int_eq:1}
\frac{\partial \chi_{i}}{\partial R_{x}} = \frac{\partial (x^{l}y^{m}z^{n}e^{-\alpha r^{2}})}
{\partial R_{x}} =  -lx^{l-1}y^{m}z^{n}e^{-\alpha r^{2}} + 2\alpha x^{l+1}y^{m}z^{n}e^{-\alpha r^{2}}
\end{equation}
Here the $x$ is expressed as:
\begin{equation}
 x = x_{e} - R_{x} 
\end{equation}
Which is the $x$ difference between the electron position and the nuclear position. 

The expression is exactly equal to the form below:
\begin{equation}
 \label{derivative_overlap_direct_int_eq:2}
\frac{\partial \chi^{lmn}_{i}}{\partial R_{x}} = -l\chi_{i}^{l-1mn} + 2\alpha\chi_{i}^{l+1mn}
\end{equation}
This result shows that the derivatives for the Gaussian primitive can be expressed as the linear 
sum of two Gaussian primitives, one is with higher angular momentum and the other with lower
angular momentum; it indicates that the derivatives for the integral of Gaussian primitives 
can be simply expressed as some summation between the integrals.

The overlap integral between two Gaussian primitives, can be expressed as:
\begin{equation}
 \label{derivative_overlap_direct_int_eq:3}
\frac{\partial\langle\chi_{i}|\chi_{j}\rangle}{\partial R_{x}} = 
\begin{cases}
 0  \\
\left\langle \frac{\partial \chi_{i} }{\partial R_{x}}|\chi_{j}\right\rangle \quad \text{or}
\quad \left\langle \chi_{i}|\frac{\partial \chi_{j} }{\partial R_{x}}\right\rangle \\
\left\langle \frac{\partial \chi_{i} }{\partial R_{x}}|\chi_{j}\right\rangle +
\left\langle \chi_{i}|\frac{\partial \chi_{j} }{\partial R_{x}}\right\rangle
\end{cases}
\end{equation}
If both $\chi_{i}$ and $\chi_{j}$ do not center on the given atom ($R_{x}$), then the derivatives
is zero; if either $\chi_{i}$ or $\chi_{j}$ center on the given atom, the derivatives will be like
the second equation; finally if both $\chi_{i}$ and $\chi_{j}$ center on the given atom, the 
derivatives will be expressed as in the third equation. Now let's go to see how to express the 
second equation in terms of the \ref{derivative_overlap_direct_int_eq:2}:
\begin{equation}
 \begin{split}
  \left\langle \frac{\partial \chi_{i} }{\partial R_{x}}|\chi_{j}\right\rangle &=
-l_{i}\langle\chi_{i}^{l_{i}-1m_{i}n_{i}}|\chi_{j}\rangle
+2\alpha_{i}\langle\chi_{i}^{l_{i}+1m_{i}n_{i}}|\chi_{j}\rangle
 \end{split}
\label{derivative_overlap_direct_int_eq:4}
\end{equation}
while in literature, the integral shown above usually expressed in the other way:
\begin{equation}
 \label{derivative_overlap_direct_int_eq:5}
\left\langle \frac{\partial \chi_{i} }{\partial R_{x}}|\chi_{j}\right\rangle
= -l_{i}\langle -1|0\rangle_{x} + 2\alpha_{i}\langle +1 | 0\rangle_{x}
\end{equation}
Here ``0'' indicates its original form (like for the $\chi_{j}$, it's not changed so we label it
as 0), and ``$+1$'' or ``$-1$'' means the angular momentum is ascending or descending. Finally,
the subscript of ``x'' means the changing of angular momentum is on the x. This expression is 
simpler than the \ref{derivative_overlap_direct_int_eq:4}. 

%%%%%%%%%%%%%%%%%%%%%%%%%%%%%%%%%%%%%%%%%%%%%%%%%%%%%%%%%%%%%%%%%%%%%%%%%%%%%
\subsection{Kinetic Energy Integral}
%
%
%
Kinetic energy integral is defined as:
\begin{equation}
 \label{kinetic_direct_int_eq:1}
\begin{split}
 T_{ij} &= \int \chi_{i}(-\frac{1}{2}\nabla^{2})\chi_{j} dr \\
&= -\frac{1}{2} \int 
x^{l_{i}}y^{m_{i}}z^{n_{i}}e^{-\alpha_{i} r_{i}^{2}}
\left( 
  \frac{\partial^{2}}{\partial x^{2}}
+ \frac{\partial^{2}}{\partial y^{2}}
+ \frac{\partial^{2}}{\partial z^{2}}\right)
x^{l_{j}}y^{m_{j}}z^{n_{j}}e^{-\alpha_{j} r_{j}^{2}}
\end{split}
\end{equation}
It's easy to see that the integral here can be restored to the overlap integrals. We can do it directly
by expanding each term inside \ref{kinetic_direct_int_eq:1}, however; as we know, the kinetic energy should
be symmetric (as physically it's Hermitian operator); hence usually we evaluate the kinetic energy 
in another way. Here below is a simple example:
\begin{equation}
 \begin{split}
 & -\frac{1}{2} \int 
x^{l_{i}}y^{m_{i}}z^{n_{i}}e^{-\alpha_{i} r_{i}^{2}} 
  \frac{\partial^{2}}{\partial x^{2}}
x^{l_{j}}y^{m_{j}}z^{n_{j}}e^{-\alpha_{j} r_{j}^{2}} dxdydz \\
&=
-\frac{1}{2}  \int \left\lbrace 
x^{l_{i}}y^{m_{i}}z^{n_{i}}e^{-\alpha_{i} r_{i}^{2}}\left. 
\frac{\partial (x^{l_{j}}y^{m_{j}}z^{n_{j}}e^{-\alpha_{j} r_{j}^{2}})}{\partial x}
\right\rbrace 
\right|^{+\infty}_{-\infty} dydz
\\
&+ \frac{1}{2}  \int 
\frac{\partial (x^{l_{i}}y^{m_{i}}z^{n_{i}}e^{-\alpha_{i} r_{i}^{2}})}{\partial x}
\frac{\partial (x^{l_{j}}y^{m_{j}}z^{n_{j}}e^{-\alpha_{j} r_{j}^{2}})}{\partial x} dxdydz \\
&= \frac{1}{2}  \int 
\frac{\partial (x^{l_{i}}y^{m_{i}}z^{n_{i}}e^{-\alpha_{i} r_{i}^{2}})}{\partial x}
\frac{\partial (x^{l_{j}}y^{m_{j}}z^{n_{j}}e^{-\alpha_{j} r_{j}^{2}})}{\partial x} dxdydz
 \end{split}
\label{kinetic_direct_int_eq:2}
\end{equation}
The \ref{kinetic_direct_int_eq:2} is done through integral by parts. Here we can see that the $T_{ij}$
becomes symmetric, and the $T_{ij}$ is always larger than zero between two Gaussian primitives.

Let's make the evaluation a bit of simpler. If we denote the integral in \ref{kinetic_direct_int_eq:2} 
as $I_{x}$, it's easy to see that :
\begin{equation}
 T_{ij} = I_{x} + I_{y} + I_{z}
\end{equation}
for $I_{x}$ we have that:
\begin{equation}
 \label{kinetic_direct_int_eq:3}
\begin{split}
 I_{x} &=\frac{1}{2}   \int 
          \left[ l_{i}x^{l_{i}-1} - 2\alpha_{i}x^{l_{i}+1}\right]
          y^{m_{i}}z^{n_{i}}e^{-\alpha_{i} r_{i}^{2}} \\
       &\times
          \left[ l_{j}x^{l_{j}-1} - 2\alpha_{j}x^{l_{j}+1}\right]
          y^{m_{j}}z^{n_{j}}e^{-\alpha_{j} r_{j}^{2}} dr \\
       &= \frac{1}{2}\Big(
          l_{i}l_{j}\langle -1|-1\rangle_{x} 
        - 2\alpha_{i}l_{j}\langle +1|-1\rangle_{x}    \\
       &- 2\alpha_{j}l_{i}\langle -1|+1\rangle_{x}
        + 4\alpha_{i}\alpha_{j}\langle +1|+1\rangle_{x} 
          \Big) 
\end{split}
\end{equation}

Similarly for the $I_{y}$ and $I_{z}$ we have:
\begin{equation}
\label{kinetic_direct_int_eq:4}
\begin{split}
I_{y} &=  \frac{1}{2}\Big(
          m_{i}m_{j}\langle -1|-1\rangle_{y} 
        - 2\alpha_{i}m_{j}\langle +1|-1\rangle_{y}   \\ 
       &- 2\alpha_{j}m_{i}\langle -1|+1\rangle_{y}
        + 4\alpha_{i}\alpha_{j}\langle +1|+1\rangle_{y} 
          \Big) \\
I_{z} &=  \frac{1}{2}\Big(
          n_{i}n_{j}\langle -1|-1\rangle_{z} 
        - 2\alpha_{i}n_{j}\langle +1|-1\rangle_{z}   \\ 
       &- 2\alpha_{j}n_{i}\langle -1|+1\rangle_{z}
        + 4\alpha_{i}\alpha_{j}\langle +1|+1\rangle_{z} 
          \Big) 
\end{split}
\end{equation}
Now the kinetic energy integral is done.

%%%%%%%%%%%%%%%%%%%%%%%%%%%%%%%%%%%%%%%%%%%%%%%%%%%%%%%%%%%%%%%%%%%%%%%%%%%%%%%%%%
\subsection{Nuclear Attraction Integral}
\label{direct_NAI_derivation}
%
%
%
The common nuclear attraction integral between two arbitrary Gaussian primitives can be expressed
as:
\begin{equation}
 \begin{split}
  V &= \int \chi_{i}(r)\frac{1}{r_{C}}\chi_{j}(r) dr \\
&= \int x^{l_{A}}_{A}y^{m_{A}}_{A}z^{n_{A}}_{A}e^{-\alpha r_{A}^{2}}
        \frac{1}{r_{C}}
        x^{l_{B}}_{B}y^{m_{B}}_{B}z^{n_{B}}_{B}e^{-\beta  r_{B}^{2}} dr
 \end{split}
\label{nuclear_attraction_direct_int_eq:1}
\end{equation}
$r_{A}, r_{B}$ and $r_{c}$ denotes the distance between electron position and the given
nuclear (A, B or C).

This difficulty to solve this integral is for the $1/r_{C}$. However, we can use standard
Laplace transformation to convert it into another form:
\begin{equation}
\label{nuclear_attraction_direct_int_eq:2}
 \frac{1}{r_{C}} = \frac{1}{\sqrt{\pi}}\int^{\infty}_{0} e^{-sr_{C}^{2}} s^{-\frac{1}{2}} ds
\end{equation}
So that we have:
\begin{equation}
 \begin{split}
V &= \frac{1}{\sqrt{\pi}}\int^{\infty}_{0} e^{-sr_{C}^{2}} s^{-\frac{1}{2}} ds 
          \int x^{l_{A}}_{A}y^{m_{A}}_{A}z^{n_{A}}_{A}e^{-\alpha r_{A}^{2}}
          x^{l_{B}}_{B}y^{m_{B}}_{B}z^{n_{B}}_{B}e^{-\beta  r_{B}^{2}} dr \\
       &= \frac{1}{\sqrt{\pi}}\int^{\infty}_{0}  s^{-\frac{1}{2}} ds  
          \int e^{-sr_{C}^{2}} 
          x^{l_{A}}_{A}y^{m_{A}}_{A}z^{n_{A}}_{A}e^{-\alpha r_{A}^{2}}
          x^{l_{B}}_{B}y^{m_{B}}_{B}z^{n_{B}}_{B}e^{-\beta  r_{B}^{2}} dr
 \end{split}
\label{nuclear_attraction_direct_int_eq:3}
\end{equation}
Basically, this is equivalent to three-center overlap integral. So now let's think about
how to derive it.

Let's begin from the shell pair. In the \ref{nuclear_attraction_direct_int_eq:3} we could combine the 
A and B center together. According to the results in the 
\ref{gaussian_product_rule_eq:19} and \ref{gaussian_product_rule_eq:20}, we can express 
the integral above as:
\begin{multline}
  V = e^{-\frac{\alpha\beta}{\alpha+\beta}\overline{AB}^{2}} 
\sum_{i=0}^{l_{A}+l_{B}}f_{i}(l_{A},l_{B},\overline{PA}_{x},\overline{PB}_{x})
\sum_{j=0}^{m_{A}+m_{B}}f_{j}(m_{A},m_{B},\overline{PA}_{y},\overline{PB}_{y}) \times \\
\sum_{k=0}^{n_{A}+n_{B}}f_{k}(n_{A},n_{B},\overline{PA}_{z},\overline{PB}_{z}) 
   \frac{1}{\sqrt{\pi}} 
   \int^{\infty}_{0} s^{-\frac{1}{2}} ds 
   \int x_{P}^{i}y_{P}^{j}z_{P}^{k} e^{-(\alpha+\beta)r_{P}^{2}}e^{-sr_{C}^{2}} dr
\label{nuclear_attraction_direct_int_eq:6}
\end{multline}
So the problem is left to solve the integral inside the \ref{nuclear_attraction_direct_int_eq:6}.

Before proceeding on solving the integrals, it's very interesting to note that the function of 
$x_{P}^{i}y_{P}^{j}z_{P}^{k} e^{-(\alpha+\beta)r_{P}^{2}}e^{-sr_{C}^{2}}$ will not be zero if 
$i$, $j$ and $k$ is even number. So this is different from the overlap integral situation. 
The reason is because of $r_{C}$. For example, 
$ x_{P}^{i} e^{-(\alpha+\beta)x_{P}^{2}}e^{-sx_{C}^{2}} $ is not 
even or odd function as $i$ is even or odd number, that's all because of the $x_{C}$ in the 
integral.

Firstly let's concentrate on the integral over $r$, which is:
\begin{equation}
  \int x_{P}^{i}y_{P}^{j}z_{P}^{k} e^{-(\alpha+\beta)r_{P}^{2}}e^{-sr_{C}^{2}} dr
\end{equation}
According to the Gaussian primitive product theorem, this term can be finally expressed 
into:
\begin{equation}
\begin{split}
&\int x_{P}^{i}y_{P}^{j}z_{P}^{k} e^{-(\alpha+\beta)r_{P}^{2}}e^{-sr_{C}^{2}} dr  \\
&=
e^{-\frac{(\alpha+\beta)s}{\alpha+\beta + s} \overline{PC}^{2}}\times
\int(x_{Q}+\overline{QP}_{x})^{i}(y_{Q}+\overline{QP}_{y})^{j}(z_{Q}+\overline{QP}_{z})^{k}
e^{-(\alpha+\beta + s)r_{Q}^{2}} dr \\ 
&=
e^{-\frac{(\alpha+\beta)s}{\alpha+\beta + s} \overline{PC}^{2}}\times
\sum_{i^{'}=0}^{i}\sum_{j^{'}=0}^{j}\sum_{k^{'}=0}^{k}
\binom{i}{i^{'}}\binom{j}{j^{'}}\binom{k}{k^{'}}
\overline{QP}_{x}^{i-i^{'}}\overline{QP}_{y}^{j-j^{'}}\overline{QP}_{z}^{k-k^{'}} \\
&\int x_{Q}^{i^{'}}y_{Q}^{j^{'}}z_{Q}^{k^{'}} e^{-(\alpha+\beta+s)r_{Q}^{2}} dr
\end{split}
 \label{nuclear_attraction_direct_int_eq:7}
\end{equation}
Here Q is determined by the P point and C point.

For the expression in \ref{nuclear_attraction_direct_int_eq:7}, we note that 
$\int x_{Q}^{i^{'}}y_{Q}^{j^{'}}z_{Q}^{k^{'}} e^{-(\alpha+\beta+s)r_{Q}^{2}} dr$ is even function; 
which requires that the $i^{'}$, $j^{'}$ and $k^{'}$ are all even numbers. Therefore the integral
in the \ref{nuclear_attraction_direct_int_eq:7} could be transformed into:
\begin{equation}
\begin{split}
&\int x_{P}^{i}y_{P}^{j}z_{P}^{k} e^{-(\alpha+\beta)r_{P}^{2}}e^{-sr_{C}^{2}} dr  \\
&=
e^{-\frac{(\alpha+\beta)s}{\alpha+\beta + s} \overline{PC}^{2}}\times
\sum_{i^{'}=0}^{\left[ \frac{i}{2}\right] }
\sum_{j^{'}=0}^{\left[ \frac{j}{2}\right]}
\sum_{k^{'}=0}^{\left[ \frac{k}{2}\right]}
\binom{i}{2i^{'}}\binom{j}{2j^{'}}\binom{k}{2k^{'}}
\overline{QP}_{x}^{i-2i^{'}}\overline{QP}_{y}^{j-2j^{'}}\overline{QP}_{z}^{k-2k^{'}} \\
&\int x_{Q}^{2i^{'}}y_{Q}^{2j^{'}}z_{Q}^{2k^{'}} e^{-(\alpha+\beta+s)r_{Q}^{2}} dr
\end{split}
 \label{nuclear_attraction_direct_int_eq:7_1}
\end{equation}

The $\overline{QP}$ is also connected with variable $s$, according to the 
\ref{gaussian_product_rule_eq:8}, it has the form that:
\begin{equation}
 \overrightarrow{QP} = \frac{s}{\alpha+\beta+s}\overrightarrow{CP}
\end{equation}
Therefore the \ref{nuclear_attraction_direct_int_eq:7_1} could be further transformed into:
\begin{equation}
\begin{split}
&\int x_{P}^{i}y_{P}^{j}z_{P}^{k} e^{-(\alpha+\beta)r_{P}^{2}}e^{-sr_{C}^{2}} dr  \\
&=
e^{-\frac{(\alpha+\beta)s}{\alpha+\beta + s} \overline{PC}^{2}}\times
\sum_{i^{'}=0}^{\left[ \frac{i}{2}\right] }
\sum_{j^{'}=0}^{\left[ \frac{j}{2}\right]}
\sum_{k^{'}=0}^{\left[ \frac{k}{2}\right]}
\left( \frac{s}{\alpha+\beta+s}\right)^{i+j+k-2(i^{'}+j^{'}+k^{'})} \\
&\binom{i}{2i^{'}}\binom{j}{2j^{'}}\binom{k}{2k^{'}}
\overline{CP}_{x}^{i-2i^{'}}\overline{CP}_{y}^{j-2j^{'}}\overline{CP}_{z}^{k-2k^{'}} \\
&\int x_{Q}^{2i^{'}}y_{Q}^{2j^{'}}z_{Q}^{2k^{'}} e^{-(\alpha+\beta+s)r_{Q}^{2}} dr
\end{split}
 \label{nuclear_attraction_direct_int_eq:7_2}
\end{equation}

The integration result for the \ref{nuclear_attraction_direct_int_eq:7_2} can be expressed as according 
to \ref{int_sec2_eq:8}:
\begin{equation}
\begin{split}
 &\int x_{P}^{i}y_{P}^{j}z_{P}^{k} e^{-(\alpha+\beta)r_{P}^{2}}e^{-sr_{C}^{2}} dr  \\
&=
e^{-\frac{(\alpha+\beta)s}{\alpha+\beta + s} \overline{PC}^{2}}\times
\sum_{i^{'}=0}^{\left[ \frac{i}{2}\right]}
\sum_{j^{'}=0}^{\left[ \frac{j}{2}\right]}
\sum_{k^{'}=0}^{\left[ \frac{k}{2}\right]}
\left( \frac{s}{\alpha+\beta+s}\right)^{i+j+k-2(i^{'}+j^{'}+k^{'})} \\
&\binom{i}{2i^{'}}\binom{j}{2j^{'}}\binom{k}{2k^{'}}
\overline{CP}_{x}^{i-2i^{'}}\overline{CP}_{y}^{j-2j^{'}}\overline{CP}_{z}^{k-2k^{'}} \\
&\left( \frac{\pi}{\alpha+\beta+s}\right)^{\frac{3}{2}}\times
\frac{(2i^{'}-1)!!(2j^{'}-1)!!(2k^{'}-1)!!}
{(2(\alpha+\beta+s))^{i^{'}+j^{'}+k^{'}}} 
\end{split}
 \label{nuclear_attraction_direct_int_eq:8}
\end{equation}

Now let's consider the outer integral which is over $s$:
\begin{equation}
\begin{split}
&\frac{1}{\sqrt{\pi}} 
 \int^{\infty}_{0} s^{-\frac{1}{2}} ds 
 \int x_{P}^{i}y_{P}^{j}z_{P}^{k} e^{-(\alpha+\beta)r_{P}^{2}}e^{-sr_{C}^{2}} dr \\
&= 
\frac{1}{\sqrt{\pi}}
\sum_{i^{'}=0}^{\left[ \frac{i}{2}\right]}
\sum_{j^{'}=0}^{\left[ \frac{j}{2}\right]}
\sum_{k^{'}=0}^{\left[ \frac{k}{2}\right]} 
\pi^{\frac{3}{2}}\binom{i}{2i^{'}}\binom{j}{2j^{'}}\binom{k}{2k^{'}}
\overline{CP}_{x}^{i-2i^{'}}\overline{CP}_{y}^{j-2j^{'}}\overline{CP}_{z}^{k-2k^{'}} \\
&\frac{(2i^{'}-1)!!(2j^{'}-1)!!(2k^{'}-1)!!}{2^{i^{'}+j^{'}+k^{'}}} 
\int^{\infty}_{0} ds s^{-\frac{1}{2}}
e^{-\frac{(\alpha+\beta)s}{\alpha+\beta + s} \overline{PC}^{2}} \\
&(\alpha+\beta+s)^{-\left( i^{'}+j^{'}+k^{'}+\frac{3}{2}\right) } 
\left( \frac{s}{\alpha+\beta+s}\right)^{i+j+k-2(i^{'}+j^{'}+k^{'})}
\end{split}
 \label{nuclear_attraction_direct_int_eq:11}
\end{equation}

Let's suggest to make variable transformation, that:
\begin{equation}
 \label{nuclear_attraction_direct_int_eq:12}
\frac{s}{\alpha+\beta + s} = t^{2}
\end{equation}
So that as $s$ goes from $0$ to $\infty$, then $t$ could be from $0$ to $1$.

If we differentiate it, we could get:
\begin{equation}
 \label{nuclear_attraction_direct_int_eq:13}
\frac{\alpha+\beta}{(\alpha+\beta + s)^{2}} ds = 2tdt \Rightarrow ds = 
2s^{\frac{1}{2}}\frac{(\alpha+\beta + s)^{\frac{3}{2}}}{\alpha+\beta} dt
\end{equation}

Now let's bring the \ref{nuclear_attraction_direct_int_eq:13} into the 
\ref{nuclear_attraction_direct_int_eq:11}. Furthermore, we write:
\begin{equation}
 \begin{split}
  i+j+k             &= L \\
  i^{'}+j^{'}+k^{'} &= L^{'}
 \end{split}
\end{equation}

Therefore we have:
\begin{equation}
 \label{nuclear_attraction_direct_int_eq:14}
\begin{split}
&\frac{1}{\sqrt{\pi}} 
 \int^{\infty}_{0} s^{-\frac{1}{2}} ds 
 \int x_{P}^{i}y_{P}^{j}z_{P}^{k} e^{-(\alpha+\beta)r_{P}^{2}}e^{-sr_{C}^{2}} dr \\
&= 
\frac{2\pi}{\alpha+\beta}
\sum_{i^{'}=0}^{\left[ \frac{i}{2}\right]}
\sum_{j^{'}=0}^{\left[ \frac{j}{2}\right]}
\sum_{k^{'}=0}^{\left[ \frac{k}{2}\right]} 
\binom{i}{2i^{'}}\binom{j}{2j^{'}}\binom{k}{2k^{'}}
\overline{CP}_{x}^{i-2i^{'}}\overline{CP}_{y}^{j-2j^{'}}\overline{CP}_{z}^{k-2k^{'}} \\
&\frac{(2i^{'}-1)!!(2j^{'}-1)!!(2k^{'}-1)!!}{2^{i^{'}+j^{'}+k^{'}}} 
\int^{1}_{0} dt
e^{-(\alpha+\beta)(\overline{PC}t)^{2}}
(\alpha+\beta+s)^{-L^{'} }t^{2(L-2L^{'})} 
\end{split}
\end{equation}

Now let's do further transformation to the $\alpha+\beta+s$, we can see that:
\begin{equation}
 \label{nuclear_attraction_direct_int_eq:15}
\alpha+\beta+s = \frac{\alpha+\beta}{1-t^{2}} \Rightarrow (\alpha+\beta+s)^{-1} = 
\frac{1-t^{2}}{\alpha+\beta}
\end{equation}

Therefore the nuclear integral in the \ref{nuclear_attraction_direct_int_eq:14}
further simplifies as:
\begin{equation}
\begin{split}
&\frac{1}{\sqrt{\pi}} 
 \int^{\infty}_{0} s^{-\frac{1}{2}} ds 
 \int x_{P}^{i}y_{P}^{j}z_{P}^{k} e^{-(\alpha+\beta)r_{P}^{2}}e^{-sr_{C}^{2}} dr \\
&= 
\sum_{i^{'}=0}^{\left[ \frac{i}{2}\right]}
\sum_{j^{'}=0}^{\left[ \frac{j}{2}\right]}
\sum_{k^{'}=0}^{\left[ \frac{k}{2}\right]} 
\frac{2\pi}{(\alpha+\beta)^{L^{'}+1}}
\binom{i}{2i^{'}}\binom{j}{2j^{'}}\binom{k}{2k^{'}}
\overline{CP}_{x}^{i-2i^{'}}\overline{CP}_{y}^{j-2j^{'}}\overline{CP}_{z}^{k-2k^{'}} \\
&\frac{(2i^{'}-1)!!(2j^{'}-1)!!(2k^{'}-1)!!}{2^{i^{'}+j^{'}+k^{'}}} 
\int^{1}_{0} dt
e^{-(\alpha+\beta)(\overline{PC}t)^{2}}
(1-t^{2})^{L^{'} }t^{2(L-2L^{'})} 
\end{split}
\label{nuclear_attraction_direct_int_eq:16}
\end{equation}

Up to this point, we note that the \ref{nuclear_attraction_direct_int_eq:16} establishes for both
$P = C$ and $P \neq C$. However, as the derivation going further, we may have to consider them
separately.

If $P \neq C$, by dropping the $\overline{PC}$, that means we set $v = \overline{PC}t$; it yields:
\begin{equation}
 \begin{split}
&\int^{1}_{0} dt
e^{-(\alpha+\beta)(\overline{PC}t)^{2}}
(1-t^{2})^{L^{'} }t^{2(L-2L^{'})} \\  
&=\frac{1}{\overline{PC}}\int^{\overline{PC}}_{0}dv e^{-(\alpha+\beta)v^{2}} 
\frac{1}{\overline{PC}^{2L^{'}}}\frac{1}{\overline{PC}^{2(L-2L^{'})}}
(\overline{PC}^{2}-v^{2})^{L^{'}}v^{2(L-2L^{'})} \\
&=\frac{1}{\overline{PC}^{2(L-L^{'})+1}}\int^{\overline{PC}}_{0}dv e^{-(\alpha+\beta)v^{2}} 
(\overline{PC}^{2}-v^{2})^{L^{'}}v^{2(L-2L^{'})}
 \end{split}
\label{nuclear_attraction_direct_int_eq:17}
\end{equation}
Then the integral can be finally transformed as:
\begin{equation}
 \label{nuclear_attraction_direct_int_eq:14}
\begin{split}
&\frac{1}{\sqrt{\pi}} 
 \int^{\infty}_{0} s^{-\frac{1}{2}} ds 
 \int x_{P}^{i}y_{P}^{j}z_{P}^{k} e^{-(\alpha+\beta)r_{P}^{2}}e^{-sr_{C}^{2}} dr \\
&= 
\sum_{i^{'}=0}^{\left[ \frac{i}{2}\right]}
\sum_{j^{'}=0}^{\left[ \frac{j}{2}\right]}
\sum_{k^{'}=0}^{\left[ \frac{k}{2}\right]} 
\frac{2\pi}{(\alpha+\beta)^{L^{'}+1}}
\binom{i}{2i^{'}}\binom{j}{2j^{'}}\binom{k}{2k^{'}}
\overline{CP}_{x}^{i-2i^{'}}\overline{CP}_{y}^{j-2j^{'}}\overline{CP}_{z}^{k-2k^{'}} \\
&\frac{(2i^{'}-1)!!(2j^{'}-1)!!(2k^{'}-1)!!}{2^{i^{'}+j^{'}+k^{'}}} 
\frac{1}{\overline{PC}^{2(L-L^{'})+1}} \\
&\int^{\overline{PC}}_{0}dv e^{-(\alpha+\beta)v^{2}} 
(\overline{PC}^{2}-v^{2})^{L^{'}}v^{2(L-2L^{'})} 
\end{split}
\label{nuclear_attraction_direct_int_eq:18}
\end{equation}

Finally, let's analyze the integral given in \ref{nuclear_attraction_direct_int_eq:18}, such integral
can be calculated on the base of:
\begin{equation}
\Gamma(x, a, m) = \int^{x}_{0} e^{-at^{2}}t^{2m} dt
 \label{nuclear_attraction_direct_int_eq:19}
\end{equation}
Where $a$ and $m$ are both positive real numbers. $a$ is $(\alpha+\beta)$, and 
$m$ is determined by the binomial expansion.

This integral could be transformed into the  ``incomplete gamma function'' by setting 
$t^{2}=w$:
\begin{align}
&\Gamma(x, a, m) = \int^{x}_{0} e^{-at^{2}}t^{2m} dt = 
 \frac{1}{2}\int^{x^{2}}_{0} e^{-aw}w^{m-\frac{1}{2}} dw \nonumber \\
&=
\frac{1}{2a^{m+\frac{1}{2}}}\int^{ax^{2}}_{0} 
e^{-y}y^{m-\frac{1}{2}} dy
\label{nuclear_attraction_direct_int_eq:20}
\end{align}
Where the fundamental integral of $\int^{ax^{2}}_{0} e^{-y}y^{m-\frac{1}{2}} dy$
could be got from standard math library. 

For the case that $P = C$, we can see that in the \ref{nuclear_attraction_direct_int_eq:16}
The CPx, CPy and CPz all become zero; it's only $2i^{'}=i$, $2j^{'}=j$ and $2k^{'}=k$
that the integral is not zero. Hence we have:
\begin{equation}
\begin{split}
&\frac{1}{\sqrt{\pi}} 
 \int^{\infty}_{0} s^{-\frac{1}{2}} ds 
 \int x_{P}^{i}y_{P}^{j}z_{P}^{k} e^{-(\alpha+\beta)r_{P}^{2}}e^{-sr_{C}^{2}} dr \\
&= 
\sum_{i^{'}=0}^{\left[ \frac{i}{2}\right]}
\sum_{j^{'}=0}^{\left[ \frac{j}{2}\right]}
\sum_{k^{'}=0}^{\left[ \frac{k}{2}\right]} 
\frac{2\pi}{(\alpha+\beta)^{L^{'}+1}}
\binom{i}{2i^{'}}\binom{j}{2j^{'}}\binom{k}{2k^{'}}
\overline{CP}_{x}^{i-2i^{'}}\overline{CP}_{y}^{j-2j^{'}}\overline{CP}_{z}^{k-2k^{'}} \\
&\frac{(2i^{'}-1)!!(2j^{'}-1)!!(2k^{'}-1)!!}{2^{i^{'}+j^{'}+k^{'}}} 
\int^{1}_{0} dt
e^{-(\alpha+\beta)(\overline{PC}t)^{2}}
(1-t^{2})^{L^{'} }t^{2(L-2L^{'})} \underrightarrow{ P = C} \\
&=\frac{2\pi}{(\alpha+\beta)^{L^{'}+1}}\frac{(i-1)!!(j-1)!!(k-1)!!}{2^{L^{'}}} 
\int^{1}_{0} dt
(1-t^{2})^{L^{'}} \\
&=\frac{2\pi}{(\alpha+\beta)^{L^{'}+1}}\frac{(i-1)!!(j-1)!!(k-1)!!}{2^{L^{'}}} 
\sum_{m=0}^{L^{'}}\binom{L^{'}}{m}(-1)^{m}\frac{1}{2m+1}
\end{split}
\label{nuclear_attraction_direct_int_eq:21}
\end{equation}


% 
% firstly set up on Jan 2012
%
% fully derived the ERI in OS framework
% derived the overlap, kinetic and nuclear integral in OS framework
%
%
%%%%%%%%%%%%%%%%%%%%%%%%%%%%%%%%%%%%%%%%%%%%%%%%%%%%%%%%%%%%%%%%%%%%%%%%%%%%%%%%
\section{OS Method}
%
%
%
%
OS method is based on two cognitions for the integral of Gaussian primitive
functions. The first cognition is that all of integral could be reduced into the
form of three body overlap integrals. The second cognition is based on
derivatives of Gaussian primitive function:
\begin{equation}
 \label{OS_general_int_eq:1}
\frac{\partial \chi}{\partial R_{x}} = \frac{\partial
(x^{l}y^{m}z^{n}e^{-\alpha r^{2}})}
{\partial R_{x}} =  -lx^{l-1}y^{m}z^{n}e^{-\alpha r^{2}} + 2\alpha
x^{l+1}y^{m}z^{n}e^{-\alpha r^{2}}
\end{equation}
Here the $x$ is expressed as:
\begin{equation}
 x = x_{e} - R_{x} 
\end{equation}
In such relation, it's clear that the $\chi(l,m,n)$, its derivatives and the
higher angular momentum one $\chi(l+1,m,n)$ are connected with each other so
that it provides an potential opportunity to link the $\chi(l+1,m,n)$
and $\chi(l,m,n)$ together through $\chi$'s derivatives.

Now following the definition in the OS method, we can express such 
relation as:
\begin{equation}
 \label{OS_general_int_eq:2}
 \frac{\partial \chi(r,\alpha,l,R)}{\partial R_{i}} = 
2\alpha\chi(r,\alpha,l+\iota_{i},R) - N_{i}(l)\chi(r,\alpha,l-\iota_{i},R)
\end{equation}
as $i = x, y, z$. Original $\chi$ is $\chi(r,\alpha,l,R)$, so $r$ is the
electron coordinate, $R$ is the nuclear coordinate, $\alpha$ is the exponent
and $l$ is the angular momentum (actually it's a three dimensional vector).
$\iota$ characterizes the arising or descending of the angular momentum,
it's actually Kronecker symbol:
\begin{equation}
 \iota_{i} = (\delta_{ix}, \delta_{iy}, \delta_{iz})
\label{OS_general_int_eq:3}
\end{equation}
Similarly, $N_{i}(l)$ is:
\begin{equation}
N_{i}(l) = 
\begin{cases}
 l_{x} & i = x \\
 l_{y} & i = y \\
 l_{z} & i = z 
\end{cases}
 \label{OS_general_int_eq:4}
\end{equation}

By moving the exponent to the left side, the \ref{OS_general_int_eq:2}
could be further expressed into:
\begin{equation}
 \label{OS_general_int_eq:5}
\chi(r,\alpha,l+\iota_{i},R) =  
\frac{1}{2\alpha}\frac{\partial \chi(r,\alpha,l,R)}{\partial R_{i}}
+ \frac{N_{i}(l)}{2\alpha}\chi(r,\alpha,l-\iota_{i},R)
\end{equation} 

%%%%%%%%%%%%%%%%%%%%%%%%%%%%%%%%%%%%%%%%%%%%%%%%%%%%%%%%%%%%%%%%%%%%%%%%%%%%%%%%
\subsection{Three Center Overlap Integral}
%
%
%
The purpose for this section, is to employ the relation in the \ref{OS_general_int_eq:5}
to derive a recursive relation for evaluating the three center overlap integral
in terms of the corresponding lower angular momentum integrals. 

Now let's suggest a three body overlap integral between $\chi_{a}, \chi_{b}$
and $\chi_{c}$:
\begin{equation}
 \begin{split}
(a|b|c) &= \int \chi_{a}(r)\chi_{b}(r)\chi_{c}(r) dr \\
&= \int x^{l_{A}}_{A}y^{m_{A}}_{A}z^{n_{A}}_{A}e^{-\alpha r_{A}^{2}}
        x^{l_{B}}_{B}y^{m_{B}}_{B}z^{n_{B}}_{B}e^{-\beta  r_{B}^{2}} 
        x^{l_{C}}_{C}y^{m_{C}}_{C}z^{n_{C}}_{C}e^{-\gamma r_{C}^{2}}dr
 \end{split}
\label{OS_three_overlap_int_eq:1}
\end{equation}
A,B and C could be the same center, or different centers. From the previous
chapter to evaluate the overlap integral, we know that we could combine
$\chi_{a}$ and $\chi_{b}$ together and then combine the new Gaussian primitive
which centers at $p$ with the $\chi_{c}$. The integral could be generally expressed as:
\begin{equation}
\label{general_expression_for_3_overlap_os}
\begin{split}
 \int \chi_{a}(r)\chi_{b}(r)\chi_{c}(r) dr &= 
\kappa_{abc}I_{abc}^{x}I_{abc}^{y}I_{abc}^{z} \\
&= e^{-\frac{\alpha\beta}{\alpha+\beta}|AB|^{2}}
e^{-\frac{(\alpha+\beta)\gamma}{\alpha+\beta+\gamma}|PC|^{2}} \\
& \int x^{l_{A}}_{A}y^{m_{A}}_{A}z^{n_{A}}_{A}
       x^{l_{B}}_{B}y^{m_{B}}_{B}z^{n_{B}}_{B}
       x^{l_{C}}_{C}y^{m_{C}}_{C}z^{n_{C}}_{C} 
e^{-(\alpha+\beta+\gamma)|r_{G}|^{2}} dr        
\end{split}
\end{equation}
In the combination of Gaussian primitives, new centers generated are given as:
\begin{equation}
 \begin{split}
  \overrightarrow{P} &= \frac{\alpha \overrightarrow{A} + 
\beta \overrightarrow{B}}{\alpha+\beta}
\Rightarrow \\
  \overrightarrow{G} &= \frac{(\alpha+\beta)\overrightarrow{P} + \gamma
\overrightarrow{C}}{ \alpha+\beta +\gamma } \\
    &= \frac{\alpha \overrightarrow{A} + \beta \overrightarrow{B} + 
\gamma \overrightarrow{C}}{ \alpha+\beta +\gamma }
 \end{split}
\label{OS_three_overlap_int_eq:2}
\end{equation}
Here we intentionally do not expand the multiplication of 
$x^{l_{A}}_{A}y^{m_{A}}_{A}z^{n_{A}}_{A}$ etc. since we want to keep it
into some symmetrical form. The essence for deriving the three center of 
overlap integral, is not to really calculate it; but gain some recursive relationship
among the integrals. Here the symmetry of the integral \ref{general_expression_for_3_overlap_os}
is some key feature.

Firstly, for the pre-factor we could express it into:
\begin{equation}
\kappa_{abc} = e^{-\frac{\alpha\beta}{\alpha+\beta}|AB|^{2}}
e^{-\frac{(\alpha+\beta)\gamma}{\alpha+\beta+\gamma}|PC|^{2}}
\label{OS_three_overlap_int_eq:3}
\end{equation}
Now we expand the \ref{OS_three_overlap_int_eq:3} into (We use $A_{i}$, $B_{i}$
and $C_{i}$ to represent it's component on X, Y or Z direction, i could be
x, y, or z):
\begin{equation}
 \begin{split}
 &e^{-\frac{\alpha\beta}{\alpha+\beta}(A_{i}-B_{i})^{2}}
e^{-\frac{(\alpha+\beta)\gamma}{\alpha+\beta+\gamma}(P_{i}-C_{i})^{2}} \\
&=e^{-\frac{\alpha\beta}{\alpha+\beta}(A_{i}-B_{i})^{2}} 
e^{-\frac{\gamma}{(\alpha+\beta)(\alpha+\beta+\gamma)}
((\alpha+\beta)C_{i}-\alpha A_{i} - \beta B_{i})^{2}} \\
 \end{split}
\label{OS_three_overlap_int_eq:4}
\end{equation}
For the exponents we have:
\begin{equation}
 \begin{split}
 & \frac{\alpha\beta}{\alpha+\beta}(A_{i}-B_{i})^{2}
+ \frac{\gamma}{(\alpha+\beta)(\alpha+\beta+\gamma)}
((\alpha+\beta)C_{i} -\alpha A_{i} - \beta B_{i})^{2} \\
&=  \frac{\alpha\beta(\alpha+\beta+\gamma)(A_{i}-B_{i})^{2}
+ \gamma((\alpha+\beta)C_{i} -\alpha A_{i} - \beta B_{i})^{2}}
{(\alpha+\beta)(\alpha+\beta+\gamma)} \\
&= \frac{\alpha^{2}\beta(A_{i}-B_{i})^{2} 
+ \alpha\beta^{2}(A_{i}-B_{i})^{2}+\alpha\beta\gamma(A_{i}-B_{i})^{2}}
{(\alpha+\beta)(\alpha+\beta+\gamma)} \\
&+ \frac{\gamma(\alpha+\beta)^{2}C_{i}^{2}+\alpha^{2}\gamma A_{i}^{2} +
\beta^{2}\gamma B_{i}^{2} }{(\alpha+\beta)(\alpha+\beta+\gamma)} \\
&+ \frac{-2\alpha\gamma(\alpha+\beta)A_{i}C_{i}
- 2\beta\gamma(\alpha+\beta)B_{i}C_{i} + 2\alpha\beta\gamma A_{i}B_{i}}
{(\alpha+\beta)(\alpha+\beta+\gamma)}  \\
 \end{split}
\label{OS_three_overlap_int_eq:5}
\end{equation}
Now let's re-arrange the terms in terms of the order of $\alpha, \beta$
and $\gamma$:
\begin{equation}
 \begin{split}
  & \frac{\alpha\beta}{\alpha+\beta}(A_{i}-B_{i})^{2}
+ \frac{\gamma}{(\alpha+\beta)(\alpha+\beta+\gamma)}
((\alpha+\beta)C_{i}-\alpha A_{i} - \beta B_{i})^{2} \\
&= \frac{\alpha^{2}\beta(A_{i}-B_{i})^{2} 
+ \alpha\beta^{2}(A_{i}-B_{i})^{2}+
\alpha^{2}\gamma(A_{i}-C_{i})^{2} 
+ \beta^{2}\gamma(B_{i}-C_{i})^{2}}{(\alpha+\beta)(\alpha+\beta+\gamma)} \\
& +\frac{
  \alpha\beta\gamma(A_{i}^{2}+B_{i}^{2}+2C_{i}^{2}-2A_{i}C_{i}-2B_{i}C_{i})}
{(\alpha+\beta)(\alpha+\beta+\gamma)}  \\
&= \frac{\alpha^{2}\beta(A_{i}-B_{i})^{2} 
+ \alpha\beta^{2}(A_{i}-B_{i})^{2}+
\alpha^{2}\gamma(A_{i}-C_{i})^{2} 
+ \beta^{2}\gamma(B_{i}-C_{i})^{2}}{(\alpha+\beta)(\alpha+\beta+\gamma)} \\
&+\frac{
  \alpha\beta\gamma\left( (C_{i}-A_{i})^{2} + (C_{i}-B_{i})^{2}\right)}
{(\alpha+\beta)(\alpha+\beta+\gamma)} \\
&= \frac{\alpha\beta(A_{i}-B_{i})^{2}(\alpha+\beta) +
\alpha\gamma(A_{i}-C_{i})^{2}(\alpha+\beta) 
+ \beta\gamma(B_{i}-C_{i})^{2}(\alpha+\beta)}
{(\alpha+\beta)(\alpha+\beta+\gamma)} \\
&= \frac{\alpha\beta(A_{i}-B_{i})^{2} +
\alpha\gamma(A_{i}-C_{i})^{2} 
+ \beta\gamma(B_{i}-C_{i})^{2}}
{(\alpha+\beta+\gamma)}
 \end{split}
\label{OS_three_overlap_int_eq:6}
\end{equation}
The expression in \ref{OS_three_overlap_int_eq:6} is symmetric. Furthermore, we 
could re-form it by using the \ref{OS_three_overlap_int_eq:2} since $G$ will also
appears in the integral part:
\begin{equation}
 \begin{split}
 &\frac{\alpha\beta(A_{i}-B_{i})^{2} + \alpha\gamma(A_{i}-C_{i})^{2}  +
\beta\gamma(B_{i}-C_{i})^{2}}
 {(\alpha+\beta+\gamma)} \\
&= -(\alpha+\beta+\gamma)\left( G_{i}^{2} - \frac{\alpha A_{i}^{2} + \beta
B_{i}^{2} + \gamma C_{i}^{2}}
{\alpha+\beta+\gamma}\right)  
 \end{split}
\label{OS_three_overlap_int_eq:7}
\end{equation}
Therefore, for the pre-factor we have:
\begin{align}
\kappa_{abc} &= e^{-\frac{\alpha\beta}{\alpha+\beta}|A-B|^{2}}
e^{-\frac{(\alpha+\beta)\gamma}{\alpha+\beta+\gamma}|P-C|^{2}} \nonumber \\
&= e^{(\alpha+\beta+\gamma)\left( |G|^{2} - \frac{\alpha |A|^{2} + \beta |B|^{2} + \gamma |C|^{2}}
{\alpha+\beta+\gamma}\right)}
\label{OS_three_overlap_int_eq:8}
\end{align}
Its derivatives for the $R_{i}$ is:
\begin{equation}
 \label{OS_three_overlap_int_eq:9}
\frac{\partial \kappa_{abc}}{R_{i}} = (2\alpha G_{Ai} - 2\alpha
A_{i})\kappa_{abc}
\end{equation}

Now let's expand the integral of $I^{x}_{abc}$:
\begin{equation}
 \begin{split}
  I^{x}_{abc}(l_{A},l_{B},l_{C}) &= \sum_{l_{1}=0}^{l_{A}}\sum_{l_{2}=0}^{l_{B}}
\sum_{l_{3}=0}^{l_{C}}  
\binom{l_{A}}{l_{1}}\binom{l_{B}}{l_{2}}\binom{l_{C}}{l_{3}} \\
&(G_{x}-A_{x})^{(l_{A}-l_{1})} 
 (G_{x}-B_{x})^{(l_{B}-l_{2})}
 (G_{x}-C_{x})^{(l_{C}-l_{3})} \\
&\int (x-G_{x})^{l_{1}+l_{2}+l_{3}}
e^{-(\alpha+\beta+\gamma)(x-G_{x})^{2}} dx \\
&= \sqrt{\frac{\pi}{\alpha+\beta+\gamma}}\sum_{l_{1}=0}^{l_{A}}\sum_{l_{2}=0}^{l_{B}}
\sum_{l_{3}=0}^{l_{C}}  
\binom{l_{A}}{l_{1}}\binom{l_{B}}{l_{2}}\binom{l_{C}}{l_{3}} \\
&(G_{x}-A_{x})^{(l_{A}-l_{1})} 
 (G_{x}-B_{x})^{(l_{B}-l_{2})}
 (G_{x}-C_{x})^{(l_{C}-l_{3})} \\
&\frac{(l_{1}+l_{2}+l_{3}-1)!!}
{\left\lbrace 2(\alpha+\beta+\gamma)\right\rbrace^{l_{1}+l_{2}+l_{3}} }
 \end{split}
\label{OS_three_overlap_int_eq:10}
\end{equation}
We note that $l_{1}+l_{2}+l_{3}$ should be even number else the integral is zero. For
the integral on the y, z direction, we have the similar result, too.
It's derivative could be expressed as:
\begin{equation}
\begin{split}
 \frac{\partial I^{x}_{abc}}{\partial R_{x}} &= 
l_{A}\left( \frac{\alpha}{\alpha+\beta+\gamma} -1\right)I^{x}_{abc}(l_{A}-1,l_{B},l_{C}) \\
&+ l_{B}\left( \frac{\alpha}{\alpha+\beta+\gamma}\right)I^{x}_{abc}(l_{A},l_{B}-1,l_{C}) \\
&+ l_{C}\left( \frac{\alpha}{\alpha+\beta+\gamma}\right)I^{x}_{abc}(l_{A},l_{B},l_{C}-1) 
\end{split}
 \label{OS_three_overlap_int_eq:11}
\end{equation}
We note that for the derivatives of $I^{i}_{abc}$ in terms of $R_{j}$ ($j \neq i$), the result
is obviously zero. 

Now let's combine the result in \ref{OS_three_overlap_int_eq:9} and \ref{OS_three_overlap_int_eq:11},
and also employing the relation in \ref{OS_general_int_eq:5}; we can arrive at some symmetrical
recursive relation for the $(a|b|c)$:
\begin{equation}
 \begin{split}
 (a+\iota_{x}|b|c) &= \frac{1}{2\alpha}\frac{\partial }{\partial R_{Ax}}(a|b|c) 
+ \frac{l_{A}}{2\alpha}(a-\iota_{x}|b|c) \\
&= (G_{Ax} - A_{x})(a|b|c) + 
l_{A}\left(\frac{1}{2(\alpha+\beta+\gamma)} -\frac{1}{2\alpha}\right)(a-\iota_{x}|b|c) \\
&+ l_{B}\left(\frac{1}{2(\alpha+\beta+\gamma)}\right)(a|b-\iota_{x}|c)  \\
&+ l_{C}\left(\frac{1}{2(\alpha+\beta+\gamma)}\right)(a|b|c-\iota_{x}) +  
\frac{l_{A}}{2\alpha}(a-\iota_{x}|b|c) \\
&= (G_{Ax} - A_{x})(a|b|c) + 
l_{A}\left(\frac{1}{2(\alpha+\beta+\gamma)}\right)(a-\iota_{x}|b|c) \\
&+ 
l_{B}\left(\frac{1}{2(\alpha+\beta+\gamma)}\right)(a|b-\iota_{x}|c) \\
&+
l_{C}\left(\frac{1}{2(\alpha+\beta+\gamma)}\right)(a|b|c-\iota_{x}) 
 \end{split}
\label{OS_three_overlap_int_eq:12}
\end{equation}
Such relation could be generalized into the derivative for $R_{Xi}$ for all of 
center of X and all of components of $i=x,y,z$. The relation is the starting point for 
all of following derivations.

At last, let's consider that if one of primitive function is ``S'' type of 
function, how to reduce the recursive relation in
\ref{OS_three_overlap_int_eq:12}. For the S type of function it does not have
the angular momentum part, the pre-factor part part still exists; but for the
integral in \ref{OS_three_overlap_int_eq:10}, the derivatives for the S type 
function will be zero. For example, suggest that $a$ is the S type of function,
then it's obvious that:
\begin{equation}
\begin{split}
 \frac{\partial I^{x}_{abc}}{\partial R_{x}} &= 
l_{B}\left(
\frac{\alpha}{\alpha+\beta+\gamma}\right)I^{x}_{abc}(l_{A},l_{B}-1,l_{C}) \\
&+ l_{C}\left(
\frac{\alpha}{\alpha+\beta+\gamma}\right)I^{x}_{abc}(l_{A},l_{B},l_{C}-1) 
\end{split}
 \label{OS_three_overlap_int_eq:13}
\end{equation}
Then combined with \ref{OS_three_overlap_int_eq:12}, it gives:
\begin{equation}
 \begin{split}
 (0+\iota_{x}|b|c) &= \frac{1}{2\alpha}\frac{\partial }{\partial R_{Ax}}(0|b|c) 
\\
&= (G_{Ax} - A_{x})(0|b|c) 
+ l_{B}\left(\frac{1}{2(\alpha+\beta+\gamma)}\right)(0|b-\iota_{x}|c) \\ 
&+ l_{C}\left(\frac{1}{2(\alpha+\beta+\gamma)}\right)(0|b|c-\iota_{x}) 
 \end{split}
\label{OS_three_overlap_int_eq:14}
\end{equation}
For the other S type centers we have the similar relation, too.

From the above derivation, the bottom integral of three center 
overlap could be expressed as:
\begin{equation}
 \begin{split}
  (0_{A}|0_{B}|0_{C}) &= \int e^{-\alpha r_{A}^{2}}e^{-\beta  r_{B}^{2}} 
             e^{-\gamma r_{C}^{2}}dr \\
          &= e^{-\frac{\alpha\beta}{\alpha+\beta}|AB|^{2}}
e^{-\frac{(\alpha+\beta)\gamma}{\alpha+\beta+\gamma}|PC|^{2}} 
\int e^{-(\alpha+\beta+\gamma)|r_{G}|^{2}} dr \\
          &= e^{-\frac{\alpha\beta}{\alpha+\beta}|AB|^{2}}
e^{-\frac{(\alpha+\beta)\gamma}{\alpha+\beta+\gamma}|PC|^{2}} 
\left( \frac{\pi}{\alpha+\beta+\gamma}\right)^{\frac{3}{2}} 
 \end{split}
 \label{OS_bottom_three_overlap_int_1}
\end{equation}

We can also write the above integral in another form:
\begin{equation}
 \begin{split}
  (0_{A}|0_{B}|0_{C}) &= e^{-\frac{\alpha\beta}{\alpha+\beta}|AB|^{2}}
e^{-\frac{(\alpha+\beta)\gamma}{\alpha+\beta+\gamma}|PC|^{2}} 
\left( \frac{\pi}{\alpha+\beta+\gamma}\right)^{\frac{3}{2}} \\
&= (0_{A}|0_{B})\left( \frac{\alpha+\beta}{\alpha+\beta+\gamma}\right)^{\frac{3}{2}}
e^{-\frac{(\alpha+\beta)\gamma}{\alpha+\beta+\gamma}|PC|^{2}} 
 \end{split}
 \label{OS_bottom_three_overlap_int_2}
\end{equation}
where $(0_{A}|0_{B})$ is the overlap integral in \ref{overlap_direct_int_eq:1}.

%%%%%%%%%%%%%%%%%%%%%%%%%%%%%%%%%%%%%%%%%%%%%%%%%%%%%%%%%%%%%%%%%%%%%%%%%%%%%%%%
\subsection{Electron Repulsion Integrals}
\label{os_eri}
%
%
%
%
Now we begin to use the conclusion got in the previous section to solve the
real problems. This section we are going to tackle down the most difficult one,
the double electrons integral.

Let's consider some electronic repulsion integral over the
Gaussian primitive functions:
\begin{equation}
 \label{OS_ERI_eq:1}
(ab|cd) = \int dr \int dr^{'} \chi_{a}(r)\chi_{b}(r)\frac{1}{|r-r^{'}|}
\chi_{c}(r^{'})\chi_{d}(r^{'})
\end{equation}
a,b,c,d are just some general Gaussian primitive functions.

Firstly we do transformation to the $\frac{1}{|r-r^{'}|}$:
\begin{equation}
 \frac{1}{|r-r^{'}|} = \frac{2}{\pi^{1/2}}\int^{\infty}_{0}
e^{(r-r^{'})^{2}u^{2}}du
\end{equation}
Here $(r-r^{'})^{2}$ is 
\begin{equation}
(r-r^{'})^{2} = (x_{r}-x_{r^{'}})^{2} +  (y_{r}-y_{r^{'}})^{2} +
(z_{r}-z_{r^{'}})^{2}
\end{equation}
We note the $r$ and $r^{'}$ are more like the A, B defined in the 
\ref{OS_three_overlap_int_eq:4}. Based on this transformation,
the integral could be reformed as:
\begin{equation}
 \begin{split}
 (ab|cd) &= \frac{2}{\pi^{1/2}}\int^{\infty}_{0} du 
\int dr \int dr^{'} \chi_{a}(r)\chi_{b}(r) e^{(r-r^{'})^{2}u^{2}}
\chi_{c}(r^{'})\chi_{d}(r^{'}) \\
&= \frac{2}{\pi^{1/2}}\int^{\infty}_{0} du
\int dr^{'}\chi_{c}(r^{'})\chi_{d}(r^{'}) 
\left( \int dr  \chi_{a}(r)\chi_{b}(r) e^{(r-r^{'})^{2}u^{2}}\right) \\ 
&= \frac{2}{\pi^{1/2}}\int^{\infty}_{0} du
\int dr^{'}\chi_{c}(r^{'})\chi_{d}(r^{'})(a|0_{r^{'}}|b) \\
&= \frac{2}{\pi^{1/2}}\int^{\infty}_{0} du (ab|u|cd)
 \end{split}
\label{OS_ERI_eq:2}
\end{equation}
Therefore, the two electron integral now is converted into the three
center overlap integral.

Before we move on, it's useful to remind us the key idea of the derivation
we are going to. We are going to use the recursive relation for the three
center overlap integral, to derive the potential recursive relation for the
ERI. Furthermore, before we really move into the ERI section, actually there are
some very fancy formulas needed to be introduced first. Suggest that there are
three variables, $u$, $\epsilon$ and $\eta$, they are all independent with each
other; we can prove that:
\begin{equation}
 \frac{1}{\epsilon+u^{2}} = \frac{1}{\epsilon}\left( 1-\frac{\rho}{\epsilon}
\frac{u^{2}}{\rho+u^{2}}\right) - \frac{1}{\epsilon+\eta}
\frac{u^{2}}{\epsilon+u^{2}}\frac{u^{2}}{\rho+u^{2}} 
\label{OS_ERI_eq:3}
\end{equation}
Where the $\rho$ is:
\begin{equation}
 \rho = \frac{\epsilon\eta}{\epsilon+\eta}
\end{equation}
This intelligent identity is not so obvious by judging from its appearance.
However, this identity possesses a key path to the final form of expression.
Now let's go to see how to prove it.
\begin{equation}
 \begin{split}
 &\frac{u^{2}}{\rho+u^{2}}\left( \frac{\rho}{\epsilon^{2}} + 
\frac{1}{\epsilon+\eta}\frac{u^{2}}{\epsilon+u^{2}}\right) \\
&= \frac{u^{2}}{\rho+u^{2}}\left(
\frac{\epsilon\eta}{\epsilon^{2}(\epsilon+\eta)} + 
\frac{1}{\epsilon+\eta}\frac{u^{2}}{\epsilon+u^{2}}\right) \\
&= \frac{u^{2}}{\rho+u^{2}}\frac{1}{\epsilon+\eta}\left(
\frac{\epsilon\eta}{\epsilon^{2}} + 
\frac{u^{2}}{\epsilon+u^{2}}\right) \\
&= \frac{u^{2}}{\rho+u^{2}}\frac{1}{\epsilon+\eta}\left(
\frac{\eta}{\epsilon} + 
\frac{u^{2}}{\epsilon+u^{2}}\right) \\
&= \frac{u^{2}}{\rho+u^{2}}\frac{1}{\epsilon+\eta}
\frac{\epsilon\eta + \eta u^{2} + \epsilon u^{2}}{\epsilon(\epsilon+u^{2})} \\
&= \frac{u^{2}}{\rho+u^{2}}\frac{1}{\epsilon+\eta}
\frac{\epsilon\eta + (\eta+ \epsilon)u^{2}}{\epsilon(\epsilon+u^{2})} \\
&= \frac{u^{2}}{\rho+u^{2}}\frac{1}{\epsilon+\eta}
\frac{(\rho + u^{2})(\eta+ \epsilon)}{\epsilon(\epsilon+u^{2})} \\
&= \frac{u^{2}}{\epsilon(\epsilon+u^{2})}
 \end{split} 
\label{OS_ERI_eq:4}
\end{equation}
Now let's combine the result in (\ref{OS_ERI_eq:4}) with the last term in
\ref{OS_ERI_eq:3}, then it gives:
\begin{equation}
 \begin{split}
  \frac{1}{\epsilon}-\frac{u^{2}}{\epsilon(\epsilon+u^{2})}  
= \frac{1}{\epsilon}\left( \frac{\epsilon}{\epsilon+u^{2}}\right) 
= \frac{1}{\epsilon+u^{2}}
 \end{split}
\label{OS_ERI_eq:5}
\end{equation}
Which is the final result in \ref{OS_ERI_eq:3}. Here we note, that the result
in the \ref{OS_ERI_eq:3} will provide us a chance to link the primitives on 
$r$ and primitives on $r^{'}$ in \ref{OS_ERI_eq:1}.

Now let's firstly solve this integral of $(a|0_{r^{'}}|b)$. We would apply
the recursive relation we got in the \ref{OS_three_overlap_int_eq:12}:
\begin{equation}
 \begin{split}
 (a+\iota_{i}|b|c) 
&= (G_{Ai} - A_{i})(a|b|c) + 
N_{i}(A)\left(\frac{1}{2(\alpha+\beta+\gamma)}\right)(a-\iota_{i}|b|c) \\
&+ 
N_{i}(B)\left(\frac{1}{2(\alpha+\beta+\gamma)}\right)(a|b-\iota_{i}|c) \\
&+
N_{i}(C)\left(\frac{1}{2(\alpha+\beta+\gamma)}\right)(a|b|c-\iota_{i}) 
 \end{split}
\label{OS_ERI_eq:6}
\end{equation}
Here we extended it to the general form, and the $\iota_{i}$ and $N_{i}$ are just
the same as \ref{OS_general_int_eq:4} etc. $i$ could be $x,y, z$.

Let's consider to fit it to $(a|0_{r^{'}}|b)$. Here the $\gamma$ is $u^{2}$, and
for the center B (it's the $r^{'}$) we only have a S type of orbital so the
expression could be further expressed as:
\begin{equation}
 \begin{split}
  (a+\iota_{i}|0_{r^{'}}|b)
&=(G_{Ai} - A_{i})(a|0_{r^{'}}|b) +
N_{i}(A)\left(\frac{1}{2(\alpha+\beta+u^{2})}\right)(a-\iota_{i}|0_{r^{'}}|b) \\
&+
N_{i}(B)\left(\frac{1}{2(\alpha+\beta+u^{2})}\right)(a|0_{r^{'}}|b-\iota_{i}) \\
&=(G_{Ai} - A_{i})(a|0_{r^{'}}|b) +
N_{i}(A)\left(\frac{1}{2(\epsilon+u^{2})}\right)(a-\iota_{i}|0_{r^{'}}|b) \\
&+
N_{i}(B)\left(\frac{1}{2(\epsilon+u^{2})}\right)(a|0_{r^{'}}|b-\iota_{i})
 \end{split}
\label{OS_ERI_eq:7}
\end{equation}

According to the \ref{OS_ERI_eq:3}, $\dfrac{1}{\epsilon+u^{2}}$ could be
directly expanded; so the result could be reformed into:
\begin{equation}
 \begin{split}
  (a+\iota_{i}|0_{r^{'}}|b)
&=(G_{Ai} - A_{i})(a|0_{r^{'}}|b) \\
&+
\frac{N_{i}(A)}{2\epsilon}\left(1-\frac{\rho}{\epsilon}
\frac{u^{2}}{\rho+u^{2}}\right)(a-\iota_{i}|0_{r^{'}}|b) \\
&-\frac{N_{i}(A)}{2}\left(\frac{1}{\epsilon+\eta}
\frac{u^{2}}{\epsilon+u^{2}}\frac{u^{2}}{\rho+u^{2}}\right)(a-\iota_{i}|0_{r^{'}}|b) \\
&+ \frac{N_{i}(B)}{2\epsilon}\left(1-\frac{\rho}{\epsilon}
\frac{u^{2}}{\rho+u^{2}}\right)(a|0_{r^{'}}|b-\iota_{i}) \\
&-\frac{N_{i}(B)}{2}\left(\frac{1}{\epsilon+\eta}
\frac{u^{2}}{\epsilon+u^{2}}\frac{u^{2}}{\rho+u^{2}}\right)(a|0_{r^{'}}|b-\iota_{i})
 \end{split}
\label{OS_ERI_eq:8}
\end{equation}
The $\eta$ and corresponding $\rho$ could be referred later. Now they are only
arbitrary numbers ($\rho$ depends on $\eta$ and $\epsilon$). 

Now we are trying to wrap up some terms in the \ref{OS_ERI_eq:8} into a new
form:
\begin{equation}
 \begin{split}
  &(a|0_{r^{'}}+\iota_{i}|b) \\
  & =(0_{r^{'}}+\iota_{i}|a|b)  \\
  &= \left( \frac{\alpha A_{i} + \beta B_{i} + u^{2}r^{'}_{i}}
{\alpha +\beta + u^{2}} - r^{'}_{i}\right)(a|0_{r^{'}}|b)\\
&+
N_{i}(A)\left(\frac{1}{2(\epsilon+u^{2})}\right)(a-\iota_{i}|0_{r^{'}}|b) \\
&+
N_{i}(B)\left(\frac{1}{2(\epsilon+u^{2})}\right)(a|0_{r^{'}}|b-\iota_{i})
 \end{split}
\label{OS_ERI_eq:9}
\end{equation}
Since
\begin{equation}
 \begin{split}
  \frac{\alpha A_{i} + \beta B_{i} + u^{2}r^{'}_{i}}
{\alpha +\beta + u^{2}} - r^{'}_{i} = 
 \frac{\alpha A_{i} + \beta B_{i} -r^{'}_{i}(\alpha+\beta)}
{\epsilon + u^{2}} = \frac{P_{i}\epsilon -r^{'}_{i}\epsilon}
{\epsilon + u^{2}}
 \end{split}
\end{equation}
where 
\begin{equation}
 P_{i} = \frac{\alpha A_{i} + \beta B_{i}}{\alpha+\beta}
\end{equation}
and $\epsilon = \alpha+\beta$, therefore the \ref{OS_ERI_eq:9} could be
expressed as:
\begin{equation}
 \begin{split}
&(a|0_{r^{'}}+\iota_{i}|b) \\
&=-\frac{\epsilon}{\epsilon + u^{2}}(r^{'}_{i} - P_{i})(a|0_{r^{'}}|b) \\
&+
N_{i}(A)\left(\frac{1}{2(\epsilon+u^{2})}\right)(a-\iota_{i}|0_{r^{'}}|b) \\
&+
N_{i}(B)\left(\frac{1}{2(\epsilon+u^{2})}\right)(a|0_{r^{'}}|b-\iota_{i})  
 \end{split}
\label{OS_ERI_eq:10}
\end{equation}
Here we note that our purpose here is try to build some expression, which
could reduce the integral into some potential recursive forms.

Now we can use the $(a|0_{r^{'}}+\iota_{i}|b)$ to transform the
\ref{OS_ERI_eq:8}:
\begin{equation}
 \begin{split}
    (a+\iota_{i}|0_{r^{'}}|b)
&=(G_{Ai} - A_{i})(a|0_{r^{'}}|b) \\
&+
\frac{N_{i}(A)}{2\epsilon}\left(1-\frac{\rho}{\epsilon}
\frac{u^{2}}{\rho+u^{2}}\right)(a-\iota_{i}|0_{r^{'}}|b) \\
&+ \frac{N_{i}(B)}{2\epsilon}\left(1-\frac{\rho}{\epsilon}
\frac{u^{2}}{\rho+u^{2}}\right)(a|0_{r^{'}}|b-\iota_{i}) \\
&-\frac{u^{2}}{\epsilon+\eta}\frac{u^{2}}{\rho+u^{2}}(a|0_{r^{'}}+\iota_{i}|b) \\
&-\frac{\epsilon}{\epsilon+u^{2}}
\frac{u^{2}}{\epsilon+\eta}\frac{u^{2}}{\rho+u^{2}}(r^{'}_{i} - P_{i})((a|0_{r^{'}}|b)) 
 \end{split}
\label{OS_ERI_eq:11}
\end{equation}

Now let's use the relation defined in the \ref{OS_ERI_eq:3}
to expand the $(G_{Ai} - A_{i})$. We note that the $G_{Ai}$ is:
\begin{equation}
 G_{Ai} = \frac{\alpha A_{i} + \beta B_{i} + u^{2} r^{'}_{i}}{\alpha + \beta +
u^{2}}
 = \frac{\alpha A_{i} + \beta B_{i} + u^{2} r^{'}_{i}}{\epsilon + u^{2}}
\end{equation}
Hence, for $(G_{Ai} - A_{i})$ according to \ref{OS_ERI_eq:3} it's:
\begin{equation}
 \begin{split}
 G_{Ai} - A_{i} &= \left( \frac{\alpha A_{i} + \beta B_{i}}{\epsilon} -
A_{i}\right) +
\frac{u^{2} r^{'}_{i}}{\epsilon} \\
&-(\alpha A_{i} + \beta B_{i} + u^{2} r^{'}_{i})\left(\frac{\rho}{\epsilon^{2}}
\frac{u^{2}}{\rho+u^{2}} + \frac{1}{\epsilon+\eta}
\frac{u^{2}}{\epsilon+u^{2}}\frac{u^{2}}{\rho+u^{2}} 
\right)  \\
&= \left( P_{i} - A_{i}\right) + \frac{u^{2} r^{'}_{i}}{\epsilon} \\
&-\left( \frac{u^{2}r^{'}_{i}}{\epsilon} + P_{i}\right)
\left(\frac{\rho}{\epsilon}
\frac{u^{2}}{\rho+u^{2}} + \frac{\epsilon}{\epsilon+\eta}
\frac{u^{2}}{\epsilon+u^{2}}\frac{u^{2}}{\rho+u^{2}} 
\right) 
 \end{split}
\label{OS_ERI_eq:12}
\end{equation}

Now let's combine the terms for the $(a|0_{r^{'}}|b)$ in the \ref{OS_ERI_eq:11}
and \ref{OS_ERI_eq:12}. For the $P_{i}$ term, we can see that:
\begin{equation}
 \left( P_{i} - A_{i}\right)(a|0_{r^{'}}|b) 
-P_{i}\frac{\rho}{\epsilon}
\frac{u^{2}}{\rho+u^{2}}(a|0_{r^{'}}|b)
\label{OS_ERI_eq:13}
\end{equation}
the terms for the $\dfrac{\epsilon}{\epsilon+\eta}
\dfrac{u^{2}}{\epsilon+u^{2}}\dfrac{u^{2}}{\rho+u^{2}} $ canceled. 

For the $r_{i}^{'}$, the remaining terms are:
\begin{equation}
 \begin{split}
&\frac{u^{2} r^{'}_{i}}{\epsilon}
-\frac{u^{2}r^{'}_{i}}{\epsilon}
\left(\frac{\rho}{\epsilon}
\frac{u^{2}}{\rho+u^{2}} + \frac{\epsilon}{\epsilon+\eta}
\frac{u^{2}}{\epsilon+u^{2}}\frac{u^{2}}{\rho+u^{2}} 
\right) \\
&-\frac{\epsilon}{\epsilon+u^{2}}
\frac{u^{2}}{\epsilon+\eta}\frac{u^{2}}{\rho+u^{2}}r^{'}_{i} 
 \end{split}
\end{equation}
We do not include the integral of $(a|0_{r^{'}}|b)$ here since they are similar
terms in terms of $(a|0_{r^{'}}|b)$.

Here we have:
\begin{equation}
\frac{\epsilon}{\epsilon+\eta}
\frac{u^{2}}{\epsilon+u^{2}}\frac{u^{2}}{\rho+u^{2}}\left(
1+\frac{u^{2}}{\epsilon} \right)r^{'}_{i} = 
\frac{u^{2}}{\epsilon+\eta}
\frac{u^{2}}{\rho+u^{2}}r^{'}_{i} 
\end{equation}
Then
\begin{equation}
\begin{split}
&\frac{u^{2}}{\epsilon+\eta}
\frac{u^{2}}{\rho+u^{2}}r^{'}_{i} + 
\frac{u^{2}r^{'}_{i}}{\epsilon}
\frac{\rho}{\epsilon}
\frac{u^{2}}{\rho+u^{2}} = \\
&{u^{2}}r^{'}_{i}
\frac{u^{2}}{\rho+u^{2}}\left(\frac{1}{\epsilon+\eta} + 
\frac{\rho}{\epsilon^{2}} \right) \\
&= {u^{2}}r^{'}_{i}
\frac{u^{2}}{\rho+u^{2}}\left(\frac{1}{\epsilon+\eta} + 
\frac{1}{\epsilon+\eta}\frac{\eta}{\epsilon} \right) \\
&= {u^{2}}r^{'}_{i}
\frac{u^{2}}{\rho+u^{2}}\frac{1}{\epsilon}
\end{split}
\end{equation}
Finally
\begin{equation}
\frac{u^{2} r^{'}_{i}}{\epsilon}
-{u^{2}}r^{'}_{i}
\frac{u^{2}}{\rho+u^{2}}\frac{1}{\epsilon} = r^{'}_{i}
\frac{u^{2}}{\rho+u^{2}}\frac{\rho}{\epsilon}
\end{equation}
Surprisingly, this term is directly corresponding to the terms for $P_{i}$
in the \ref{OS_ERI_eq:13}, so totally we have the result that:
\begin{equation}
\begin{split}
&(G_{Ai} - A_{i})(a|0_{r^{'}}|b)
-\frac{\epsilon}{\epsilon+u^{2}}
\frac{u^{2}}{\epsilon+\eta}\frac{u^{2}}{\rho+u^{2}}(r^{'}_{i} -
P_{i})(a|0_{r^{'}}|b)  \\
&= \left( P_{i} - A_{i}\right)(a|0_{r^{'}}|b) + 
(r^{'}_{i}-P_{i})\frac{\rho}{\epsilon}
\frac{u^{2}}{\rho+u^{2}}(a|0_{r^{'}}|b)
\end{split}
\label{OS_ERI_eq:14}
\end{equation}
 
By using the \ref{OS_ERI_eq:14}, the \ref{OS_ERI_eq:8} could be finally
expressed as:
\begin{equation}
 \begin{split}
  (a+\iota_{i}|0_{r^{'}}|b)
&=(P_{i} - A_{i})(a|0_{r^{'}}|b) \\
&+
\frac{N_{i}(A)}{2\epsilon}\left(1-\frac{\rho}{\epsilon}
\frac{u^{2}}{\rho+u^{2}}\right)(a-\iota_{i}|0_{r^{'}}|b) \\
&+ \frac{N_{i}(B)}{2\epsilon}\left(1-\frac{\rho}{\epsilon}
\frac{u^{2}}{\rho+u^{2}}\right)(a|0_{r^{'}}|b-\iota_{i}) \\
&+(r^{'}_{i}-P_{i})\frac{\rho}{\epsilon}
\frac{u^{2}}{\rho+u^{2}}(a|0_{r^{'}}|b) 
-\frac{u^{2}}{\epsilon+\eta}\frac{u^{2}}{\rho+u^{2}}(a|0_{r^{'}}+\iota_{i}|b)
 \end{split}
\label{OS_ERI_eq:15}
\end{equation}

Now let's set the value of $\eta$ and $\rho$, so that to make the $(a+\iota_{i}|0_{r^{'}}|b)$
related to the outer integral on $\chi_{c}$ and $\chi_{d}$:
\begin{equation}
 \label{OS_ERI_eq:16}
\eta = \alpha^{'} + \beta^{'} 
\end{equation}
where $\alpha^{'}$ is the exponent factor for $\chi_{c}$, and $\beta^{'}$ is the exponent
factor for $\chi_{d}$. Therefore, the $\rho$ could be expressed as:
\begin{equation}
 \label{OS_ERI_eq:17}
\rho = \frac{(\alpha+\beta)(\alpha^{'} + \beta^{'})}{(\alpha+\beta)+(\alpha^{'} + \beta^{'})} 
\end{equation}

Let's consider the outer integral over $r^{'}$, it's easy to see that:
\begin{equation}
\label{OS_ERI_eq:18}
 -\int dr^{'} \chi_{c}(r^{'})\chi_{d}(r^{'})(a|0_{r^{'}} +\iota_{i} |b)
= \int dr \chi_{a}(r)\chi_{b}(r)(c|0_{r} + \iota_{i}|d)
\end{equation}
This is because that for the term of $0_{r} + \iota_{i}$, which is converted from
$r^{'}_{i} - r_{i}$ into $r_{i}-r^{'}_{i}$.Therefore, for the term of 
$\dfrac{u^{2}}{\epsilon+\eta}\dfrac{u^{2}}
{\rho+u^{2}}(a|0_{r^{'}}+\iota_{i}|b)$ in the integral it could be converted into:
\begin{equation}
 \label{OS_ERI_eq:19}
\begin{split}
&-\frac{u^{2}}{\epsilon+\eta}\frac{u^{2}}
{\rho+u^{2}}\int dr^{'} \chi_{c}(r^{'})\chi_{d}(r^{'})(a|0_{r^{'}}+\iota_{i}|b) \\
&=\frac{1}{\epsilon+\eta}\frac{u^{2}}
{\rho+u^{2}}\int dr \chi_{a}(r)\chi_{b}(r)u^{2}(c|0_{r}+\iota_{i}|d)
\end{split}
\end{equation}
While according to the \ref{OS_ERI_eq:10}, the term of $u^{2}(c|0_{r}+\iota_{i}|d)$
is:
\begin{equation}
 \begin{split}
  u^{2}(c|0_{r}+\iota_{i}|d) &= 
-\eta(r_{i} - Q_{i})(c|0_{r}|d) \\
&+
\left(\frac{N_{i}(C)}{2}\right)(c-\iota_{i}|0_{r}|d) \\
&+
\left(\frac{N_{i}(D)}{2}\right)(c|0_{r}|d-\iota_{i}) \\
&-\eta(c|0_{r}+\iota_{i}|d)
 \end{split}
\label{OS_ERI_eq:20}
\end{equation}
and 
\begin{equation}
 Q_{i} = \frac{\alpha^{'}C + \beta^{'}D}{\alpha^{'} + \beta^{'}}
\end{equation}

Now let's take the result of \ref{OS_ERI_eq:20} into \ref{OS_ERI_eq:19}, it's:
\begin{equation}
 \begin{split}
  &-\frac{u^{2}}{\epsilon+\eta}\frac{u^{2}}
{\rho+u^{2}}\int dr^{'} \chi_{c}(r^{'})\chi_{d}(r^{'})(a|0_{r^{'}}+\iota_{i}|b) \\
&=Q_{i}\frac{\eta}{\epsilon+\eta}\frac{u^{2}}
{\rho+u^{2}}\int dr \chi_{a}(r)\chi_{b}(r)(c|0_{r}|d) \\
&+
\left(\frac{N_{i}(C)}{2}\right)\frac{1}{\epsilon+\eta}\frac{u^{2}}
{\rho+u^{2}}\int dr \chi_{a}(r)\chi_{b}(r)(c-\iota_{i}|0_{r}|d) \\
&+
\left(\frac{N_{i}(D)}{2}\right)\frac{1}{\epsilon+\eta}\frac{u^{2}}
{\rho+u^{2}}\int dr \chi_{a}(r)\chi_{b}(r)(c|0_{r}|d-\iota_{i}) \\
&-\eta \frac{1}{\epsilon+\eta}\frac{u^{2}}
{\rho+u^{2}}\int dr \chi_{a}(r)\chi_{b}(r) r_{i} (c|0_{r}|d) \\
&-\eta\frac{1}{\epsilon+\eta}\frac{u^{2}}
{\rho+u^{2}}\int dr \chi_{a}(r)\chi_{b}(r)(c|0_{r}+\iota_{i}|d)
 \end{split}
\label{OS_ERI_eq:21}
\end{equation}

Now let's look into the details of \ref{OS_ERI_eq:21}. In terms of its last
two terms, actually we have:
\begin{equation}
 \begin{split}
 r_{i}(c|0_{r}|d) +(c|0_{r}+\iota_{i}|d) 
&=
\int dr^{'} \chi_{c}(r^{'})\chi_{d}(r^{'})
r_{i}e^{u^{2}(r^{'}-r)^{2}} \\
&+ 
\int dr^{'} \chi_{c}(r^{'})\chi_{d}(r^{'})
(r^{'}_{i}-r_{i})e^{u^{2}(r^{'}-r)^{2}}
 \end{split}
\label{OS_ERI_eq:22}
\end{equation}
Here according to our traditional notation (see \ref{OS_three_overlap_int_eq:4}),
$r$ and $r^{'}$ represent the points in the space, and $i$ denotes its components
on the X, Y or Z axis. Therefore, it's clear that:
\begin{equation}
  r_{i}(c|0_{r}|d) +(c|0_{r}+\iota_{i}|d) =
\int dr^{'} \chi_{c}(r^{'})\chi_{d}(r^{'})
r^{'}_{i}e^{u^{2}(r^{'}-r)^{2}}
\label{OS_ERI_eq:23}
\end{equation}  
If we go over integral by $r$, it yields:
\begin{equation}
\begin{split}
&\int dr \chi_{a}(r)\chi_{b}(r) \int dr^{'} \chi_{c}(r^{'})\chi_{d}(r^{'})
r^{'}_{i}e^{(r^{'}-r)^{2}} \\
&= \int dr^{'} \chi_{c}(r^{'})\chi_{d}(r^{'}) r^{'}_{i}
\int dr \chi_{a}(r)\chi_{b}(r) e^{u^{2}(r^{'}-r)^{2}} \\
&= \int dr^{'} \chi_{c}(r^{'})\chi_{d}(r^{'}) r^{'}_{i}(a|0_{r^{'}}|b)
\end{split}
\label{OS_ERI_eq:24}
\end{equation}
Therefore the \ref{OS_ERI_eq:21} could be converted into:
\begin{equation}
 \begin{split}
  &-\frac{u^{2}}{\epsilon+\eta}\frac{u^{2}}
{\rho+u^{2}}\int dr^{'} \chi_{c}(r^{'})\chi_{d}(r^{'})(a|0_{r^{'}}+\iota_{i}|b) \\
&=Q_{i}\frac{\eta}{\epsilon+\eta}\frac{u^{2}}
{\rho+u^{2}}\int dr \chi_{a}(r)\chi_{b}(r)(c|0_{r}|d) \\
&+
\left(\frac{N_{i}(C)}{2}\right)\frac{1}{\epsilon+\eta}\frac{u^{2}}
{\rho+u^{2}}\int dr \chi_{a}(r)\chi_{b}(r)(c-\iota_{i}|0_{r}|d) \\
&+
\left(\frac{N_{i}(D)}{2}\right)\frac{1}{\epsilon+\eta}\frac{u^{2}}
{\rho+u^{2}}\int dr \chi_{a}(r)\chi_{b}(r)(c|0_{r}|d-\iota_{i}) \\
&-\eta \frac{1}{\epsilon+\eta}\frac{u^{2}}
{\rho+u^{2}}\int dr^{'} \chi_{c}(r^{'})\chi_{d}(r^{'}) r^{'}_{i}(a|0_{r^{'}}|b)
 \end{split}
\label{OS_ERI_eq:25}
\end{equation}

Now let's combine the result in \ref{OS_ERI_eq:25} with the result in \ref{OS_ERI_eq:15}.
By multiplying $\chi_{c}$ and $\chi_{d}$ and integrate over $r^{'}$ in 
\ref{OS_ERI_eq:15}, we can get:
\begin{equation}
 \begin{split}
&\int dr^{'} \chi_{c}(r^{'})\chi_{d}(r^{'})(a+\iota_{i}|0_{r^{'}}|b) \\
&=(P_{i} - A_{i})\int dr^{'} \chi_{c}(r^{'})\chi_{d}(r^{'})(a|0_{r^{'}}|b) \\
&+
\frac{N_{i}(A)}{2\epsilon}\left(1-\frac{\rho}{\epsilon}
\frac{u^{2}}{\rho+u^{2}}\right)\int dr^{'} \chi_{c}(r^{'})\chi_{d}(r^{'})
(a-\iota_{i}|0_{r^{'}}|b) \\
&+ \frac{N_{i}(B)}{2\epsilon}\left(1-\frac{\rho}{\epsilon}
\frac{u^{2}}{\rho+u^{2}}\right)\int dr^{'} \chi_{c}(r^{'})\chi_{d}(r^{'})
(a|0_{r^{'}}|b-\iota_{i}) \\
&-P_{i}\frac{\rho}{\epsilon}
\frac{u^{2}}{\rho+u^{2}}\int dr^{'} \chi_{c}(r^{'})\chi_{d}(r^{'})(a|0_{r^{'}}|b) \\
&+\frac{\rho}{\epsilon}
\frac{u^{2}}{\rho+u^{2}}\int dr^{'} \chi_{c}(r^{'})\chi_{d}(r^{'})r^{'}_{i}(a|0_{r^{'}}|b) \\
&+Q_{i}\frac{\eta}{\epsilon+\eta}\frac{u^{2}}
{\rho+u^{2}}\int dr \chi_{a}(r)\chi_{b}(r)(c|0_{r}|d) \\
&+
\left(\frac{N_{i}(C)}{2}\right)\frac{1}{\epsilon+\eta}\frac{u^{2}}
{\rho+u^{2}}\int dr \chi_{a}(r)\chi_{b}(r)(c-\iota_{i}|0_{r}|d) \\
&+
\left(\frac{N_{i}(D)}{2}\right)\frac{1}{\epsilon+\eta}\frac{u^{2}}
{\rho+u^{2}}\int dr \chi_{a}(r)\chi_{b}(r)(c|0_{r}|d-\iota_{i}) \\
&-\eta \frac{1}{\epsilon+\eta}\frac{u^{2}}
{\rho+u^{2}}\int dr^{'} \chi_{c}(r^{'})\chi_{d}(r^{'}) r^{'}_{i}(a|0_{r^{'}}|b) \\
&= (P_{i} - A_{i})\int dr^{'} \chi_{c}(r^{'})\chi_{d}(r^{'})(a|0_{r^{'}}|b) \\
&+
\frac{N_{i}(A)}{2\epsilon}\left(1-\frac{\rho}{\epsilon}
\frac{u^{2}}{\rho+u^{2}}\right)\int dr^{'} \chi_{c}(r^{'})\chi_{d}(r^{'})
(a-\iota_{i}|0_{r^{'}}|b) \\
&+ \frac{N_{i}(B)}{2\epsilon}\left(1-\frac{\rho}{\epsilon}
\frac{u^{2}}{\rho+u^{2}}\right)\int dr^{'} \chi_{c}(r^{'})\chi_{d}(r^{'})
(a|0_{r^{'}}|b-\iota_{i}) \\
&+\left(W_{i} -P_{i}\right)
\frac{u^{2}}{\rho+u^{2}}\int dr^{'} \chi_{c}(r^{'})\chi_{d}(r^{'})(a|0_{r^{'}}|b) \\
&+
\left(\frac{N_{i}(C)}{2}\right)\frac{1}{\epsilon+\eta}\frac{u^{2}}
{\rho+u^{2}}\int dr \chi_{a}(r)\chi_{b}(r)(c-\iota_{i}|0_{r}|d) \\
&+
\left(\frac{N_{i}(D)}{2}\right)\frac{1}{\epsilon+\eta}\frac{u^{2}}
{\rho+u^{2}}\int dr \chi_{a}(r)\chi_{b}(r)(c|0_{r}|d-\iota_{i}) 
 \end{split}
\label{OS_ERI_eq:26}
\end{equation}
Where in this long expression, it's clear that the integral term containing
$r^{'}_{i}$ vanished, and we combine the terms of $P_{i}$ and $Q_{i}$ together
so to give $W_{i}$:
\begin{equation}
 W_{i} = \frac{\epsilon P_{i} + \eta Q_{i}}{\epsilon + \eta}
\end{equation}
We note, that this surprising result has very beautiful symmetry among the 
resulting integrals.

By using the expression of $(ab|u|cd)$ in \ref{OS_ERI_eq:2}, the expression for
the \ref{OS_ERI_eq:26} could be further simplified as:
\begin{equation}
 \begin{split}
((a+\iota_{i})b|u|cd) &= (P_{i} - A_{i})(ab|u|cd) +
\left(W_{i} -P_{i}\right)
\frac{u^{2}}{\rho+u^{2}}(ab|u|cd) \\
&+\frac{N_{i}(A)}{2\epsilon}\left(1-\frac{\rho}{\epsilon}
\frac{u^{2}}{\rho+u^{2}}\right)
((a-\iota_{i})b|u|cd) \\
&+\frac{N_{i}(B)}{2\epsilon}\left(1-\frac{\rho}{\epsilon}
\frac{u^{2}}{\rho+u^{2}}\right)
(a(b-\iota_{i})|u|cd) \\
&+\left(\frac{N_{i}(C)}{2}\right)\frac{1}{\epsilon+\eta}\frac{u^{2}}
{\rho+u^{2}}(ab|u|(c-\iota_{i})d) \\
&+\left(\frac{N_{i}(D)}{2}\right)\frac{1}{\epsilon+\eta}\frac{u^{2}}
{\rho+u^{2}}(ab|u|c(d-\iota_{i}))
\end{split}
\label{OS_ERI_eq:27}
\end{equation}

Now let's define some auxiliary integral function:
\begin{equation}
\label{OS_ERI_eq:28}
 (ab|cd)^{(m)} = \frac{2}{\sqrt{\pi}}\int^{\infty}_{0} du \left( \frac{u^{2}}
{\rho+u^{2}}\right)^{m}(ab|u|cd) 
\end{equation}
Then by multiplying with $\left( \dfrac{u^{2}}
{\rho+u^{2}}\right)^{m}$ and integrating over $u$
the above result could be written as:
\begin{equation}
 \begin{split}
((a+\iota_{i})b|cd)^{(m)} &= (P_{i} - A_{i})(ab|cd)^{(m)} +
\left(W_{i} -P_{i}\right)(ab|cd)^{(m+1)} \\
&+\frac{N_{i}(A)}{2\epsilon}\left(((a-\iota_{i})b|cd)^{(m)}-\frac{\rho}{
\epsilon }((a-\iota_{i})b|cd)^{(m+1)}\right)  \\
&+\frac{N_{i}(B)}{2\epsilon}\left((a(b-\iota_{i})|cd)^{(m)}-\frac{\rho}{
\epsilon }(a(b-\iota_{i})|cd)^{(m+1)}\right)  \\
&+\left(\frac{N_{i}(C)}{2}\right)\frac{1}{\epsilon+\eta}
(ab|(c-\iota_{i})d)^{(m+1)} \\
&+\left(\frac{N_{i}(D)}{2}\right)\frac{1}{\epsilon+\eta}
(ab|c(d-\iota_{i}))^{(m+1)}
\end{split}
\label{OS_ERI_result}
\end{equation}
We note that the true ERI is $(ab|cd)^{(0)}$. This is the final result 
for deriving the ERI in the OS framework.

Finally, let's make some complementary work. In the \ref{OS_ERI_result},
it's clear all of integrals could be derived from the basic integral of
$(00|00)^{(m)}$, which is in form of:
\begin{equation}
 \begin{split}
 (00|00)^{(m)} &= \frac{2}{\sqrt{\pi}} \int_{0}^{\infty} du \left( \frac{u^{2}}
{\rho+u^{2}}\right)^{m} (00|u|00) \\
&=\frac{2}{\sqrt{\pi}} \int_{0}^{\infty} du \left( \frac{u^{2}}
{\rho+u^{2}}\right)^{m} 
\int dr^{'}  e^{-\alpha^{'} (r^{'}_{C})^{2}}e^{-\beta^{'} (r^{'}_{D})^{2}} \\
&\int dr  e^{-\alpha r_{A}^{2}} e^{-\beta r_{B}^{2}} e^{(r-r^{'})^{2}u^{2}} \\
\end{split}
\end{equation}
As for the integral over $r$, by using the three body integral result in
\ref{OS_bottom_three_overlap_int_2}, we can rewrite it as:
\begin{equation}
 \begin{split}
  \int dr  e^{-\alpha r_{A}^{2}} e^{-\beta r_{B}^{2}} e^{(r-r^{'})^{2}u^{2}} &=
  (0_{A}|0_{B})\left( \frac{\alpha+\beta}{\alpha+\beta+u^{2}}\right)^{\frac{3}{2}}
e^{-\frac{(\alpha+\beta)u^{2}}{\alpha+\beta+u^{2}}|r^{'}-P|^{2}} 
 \end{split}
 \label{OS_ERI_eq:30}
\end{equation}
This is equivalent to treat $e^{(r-r^{'})^{2}u^{2}}$ as a S type of Gaussian, where
the electron in $r$ centers on ``nuclei'' of $r^{'}$.

The RHS of \ref{OS_ERI_eq:30} is still like a Gaussian of $r^{'}$ centering
on $P$, so once again by using the \ref{OS_bottom_three_overlap_int_2};
it's able to integrate over $r^{'}$. Before we move on, let's do some prepare
work. Firstly, considering the combination of Gaussian on $P$ and Gaussian
on $C$ and $D$, for the exponent coefficient it gives:
\begin{equation}
 \begin{split}
  \frac{\frac{(\alpha+\beta)u^{2}}{\alpha+\beta+u^{2}}\times
  (\alpha^{'}+\beta^{'})}{\frac{(\alpha+\beta)u^{2}}{\alpha+\beta+u^{2}} 
  +\alpha^{'}+\beta^{'}} 
  &= \frac{(\alpha^{'}+\beta^{'})(\alpha+\beta)u^{2}}
  {(\alpha+\beta+u^{2})(\alpha^{'}+\beta^{'})+(\alpha+\beta)u^{2}} \\
  &= \frac{u^{2}}{1+\frac{u^{2}}{\alpha+\beta}+\frac{u^{2}}{\alpha^{'}+\beta^{'}}} \\
  &= \frac{u^{2}}{1+u^{2}\left( \frac{1}{\alpha+\beta}+\frac{1}{\alpha^{'}+\beta^{'}}
  \right) } \\
  &= \frac{u^{2}}{1+\frac{u^{2}}{\rho}} \\
  &= \frac{\rho u^{2}}{\rho+ u^{2}}
 \end{split}
 \label{OS_ERI_eq:31}
 \end{equation}
 Here we use the definition for variable of $\rho$ in \ref{OS_ERI_eq:17}.
 
 So the integral over $r^{'}$ becomes:
\begin{equation}
 \begin{split}
   &\int dr^{'}  e^{-\alpha^{'} (r^{'}_{C})^{2}}e^{-\beta^{'} (r^{'}_{D})^{2}} 
   \int dr  e^{-\alpha r_{A}^{2}} e^{-\beta r_{B}^{2}} e^{(r-r^{'})^{2}u^{2}} \\
&= (0_{A}|0_{B})\left( \frac{\alpha+\beta}{\alpha+\beta+u^{2}}\right)^{\frac{3}{2}}   
  \int dr^{'}  e^{-\alpha^{'} (r^{'}_{C})^{2}}e^{-\beta^{'} (r^{'}_{D})^{2}} 
   e^{-\frac{(\alpha+\beta)u^{2}}{\alpha+\beta+u^{2}}|r^{'}-P|^{2}} \\
&= (0_{A}|0_{B})(0_{C}|0_{D})
\left( \frac{\alpha+\beta}{\alpha+\beta+u^{2}}\right)^{\frac{3}{2}} 
\left( \frac{\alpha^{'}+\beta^{'}}{\alpha^{'}+\beta^{'}+ 
\frac{(\alpha+\beta)u^{2}}{\alpha+\beta+u^{2}}}\right)^{\frac{3}{2}}
e^{-\frac{\rho u^{2}}{\rho+ u^{2}}|PQ|^{2}}
\end{split} 
\label{OS_ERI_eq:32}
\end{equation}
This is using the result of \ref{OS_ERI_eq:31}. Combing the term in \ref{OS_ERI_eq:32},
it gives:
\begin{equation}
\begin{split}
 &\left( \frac{\alpha+\beta}{\alpha+\beta+u^{2}}\right)^{\frac{3}{2}} 
\left( \frac{\alpha^{'}+\beta^{'}}{\alpha^{'}+\beta^{'}+ 
\frac{(\alpha+\beta)u^{2}}{\alpha+\beta+u^{2}}}\right)^{\frac{3}{2}} \\
&= \left( \frac{(\alpha+\beta)(\alpha^{'}+\beta^{'})}
{(\alpha+\beta+u^{2})(\alpha^{'}+\beta^{'})+(\alpha+\beta)u^{2}}\right)^{\frac{3}{2}}\\
&= \left(\frac{1}{1+\frac{u^{2}}{\rho}}\right)^{\frac{3}{2}} \\
&= \left(\frac{\rho}{\rho+u^{2}}\right)^{\frac{3}{2}}
\end{split}
 \label{OS_ERI_eq:33}
\end{equation}
So after integrate over $r$ and $r^{'}$ the bottom integral becomes:
\begin{equation}
 \begin{split}
 (00|00)^{(m)} &=\frac{2}{\sqrt{\pi}}(0_{A}|0_{B})(0_{C}|0_{D})
 \int_{0}^{\infty} du \left( \frac{u^{2}}{\rho+u^{2}}\right)^{m}
 \left(\frac{\rho}{\rho+u^{2}}\right)^{\frac{3}{2}}
 e^{-\frac{\rho u^{2}}{\rho+ u^{2}}|PQ|^{2}} 
 \end{split}
\label{OS_ERI_eq:34}
\end{equation}
If we set $t^{2} = \frac{u^{2}}{\rho+ u^{2}}$, and it's easy to know
that 
\begin{equation}
 du = (1-t^{2})^{-\frac{3}{2}}\rho^{1/2} dt
 \label{OS_ERI_eq:35}
\end{equation}
so \ref{OS_ERI_eq:34} becomes:
\begin{equation}
 \begin{split}
 (00|00)^{(m)} &= 2\left( \frac{\rho}{\pi}\right)^{\frac{1}{2}}(0_{A}|0_{B})
(0_{C}|0_{D})\int^{1}_{0} t^{2m} e^{-(\rho|PQ|^{2})t^{2}} dt 
 \end{split}
\label{OS_ERI_complementary_result}
\end{equation}

Where we have the arguments as:
\begin{align}
 \overrightarrow{P} &= \frac{\alpha \overrightarrow{A} + \beta
\overrightarrow{B}}{\alpha + \beta} \nonumber \\
\overrightarrow{Q} &= \frac{\alpha^{'} \overrightarrow{C} + \beta^{'}
\overrightarrow{D}}{\alpha^{'} + \beta^{'}} \nonumber \\
\rho &= \frac{(\alpha+\beta)(\alpha^{'}+\beta^{'})}
{(\alpha+\beta)+(\alpha^{'}+\beta^{'})}
\end{align}
    
%%%%%%%%%%%%%%%%%%%%%%%%%%%%%%%%%%%%%%%%%%%%%%%%%%%%%%%%%%%%%%%%%%%%%%%%%%%%%%%%
\subsection{Overlap Integral}
%
%
%
the basic overlap integral is:
\begin{equation}
 \label{OS_overlap_eq:1}
I = \int dr \chi_{a}(r) \chi_{b}(r)
\end{equation}

By using the recursive relation derived in the three center overlap
integral (see \ref{OS_three_overlap_int_eq:12}), set the middle Gaussian
primitive to be zero; we can get the recursive relation for overlap:
\begin{equation}
\begin{split}
(a+\iota_{i}|b) &= 
(P_{i} - A_{i})(a|b) + 
N_{i}(A)\left(\frac{1}{2(\alpha+\beta)}\right)(a-\iota_{i}|b) \\
&+ 
N_{i}(B)\left(\frac{1}{2(\alpha+\beta)}\right)(a|b-\iota_{i})  
\end{split}
\label{OS_overlap_result}
\end{equation}

%%%%%%%%%%%%%%%%%%%%%%%%%%%%%%%%%%%%%%%%%%%%%%%%%%%%%%%%%%%%%%%%%%%%%%%%%%%%%%%%
\subsection{Derivatives of Basis Sets in Overlap}
%
%
%
Before we derive the recursive relation for kinetic energy, let's firstly 
go to see another type of integral of 
$\left( \dfrac{\partial a}{\partial x}|b\right)$.

Firstly, we note that the derivatives can not directly be applied to the 
recursive relation like \ref{OS_overlap_result}:
\begin{equation}
\begin{split}
(\frac{\partial (a+\iota_{i})}{\partial x}|b) & \neq 
(P_{i} - A_{i})(\frac{\partial a}{\partial x}|b) + 
N_{i}(A)\left(\frac{1}{2(\alpha+\beta)}\right)(\frac{\partial(a-\iota_{i})}{\partial x}|b) \\
&+ 
N_{i}(B)\left(\frac{1}{2(\alpha+\beta)}\right)(\frac{\partial a}{\partial x}|b-\iota_{i})  
\end{split}
\end{equation}

Let's prove this with some simple example. Suggest we have two Gaussian 
primitives:
\begin{align}
 \chi_{i} &= x^{l_{i}}y^{m_{i}}z^{n_{i}} e^{-\alpha_{i} r^{2}} \nonumber \\
 \chi_{j} &= x^{l_{j}}y^{m_{j}}z^{n_{j}} e^{-\alpha_{j} r^{2}} \nonumber \\
 \gamma   &= \alpha_{i} + \alpha_{j}
 \end{align}

According to the \ref{derivative_overlap_direct_int_eq:2} and recursive relation
for the overlap integral, we can have:
\begin{equation}
 \begin{split}
 \left(\frac{\partial \chi_{i}}{\partial x}|\chi_{j}\right)  &= 
 l_{i}(\chi_{i}^{l-1mn}|\chi_{j}) - 2\alpha_{i} (\chi_{i}^{l+1mn}|\chi_{j}) \\
 &= l_{i}\left[ PA_{k}(\chi_{i}^{l-1mn}-\iota_{k}|\chi_{j}) + 
 \frac{N_{k}(A)}{2 \gamma}(\chi_{i}^{l-1mn}-2\iota_{k}|\chi_{j})  \right. \\ 
 &+ \left. \frac{N_{k}(B)}{2 \gamma}(\chi_{i}^{l-1mn}-\iota_{k}|\chi_{j}-\iota_{k})
 \right] \\
 &-2\alpha_{i} \left[ PA_{k}(\chi_{i}^{l+1mn}-\iota_{k}|\chi_{j}) + 
 \frac{N_{k}(A)}{2 \gamma}(\chi_{i}^{l+1mn}-2\iota_{k}|\chi_{j}) \right. \\ 
 &+ \left. \frac{N_{k}(B)}{2 \gamma}(\chi_{i}^{l+1mn}-\iota_{k}|\chi_{j}-\iota_{k})
 \right] 
 \end{split}
 \label{deriv_os_overlap:1}
\end{equation}
Here the index of $i$ and $j$ are omitted for l, m and n. If the l, m and n do not 
change, they are not listed neither.

It's easy to see that if $k = y$ or $k = z$ then the above equation could be 
written as:
\begin{equation}
 \begin{split}
  \left(\frac{\partial \chi_{i}}{\partial x}|\chi_{j}\right)  &= 
 l_{i}(\chi_{i}^{l-1mn}|\chi_{j}) - 2\alpha_{i} (\chi_{i}^{l+1mn}|\chi_{j}) \\ 
 &=  PA_{k}(\frac{\partial(\chi_{i}^{lmn}-\iota_{k})}{\partial x}|\chi_{j}) + 
 \frac{N_{k}(A)}{2 \gamma}(\frac{\partial(\chi_{i}^{lmn}-2\iota_{k})}{\partial x}|\chi_{j}) 
 \\ 
 &+ \frac{N_{k}(B)}{2 \gamma}
 (\frac{\partial (\chi_{i}^{lmn}-\iota_{k}) }{\partial x}|\chi_{j}-\iota_{k}) 
 \end{split}
\end{equation}
However, if $k = x$, the above equation does not hold.

For $k = x$, the \ref{deriv_os_overlap:1} could be expressed as:
\begin{equation}
 \begin{split}
 \left(\frac{\partial \chi_{i}}{\partial x}|\chi_{j}\right)  &= 
 l_{i}(\chi_{i}^{l-1mn}|\chi_{j}) - 2\alpha_{i} (\chi_{i}^{l+1mn}|\chi_{j}) \\
 &= l_{i}\left[ PA_{x}(\chi_{i}^{l-2mn}|\chi_{j}) + 
 \frac{N_{x}(A)}{2 \gamma}(\chi_{i}^{l-3mn}|\chi_{j})  \right. \\ 
 &+ \left. \frac{N_{x}(B)}{2 \gamma}(\chi_{i}^{l-2mn}|\chi_{j}^{l-1mn})
 \right] \\
 &-2\alpha_{i} \left[ PA_{x}(\chi_{i}|\chi_{j}) + 
 \frac{N_{x}(A)}{2 \gamma}(\chi_{i}^{l-1mn}|\chi_{j}) \right. \\ 
 &+ \left. \frac{N_{x}(B)}{2 \gamma}(\chi_{i}|\chi_{j}^{l-1mn})
 \right] \\
 &= PA_{x} \left( l_{i}(\chi_{i}^{l-2mn}|\chi_{j}) -2\alpha_{i} (\chi_{i}|\chi_{j}) 
 \right) \\
 &+ \frac{N_{x}(A)}{2 \gamma}\left(
 l_{i}(\chi_{i}^{l-3mn}|\chi_{j}) -2\alpha_{i} (\chi_{i}^{l-1mn}|\chi_{j})
 \right) \\
 &+ \frac{N_{x}(B)}{2 \gamma}\left(
 l_{i}(\chi_{i}^{l-2mn}|\chi_{j}^{l-1mn}) -2\alpha_{i} (\chi_{i}|\chi_{j}^{l-1mn})
 \right) \\
 &= PA_{x} (\frac{\partial \chi_{i}^{l-1mn}}{\partial x}|\chi_{j}) + 
 PA_{x} (\chi_{i}^{l-1mn}|\chi_{j}) \\
 &+ \frac{N_{x}(A)}{2 \gamma}(\frac{\partial \chi_{i}^{l-2mn}}{\partial x}|\chi_{j})
 + 2\frac{N_{x}(A)}{2 \gamma}(\chi_{i}^{l-2mn}|\chi_{j}) \\
 &+ \frac{N_{x}(B)}{2 \gamma}(\frac{\partial \chi_{i}^{l-1mn}}{\partial x}|\chi_{j}^{l-1mn})
 + \frac{N_{x}(B)}{2 \gamma}(\chi_{i}^{l-1mn}|\chi_{j}^{l-1mn}) \\
 &= PA_{x} (\frac{\partial \chi_{i}^{l-1mn}}{\partial x}|\chi_{j}) + 
 \frac{N_{x}(A)}{2 \gamma}(\frac{\partial \chi_{i}^{l-2mn}}{\partial x}|\chi_{j}) \\
 &+ \frac{N_{x}(B)}{2 \gamma}(\frac{\partial \chi_{i}^{l-1mn}}{\partial x}|\chi_{j}^{l-1mn}) \\
 &+ PA_{x} (\chi_{i}^{l-1mn}|\chi_{j}) + \frac{N_{x}(A)}{2 \gamma}(\chi_{i}^{l-2mn}|\chi_{j}) \\
 &+ \frac{N_{x}(B)}{2 \gamma}(\chi_{i}^{l-1mn}|\chi_{j}^{l-1mn}) \\
 &+ \frac{N_{x}(A)}{2 \gamma}(\chi_{i}^{l-2mn}|\chi_{j})
 \end{split}
 \label{deriv_os_overlap:2}
\end{equation}
Therefore, we have additional terms in \ref{deriv_os_overlap:2}.


%%%%%%%%%%%%%%%%%%%%%%%%%%%%%%%%%%%%%%%%%%%%%%%%%%%%%%%%%%%%%%%%%%%%%%%%%%%%%%%%
\subsection{Kinetic Energy Integral}
%
%
%
According to the \ref{kinetic_direct_int_eq:2}, the kinetic energy integral is:
\begin{align}
 \label{OS_kinetic_eq:1}
I &= \frac{1}{2}\int dr (\nabla\chi_{i}(r)) \cdotp (\nabla\chi_{j}(r)) \nonumber \\
  &= \frac{1}{2}\int d x \frac{\partial \chi_{i}(r)}{\partial x}
  \frac{\partial \chi_{j}(r)}{\partial x} 
  +  \frac{1}{2}\int d y \frac{\partial \chi_{i}(r)}{\partial y}
  \frac{\partial \chi_{j}(r)}{\partial y} 
  +  \frac{1}{2}\int d z \frac{\partial \chi_{i}(r)}{\partial z}
  \frac{\partial \chi_{j}(r)}{\partial z} 
\end{align}
So the kinetic integral is actually a linear combination of overlap integrals,
and the kinetic integral itself could be decomposed into three integrals sum:
\begin{equation}
 I = I_{x} + I_{y} + I_{z}
\end{equation}
In the following derivation, we will concentrate on one component, which is 
in general named as $I_{i}$.

For a general Gaussian primitive function of $\chi = x^{l}y^{m}z^{n}e^{-\alpha r^{2}}$, who 
centers on $A$; we can express its derivatives as:
\begin{align}
 \frac{\partial \chi^{lmn}_{A}}{\partial v} &= 
 N_{i}(A)(\chi_{A}^{lmn}-\iota_{i}) - 2\alpha(\chi_{A}^{lmn}+\iota_{i}) \nonumber \\
 &= N_{i}(A)(-1)_{i} - 2\alpha(+1)_{i}
 \label{OS_kinetic_eq:0}
\end{align}
$v$ denotes the possible derivatives, could be x, y or z. 

Therefore, suggest that in \ref{OS_kinetic_eq:1} the $\chi_{i}$ resides 
on $A$ with $\alpha$ exponential factor, and $\chi_{j}$ is on $B$ with
$\beta$ exponential factor; we can have:
\begin{equation}
\begin{split}
\left(  \frac{\partial \chi_{A}}{\partial v}|\frac{\partial \chi_{B}}{\partial v}
\right)  &= N_{i}(A)N_{i}(B)(-1|-1)_{i} -2\alpha N_{i}(B)(+1|-1)_{i} \\
&-2\beta N_{i}(A)(-1|+1)_{i} + 4\alpha\beta(+1|+1)_{i}
\end{split}
\label{OS_kinetic_eq:2}
\end{equation}
the $i$ is determined from the v. If v is x, then $N_{i}(A) = l_{A}$ and 
$N_{i}(B) = l_{B}$; if v is y, then $N_{i}(A) = m_{A}$ and 
$N_{i}(B) = m_{B}$ and if v is z, then $N_{i}(A) = n_{A}$ and 
$N_{i}(B) = n_{B}$. 

By expanding all of overlap integrals into its RR form, we can have:
\begin{equation}
 \begin{split}
 \left(  \frac{\partial \chi_{A}}{\partial v}|\frac{\partial \chi_{B}}{\partial v}
\right)  &=  N_{i}(A)N_{i}(B)
\left[ PA_{k}(-1-\iota_{k}|-1)_{i} + \frac{N_{k}(A)}{2 \gamma}(-1-2\iota_{k}|-1)_{i} \right. \\ 
 &+ \left. \frac{N_{k}(B)}{2 \gamma}(-1-\iota_{k}|-1-\iota_{k})_{i} \right] \\
&- 2\alpha N_{i}(B)
\left[ PA_{k}(+1-\iota_{k}|-1)_{i} + \frac{N_{k}(A)}{2 \gamma}(+1-2\iota_{k}|-1)_{i} \right. \\ 
 &+ \left. \frac{N_{k}(B)}{2 \gamma}(+1-\iota_{k}|-1-\iota_{k})_{i} \right] \\ 
&- 2\beta  N_{i}(A)
\left[ PA_{k}(-1-\iota_{k}|+1)_{i} + \frac{N_{k}(A)}{2 \gamma}(-1-2\iota_{k}|+1)_{i} \right. \\ 
 &+ \left. \frac{N_{k}(B)}{2 \gamma}(-1-\iota_{k}|+1-\iota_{k})_{i} \right] \\  
&+ 4\alpha\beta
\left[ PA_{k}(+1-\iota_{k}|+1)_{i} + \frac{N_{k}(A)}{2 \gamma}(+1-2\iota_{k}|+1)_{i} \right. \\ 
 &+ \left. \frac{N_{k}(B)}{2 \gamma}(+1-\iota_{k}|+1-\iota_{k})_{i} \right] 
 \end{split}
 \label{OS_kinetic_eq:3}
\end{equation}
where $\gamma = \alpha + \beta$. $k$ denotes possible RR direction, which is 
also x, y or z. $k$ is different from the $v$ here.

Firstly, let's go to see a simple case that the derivatives direction $i$ is not
same with RR direction $k$. It's easy to see, in this case the above result
turns into:
\begin{equation}
 \begin{split}
 \left(  \frac{\partial \chi_{A}}{\partial v}|\frac{\partial \chi_{B}}{\partial v}
\right)  &= PA_{k}
 \left(\frac{\partial (\chi_{A}-\iota_{k})}{\partial v}|
 \frac{\partial \chi_{B}}{\partial v} \right) + 
\frac{N_{k}(A)}{2 \gamma}
 \left(\frac{\partial (\chi_{A}-2\iota_{k})}{\partial v}|
 \frac{\partial \chi_{B}}{\partial v} \right) \\
&+\frac{N_{k}(B)}{2 \gamma}
 \left(\frac{\partial (\chi_{A}-\iota_{k})}{\partial v}|
 \frac{\partial(\chi_{B}-\iota_{k})}{\partial v} \right) 
 \end{split}
 \label{OS_kinetic_eq:4} 
\end{equation}

For the case that RR direction $k$ is same with derivatives direction of $i$, according
to the equation of \ref{OS_kinetic_eq:3}, we have:
\begin{equation}
 \begin{split}
 \left(  \frac{\partial \chi_{A}}{\partial v_{i=k}}|
 \frac{\partial \chi_{B}}{\partial v_{i=k}} \right)  &=  N_{i}(A)N_{i}(B)
\left[ PA_{i}(-2|-1)_{i} + \frac{N_{i}(A)-2}{2 \gamma}(-3|-1)_{i} \right. \\ 
 &+ \left. \frac{N_{i}(B)-1}{2 \gamma}(-2|-2)_{i} \right] \\
&- 2\alpha N_{i}(B)
\left[ PA_{i}(0|-1)_{i} + \frac{N_{i}(A)}{2 \gamma}(-1|-1)_{i} \right. \\ 
 &+ \left. \frac{N_{i}(B)-1}{2 \gamma}(0|-2)_{i} \right] \\ 
&- 2\beta  N_{i}(A)
\left[ PA_{i}(-2|+1)_{i} + \frac{N_{i}(A)-2}{2 \gamma}(-3|+1)_{i} \right. \\ 
 &+ \left. \frac{N_{i}(B)+1}{2 \gamma}(-2|0)_{i} \right] \\  
&+ 4\alpha\beta
\left[ PA_{i}(0|+1)_{i} + \frac{N_{i}(A)}{2 \gamma}(-1|+1)_{i} \right. \\ 
 &+ \left. \frac{N_{i}(B)+1}{2 \gamma}(0|0)_{i} \right] \\
&= PA_{i} \Bigg\{   
(N_{i}(A)-1)N_{i}(B)(-2|-1)_{i} - 2\alpha N_{i}(B)(0|-1)_{i}  \\
&- 2\beta(N_{i}(A)-1)(-2|+1)_{i} +4\alpha\beta (0|+1)_{i} \Bigg\}  \\
&+ PA_{i}\Bigg\{ N_{i}(B)(-2|-1)_{i} -2\beta(-2|+1)_{i} \Bigg\} \\
&+\frac{N_{i}(A)}{2 \gamma} \Bigg\{ 
(N_{i}(A)-2)N_{i}(B)(-3|-1)_{i} - 2\alpha N_{i}(B)(-1|-1)_{i}  \\
&-2\beta(N_{i}(A)-2)(-3|+1)_{i} + 4\alpha\beta(-1|+1)_{i} \Bigg \} \\
&+\frac{N_{i}(B)}{2 \gamma} \Bigg\{ 
(N_{i}(A)-1)(N_{i}(B)-1)(-2|-2)_{i} - 2\alpha (N_{i}(B)-1)(0|-2)_{i} \\ 
&- 2\beta(N_{i}(A)-1)(-2|0)_{i} + 4\alpha\beta(0|0)_{i} \Bigg\} \\
&+\frac{N_{i}(B)(N_{i}(B)-1)}{2 \gamma}(-2|-2)_{i} - 
\frac{2\beta N_{i}(A)}{2 \gamma} (-2|0)_{i} \\
&-\frac{2\beta N_{i}(B)}{2 \gamma} (-2|0)_{i}
\end{split}
 \label{OS_kinetic_eq:5}
\end{equation}

The above expression could be re-fomulated into more clearly way:
\begin{equation}
 \begin{split}
 \left(  \frac{\partial \chi_{A}}{\partial v_{i=k}}|
 \frac{\partial \chi_{B}}{\partial v_{i=k}} \right)  &=
 PA_{i}
 \left(\frac{\partial (\chi_{A}-\iota_{i})}{\partial v}|
 \frac{\partial \chi_{B}}{\partial v} \right) + 
\frac{N_{i}(A)}{2 \gamma}
 \left(\frac{\partial (\chi_{A}-2\iota_{i})}{\partial v}|
 \frac{\partial \chi_{B}}{\partial v} \right) \\
&+\frac{N_{i}(B)}{2 \gamma}
 \left(\frac{\partial (\chi_{A}-\iota_{i})}{\partial v}|
 \frac{\partial(\chi_{B}-\iota_{i})}{\partial v} \right) \\
&+ PA_{i}\left[ N_{i}(B)(-2|-1)_{i} -2\beta(-2|+1)_{i} \right] \\
&+\frac{N_{i}(B)(N_{i}(B)-1)}{2 \gamma}(-2|-2)_{i} - 
\frac{2\beta N_{i}(A)}{2 \gamma} (-2|0)_{i} \\
&-\frac{2\beta N_{i}(B)}{2 \gamma} (-2|0)_{i}
\end{split}
 \label{OS_kinetic_eq:6}
\end{equation}


\begin{comment}


%

\begin{equation}
 \begin{split}
 \left(  \frac{\partial \chi_{A}}{\partial v_{i=k}}|
 \frac{\partial \chi_{B}}{\partial v_{i=k}} \right)  &=  N_{i}(A)N_{i}(B)
\left[ PA_{i}(-2|-1)_{i} + \frac{N_{i}(A)-2}{2 \gamma}(-3|-1)_{i} \right. \\ 
 &+ \left. \frac{N_{i}(B)-1}{2 \gamma}(-2|-2)_{i} \right] \\
&- 2\alpha N_{i}(B)
\left[ PA_{i}(0|-1)_{i} + \frac{N_{i}(A)}{2 \gamma}(-1|-1)_{i} \right. \\ 
 &+ \left. \frac{N_{i}(B)-1}{2 \gamma}(0|-2)_{i} \right] \\ 
&- 2\beta  N_{i}(A)
\left[ PA_{i}(-2|+1)_{i} + \frac{N_{i}(A)-2}{2 \gamma}(-3|+1)_{i} \right. \\ 
 &+ \left. \frac{N_{i}(B)+1}{2 \gamma}(-2|0)_{i} \right] \\  
&+ 4\alpha\beta
\left[ PA_{i}(0|+1)_{i} + \frac{N_{i}(A)}{2 \gamma}(-1|+1)_{i} \right. \\ 
 &+ \left. \frac{N_{i}(B)+1}{2 \gamma}(0|0)_{i} \right] \\
&= PA_{i}
 \left(\frac{\partial (\chi_{A}-\iota_{i})}{\partial v_{i}}|
 \frac{\partial \chi_{B}}{\partial v_{i}} \right) + 
\frac{N_{i}(A)}{2 \gamma}
 \left(\frac{\partial (\chi_{A}-2\iota_{i})}{\partial v_{i}}|
 \frac{\partial \chi_{B}}{\partial v_{i}} \right) \\
&+\frac{N_{i}(B)}{2 \gamma}
 \left(\frac{\partial (\chi_{A}-\iota_{i})}{\partial v_{i}}|
 \frac{\partial(\chi_{B}-\iota_{i})}{\partial v_{i}} \right) \\
&+\Biggl( 
-\frac{2 N_{i}(A)N_{i}(B)}{2\gamma}(-3|-1)_{i} - 
\frac{N_{i}(A)N_{i}(B)}{2\gamma}(-2|-2)_{i} \\
&-
\frac{2 \alpha N_{i}(B)}{2\gamma}(0|-2)_{i} +
\frac{4 \beta N_{i}(A)}{2\gamma}(-3|+1)_{i} \\
&-
\frac{2 \beta N_{i}(A)}{2\gamma}(-2|0)_{i}  +
\frac{4 \alpha\beta}{2\gamma}(0|0)_{i}
\Biggr)
 \end{split}
 \label{OS_kinetic_eq:5}
\end{equation}

\end{comment}

%%%%%%%%%%%%%%%%%%%%%%%%%%%%%%%%%%%%%%%%%%%%%%%%%%%%%%%%%%%%%%%%%%%%%%%%%%%%%%%%
\subsection{Three Body Kinetic Integral}
%
%
%
According to the \ref{kinetic_direct_int_eq:2}, the three body kinetic enegy
could also be expressed as:
\begin{equation}
\frac{1}{2}\sum_{v = x, y, z}\int \chi_{A}(r)\chi_{B}(r)
\frac{\partial^{2}}{\partial v^{2}} \chi_{C}(r) dr
= \frac{1}{2}\int \nabla(\chi_{A}(r)\chi_{B}(r)) \cdot \nabla\chi_{C}(r) dr
 \label{three_body_kinetic_integral_eq:1}
\end{equation}

According to the \ref{OS_kinetic_eq:0}, we have:
\begin{align}
 \frac{\partial \chi^{lmn}_{A}}{\partial v} &= 
 N_{i}(A)(\chi_{A}^{lmn}-\iota_{i}) - 2\alpha(\chi_{A}^{lmn}+\iota_{i}) \nonumber \\
 &= N_{i}(A)(-1)_{i} - 2\alpha(+1)_{i}
 \label{three_body_kinetic_integral_eq:2}
\end{align}
We can further generate derivtives for \ref{three_body_kinetic_integral_eq:1}
as:
\begin{equation}
 \begin{split}
&\int \nabla(\chi_{A}(r)\chi_{B}(r)) \cdot \nabla\chi_{C}(r) dr \\
&= \int \left[ (\nabla\chi_{A}(r)\chi_{B}(r) + \chi_{A}(r)\nabla\chi_{B}(r)\right] 
\cdot \nabla\chi_{C}(r) dr \\
&= \sum_{i = x, y, z} \int
[ N_{i}(A)(\chi_{A}-\iota_{i}) - 2\alpha(\chi_{A}+\iota_{i})] \chi_{B} \\
&[ N_{i}(C)(\chi_{C}-\iota_{i}) - 2\gamma(\chi_{C}+\iota_{i})] dr \\
&+ \sum_{i = x, y, z} \int 
\chi_{A}(r) \left[ 
N_{i}(B)(\chi_{B}-\iota_{i}) - 2\beta(\chi_{B}+\iota_{i})\right] \\
&\left[ N_{i}(C)(\chi_{C}-\iota_{i}) - 2\gamma(\chi_{C}+\iota_{i})\right]  dr
 \end{split}
\label{three_body_kinetic_integral_eq:3}
\end{equation}
$\alpha$ is the expotential factor for $\chi_{A}$, $\beta$ is for $\chi_{B}$
and $\gamma$ is for $\chi_{C}$.

Now let's expand all the terms:
\begin{equation}
 \begin{split}
&\int \nabla(\chi_{A}(r)\chi_{B}(r)) \cdot \nabla\chi_{C}(r) dr \\
&= \sum_{i = x, y, z} \int
[ N_{i}(A)(\chi_{A}-\iota_{i}) - 2\alpha(\chi_{A}+\iota_{i})] \chi_{B} \\
&[ N_{i}(C)(\chi_{C}-\iota_{i}) - 2\gamma(\chi_{C}+\iota_{i})] dr \\
&+ \sum_{i = x, y, z} \int 
\chi_{A}(r) \left[ 
N_{i}(B)(\chi_{B}-\iota_{i}) - 2\beta(\chi_{B}+\iota_{i})\right] \\
&\left[ N_{i}(C)(\chi_{C}-\iota_{i}) - 2\gamma(\chi_{C}+\iota_{i})\right]  dr \\
&= \sum_{i = x, y, z} \int 
 N_{i}(A)N_{i}(C)(\chi_{A}-\iota_{i})\chi_{B}(\chi_{C}-\iota_{i}) dr \\
&- \sum_{i = x, y, z} \int 
 2\alpha N_{i}(C)(\chi_{A}+\iota_{i})\chi_{B}(\chi_{C}-\iota_{i}) dr \\
&- \sum_{i = x, y, z} \int 
 2\gamma N_{i}(A)(\chi_{A}-\iota_{i})\chi_{B}(\chi_{C}+\iota_{i}) dr \\
&+ \sum_{i = x, y, z} \int 
 4\alpha\gamma   (\chi_{A}+\iota_{i})\chi_{B}(\chi_{C}+\iota_{i}) dr \\ 
&+ \sum_{i = x, y, z} \int 
 N_{i}(B)N_{i}(C)\chi_{A}(\chi_{B}-\iota_{i})(\chi_{C}-\iota_{i}) dr \\
&- \sum_{i = x, y, z} \int 
 2\beta  N_{i}(C)\chi_{A}(\chi_{B}+\iota_{i})(\chi_{C}-\iota_{i}) dr \\
&- \sum_{i = x, y, z} \int 
 2\gamma N_{i}(B)\chi_{A}(\chi_{B}-\iota_{i})(\chi_{C}+\iota_{i}) dr \\
&+ \sum_{i = x, y, z} \int 
 4\beta\gamma    \chi_{A}(\chi_{B}+\iota_{i})(\chi_{C}+\iota_{i}) dr  
 \end{split}
\label{three_body_kinetic_integral_eq:4}
\end{equation}

Particularly, for the bottom $SSS$ type integrals, it has the expression as:
\begin{equation}
\begin{split}
 &\int \nabla(0_{A}0_{B}) \cdot \nabla 0_{C} dr \\
 &= \sum_{i = x, y, z} \int 
 4\alpha\gamma   (0_{A}+\iota_{i})0_{B}(0_{C}+\iota_{i}) dr \\ 
 &+ \sum_{i = x, y, z} \int 
 4\beta\gamma    0_{A}(0_{B}+\iota_{i})(0_{C}+\iota_{i}) dr \\
\end{split}
\label{three_body_kinetic_integral_eq:5}
\end{equation}
By using the RR expression for three body overlap integral, it becomes:
\begin{equation}
 \begin{split}
 &\int \nabla(0_{A}0_{B}) \cdot \nabla 0_{C} dr \\
 &= \sum_{i = x, y, z} \int 
 4\alpha\gamma   (0_{A}+\iota_{i})0_{B}(0_{C}+\iota_{i}) dr \\ 
 &+ \sum_{i = x, y, z} \int 
 4\beta\gamma    0_{A}(0_{B}+\iota_{i})(0_{C}+\iota_{i}) dr \\
 &= \sum_{i = x, y, z} \int 
 4\alpha\gamma\left\lbrace (G_{i}-C_{i})(0_{A}+\iota_{i})0_{B}0_{C}
 +\frac{1}{2(\alpha+\beta+\gamma)}0_{A}0_{B}0_{C} \right\rbrace dr \\
 &+ \sum_{i = x, y, z} \int 
 4\beta\gamma \left\lbrace (G_{i}-C_{i})0_{A}(0_{B}+\iota_{i})0_{C}
 +\frac{1}{2(\alpha+\beta+\gamma)}0_{A}0_{B}0_{C} \right\rbrace dr \\
 &= \sum_{i = x, y, z} \int 
 4\alpha\gamma\left\lbrace (G_{i}-C_{i})(G_{i}-A_{i})0_{A}0_{B}0_{C}
 +\frac{1}{2(\alpha+\beta+\gamma)}0_{A}0_{B}0_{C} \right\rbrace dr \\
 &+ \sum_{i = x, y, z} \int 
 4\beta\gamma \left\lbrace (G_{i}-C_{i})(G_{i}-B_{i})0_{A}0_{B}0_{C}
 +\frac{1}{2(\alpha+\beta+\gamma)}0_{A}0_{B}0_{C} \right\rbrace dr \\ 
 &= \sum_{i = x, y, z} \left\lbrace 
 4\gamma(G_{i}-C_{i}) \left[ \alpha(G_{i}-A_{i}) + \beta(G_{i}-B_{i})\right] 
 + \frac{2\gamma(\alpha+\beta)}{\alpha+\beta+\gamma}
 \right\rbrace \int 0_{A}0_{B}0_{C} dr
\end{split}
\label{three_body_kinetic_integral_eq:6}
\end{equation}


\begin{comment}
 &+ \sum_{i = x, y, z} \int \Bigl{ 
\chi_{A}(r) \left[ 
N_{i}(B)(\chi_{B}-\iota_{i}) - 2\beta(\chi_{B}+\iota_{i})\right]
\left[ N_{i}(C)(\chi_{C}-\iota_{i}) - 2\gamma(\chi_{C}+\iota_{i})\right] \Bigr} dr
\end{comment}




%%%%%%%%%%%%%%%%%%%%%%%%%%%%%%%%%%%%%%%%%%%%%%%%%%%%%%%%%%%%%%%%%%%%%%%%%%%%%%%%
\subsection{Nuclear Attraction Integral}
%
%
%
In the section of \ref{direct_NAI_derivation}, we have derived the nuclear
attraction integral explicitly. Here in this section, we will use the recursive
relation to derive a simple algorithm to calculate the NAI.

As we know, a general NAI is expressed as:
\begin{equation}
 \begin{split}
  (a|\Lambda_{c}|b) &= \int \chi_{i}(r)\frac{1}{r_{C}}\chi_{j}(r) dr \\
&= \int x^{l_{A}}_{A}y^{m_{A}}_{A}z^{n_{A}}_{A}e^{-\alpha r_{A}^{2}}
        \frac{1}{r_{C}}
        x^{l_{B}}_{B}y^{m_{B}}_{B}z^{n_{B}}_{B}e^{-\beta  r_{B}^{2}} dr
 \end{split}
\end{equation}
$r_{A}, r_{B}$ and $r_{c}$ denotes the distance between electron position and
the given nuclear (A, B or C).

Through transformation, the $\frac{1}{r_{C}}$ becomes:
\begin{equation}
\label{OS_nuclear_eq:1}
\frac{1}{r_{C}} = \frac{2}{\pi^{1/2}}\int^{\infty}_{0}e^{-u^{2}(r-C)^{2}} du
\end{equation}
Then it's fully converted into a S type of primitive function.

Therefore, for the $(a|0_{c}|b)$ it could be expressed as:
\begin{equation}
 \label{OS_nuclear_eq:2}
\begin{split}
(a|\Lambda_{c}|b) &= \frac{2}{\pi^{1/2}}\int^{\infty}_{0} du
 \int dr x^{l_{A}}_{A}y^{m_{A}}_{A}z^{n_{A}}_{A}e^{-\alpha r_{A}^{2}}
        e^{-u^{2}(r-C)^{2}}
        x^{l_{B}}_{B}y^{m_{B}}_{B}z^{n_{B}}_{B}e^{-\beta  r_{B}^{2}} \\
&= \frac{2}{\pi^{1/2}}\int^{\infty}_{0} du  (a|0_{c}|b)
\end{split}
\end{equation}
Then the NAI is converted back to the three center overlap integral.

According to the recursive relation in \ref{OS_three_overlap_int_eq:12},
the $(a|0_{c}|b)$ could be expressed as:
\begin{equation}
 \begin{split}
  (a+\iota_{i}|0_{c}|b) 
&= \left( \frac{\alpha A_{i} + \beta B_{i} + u^{2}C_{i}}{\alpha+\beta+u^{2}} -
A_{i}\right) (a|0_{c}|b)  \\
&+ 
N_{i}(A)\left(\frac{1}{2(\alpha+\beta+u^{2})}\right)(a-\iota_{i}|0_{c}|b) \\
&+
N_{i}(C)\left(\frac{1}{2(\alpha+\beta+u^{2})}\right)(a|0_{c}|b-\iota_{i}) \\
&= \left( P_{i}  - A_{i}\right) (a|0_{c}|b) - (P_{i} - C_{i})
\frac{u^{2}}{\alpha+\beta+u^{2}}(a|0_{c}|b)  \\
&+ 
\frac{N_{i}(A)}{2(\alpha+\beta)}\left(1-\frac{u^{2}}{(\alpha+\beta+u^{2})}
\right)(a-\iota_{i}|0_{ c }|b) \\
&+
\frac{N_{i}(C)}{2(\alpha+\beta)}\left(1-\frac{u^{2}}{(\alpha+\beta+u^{2})}
\right)(a|0_{c}|b-\iota_{i}) \\
 \end{split}
\label{OS_nuclear_eq:3}
\end{equation}
Where $P_{i}$ is:
\begin{equation}
 P_{i} = \frac{\alpha A_{i} + \beta B_{i}}{\alpha+\beta}
\end{equation}
Here, we are carefully modifying the expression so that to make the result
similar to the one we derived in ERI(see \ref{OS_ERI_eq:27}). Therefore,
here for \ref{OS_nuclear_eq:3} a similar auxiliary function (see
\ref{OS_ERI_eq:28}) could be defined($\epsilon = \alpha+\beta$):
\begin{equation}
\label{OS_nuclear_eq:4}
  (a|0_{c}|b)^{(m)} = \frac{2}{\sqrt{\pi}}\int du \left( \frac{u^{2}}
{\epsilon+u^{2}}\right)^{m}(ab|u|cd) 
\end{equation}
and we have:
\begin{equation}
 \begin{split}
  (a+\iota_{i}|0_{c}|b)^{(m)} &=  
\left( P_{i}  - A_{i}\right) (a|0_{c}|b)^{(m)} - (P_{i} - C_{i})
(a|0_{c}|b)^{(m+1)}  \\
&+ 
\frac{N_{i}(A)}{2\epsilon}\left(
(a-\iota_{i}|0_{ c }|b)^{(m)}-
(a-\iota_{i}|0_{ c }|b)^{(m+1)}
\right) \\
&+
\frac{N_{i}(C)}{2\epsilon}\left(
(a|0_{c}|b-\iota_{i})^{(m)}-
(a|0_{c}|b-\iota_{i})^{(m+1)}
\right) \\
 \end{split}
\label{OS_nuclear_eq:5}
\end{equation}
The corresponding integral of $(0_{a}|0_{c}|0_{b})^{(m)}$ is:
\begin{equation}
\label{OS_nuclear_eq:6}
 (0_{a}|0_{c}|0_{b})^{(m)} = 2\left(
\frac{\epsilon}{\pi}\right)^{1/2}(0_{a}|0_{b})
\int^{1}_{0} t^{2m} e^{-(\epsilon|PC|^{2})t^{2}} dt 
\end{equation}


%%%%%%%%%%%%%%%%%%%%%%%%%%%%%%%%%%%%%%%%%%%%%%%%%%%%%%%%%%%%%%%%%%%%%%%%%%%%%%%%
\begin{comment}
%%%
%%%  the description here is not quite what we want. So just screen the
%    contents.
%%%

\subsection{Implementation Considerations}
%
%
%
Now let's give some deep discussion on how to realize the algorithm. For the
OS and its derived algorithm, the first thing we can note is that the recursive
relationship is actually expressed based on ``shell'' rather than the basis set
orders (for example, shell P has three basis set orders; Px, Py and Pz).

This is easy to understand. Suggest for an arbitrary overlap integral,
$(a|b)$, there will be $n_{a}n_{b}$ basis set pairs ($n$ is the number of
Cartesian basis set). For the $(a+\iota_{i}|b)$, depending on the literal 
meaning the basis set pairs would be $3n_{a}n_{b}$ since $\iota_{i}$ loop
over x, y and z. However, if shell a is incremented to the shell a+1 (like
from shell P to shell D), the basis set pairs would be:
\begin{equation}
 N_{1} = n_{b}\left(\frac{(L+2)(L+3)(L+4)-(L+1)(L+2)(L+3)}{6} \right) 
\end{equation}
$L$ is the angular momentum number for the shell a. If we expand the
$3n_{a}n_{b}$ in terms of $L$, it's:
\begin{equation}
N_{2} = 3n_{b}\left(\frac{(L+1)(L+2)(L+3)-(L+1)(L+2)L}{6} \right) 
\end{equation}
Therefore $N_{1} - N_{2}$ gives:
\begin{equation}
 \begin{split}
 N_{1} - N_{2} &= n_{b}\left( \frac{(L+2)(L+3)(L+4)-(L+1)(L+2)(L+3)}{6}\right.
\\
&-\left. \frac{3(L+1)(L+2)(L+3)-3(L+1)(L+2)L}{6}\right) \\
&=  n_{b}(L+2)\left( \frac{(L+3)(L+4)-(L+1)(L+3)}{6}\right. \\
&-\left.  \frac{3(L+1)(L+3)-3(L+1)L}{6}\right) \\
&= n_{b}(L+2)\left( \frac{(L+3)(L+4)+3(L+1)L-4(L+1)(L+3)}{6}\right) \\
&= n_{b}(L+2)\left( \frac{(L+3)(-3L)+3(L+1)L}{6}\right) \\
&= n_{b}(L+2)L\left( \frac{-3(L+3)+3(L+1)}{6}\right) \\
&= n_{b}(L+2)L\left( \frac{-6}{6}\right) \\
&< 0
 \end{split}
\end{equation}
Therefore, The incremental on the integral could be fully expressed by shell
rather than basis set orders. What's more, it also indicates that for all of 
recursive relations in the OS framework, the integral is meaningful to the
``shells'', we need not to consider it in the basis set order level.

\end{comment}



% 
% firstly set up on Jan 2012
%
%
%
%
%%%%%%%%%%%%%%%%%%%%%%%%%%%%%%%%%%%%%%%%%%%%%%%%%%%%%%%%%%%%%%%%%%%%%%%%%%%%%%%%
\section{HGP Method}
%
%
%
%
HGP method\cite{HGP} is a revised method based on OS\cite{OS1986} framework.
It's central idea is to employ the horizontal recurrence relation so that 
to take the contraction of coefficients into consideration. 

Before we step into the discussion, let's define some expressions. The general
un-contracted integral, for example; the electron repulsion integral is
expressed as:
\begin{equation}
 (ab|cd) = \int dr \int dr^{'} \chi_{a}(r)\chi_{b}(r)\frac{1}{|r-r^{'}|}
\chi_{c}(r^{'})\chi_{d}(r^{'})
\label{HGP:1}
\end{equation}
This is same with the definition in \ref{OS_ERI_eq:1}. As we have noted in the
previous section that the integral are actually based on shell so we only need
to specify the shells of a, b, c, d. Therefore, the integral in the \ref{HGP:1}
is also called ``shell quartet''.

For the contracted integrals, we express it as:
\begin{align}
 \label{HGP:2}
[ij|kl] &=
\sum_{\mu}^{K_{\mu}}\sum_{\nu}^{K_{\nu}}\sum_{\lambda}^{K_{\lambda}}
\sum_{\eta}^{K_{\eta}}c_{\mu i}c_{\nu j}c_{\lambda k}c_{\eta l} \nonumber \\
&\int dr \int dr^{'} \chi_{\mu}(r)\chi_{\nu}(r)\frac{1}{|r-r^{'}|}
\chi_{\lambda}(r^{'})\chi_{\eta}(r^{'})
\end{align}
The Greek letters are the indices over AO space, the i,j,k,l are indices over
the basis set function space. $K$ denotes the contraction degree.
 

%%%%%%%%%%%%%%%%%%%%%%%%%%%%%%%%%%%%%%%%%%%%%%%%%%%%%%%%%%%%%%%%%%%%%%%%%%%%%%%%
\subsection{Horizontal Recurrence Relation}
In the OS framework, we already derived the recursive relation for the electron
repulsion integral:
\begin{equation}
 \begin{split}
 ((a+\iota_{i})b|cd)^{(m)} &= (P_{i} - A_{i})(ab|cd)^{(m)} +
\left(W_{i} -P_{i}\right)(ab|cd)^{(m+1)} \\
&+\frac{N_{i}(A)}{2\epsilon}\left(((a-\iota_{i})b|cd)^{(m)}-\frac{\rho}{
\epsilon }((a-\iota_{i})b|cd)^{(m+1)}\right)  \\
&+\frac{N_{i}(B)}{2\epsilon}\left((a(b-\iota_{i})|cd)^{(m)}-\frac{\rho}{
\epsilon }(a(b-\iota_{i})|cd)^{(m+1)}\right)  \\
&+\left(\frac{N_{i}(C)}{2}\right)\frac{1}{\epsilon+\eta}
(ab|(c-\iota_{i})d)^{(m+1)} \\
&+\left(\frac{N_{i}(D)}{2}\right)\frac{1}{\epsilon+\eta}
(ab|c(d-\iota_{i}))^{(m+1)} 
 \end{split}
\label{HGP_ERI_eq:1}
\end{equation}
Here we noted that in such relation, the shell $a$ and shell $b$ are in some
``symmetrical'' position thus if we consider the expression of
$(a(b+\iota_{i})|cd)^{(m)}$, it's:
\begin{equation}
\begin{split}
 (a(b+\iota_{i})|cd)^{(m)} &= (P_{i} - B_{i})(ab|cd)^{(m)} +
\left(W_{i} -P_{i}\right)(ab|cd)^{(m+1)} \\
&+\frac{N_{i}(A)}{2\epsilon}\left(((a-\iota_{i})b|cd)^{(m)}-\frac{\rho}{
\epsilon }((a-\iota_{i})b|cd)^{(m+1)}\right)  \\
&+\frac{N_{i}(B)}{2\epsilon}\left((a(b-\iota_{i})|cd)^{(m)}-\frac{\rho}{
\epsilon }(a(b-\iota_{i})|cd)^{(m+1)}\right)  \\
&+\left(\frac{N_{i}(C)}{2}\right)\frac{1}{\epsilon+\eta}
(ab|(c-\iota_{i})d)^{(m+1)} \\
&+\left(\frac{N_{i}(D)}{2}\right)\frac{1}{\epsilon+\eta}
(ab|c(d-\iota_{i}))^{(m+1)} 
 \end{split}
\label{HGP_ERI_eq:2} 
\end{equation}
We can see that most of the terms in \ref{HGP_ERI_eq:2} are same with
\ref{HGP_ERI_eq:1}. Then if we subtract the two terms, it gives:
\begin{equation}
 \label{HGP_ERI_HRR}
 (a(b+\iota_{i})|cd)^{(m)} = ((a+\iota_{i})b|cd)^{(m)} + 
(A_{i} - B_{i})(ab|cd)^{(m)}
\end{equation}
The new recursive relation in \ref{HGP_ERI_HRR} is called ``horizontal
recurrence relation''(HRR). The relation in \ref{HGP_ERI_eq:1} and 
\ref{HGP_ERI_eq:2} are called ``vertical recurrence relation''(VRR). 
Here it has an important feature, that for the \ref{HGP_ERI_HRR} 
we could contract it so that to transform it into:
\begin{equation}
 \label{HGP_ERI_eq:3} 
 [a(b+\iota_{i})|cd]^{(m)} = [(a+\iota_{i})b|cd]^{(m)} + 
(A_{i} - B_{i})[ab|cd]^{(m)}
\end{equation}
Therefore, the advantages for the HRR relation is that it reduces the
calculation inside the loop of primitive function pairs. For example, for the
$[pp|pp]$ primitives, now we just need to calculate the $[ds|ds]$. Therefore,
much less calculation is done with VRR.

What's more, we note that the HRR relation in \ref{HGP_ERI_HRR} is also applied 
to the other OS recursive relations. For example, for the three center overlap
integral whose relationship is:
\begin{equation}
 \begin{split}
 (a+\iota_{i}|b|c) 
&= (G_{Ai} - A_{i})(a|b|c) + 
N_{i}(A)\left(\frac{1}{2(\alpha+\beta+\gamma)}\right)(a-\iota_{i}|b|c) \\
&+ 
N_{i}(B)\left(\frac{1}{2(\alpha+\beta+\gamma)}\right)(a|b-\iota_{i}|c) \\
&+
N_{i}(C)\left(\frac{1}{2(\alpha+\beta+\gamma)}\right)(a|b|c-\iota_{i}) 
 \end{split}
\end{equation}
It's easy to see that the corresponding HRR is:
\begin{equation}
\label{HGP_ERI_eq:4}
 (a|b+\iota_{i}|c) = (a+\iota_{i}|b|c) + 
(A_{i} - B_{i})(a|b|c)
\end{equation}
Therefore, the same formula retains.


%%%%%%%%%%%%%%%%%%%%%%%%%%%%%%%%%%%%%%%%%%%%%%%%%%%%%%%%%%%%%%%%%%%%%%%%%%%%%%%%
\subsection{HGP Algorithm to Evaluate Integrals}
%
%
%
Based on the HRR, we have some new algorithm to evaluate the integrals(the
real calculation will be in reverse order):
\begin{description}
 \item[HRR step:]
 For a general shell quartet of $[ab|cd]$, through the HRR it could be
 evidently reduced to the integrals of $[e0|f0]$; where for the $e$ the angular
 momentum would be the sum of $a$ and $b$, for the $f$ it would be sum of $c$
 and $d$.
 \item[CONTRACTION step:]
 Perform the contraction from $(e0|f0)$ to $[e0|f0]$.
 \item[VRR step:]
 Calculate the primitive integral of $(e0|f0)$ in VRR. Each of such integrals
 will be reduced to the $(ss|ss)^{(m)}$, while for $(ab|cd)$ $m$ ranges from
 0 to $a+b+c+d$. 
 \end{description}

How to evaluate the HGP algorithm? By using the HRR, a lot of calculations
within the loop of primitives pairs now could be moved out of the contraction
loop (this loop is scaled with $K^{4}$). However, in HRR the angular momentum
increases in the right hand side so it implies that more shell quartets
are needed to be generated; which is sacrifice in terms of the balance. 

Now let's give an example to see how the HGP works. Suggest that for the 
shell quartet of $[FF|FF]$, by using the HRR we can have:
\begin{enumerate}
 \item $[FF|FF]$ is reduced to $[GD|FF]$ and $[FD|FF]$;
 \item $[GD|FF]$ is reduced to $[HP|FF]$ and $[GP|FF]$;
 \item $[FD|FF]$ is reduced to $[GP|FF]$ and $[FP|FF]$;
 \item $[HP|FF]$ is reduced to $[IS|FF]$ and $[HS|FF]$;
 \item $[GP|FF]$ is reduced to $[HS|FF]$ and $[GS|FF]$;
 \item $[FP|FF]$ is reduced to $[GS|FF]$ and $[FS|FF]$
\end{enumerate}
Therefore, there are only four elementary shell quartets, namely the 
$[FS|FF]$, $[GS|FF]$, $[HS|FF]$ and $[IS|FF]$. On the other hand, similar
treatment to the ket side will yields $[FF|FS]$, $[FF|GS]$, $[FF|HS]$ and
$[FF|SI]$ therefore only their combinations are needed:
\begin{enumerate}
 \item $[FS|FS]$, $[FS|GS]$, $[FS|HS]$, $[FS|IS]$
 \item $[GS|FS]$, $[GS|GS]$, $[GS|HS]$, $[GS|IS]$
 \item $[HS|FS]$, $[HS|GS]$, $[HS|HS]$, $[HS|IS]$
 \item $[IS|FS]$, $[IS|GS]$, $[IS|HS]$, $[IS|IS]$
\end{enumerate}
Then in the VRR step we could calculate all of these integrals.



%
% set up Oct. 2013
%
\section{Calculating $(00|r_{12}|00)^{m}$ integrals}
%
%
The $(00|r_{12}|00)^{m}$ type of integrals is fundamental
for recursive relation calculation. This type of integrals
is needed to be calculated directly from incomplete Gamma 
function (see \ref{nuclear_attraction_direct_int_eq:20} and 
\ref{OS_ERI_complementary_result}):
\begin{equation}
 (0_{A}0_{B}|r_{12}|0_{C}0_{D})^{(m)} = 
 2\left( \frac{\rho}{\pi}\right)^{\frac{1}{2}}(0_{A}|0_{B})
(0_{C}|0_{D})\int^{1}_{0} t^{2m} e^{-(\rho|PQ|^{2})t^{2}} dt 
\label{fm_ssssm_eq:0}
\end{equation}
Where we have the arguments as:
\begin{align}
 \overrightarrow{P} &= \frac{\alpha \overrightarrow{A} + \beta
\overrightarrow{B}}{\alpha + \beta} \nonumber \\
\overrightarrow{Q} &= \frac{\gamma \overrightarrow{C} + \delta
\overrightarrow{D}}{\gamma + \delta} \nonumber \\
\rho &= \frac{(\alpha+\beta)(\gamma + \delta)}
{(\alpha+\beta)+(\gamma + \delta)}
\end{align}

The integral in \ref{fm_ssssm_eq:0} could be directly calculated through
incomplete Gamma function. However, such way is expensive because the 
$(00|r_{12}|00)^{(m)}$ is inside the primitive Gaussian loop. Suggest
we have 1000 basis sets, and each basis set's contraction is 3; then
for producing the normal four center ERI integrals the number $(00|r_{12}|00)^{(m)}$
for each $m$ could be approximated as: $\dfrac{1}{8}*1000^{4}*3^{4} = 
1.0125\times 10^{13}$! Therefore, even a small increasing of the calculation
cost to the $(00|r_{12}|00)^{(m)}$ will lead to a dramatical cost increase
to the integral calculation. Therefore, it's necessary to have a simpler
way to calculate the $(00|r_{12}|00)^{(m)}$ integrals.

There has been a lot of literatures to discuss the integral calculation here(see 
\cite{harris1983sssm, gill1991two} etc. and the paper cited by them). The discussion
made in this section is mainly based on the results in the above reference.

\subsection{$f_{m}(t)$}
%
%
Firstly, let's discuss the integral inside the \ref{fm_ssssm_eq:0}:
\begin{equation}\label{fm_ssssm_fmt_eq:1}
 f_{m}(t) = \frac{2}{\sqrt{\pi}}\int^{1}_{0} u^{2m} e^{-tu^{2}} du 
\end{equation}

When the $m=0$, this function is back to the error function:
\begin{equation}
\begin{split}
 f_{0}(t) &= \frac{2}{\sqrt{\pi}}\int^{1}_{0} e^{-tu^{2}} du  \\
          &= t^{-\frac{1}{2}}\frac{2}{\sqrt{\pi}}\int^{1}_{0} 
          e^{-(\sqrt{t}u)^{2}} d (\sqrt{t}u) \\
          &= t^{-\frac{1}{2}}\frac{2}{\sqrt{\pi}}\int^{\sqrt{t}}_{0} e^{-x^{2}} dx \\
          &= t^{-\frac{1}{2}} erf(t^{\frac{1}{2}})
\end{split}
\label{fm_ssssm_fmt_eq:2}
\end{equation}
The error function is fast to converge(this function in standard C++ library is several
hundreds time faster than the incomplete gamma function defined in boost library), and 
we note; this function can be found 
in standard C++ library\footnote{We note that the standard efc function in C library has 
the $\dfrac{2}{\sqrt{\pi}}$}.

For $m>0$, by integration by parts it's easy to find a recursive relation
for compute $f_{m}(t)$:
\begin{equation}
 \begin{split}
  f_{m}(t) &= \frac{2}{\sqrt{\pi}}\int^{1}_{0} u^{2m} e^{-tu^{2}} du \\ 
           &= -\frac{1}{t\sqrt{\pi}}\int^{1}_{0} u^{2m-1} e^{-tu^{2}} d(-tu^{2}) \\
           &= -\frac{1}{t\sqrt{\pi}}\int^{1}_{0} u^{2m-1} d\left( e^{-tu^{2}} \right) \\         
           &= -\left.\frac{1}{t\sqrt{\pi}}u^{2m-1} e^{-tu^{2}}\right|^{1}_{0} + 
           \frac{2m-1}{t\sqrt{\pi}}\int^{1}_{0} u^{2m-2} e^{-tu^{2}} du \\  
           &= -\frac{1}{t\sqrt{\pi}}e^{-t} + \frac{2m-1}{2t} f_{m-1}(t) \Rightarrow \\
           &= \frac{1}{2t}\left( (2m-1)f_{m-1}(t) - \frac{2}{\sqrt{\pi}}e^{-t}\right)    
 \end{split}
 \label{fm_ssssm_fmt_eq:3}
\end{equation}
This function provides the easiest way to compute the $f_{m}(t)$. However, if we use
such recursive relation to direct compute $f_{m}(t)$ from the $f_{0}(t)$, it's 
easy to see the error is propagating very quickly (see the example in 
\ref{error_propagation_numerical} for more details). Therefore, to directly use the 
\ref{fm_ssssm_fmt_eq:3} from $f_{0}(t)$ for computing is not applicable.

The error propagation in this example is because we have 
$f_{m}(t) > f_{m+1}(t)$\footnote{such expression could be easily derived from 
Cauchy–Schwarz inequality}. It's easy to see that $f_{m}(t)$ is always larger than 0.
As the $m$ grows larger, the difference between $f_{m}(t)$ and $f_{m+1}(t)$ is getting 
smaller. This is the reason why error begins propagating when we calculate $f_{m+1}(t)$
from $f_{m}(t)$.

However, such recursive relation may be applicable for some given $m$ and $t$ combinations,
as long as the error is within some small range. For testify this idea, we made an
investigation for all of $m$ and $t$ combinations within error limit of $1.0E^{-10}$. 
The results indicate that for $m=1$ to $m=10$ and $t>1$, by using this ``UP'' 
recursive relation we could get the accurate results(within error of $1.0E^{-10}$) 
by climbing from error function\footnote{for a detailed testing, we study the $t$
from $1.0$ to $30.0$(when $t>30.0$ we can use simpler form to express $f_{m}(t)$, see
the discussion below) with step length of $1.0E^{-6}$ for $m=1$ to $m=10$. All of results
are comparing with the one calculated from incomplete gamma function from boost library
in terms of error limit in $1.0E^{-10}$}.

On the other hand, such circumstance indicates that the reverse recursive calculation
is applicable for calculation, that is; to compute the $f_{m}(t)$ from $f_{m+1}(t)$. 
It's easy to show, that through the ``downward recursive relation'' the computation
of $f_{m}(t)$ will never lose the accuracy\footnote{We tested the cases for $m=1$ 
to $m=10$ and $t$ ranges from $0.0$ to $30.0$ with same step length etc. shown above}. 
Therefore, in real application we will use the following expression:
\begin{equation}
 \label{fm_ssssm_fmt_eq:4}
 f_{m}(t) = \frac{1}{2m+1}\left( 2tf_{m+1}(t) + \frac{2}{\sqrt{\pi}}e^{-t}\right) 
\end{equation}
We have tested for $m=0$ to $m=30$ for t ranging from $0.001$ to $30.00$ (with 
step wide of 0.001 too), by using 
the downward recursive relation all of results could be accurate within the error 
range of $1.0^{-10}$. Therefore, the problem here is that how we compute the 
$f_{m_{max}}(t)$.

The most easiest way to calculate $f_{m}(t)$ is perhaps through it's series expansion. In 
paper of Harris\cite{harris1983sssm}, equation 9 provides a series expansion:
\begin{equation}
 \label{fm_ssssm_fmt_eq:5}
 f_{m}(t) = \frac{2}{\sqrt{\pi}}e^{-t}\sum_{k=0}^{\infty}\frac{(2m-1)!!}{(2m+2k+1)!!}
 (2t)^{k}
\end{equation}
We can see that for small t (perhaps $t<=1$), such expression will converge in 
a very quick way. For larger $t$, it's obvious that it needs more term for 
converging\footnote{We have tested the $m=1$ to $m=10$ cases with respect to 
$t<=1$ in same testing condition mentioned above. All of results are in good
accuracy within error range of $1.0^{-10}$ with 12 terms expansion}.

However, for the large $t$ situation, the integral of $f_{m}(t)$ should be faster
approaching to zero compared with these with small $t$. From the paper 
\cite{harris1983sssm}, it has another series expansion to understand it:
\begin{equation}
\begin{split}
 f_{m}(t) &= \frac{(2m-1)!!}{(2t)^{m}t^{1/2}}-\frac{2}{\sqrt{\pi}}\frac{e^{-t}}{2t} 
 \left( 1+\frac{2m-1}{2t} \right. \\
 &+ \left. \frac{(2m-1)(2m-3)}{(2t)^{2}} + \frac{(2m-1)(2m-3)(2m-5)}{(2t)^{3}}+ \cdots 
 \right) 
\end{split}
 \label{fm_ssssm_fmt_eq:6}
\end{equation}
It's easy to see, when $t>25$; then $\dfrac{2}{\sqrt{\pi}}\dfrac{e^{-t}}{2t}$
will be less than $10^{-12}$ therefore we can always omit the series inside 
\ref{fm_ssssm_fmt_eq:6} so that when $t>=25$, the $f_{m}(t)$ could be expressed 
as:
\begin{equation}
 \label{fm_ssssm_fmt_eq:7}
 f_{m}(t) = \frac{(2m-1)!!}{(2t)^{m}t^{1/2}}
\end{equation}

Now let's give a summary to calculate the $f_{m}(t)$:
\begin{enumerate}
 \item if $M_{max} = 0$, use erf function;
 \item if $M_{max} >= 1$ and $M_{max} <= 10$:
 \begin{enumerate}
  \item if $t<=1$, calculate $f_{Mmax}(t)$ by using power series in 
  \ref{fm_ssssm_fmt_eq:5} with 12 terms, then use down recursive relation
  in \ref{fm_ssssm_fmt_eq:4} for the rest of $f_{m}(t)$;
  \item if $t>1$ and $t<25$, calculate $f_{0}(t)$ with error function and 
  use up recursive relation in \ref{fm_ssssm_fmt_eq:3} to derive other $f_{m}(t)$;
  \item if $t>=25$, calculate $f_{Mmax}(t)$ with \ref{fm_ssssm_fmt_eq:7} and use down 
  recursive relation in \ref{fm_ssssm_fmt_eq:4} to derive other $f_{m}(t)$
 \end{enumerate}
 \item if $M_{max} >= 11$:
 \begin{enumerate}
  \item if $t<25$, calculate $f_{Mmax}(t)$ from incomplete gamma function (for example,
  from boost library function), and use down 
  recursive relation in \ref{fm_ssssm_fmt_eq:4} to derive other $f_{m}(t)$
  \item if $t>=25$, calculate $f_{Mmax}(t)$ with \ref{fm_ssssm_fmt_eq:7} and use down 
  recursive relation in \ref{fm_ssssm_fmt_eq:4} to derive other $f_{m}(t)$
 \end{enumerate}
\end{enumerate}
We note, that with such arrangement, all of S, P and D shell integrals (up to second 
derivatives order) could be derived without incomplete gamma function, which is 
expensive. for $m>11$, the down recursive relation grantees that only one $f_{m}(t)$
is calculated, which also saves a lot of times.

\subsection{Compute Bottom Integrals}
%
%
Now let's use the above results and consider the full expression of 
\ref{fm_ssssm_eq:0}:
\begin{equation}
 (00|r_{12}|00)^{(m)} = O\rho^{\frac{1}{2}}f_{m}(T)
\label{fm_ssssm_eq:1}
\end{equation}
Where the parameters are given as\footnote{``O'' here indicates 
overlap integrals}:
\begin{equation}
 \begin{split}
  \sigma_{P} &= \frac{1}{\alpha + \beta} \\
  \bm{P}  &=(\alpha{A} + \beta\bm{B})\sigma_{P} \\
  O_{P} &= (\pi\sigma_{P})^{\frac{3}{2}}e^{-\alpha\beta\sigma_{P}
  |\bm{A}-\bm{B}|^{2}} \\
   \sigma_{Q} &= \frac{1}{\gamma + \delta} \\
  \bm{Q}  &= (\gamma\bm{C} + \delta\bm{D})\sigma_{Q} \\
  O_{Q} &= (\pi\sigma_{Q})^{\frac{3}{2}}e^{-\gamma\delta\sigma_{Q}
  |\bm{C}-\bm{D}|^{2}} \\
  O &= O_{P}O_{Q} \\
  \rho &= \frac{1}{\sigma_{P} + \sigma_{Q}} \\
  R &= |\bm{P}-\bm{Q}| \\
  T &= \rho R^{2}
 \end{split}
 \label{fm_ssssm_eq:2}
\end{equation}
Here all of the parameters except $\rho$, $O$, $R$ and $T$ could be 
pre-computed outside the integral loop.

For the $m=0$, the $(00|r_{12}|00)^{(0)}$ is:
\begin{equation}
 \begin{split}
  (00|r_{12}|00)^{(0)} &= O\rho^{\frac{1}{2}}f_{0}(T) \\
  &= O\rho^{\frac{1}{2}} T^{-\frac{1}{2}} erf(T^{\frac{1}{2}}) \\
  &= \frac{O}{R}erf(T^{\frac{1}{2}}) 
 \end{split}
 \label{fm_ssssm_eq:3}
\end{equation}

For the $m>0$ case, the down recursive relation is:
\begin{equation}
\begin{split}
 f_{m}(T) &=\frac{1}{2m+1}\left( 2Tf_{m+1}(T) + \frac{2}{\sqrt{\pi}}e^{-T}\right)
 \Rightarrow \\
 (00|r_{12}|00)^{(m)} &= \frac{1}{2m+1}
 \left( 2t(00|r_{12}|00)^{(m+1)} + \frac{2}{\sqrt{\pi}}O\rho^{\frac{1}{2}}e^{-T} \right)
\end{split}
\label{fm_ssssm_eq:4}
\end{equation}
Here the factor of $\frac{1}{T\sqrt{\pi}}$ and term 
$O\rho^{\frac{1}{2}}e^{-T}$ need to 
be computed in the integral loop (but only once!). 

The up recursive relation is:
\begin{equation}
 \begin{split}
  f_{m}(T) &= \frac{1}{2T}\left( (2m-1)f_{m-1}(T) - \frac{2}{\sqrt{\pi}}e^{-T}\right) 
  \Rightarrow \\    
  (00|r_{12}|00)^{(m)} &= \frac{1}{2T}\left( (2m-1)(00|r_{12}|00)^{(m-1)} - 
  \frac{2}{\sqrt{\pi}}O\rho^{\frac{1}{2}}e^{-T}\right)
 \end{split}
 \label{fm_ssssm_eq:5}
\end{equation}

The power series becomes:
\begin{equation}
  (00|r_{12}|00)^{(m)} = \frac{2}{\sqrt{\pi}}O\rho^{\frac{1}{2}}e^{-T}
  \sum_{k=0}^{\infty}\frac{(2m-1)!!}{(2m+2k+1)!!}(2T)^{k}
 \label{fm_ssssm_eq:6}
\end{equation}
and the \ref{fm_ssssm_fmt_eq:7} becomes:
\begin{equation}
 \label{fm_ssssm_fmt_eq:7}
 (00|r_{12}|00)^{(m)} = O\frac{(2m-1)!!}{(2T)^{m}R}
\end{equation}




%%% Local Variables:
%%% mode: latex
%%% TeX-master: "../../note"
%%% End:
