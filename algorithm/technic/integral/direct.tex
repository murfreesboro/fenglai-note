% 
% firstly set up on May, 2011
%
% for direct calculation:
% checked the results for kinetic integral,  overlap integral, 
% and derivations for overlap.   June 2ed
% checked the derivation for nuclear attraction integral, fine.  June 13th
%
%

%%%%%%%%%%%%%%%%%%%%%%%%%%%%%%%%%%%%%%%%%%%%%%%%%%%%%%%%%%%%%%%%%%%%%%%%%%%%%%%%%%%%%%%%%%%%
\section{Direct Integrals Calculation}
%
%
%
%
In this section, we will show how to derive the one electron integrals in terms of the 
basic form of shell pair data shown in \ref{basic_shell_pair_data_form}. Generally, such
way of derivation is simple and the efficiency is low; therefore in the modern quantum 
chemistry package it's not used anymore. However, it's still listed here for archive reason.
Furthermore, from the derivation we could clearly see how to make integral from the basic
way.

%%%%%%%%%%%%%%%%%%%%%%%%%%%%%%%%%%%%%%%%%%%%%%%%%%%%%%%%%%%%%%%%%%%%%%%%%%%%%%%%%%%%%%%%%%%%
\subsection{Overlap Integrals}
%
%
%
%
Now let's step into overlap integral. Firstly let's consider the simple S type of integral 
without contraction:
\begin{equation}
 \label{overlap_direct_int_eq:1}
\begin{split}
S_{\mu\nu}  &= \int \phi_{\mu}^{*}\phi_{\nu} dr \\
&= \int e^{-\alpha r_{A}^{2} - \beta r_{B}^{2}} dr \\
&= e^{-\frac{\alpha\beta}{\alpha+\beta}\overline{AB}^{2}}\int e^{-(\alpha+\beta)r_{P}^{2}} dr \\
&= e^{-\frac{\alpha\beta}{\alpha+\beta}\overline{AB}^{2}} \int e^{-(\alpha+\beta)x_{P}^{2}} dx
\int e^{-(\alpha+\beta)y_{P}^{2}} dy \int e^{-(\alpha+\beta)z_{P}^{2}} dz \\
&= e^{-\frac{\alpha\beta}{\alpha+\beta}\overline{AB}^{2}}\left( \frac{\pi}{\alpha+\beta}\right)
^{\frac{3}{2}}   
\end{split}
\end{equation}

The more common form of overlap integral between two arbitrary Gaussian primitives without 
contraction, will be given according to the \ref{gaussian_product_rule_eq:19} and 
\ref{gaussian_product_rule_eq:20}:
\begin{equation}
 \label{overlap_direct_int_eq:2}
\begin{split}
S_{\mu\nu}  &= \int \phi_{\mu}^{*}\phi_{\nu} dr \\
&= \int x_{A}^{l_{1}}y_{A}^{m_{1}}z_{A}^{n_{1}}
        x_{B}^{l_{2}}y_{B}^{m_{2}}z_{B}^{n_{2}}
        e^{-\alpha r_{A}^{2} - \beta r_{B}^{2}} dr \\
&= \int \sum_{i=0}^{l_{1}+l_{2}}\sum_{j=0}^{m_{1}+m_{2}}\sum_{k=0}^{n_{1}+n_{2}}
   x_{P}^{i}y_{P}^{j}z_{P}^{k}
   e^{-(\alpha+\beta)r_{P}^{2}} e^{-\frac{\alpha\beta}{\alpha+\beta}\overline{AB}^{2}}
   F_{ijk} dr \\
&= e^{-\frac{\alpha\beta}{\alpha+\beta}\overline{AB}^{2}}
   \sum_{i=0}^{l_{1}+l_{2}}\sum_{j=0}^{m_{1}+m_{2}}\sum_{k=0}^{n_{1}+n_{2}}F_{ijk}
   \int x_{P}^{i}y_{P}^{j}z_{P}^{k} e^{-(\alpha+\beta)r_{P}^{2}} dr \\
&= e^{-\frac{\alpha\beta}{\alpha+\beta}\overline{AB}^{2}}
   \sum_{i=0}^{l_{1}+l_{2}}\sum_{j=0}^{m_{1}+m_{2}}\sum_{k=0}^{n_{1}+n_{2}}F_{ijk}
   \int x_{P}^{i}e^{-(\alpha+\beta)x_{P}^{2}} dx \\
&\times
   \int y_{P}^{j}e^{-(\alpha+\beta)y_{P}^{2}} dy \int z_{P}^{k}e^{-(\alpha+\beta)z_{P}^{2}} dz  
\end{split}
\end{equation}

Now the integral has been retreated into simpler form. Now let's consider the integral form
of:
\begin{equation}
 \label{overlap_direct_int_eq:3}
 \int x^{i}e^{-\gamma x^{2}} dx
\end{equation}

This is the general form of integrals appearing in the \ref{overlap_direct_int_eq:2}. 
Firstly, we can
see that this integral only exists when $i$ is even number; that is to say:
\begin{equation}
\label{overlap_direct_int_eq:even}
 \int x^{i}e^{-\gamma x^{2}} dx = 0 \quad i = 2k+1
\end{equation}
This is because the function of $x^{i}e^{-\gamma x^{2}}$ is odd function when $i = 2k+1$. Hence 
here we have to make additional treatment, that to replace all of $i$, $j$ and $k$ in 
\ref{overlap_direct_int_eq:2} into $i^{'}$, $j^{'}$ and $k^{'}$:
\begin{equation}
 \begin{split}
  i &= 2*i^{'} \\ 
  j &= 2*j^{'} \\
  k &= 2*k^{'} 
 \end{split}
 \label{overlap_direct_int_eq:4}
\end{equation}

So the result in the \ref{overlap_direct_int_eq:2} can be rewritten as:
\begin{equation}
 \label{overlap_direct_int_eq:5}
\begin{split}
S_{\mu\nu}  &= \int \phi_{\mu}^{*}\phi_{\nu} dr \\
&= e^{-\frac{\alpha\beta}{\alpha+\beta}\overline{AB}^{2}}
   \sum_{i^{'}=0}^{\left[ \frac{l_{1}+l_{2}}{2}\right] }
   \sum_{j^{'}=0}^{\left[ \frac{m_{1}+m_{2}}{2}\right] }
   \sum_{k^{'}=0}^{\left[ \frac{n_{1}+n_{2}}{2}\right] }
   F_{2i^{'},2j^{'},2k^{'}}
   \int x_{P}^{2i^{'}}e^{-(\alpha+\beta)x_{P}^{2}} dx \\
&\times
   \int y_{P}^{2j^{'}}e^{-(\alpha+\beta)y_{P}^{2}} dy 
   \int z_{P}^{2k^{'}}e^{-(\alpha+\beta)z_{P}^{2}} dz  
\end{split}
\end{equation}

This integral, actually can be decomposed into three pieces:
\begin{equation}
 \label{overlap_direct_int_eq:6}
S_{\mu\nu} = e^{-\frac{\alpha\beta}{\alpha+\beta}\overline{AB}^{2}}
S_{\mu\nu}^{x}S_{\mu\nu}^{y}S_{\mu\nu}^{z}
\end{equation}

All of the three pieces are sharing the same structure:
\begin{align}
 \label{overlap_direct_int_eq:7}
S_{\mu\nu}^{x} &= \sum_{i^{'}=0}^{\left[ \frac{l_{1}+l_{2}}{2}\right] }
                  f_{2i^{'}}(l_{1},l_{2},\overline{PA}_{x},\overline{PB}_{x})
                  \int x_{P}^{2i^{'}}e^{-(\alpha+\beta)x_{P}^{2}} dx \nonumber \\
S_{\mu\nu}^{y} &= \sum_{j^{'}=0}^{\left[ \frac{m_{1}+m_{2}}{2}\right] }
                  f_{2j^{'}}(m_{1},m_{2},\overline{PA}_{y},\overline{PB}_{y})
                  \int y_{P}^{2j^{'}}e^{-(\alpha+\beta)y_{P}^{2}} dy \nonumber \\
S_{\mu\nu}^{z} &= \sum_{k^{'}=0}^{\left[ \frac{n_{1}+n_{2}}{2}\right] }
                  f_{2k^{'}}(n_{1},n_{2},\overline{PA}_{z},\overline{PB}_{z})
                  \int z_{P}^{2k^{'}}e^{-(\alpha+\beta)z_{P}^{2}} dz
\end{align}
According to the \ref{int_sec2_eq:8}, where the integral can be expressed as:
\begin{equation}
\int x^{2k}e^{-\alpha x^{2}} dx = 
\frac{(2k-1)!!\sqrt{\pi}}{(2\alpha)^{k}\sqrt{\alpha}}
\end{equation}
we can have that:
\begin{align}
 \label{overlap_direct_int_eq:8}
S_{\mu\nu}^{x} &= \sum_{i^{'}=0}^{\left[ \frac{l_{1}+l_{2}}{2}\right] }
                  f_{2i^{'}}(l_{1},l_{2},\overline{PA}_{x},\overline{PB}_{x})
                  \frac{(2i^{'}-1)!!\sqrt{\pi}}{(2(\alpha+\beta))^{i^{'}}\sqrt{\alpha+\beta}}
                  \nonumber \\
S_{\mu\nu}^{y} &= \sum_{j^{'}=0}^{\left[ \frac{m_{1}+m_{2}}{2}\right] }
                  f_{2j^{'}}(m_{1},m_{2},\overline{PA}_{y},\overline{PB}_{y})
                  \frac{(2j^{'}-1)!!\sqrt{\pi}}{(2(\alpha+\beta))^{j^{'}}\sqrt{\alpha+\beta}}
                  \nonumber \\
S_{\mu\nu}^{z} &= \sum_{k^{'}=0}^{\left[ \frac{n_{1}+n_{2}}{2}\right] }
                  f_{2k^{'}}(n_{1},n_{2},\overline{PA}_{z},\overline{PB}_{z})
                  \frac{(2k^{'}-1)!!\sqrt{\pi}}{(2(\alpha+\beta))^{k^{'}}\sqrt{\alpha+\beta}}
\end{align}
Now this is the result for the overlap integral between two arbitrary Gaussian primitives with
different centers.

For the overlap integral on the same center, according to the result in 
\ref{gaussian_product_rule_eq:21}, we will have that ($A=B$):
\begin{equation}
 \begin{split}
&\int x_{A}^{l_{1}}y_{A}^{m_{1}}z_{A}^{n_{1}}
        x_{B}^{l_{2}}y_{B}^{m_{2}}z_{B}^{n_{2}}
        e^{-\alpha r_{A}^{2} - \beta r_{B}^{2}} dr \\
&= \int x_{A}^{l_{1}+l_{2}}y_{A}^{m_{1}+m_{2}}z_{A}^{n_{1}+n_{2}}e^{-(\alpha+\beta)r_{A}^{2}} dr\\
&= 
\int x_{A}^{l_{1}+l_{2}}e^{-(\alpha+\beta)x_{A}^{2}} dx
\int y_{A}^{m_{1}+m_{2}}e^{-(\alpha+\beta)y_{A}^{2}} dy
\int z_{A}^{n_{1}+n_{2}}e^{-(\alpha+\beta)z_{A}^{2}} dz
 \end{split}
\label{overlap_direct_int_eq:9}
\end{equation}
According to the result in \ref{overlap_direct_int_eq:even}, the result will 
be quite interesting that
this integral is not zero only if $l_{1}+l_{2} = 2i$, $m_{1}+m_{2} = 2j$, $n_{1}+n_{2} = 2k$. 
Hence it's easy to see that integrals between S and P shell etc. are all zero. However, it does 
not mean that different shell types in the same center are orthogonal. For example, the integral
between PX and FXY2 are not zero.

For the integral, according to the result in \ref{int_sec2_eq:8}; the final result 
between same center could be expressed as:
\begin{equation}
 \label{overlap_direct_int_eq:10}
\begin{split}
 & 
\int x_{A}^{l_{1}+l_{2}}e^{-(\alpha+\beta)x_{A}^{2}} dx
\int y_{A}^{m_{1}+m_{2}}e^{-(\alpha+\beta)y_{A}^{2}} dy
\int z_{A}^{n_{1}+n_{2}}e^{-(\alpha+\beta)z_{A}^{2}} dz \\
&= \left( \frac{\pi}{\alpha+\beta}\right)^{\frac{3}{2}} 
\frac{(l_{1}+l_{2}-1)!!(m_{1}+m_{2}-1)!!(n_{1}+n_{2}-1)!!}
{(2(\alpha+\beta))^{\frac{l_{1}+l_{2}+m_{1}+m_{2}+n_{1}+n_{2}}{2}}}
\end{split}
\end{equation}
 
Finally, we step into the overlap integral for an arbitrary basis function of $\phi$, which is 
a set of contracted Gaussian primitives. Firstly we can record the result in 
\ref{overlap_direct_int_eq:6}
as $\langle\chi_{\mu}|\chi_{\nu}\rangle$ (but usually we express it as 
$\langle\chi_{i}|\chi_{j}\rangle$, since the $\mu$, $\nu$ label are related to the indices of 
basis set functions). The basis set function can be expressed as:
\begin{equation}
\begin{split}
\label{overlap_direct_int_eq:11}
 \phi_{\mu} &= \sum_{i}d_{\mu i}\chi_{i} \\
 \phi_{\nu} &= \sum_{j}d_{\mu j}\chi_{j} 
\end{split}
\end{equation}

Then based on the results in \ref{overlap_direct_int_eq:6}, \ref{overlap_direct_int_eq:7} 
and \ref{overlap_direct_int_eq:8}
we have:
\begin{equation}
 \label{overlap_direct_int_eq:12}
 S_{\mu\nu} = \sum_{i}\sum_{j}d_{\mu i}d_{\mu j}\langle\chi_{i}|\chi_{j}\rangle
\end{equation}
This is the final result for overlap integral.

%%%%%%%%%%%%%%%%%%%%%%%%%%%%%%%%%%%%%%%%%%%%%%%%%%%%%%%%%%%%%%%%%%%%%%%%%%%
\subsection{Derivatives for Overlap Integral} 

There's more advantages if we use Gaussian primitive function rather than the Slater type of function
in getting the derivatives of the integral. For a common Gaussian primitive, let's derive its 
derivatives we can see that:
\begin{equation}
 \label{derivative_overlap_direct_int_eq:1}
\frac{\partial \chi_{i}}{\partial R_{x}} = \frac{\partial (x^{l}y^{m}z^{n}e^{-\alpha r^{2}})}
{\partial R_{x}} =  -lx^{l-1}y^{m}z^{n}e^{-\alpha r^{2}} + 2\alpha x^{l+1}y^{m}z^{n}e^{-\alpha r^{2}}
\end{equation}
Here the $x$ is expressed as:
\begin{equation}
 x = x_{e} - R_{x} 
\end{equation}
Which is the $x$ difference between the electron position and the nuclear position. 

The expression is exactly equal to the form below:
\begin{equation}
 \label{derivative_overlap_direct_int_eq:2}
\frac{\partial \chi^{lmn}_{i}}{\partial R_{x}} = -l\chi_{i}^{l-1mn} + 2\alpha\chi_{i}^{l+1mn}
\end{equation}
This result shows that the derivatives for the Gaussian primitive can be expressed as the linear 
sum of two Gaussian primitives, one is with higher angular momentum and the other with lower
angular momentum; it indicates that the derivatives for the integral of Gaussian primitives 
can be simply expressed as some summation between the integrals.

The overlap integral between two Gaussian primitives, can be expressed as:
\begin{equation}
 \label{derivative_overlap_direct_int_eq:3}
\frac{\partial\langle\chi_{i}|\chi_{j}\rangle}{\partial R_{x}} = 
\begin{cases}
 0  \\
\left\langle \frac{\partial \chi_{i} }{\partial R_{x}}|\chi_{j}\right\rangle \quad \text{or}
\quad \left\langle \chi_{i}|\frac{\partial \chi_{j} }{\partial R_{x}}\right\rangle \\
\left\langle \frac{\partial \chi_{i} }{\partial R_{x}}|\chi_{j}\right\rangle +
\left\langle \chi_{i}|\frac{\partial \chi_{j} }{\partial R_{x}}\right\rangle
\end{cases}
\end{equation}
If both $\chi_{i}$ and $\chi_{j}$ do not center on the given atom ($R_{x}$), then the derivatives
is zero; if either $\chi_{i}$ or $\chi_{j}$ center on the given atom, the derivatives will be like
the second equation; finally if both $\chi_{i}$ and $\chi_{j}$ center on the given atom, the 
derivatives will be expressed as in the third equation. Now let's go to see how to express the 
second equation in terms of the \ref{derivative_overlap_direct_int_eq:2}:
\begin{equation}
 \begin{split}
  \left\langle \frac{\partial \chi_{i} }{\partial R_{x}}|\chi_{j}\right\rangle &=
-l_{i}\langle\chi_{i}^{l_{i}-1m_{i}n_{i}}|\chi_{j}\rangle
+2\alpha_{i}\langle\chi_{i}^{l_{i}+1m_{i}n_{i}}|\chi_{j}\rangle
 \end{split}
\label{derivative_overlap_direct_int_eq:4}
\end{equation}
while in literature, the integral shown above usually expressed in the other way:
\begin{equation}
 \label{derivative_overlap_direct_int_eq:5}
\left\langle \frac{\partial \chi_{i} }{\partial R_{x}}|\chi_{j}\right\rangle
= -l_{i}\langle -1|0\rangle_{x} + 2\alpha_{i}\langle +1 | 0\rangle_{x}
\end{equation}
Here ``0'' indicates its original form (like for the $\chi_{j}$, it's not changed so we label it
as 0), and ``$+1$'' or ``$-1$'' means the angular momentum is ascending or descending. Finally,
the subscript of ``x'' means the changing of angular momentum is on the x. This expression is 
simpler than the \ref{derivative_overlap_direct_int_eq:4}. 

%%%%%%%%%%%%%%%%%%%%%%%%%%%%%%%%%%%%%%%%%%%%%%%%%%%%%%%%%%%%%%%%%%%%%%%%%%%%%
\subsection{Kinetic Energy Integral}
%
%
%
Kinetic energy integral is defined as:
\begin{equation}
 \label{kinetic_direct_int_eq:1}
\begin{split}
 T_{ij} &= \int \chi_{i}(-\frac{1}{2}\nabla^{2})\chi_{j} dr \\
&= -\frac{1}{2} \int 
x^{l_{i}}y^{m_{i}}z^{n_{i}}e^{-\alpha_{i} r_{i}^{2}}
\left( 
  \frac{\partial^{2}}{\partial x^{2}}
+ \frac{\partial^{2}}{\partial y^{2}}
+ \frac{\partial^{2}}{\partial z^{2}}\right)
x^{l_{j}}y^{m_{j}}z^{n_{j}}e^{-\alpha_{j} r_{j}^{2}}
\end{split}
\end{equation}
It's easy to see that the integral here can be restored to the overlap integrals. We can do it directly
by expanding each term inside \ref{kinetic_direct_int_eq:1}, however; as we know, the kinetic energy should
be symmetric (as physically it's Hermitian operator); hence usually we evaluate the kinetic energy 
in another way. Here below is a simple example:
\begin{equation}
 \begin{split}
 & -\frac{1}{2} \int 
x^{l_{i}}y^{m_{i}}z^{n_{i}}e^{-\alpha_{i} r_{i}^{2}} 
  \frac{\partial^{2}}{\partial x^{2}}
x^{l_{j}}y^{m_{j}}z^{n_{j}}e^{-\alpha_{j} r_{j}^{2}} dxdydz \\
&=
-\frac{1}{2}  \int \left\lbrace 
x^{l_{i}}y^{m_{i}}z^{n_{i}}e^{-\alpha_{i} r_{i}^{2}}\left. 
\frac{\partial (x^{l_{j}}y^{m_{j}}z^{n_{j}}e^{-\alpha_{j} r_{j}^{2}})}{\partial x}
\right\rbrace 
\right|^{+\infty}_{-\infty} dydz
\\
&+ \frac{1}{2}  \int 
\frac{\partial (x^{l_{i}}y^{m_{i}}z^{n_{i}}e^{-\alpha_{i} r_{i}^{2}})}{\partial x}
\frac{\partial (x^{l_{j}}y^{m_{j}}z^{n_{j}}e^{-\alpha_{j} r_{j}^{2}})}{\partial x} dxdydz \\
&= \frac{1}{2}  \int 
\frac{\partial (x^{l_{i}}y^{m_{i}}z^{n_{i}}e^{-\alpha_{i} r_{i}^{2}})}{\partial x}
\frac{\partial (x^{l_{j}}y^{m_{j}}z^{n_{j}}e^{-\alpha_{j} r_{j}^{2}})}{\partial x} dxdydz
 \end{split}
\label{kinetic_direct_int_eq:2}
\end{equation}
The \ref{kinetic_direct_int_eq:2} is done through integral by parts. Here we can see that the $T_{ij}$
becomes symmetric, and the $T_{ij}$ is always larger than zero between two Gaussian primitives.

Let's make the evaluation a bit of simpler. If we denote the integral in \ref{kinetic_direct_int_eq:2} 
as $I_{x}$, it's easy to see that :
\begin{equation}
 T_{ij} = I_{x} + I_{y} + I_{z}
\end{equation}
for $I_{x}$ we have that:
\begin{equation}
 \label{kinetic_direct_int_eq:3}
\begin{split}
 I_{x} &=\frac{1}{2}   \int 
          \left[ l_{i}x^{l_{i}-1} - 2\alpha_{i}x^{l_{i}+1}\right]
          y^{m_{i}}z^{n_{i}}e^{-\alpha_{i} r_{i}^{2}} \\
       &\times
          \left[ l_{j}x^{l_{j}-1} - 2\alpha_{j}x^{l_{j}+1}\right]
          y^{m_{j}}z^{n_{j}}e^{-\alpha_{j} r_{j}^{2}} dr \\
       &= \frac{1}{2}\Big(
          l_{i}l_{j}\langle -1|-1\rangle_{x} 
        - 2\alpha_{i}l_{j}\langle +1|-1\rangle_{x}    \\
       &- 2\alpha_{j}l_{i}\langle -1|+1\rangle_{x}
        + 4\alpha_{i}\alpha_{j}\langle +1|+1\rangle_{x} 
          \Big) 
\end{split}
\end{equation}

Similarly for the $I_{y}$ and $I_{z}$ we have:
\begin{equation}
\label{kinetic_direct_int_eq:4}
\begin{split}
I_{y} &=  \frac{1}{2}\Big(
          m_{i}m_{j}\langle -1|-1\rangle_{y} 
        - 2\alpha_{i}m_{j}\langle +1|-1\rangle_{y}   \\ 
       &- 2\alpha_{j}m_{i}\langle -1|+1\rangle_{y}
        + 4\alpha_{i}\alpha_{j}\langle +1|+1\rangle_{y} 
          \Big) \\
I_{z} &=  \frac{1}{2}\Big(
          n_{i}n_{j}\langle -1|-1\rangle_{z} 
        - 2\alpha_{i}n_{j}\langle +1|-1\rangle_{z}   \\ 
       &- 2\alpha_{j}n_{i}\langle -1|+1\rangle_{z}
        + 4\alpha_{i}\alpha_{j}\langle +1|+1\rangle_{z} 
          \Big) 
\end{split}
\end{equation}
Now the kinetic energy integral is done.

%%%%%%%%%%%%%%%%%%%%%%%%%%%%%%%%%%%%%%%%%%%%%%%%%%%%%%%%%%%%%%%%%%%%%%%%%%%%%%%%%%
\subsection{Nuclear Attraction Integral}
\label{direct_NAI_derivation}
%
%
%
The common nuclear attraction integral between two arbitrary Gaussian primitives can be expressed
as:
\begin{equation}
 \begin{split}
  V &= \int \chi_{i}(r)\frac{1}{r_{C}}\chi_{j}(r) dr \\
&= \int x^{l_{A}}_{A}y^{m_{A}}_{A}z^{n_{A}}_{A}e^{-\alpha r_{A}^{2}}
        \frac{1}{r_{C}}
        x^{l_{B}}_{B}y^{m_{B}}_{B}z^{n_{B}}_{B}e^{-\beta  r_{B}^{2}} dr
 \end{split}
\label{nuclear_attraction_direct_int_eq:1}
\end{equation}
$r_{A}, r_{B}$ and $r_{c}$ denotes the distance between electron position and the given
nuclear (A, B or C).

This difficulty to solve this integral is for the $1/r_{C}$. However, we can use standard
Laplace transformation to convert it into another form:
\begin{equation}
\label{nuclear_attraction_direct_int_eq:2}
 \frac{1}{r_{C}} = \frac{1}{\sqrt{\pi}}\int^{\infty}_{0} e^{-sr_{C}^{2}} s^{-\frac{1}{2}} ds
\end{equation}
So that we have:
\begin{equation}
 \begin{split}
V &= \frac{1}{\sqrt{\pi}}\int^{\infty}_{0} e^{-sr_{C}^{2}} s^{-\frac{1}{2}} ds 
          \int x^{l_{A}}_{A}y^{m_{A}}_{A}z^{n_{A}}_{A}e^{-\alpha r_{A}^{2}}
          x^{l_{B}}_{B}y^{m_{B}}_{B}z^{n_{B}}_{B}e^{-\beta  r_{B}^{2}} dr \\
       &= \frac{1}{\sqrt{\pi}}\int^{\infty}_{0}  s^{-\frac{1}{2}} ds  
          \int e^{-sr_{C}^{2}} 
          x^{l_{A}}_{A}y^{m_{A}}_{A}z^{n_{A}}_{A}e^{-\alpha r_{A}^{2}}
          x^{l_{B}}_{B}y^{m_{B}}_{B}z^{n_{B}}_{B}e^{-\beta  r_{B}^{2}} dr
 \end{split}
\label{nuclear_attraction_direct_int_eq:3}
\end{equation}
Basically, this is equivalent to three-center overlap integral. So now let's think about
how to derive it.

Let's begin from the shell pair. In the \ref{nuclear_attraction_direct_int_eq:3} we could combine the 
A and B center together. According to the results in the 
\ref{gaussian_product_rule_eq:19} and \ref{gaussian_product_rule_eq:20}, we can express 
the integral above as:
\begin{multline}
  V = e^{-\frac{\alpha\beta}{\alpha+\beta}\overline{AB}^{2}} 
\sum_{i=0}^{l_{A}+l_{B}}f_{i}(l_{A},l_{B},\overline{PA}_{x},\overline{PB}_{x})
\sum_{j=0}^{m_{A}+m_{B}}f_{j}(m_{A},m_{B},\overline{PA}_{y},\overline{PB}_{y}) \times \\
\sum_{k=0}^{n_{A}+n_{B}}f_{k}(n_{A},n_{B},\overline{PA}_{z},\overline{PB}_{z}) 
   \frac{1}{\sqrt{\pi}} 
   \int^{\infty}_{0} s^{-\frac{1}{2}} ds 
   \int x_{P}^{i}y_{P}^{j}z_{P}^{k} e^{-(\alpha+\beta)r_{P}^{2}}e^{-sr_{C}^{2}} dr
\label{nuclear_attraction_direct_int_eq:6}
\end{multline}
So the problem is left to solve the integral inside the \ref{nuclear_attraction_direct_int_eq:6}.

Before proceeding on solving the integrals, it's very interesting to note that the function of 
$x_{P}^{i}y_{P}^{j}z_{P}^{k} e^{-(\alpha+\beta)r_{P}^{2}}e^{-sr_{C}^{2}}$ will not be zero if 
$i$, $j$ and $k$ is even number. So this is different from the overlap integral situation. 
The reason is because of $r_{C}$. For example, 
$ x_{P}^{i} e^{-(\alpha+\beta)x_{P}^{2}}e^{-sx_{C}^{2}} $ is not 
even or odd function as $i$ is even or odd number, that's all because of the $x_{C}$ in the 
integral.

Firstly let's concentrate on the integral over $r$, which is:
\begin{equation}
  \int x_{P}^{i}y_{P}^{j}z_{P}^{k} e^{-(\alpha+\beta)r_{P}^{2}}e^{-sr_{C}^{2}} dr
\end{equation}
According to the Gaussian primitive product theorem, this term can be finally expressed 
into:
\begin{equation}
\begin{split}
&\int x_{P}^{i}y_{P}^{j}z_{P}^{k} e^{-(\alpha+\beta)r_{P}^{2}}e^{-sr_{C}^{2}} dr  \\
&=
e^{-\frac{(\alpha+\beta)s}{\alpha+\beta + s} \overline{PC}^{2}}\times
\int(x_{Q}+\overline{QP}_{x})^{i}(y_{Q}+\overline{QP}_{y})^{j}(z_{Q}+\overline{QP}_{z})^{k}
e^{-(\alpha+\beta + s)r_{Q}^{2}} dr \\ 
&=
e^{-\frac{(\alpha+\beta)s}{\alpha+\beta + s} \overline{PC}^{2}}\times
\sum_{i^{'}=0}^{i}\sum_{j^{'}=0}^{j}\sum_{k^{'}=0}^{k}
\binom{i}{i^{'}}\binom{j}{j^{'}}\binom{k}{k^{'}}
\overline{QP}_{x}^{i-i^{'}}\overline{QP}_{y}^{j-j^{'}}\overline{QP}_{z}^{k-k^{'}} \\
&\int x_{Q}^{i^{'}}y_{Q}^{j^{'}}z_{Q}^{k^{'}} e^{-(\alpha+\beta+s)r_{Q}^{2}} dr
\end{split}
 \label{nuclear_attraction_direct_int_eq:7}
\end{equation}
Here Q is determined by the P point and C point.

For the expression in \ref{nuclear_attraction_direct_int_eq:7}, we note that 
$\int x_{Q}^{i^{'}}y_{Q}^{j^{'}}z_{Q}^{k^{'}} e^{-(\alpha+\beta+s)r_{Q}^{2}} dr$ is even function; 
which requires that the $i^{'}$, $j^{'}$ and $k^{'}$ are all even numbers. Therefore the integral
in the \ref{nuclear_attraction_direct_int_eq:7} could be transformed into:
\begin{equation}
\begin{split}
&\int x_{P}^{i}y_{P}^{j}z_{P}^{k} e^{-(\alpha+\beta)r_{P}^{2}}e^{-sr_{C}^{2}} dr  \\
&=
e^{-\frac{(\alpha+\beta)s}{\alpha+\beta + s} \overline{PC}^{2}}\times
\sum_{i^{'}=0}^{\left[ \frac{i}{2}\right] }
\sum_{j^{'}=0}^{\left[ \frac{j}{2}\right]}
\sum_{k^{'}=0}^{\left[ \frac{k}{2}\right]}
\binom{i}{2i^{'}}\binom{j}{2j^{'}}\binom{k}{2k^{'}}
\overline{QP}_{x}^{i-2i^{'}}\overline{QP}_{y}^{j-2j^{'}}\overline{QP}_{z}^{k-2k^{'}} \\
&\int x_{Q}^{2i^{'}}y_{Q}^{2j^{'}}z_{Q}^{2k^{'}} e^{-(\alpha+\beta+s)r_{Q}^{2}} dr
\end{split}
 \label{nuclear_attraction_direct_int_eq:7_1}
\end{equation}

The $\overline{QP}$ is also connected with variable $s$, according to the 
\ref{gaussian_product_rule_eq:8}, it has the form that:
\begin{equation}
 \overrightarrow{QP} = \frac{s}{\alpha+\beta+s}\overrightarrow{CP}
\end{equation}
Therefore the \ref{nuclear_attraction_direct_int_eq:7_1} could be further transformed into:
\begin{equation}
\begin{split}
&\int x_{P}^{i}y_{P}^{j}z_{P}^{k} e^{-(\alpha+\beta)r_{P}^{2}}e^{-sr_{C}^{2}} dr  \\
&=
e^{-\frac{(\alpha+\beta)s}{\alpha+\beta + s} \overline{PC}^{2}}\times
\sum_{i^{'}=0}^{\left[ \frac{i}{2}\right] }
\sum_{j^{'}=0}^{\left[ \frac{j}{2}\right]}
\sum_{k^{'}=0}^{\left[ \frac{k}{2}\right]}
\left( \frac{s}{\alpha+\beta+s}\right)^{i+j+k-2(i^{'}+j^{'}+k^{'})} \\
&\binom{i}{2i^{'}}\binom{j}{2j^{'}}\binom{k}{2k^{'}}
\overline{CP}_{x}^{i-2i^{'}}\overline{CP}_{y}^{j-2j^{'}}\overline{CP}_{z}^{k-2k^{'}} \\
&\int x_{Q}^{2i^{'}}y_{Q}^{2j^{'}}z_{Q}^{2k^{'}} e^{-(\alpha+\beta+s)r_{Q}^{2}} dr
\end{split}
 \label{nuclear_attraction_direct_int_eq:7_2}
\end{equation}

The integration result for the \ref{nuclear_attraction_direct_int_eq:7_2} can be expressed as according 
to \ref{int_sec2_eq:8}:
\begin{equation}
\begin{split}
 &\int x_{P}^{i}y_{P}^{j}z_{P}^{k} e^{-(\alpha+\beta)r_{P}^{2}}e^{-sr_{C}^{2}} dr  \\
&=
e^{-\frac{(\alpha+\beta)s}{\alpha+\beta + s} \overline{PC}^{2}}\times
\sum_{i^{'}=0}^{\left[ \frac{i}{2}\right]}
\sum_{j^{'}=0}^{\left[ \frac{j}{2}\right]}
\sum_{k^{'}=0}^{\left[ \frac{k}{2}\right]}
\left( \frac{s}{\alpha+\beta+s}\right)^{i+j+k-2(i^{'}+j^{'}+k^{'})} \\
&\binom{i}{2i^{'}}\binom{j}{2j^{'}}\binom{k}{2k^{'}}
\overline{CP}_{x}^{i-2i^{'}}\overline{CP}_{y}^{j-2j^{'}}\overline{CP}_{z}^{k-2k^{'}} \\
&\left( \frac{\pi}{\alpha+\beta+s}\right)^{\frac{3}{2}}\times
\frac{(2i^{'}-1)!!(2j^{'}-1)!!(2k^{'}-1)!!}
{(2(\alpha+\beta+s))^{i^{'}+j^{'}+k^{'}}} 
\end{split}
 \label{nuclear_attraction_direct_int_eq:8}
\end{equation}

Now let's consider the outer integral which is over $s$:
\begin{equation}
\begin{split}
&\frac{1}{\sqrt{\pi}} 
 \int^{\infty}_{0} s^{-\frac{1}{2}} ds 
 \int x_{P}^{i}y_{P}^{j}z_{P}^{k} e^{-(\alpha+\beta)r_{P}^{2}}e^{-sr_{C}^{2}} dr \\
&= 
\frac{1}{\sqrt{\pi}}
\sum_{i^{'}=0}^{\left[ \frac{i}{2}\right]}
\sum_{j^{'}=0}^{\left[ \frac{j}{2}\right]}
\sum_{k^{'}=0}^{\left[ \frac{k}{2}\right]} 
\pi^{\frac{3}{2}}\binom{i}{2i^{'}}\binom{j}{2j^{'}}\binom{k}{2k^{'}}
\overline{CP}_{x}^{i-2i^{'}}\overline{CP}_{y}^{j-2j^{'}}\overline{CP}_{z}^{k-2k^{'}} \\
&\frac{(2i^{'}-1)!!(2j^{'}-1)!!(2k^{'}-1)!!}{2^{i^{'}+j^{'}+k^{'}}} 
\int^{\infty}_{0} ds s^{-\frac{1}{2}}
e^{-\frac{(\alpha+\beta)s}{\alpha+\beta + s} \overline{PC}^{2}} \\
&(\alpha+\beta+s)^{-\left( i^{'}+j^{'}+k^{'}+\frac{3}{2}\right) } 
\left( \frac{s}{\alpha+\beta+s}\right)^{i+j+k-2(i^{'}+j^{'}+k^{'})}
\end{split}
 \label{nuclear_attraction_direct_int_eq:11}
\end{equation}

Let's suggest to make variable transformation, that:
\begin{equation}
 \label{nuclear_attraction_direct_int_eq:12}
\frac{s}{\alpha+\beta + s} = t^{2}
\end{equation}
So that as $s$ goes from $0$ to $\infty$, then $t$ could be from $0$ to $1$.

If we differentiate it, we could get:
\begin{equation}
 \label{nuclear_attraction_direct_int_eq:13}
\frac{\alpha+\beta}{(\alpha+\beta + s)^{2}} ds = 2tdt \Rightarrow ds = 
2s^{\frac{1}{2}}\frac{(\alpha+\beta + s)^{\frac{3}{2}}}{\alpha+\beta} dt
\end{equation}

Now let's bring the \ref{nuclear_attraction_direct_int_eq:13} into the 
\ref{nuclear_attraction_direct_int_eq:11}. Furthermore, we write:
\begin{equation}
 \begin{split}
  i+j+k             &= L \\
  i^{'}+j^{'}+k^{'} &= L^{'}
 \end{split}
\end{equation}

Therefore we have:
\begin{equation}
 \label{nuclear_attraction_direct_int_eq:14}
\begin{split}
&\frac{1}{\sqrt{\pi}} 
 \int^{\infty}_{0} s^{-\frac{1}{2}} ds 
 \int x_{P}^{i}y_{P}^{j}z_{P}^{k} e^{-(\alpha+\beta)r_{P}^{2}}e^{-sr_{C}^{2}} dr \\
&= 
\frac{2\pi}{\alpha+\beta}
\sum_{i^{'}=0}^{\left[ \frac{i}{2}\right]}
\sum_{j^{'}=0}^{\left[ \frac{j}{2}\right]}
\sum_{k^{'}=0}^{\left[ \frac{k}{2}\right]} 
\binom{i}{2i^{'}}\binom{j}{2j^{'}}\binom{k}{2k^{'}}
\overline{CP}_{x}^{i-2i^{'}}\overline{CP}_{y}^{j-2j^{'}}\overline{CP}_{z}^{k-2k^{'}} \\
&\frac{(2i^{'}-1)!!(2j^{'}-1)!!(2k^{'}-1)!!}{2^{i^{'}+j^{'}+k^{'}}} 
\int^{1}_{0} dt
e^{-(\alpha+\beta)(\overline{PC}t)^{2}}
(\alpha+\beta+s)^{-L^{'} }t^{2(L-2L^{'})} 
\end{split}
\end{equation}

Now let's do further transformation to the $\alpha+\beta+s$, we can see that:
\begin{equation}
 \label{nuclear_attraction_direct_int_eq:15}
\alpha+\beta+s = \frac{\alpha+\beta}{1-t^{2}} \Rightarrow (\alpha+\beta+s)^{-1} = 
\frac{1-t^{2}}{\alpha+\beta}
\end{equation}

Therefore the nuclear integral in the \ref{nuclear_attraction_direct_int_eq:14}
further simplifies as:
\begin{equation}
\begin{split}
&\frac{1}{\sqrt{\pi}} 
 \int^{\infty}_{0} s^{-\frac{1}{2}} ds 
 \int x_{P}^{i}y_{P}^{j}z_{P}^{k} e^{-(\alpha+\beta)r_{P}^{2}}e^{-sr_{C}^{2}} dr \\
&= 
\sum_{i^{'}=0}^{\left[ \frac{i}{2}\right]}
\sum_{j^{'}=0}^{\left[ \frac{j}{2}\right]}
\sum_{k^{'}=0}^{\left[ \frac{k}{2}\right]} 
\frac{2\pi}{(\alpha+\beta)^{L^{'}+1}}
\binom{i}{2i^{'}}\binom{j}{2j^{'}}\binom{k}{2k^{'}}
\overline{CP}_{x}^{i-2i^{'}}\overline{CP}_{y}^{j-2j^{'}}\overline{CP}_{z}^{k-2k^{'}} \\
&\frac{(2i^{'}-1)!!(2j^{'}-1)!!(2k^{'}-1)!!}{2^{i^{'}+j^{'}+k^{'}}} 
\int^{1}_{0} dt
e^{-(\alpha+\beta)(\overline{PC}t)^{2}}
(1-t^{2})^{L^{'} }t^{2(L-2L^{'})} 
\end{split}
\label{nuclear_attraction_direct_int_eq:16}
\end{equation}

Up to this point, we note that the \ref{nuclear_attraction_direct_int_eq:16} establishes for both
$P = C$ and $P \neq C$. However, as the derivation going further, we may have to consider them
separately.

If $P \neq C$, by dropping the $\overline{PC}$, that means we set $v = \overline{PC}t$; it yields:
\begin{equation}
 \begin{split}
&\int^{1}_{0} dt
e^{-(\alpha+\beta)(\overline{PC}t)^{2}}
(1-t^{2})^{L^{'} }t^{2(L-2L^{'})} \\  
&=\frac{1}{\overline{PC}}\int^{\overline{PC}}_{0}dv e^{-(\alpha+\beta)v^{2}} 
\frac{1}{\overline{PC}^{2L^{'}}}\frac{1}{\overline{PC}^{2(L-2L^{'})}}
(\overline{PC}^{2}-v^{2})^{L^{'}}v^{2(L-2L^{'})} \\
&=\frac{1}{\overline{PC}^{2(L-L^{'})+1}}\int^{\overline{PC}}_{0}dv e^{-(\alpha+\beta)v^{2}} 
(\overline{PC}^{2}-v^{2})^{L^{'}}v^{2(L-2L^{'})}
 \end{split}
\label{nuclear_attraction_direct_int_eq:17}
\end{equation}
Then the integral can be finally transformed as:
\begin{equation}
 \label{nuclear_attraction_direct_int_eq:14}
\begin{split}
&\frac{1}{\sqrt{\pi}} 
 \int^{\infty}_{0} s^{-\frac{1}{2}} ds 
 \int x_{P}^{i}y_{P}^{j}z_{P}^{k} e^{-(\alpha+\beta)r_{P}^{2}}e^{-sr_{C}^{2}} dr \\
&= 
\sum_{i^{'}=0}^{\left[ \frac{i}{2}\right]}
\sum_{j^{'}=0}^{\left[ \frac{j}{2}\right]}
\sum_{k^{'}=0}^{\left[ \frac{k}{2}\right]} 
\frac{2\pi}{(\alpha+\beta)^{L^{'}+1}}
\binom{i}{2i^{'}}\binom{j}{2j^{'}}\binom{k}{2k^{'}}
\overline{CP}_{x}^{i-2i^{'}}\overline{CP}_{y}^{j-2j^{'}}\overline{CP}_{z}^{k-2k^{'}} \\
&\frac{(2i^{'}-1)!!(2j^{'}-1)!!(2k^{'}-1)!!}{2^{i^{'}+j^{'}+k^{'}}} 
\frac{1}{\overline{PC}^{2(L-L^{'})+1}} \\
&\int^{\overline{PC}}_{0}dv e^{-(\alpha+\beta)v^{2}} 
(\overline{PC}^{2}-v^{2})^{L^{'}}v^{2(L-2L^{'})} 
\end{split}
\label{nuclear_attraction_direct_int_eq:18}
\end{equation}

Finally, let's analyze the integral given in \ref{nuclear_attraction_direct_int_eq:18}, such integral
can be calculated on the base of:
\begin{equation}
\Gamma(x, a, m) = \int^{x}_{0} e^{-at^{2}}t^{2m} dt
 \label{nuclear_attraction_direct_int_eq:19}
\end{equation}
Where $a$ and $m$ are both positive real numbers. $a$ is $(\alpha+\beta)$, and 
$m$ is determined by the binomial expansion.

This integral could be transformed into the  ``incomplete gamma function'' by setting 
$t^{2}=w$:
\begin{align}
&\Gamma(x, a, m) = \int^{x}_{0} e^{-at^{2}}t^{2m} dt = 
 \frac{1}{2}\int^{x^{2}}_{0} e^{-aw}w^{m-\frac{1}{2}} dw \nonumber \\
&=
\frac{1}{2a^{m+\frac{1}{2}}}\int^{ax^{2}}_{0} 
e^{-y}y^{m-\frac{1}{2}} dy
\label{nuclear_attraction_direct_int_eq:20}
\end{align}
Where the fundamental integral of $\int^{ax^{2}}_{0} e^{-y}y^{m-\frac{1}{2}} dy$
could be got from standard math library. 

For the case that $P = C$, we can see that in the \ref{nuclear_attraction_direct_int_eq:16}
The CPx, CPy and CPz all become zero; it's only $2i^{'}=i$, $2j^{'}=j$ and $2k^{'}=k$
that the integral is not zero. Hence we have:
\begin{equation}
\begin{split}
&\frac{1}{\sqrt{\pi}} 
 \int^{\infty}_{0} s^{-\frac{1}{2}} ds 
 \int x_{P}^{i}y_{P}^{j}z_{P}^{k} e^{-(\alpha+\beta)r_{P}^{2}}e^{-sr_{C}^{2}} dr \\
&= 
\sum_{i^{'}=0}^{\left[ \frac{i}{2}\right]}
\sum_{j^{'}=0}^{\left[ \frac{j}{2}\right]}
\sum_{k^{'}=0}^{\left[ \frac{k}{2}\right]} 
\frac{2\pi}{(\alpha+\beta)^{L^{'}+1}}
\binom{i}{2i^{'}}\binom{j}{2j^{'}}\binom{k}{2k^{'}}
\overline{CP}_{x}^{i-2i^{'}}\overline{CP}_{y}^{j-2j^{'}}\overline{CP}_{z}^{k-2k^{'}} \\
&\frac{(2i^{'}-1)!!(2j^{'}-1)!!(2k^{'}-1)!!}{2^{i^{'}+j^{'}+k^{'}}} 
\int^{1}_{0} dt
e^{-(\alpha+\beta)(\overline{PC}t)^{2}}
(1-t^{2})^{L^{'} }t^{2(L-2L^{'})} \underrightarrow{ P = C} \\
&=\frac{2\pi}{(\alpha+\beta)^{L^{'}+1}}\frac{(i-1)!!(j-1)!!(k-1)!!}{2^{L^{'}}} 
\int^{1}_{0} dt
(1-t^{2})^{L^{'}} \\
&=\frac{2\pi}{(\alpha+\beta)^{L^{'}+1}}\frac{(i-1)!!(j-1)!!(k-1)!!}{2^{L^{'}}} 
\sum_{m=0}^{L^{'}}\binom{L^{'}}{m}(-1)^{m}\frac{1}{2m+1}
\end{split}
\label{nuclear_attraction_direct_int_eq:21}
\end{equation}

