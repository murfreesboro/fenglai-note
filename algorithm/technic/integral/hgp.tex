% 
% firstly set up on Jan 2012
%
%
%
%
%%%%%%%%%%%%%%%%%%%%%%%%%%%%%%%%%%%%%%%%%%%%%%%%%%%%%%%%%%%%%%%%%%%%%%%%%%%%%%%%
\section{HGP Method}
%
%
%
%
HGP method\cite{HGP} is a revised method based on OS\cite{OS1986} framework.
It's central idea is to employ the horizontal recurrence relation so that 
to take the contraction of coefficients into consideration. 

Before we step into the discussion, let's define some expressions. The general
un-contracted integral, for example; the electron repulsion integral is
expressed as:
\begin{equation}
 (ab|cd) = \int dr \int dr^{'} \chi_{a}(r)\chi_{b}(r)\frac{1}{|r-r^{'}|}
\chi_{c}(r^{'})\chi_{d}(r^{'})
\label{HGP:1}
\end{equation}
This is same with the definition in \ref{OS_ERI_eq:1}. As we have noted in the
previous section that the integral are actually based on shell so we only need
to specify the shells of a, b, c, d. Therefore, the integral in the \ref{HGP:1}
is also called ``shell quartet''.

For the contracted integrals, we express it as:
\begin{align}
 \label{HGP:2}
[ij|kl] &=
\sum_{\mu}^{K_{\mu}}\sum_{\nu}^{K_{\nu}}\sum_{\lambda}^{K_{\lambda}}
\sum_{\eta}^{K_{\eta}}c_{\mu i}c_{\nu j}c_{\lambda k}c_{\eta l} \nonumber \\
&\int dr \int dr^{'} \chi_{\mu}(r)\chi_{\nu}(r)\frac{1}{|r-r^{'}|}
\chi_{\lambda}(r^{'})\chi_{\eta}(r^{'})
\end{align}
The Greek letters are the indices over AO space, the i,j,k,l are indices over
the basis set function space. $K$ denotes the contraction degree.
 

%%%%%%%%%%%%%%%%%%%%%%%%%%%%%%%%%%%%%%%%%%%%%%%%%%%%%%%%%%%%%%%%%%%%%%%%%%%%%%%%
\subsection{Horizontal Recurrence Relation}
In the OS framework, we already derived the recursive relation for the electron
repulsion integral:
\begin{equation}
 \begin{split}
 ((a+\iota_{i})b|cd)^{(m)} &= (P_{i} - A_{i})(ab|cd)^{(m)} +
\left(W_{i} -P_{i}\right)(ab|cd)^{(m+1)} \\
&+\frac{N_{i}(A)}{2\epsilon}\left(((a-\iota_{i})b|cd)^{(m)}-\frac{\rho}{
\epsilon }((a-\iota_{i})b|cd)^{(m+1)}\right)  \\
&+\frac{N_{i}(B)}{2\epsilon}\left((a(b-\iota_{i})|cd)^{(m)}-\frac{\rho}{
\epsilon }(a(b-\iota_{i})|cd)^{(m+1)}\right)  \\
&+\left(\frac{N_{i}(C)}{2}\right)\frac{1}{\epsilon+\eta}
(ab|(c-\iota_{i})d)^{(m+1)} \\
&+\left(\frac{N_{i}(D)}{2}\right)\frac{1}{\epsilon+\eta}
(ab|c(d-\iota_{i}))^{(m+1)} 
 \end{split}
\label{HGP_ERI_eq:1}
\end{equation}
Here we noted that in such relation, the shell $a$ and shell $b$ are in some
``symmetrical'' position thus if we consider the expression of
$(a(b+\iota_{i})|cd)^{(m)}$, it's:
\begin{equation}
\begin{split}
 (a(b+\iota_{i})|cd)^{(m)} &= (P_{i} - B_{i})(ab|cd)^{(m)} +
\left(W_{i} -P_{i}\right)(ab|cd)^{(m+1)} \\
&+\frac{N_{i}(A)}{2\epsilon}\left(((a-\iota_{i})b|cd)^{(m)}-\frac{\rho}{
\epsilon }((a-\iota_{i})b|cd)^{(m+1)}\right)  \\
&+\frac{N_{i}(B)}{2\epsilon}\left((a(b-\iota_{i})|cd)^{(m)}-\frac{\rho}{
\epsilon }(a(b-\iota_{i})|cd)^{(m+1)}\right)  \\
&+\left(\frac{N_{i}(C)}{2}\right)\frac{1}{\epsilon+\eta}
(ab|(c-\iota_{i})d)^{(m+1)} \\
&+\left(\frac{N_{i}(D)}{2}\right)\frac{1}{\epsilon+\eta}
(ab|c(d-\iota_{i}))^{(m+1)} 
 \end{split}
\label{HGP_ERI_eq:2} 
\end{equation}
We can see that most of the terms in \ref{HGP_ERI_eq:2} are same with
\ref{HGP_ERI_eq:1}. Then if we subtract the two terms, it gives:
\begin{equation}
 \label{HGP_ERI_HRR}
 (a(b+\iota_{i})|cd)^{(m)} = ((a+\iota_{i})b|cd)^{(m)} + 
(A_{i} - B_{i})(ab|cd)^{(m)}
\end{equation}
The new recursive relation in \ref{HGP_ERI_HRR} is called ``horizontal
recurrence relation''(HRR). The relation in \ref{HGP_ERI_eq:1} and 
\ref{HGP_ERI_eq:2} are called ``vertical recurrence relation''(VRR). 
Here it has an important feature, that for the \ref{HGP_ERI_HRR} 
we could contract it so that to transform it into:
\begin{equation}
 \label{HGP_ERI_eq:3} 
 [a(b+\iota_{i})|cd]^{(m)} = [(a+\iota_{i})b|cd]^{(m)} + 
(A_{i} - B_{i})[ab|cd]^{(m)}
\end{equation}
Therefore, the advantages for the HRR relation is that it reduces the
calculation inside the loop of primitive function pairs. For example, for the
$[pp|pp]$ primitives, now we just need to calculate the $[ds|ds]$. Therefore,
much less calculation is done with VRR.

What's more, we note that the HRR relation in \ref{HGP_ERI_HRR} is also applied 
to the other OS recursive relations. For example, for the three center overlap
integral whose relationship is:
\begin{equation}
 \begin{split}
 (a+\iota_{i}|b|c) 
&= (G_{Ai} - A_{i})(a|b|c) + 
N_{i}(A)\left(\frac{1}{2(\alpha+\beta+\gamma)}\right)(a-\iota_{i}|b|c) \\
&+ 
N_{i}(B)\left(\frac{1}{2(\alpha+\beta+\gamma)}\right)(a|b-\iota_{i}|c) \\
&+
N_{i}(C)\left(\frac{1}{2(\alpha+\beta+\gamma)}\right)(a|b|c-\iota_{i}) 
 \end{split}
\end{equation}
It's easy to see that the corresponding HRR is:
\begin{equation}
\label{HGP_ERI_eq:4}
 (a|b+\iota_{i}|c) = (a+\iota_{i}|b|c) + 
(A_{i} - B_{i})(a|b|c)
\end{equation}
Therefore, the same formula retains.


%%%%%%%%%%%%%%%%%%%%%%%%%%%%%%%%%%%%%%%%%%%%%%%%%%%%%%%%%%%%%%%%%%%%%%%%%%%%%%%%
\subsection{HGP Algorithm to Evaluate Integrals}
%
%
%
Based on the HRR, we have some new algorithm to evaluate the integrals(the
real calculation will be in reverse order):
\begin{description}
 \item[HRR step:]
 For a general shell quartet of $[ab|cd]$, through the HRR it could be
 evidently reduced to the integrals of $[e0|f0]$; where for the $e$ the angular
 momentum would be the sum of $a$ and $b$, for the $f$ it would be sum of $c$
 and $d$.
 \item[CONTRACTION step:]
 Perform the contraction from $(e0|f0)$ to $[e0|f0]$.
 \item[VRR step:]
 Calculate the primitive integral of $(e0|f0)$ in VRR. Each of such integrals
 will be reduced to the $(ss|ss)^{(m)}$, while for $(ab|cd)$ $m$ ranges from
 0 to $a+b+c+d$. 
 \end{description}

How to evaluate the HGP algorithm? By using the HRR, a lot of calculations
within the loop of primitives pairs now could be moved out of the contraction
loop (this loop is scaled with $K^{4}$). However, in HRR the angular momentum
increases in the right hand side so it implies that more shell quartets
are needed to be generated; which is sacrifice in terms of the balance. 

Now let's give an example to see how the HGP works. Suggest that for the 
shell quartet of $[FF|FF]$, by using the HRR we can have:
\begin{enumerate}
 \item $[FF|FF]$ is reduced to $[GD|FF]$ and $[FD|FF]$;
 \item $[GD|FF]$ is reduced to $[HP|FF]$ and $[GP|FF]$;
 \item $[FD|FF]$ is reduced to $[GP|FF]$ and $[FP|FF]$;
 \item $[HP|FF]$ is reduced to $[IS|FF]$ and $[HS|FF]$;
 \item $[GP|FF]$ is reduced to $[HS|FF]$ and $[GS|FF]$;
 \item $[FP|FF]$ is reduced to $[GS|FF]$ and $[FS|FF]$
\end{enumerate}
Therefore, there are only four elementary shell quartets, namely the 
$[FS|FF]$, $[GS|FF]$, $[HS|FF]$ and $[IS|FF]$. On the other hand, similar
treatment to the ket side will yields $[FF|FS]$, $[FF|GS]$, $[FF|HS]$ and
$[FF|SI]$ therefore only their combinations are needed:
\begin{enumerate}
 \item $[FS|FS]$, $[FS|GS]$, $[FS|HS]$, $[FS|IS]$
 \item $[GS|FS]$, $[GS|GS]$, $[GS|HS]$, $[GS|IS]$
 \item $[HS|FS]$, $[HS|GS]$, $[HS|HS]$, $[HS|IS]$
 \item $[IS|FS]$, $[IS|GS]$, $[IS|HS]$, $[IS|IS]$
\end{enumerate}
Then in the VRR step we could calculate all of these integrals.


