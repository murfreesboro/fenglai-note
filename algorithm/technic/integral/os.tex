% 
% firstly set up on Jan 2012
%
% fully derived the ERI in OS framework
% derived the overlap, kinetic and nuclear integral in OS framework
%
%
%%%%%%%%%%%%%%%%%%%%%%%%%%%%%%%%%%%%%%%%%%%%%%%%%%%%%%%%%%%%%%%%%%%%%%%%%%%%%%%%
\section{OS Method}
%
%
%
%
OS method is based on two cognitions for the integral of Gaussian primitive
functions. The first cognition is that all of integral could be reduced into the
form of three body overlap integrals. The second cognition is based on
derivatives of Gaussian primitive function:
\begin{equation}
 \label{OS_general_int_eq:1}
\frac{\partial \chi}{\partial R_{x}} = \frac{\partial
(x^{l}y^{m}z^{n}e^{-\alpha r^{2}})}
{\partial R_{x}} =  -lx^{l-1}y^{m}z^{n}e^{-\alpha r^{2}} + 2\alpha
x^{l+1}y^{m}z^{n}e^{-\alpha r^{2}}
\end{equation}
Here the $x$ is expressed as:
\begin{equation}
 x = x_{e} - R_{x} 
\end{equation}
In such relation, it's clear that the $\chi(l,m,n)$, its derivatives and the
higher angular momentum one $\chi(l+1,m,n)$ are connected with each other so
that it provides an potential opportunity to link the $\chi(l+1,m,n)$
and $\chi(l,m,n)$ together through $\chi$'s derivatives.

Now following the definition in the OS method, we can express such 
relation as:
\begin{equation}
 \label{OS_general_int_eq:2}
 \frac{\partial \chi(r,\alpha,l,R)}{\partial R_{i}} = 
2\alpha\chi(r,\alpha,l+\iota_{i},R) - N_{i}(l)\chi(r,\alpha,l-\iota_{i},R)
\end{equation}
as $i = x, y, z$. Original $\chi$ is $\chi(r,\alpha,l,R)$, so $r$ is the
electron coordinate, $R$ is the nuclear coordinate, $\alpha$ is the exponent
and $l$ is the angular momentum (actually it's a three dimensional vector).
$\iota$ characterizes the arising or descending of the angular momentum,
it's actually Kronecker symbol:
\begin{equation}
 \iota_{i} = (\delta_{ix}, \delta_{iy}, \delta_{iz})
\label{OS_general_int_eq:3}
\end{equation}
Similarly, $N_{i}(l)$ is:
\begin{equation}
N_{i}(l) = 
\begin{cases}
 l_{x} & i = x \\
 l_{y} & i = y \\
 l_{z} & i = z 
\end{cases}
 \label{OS_general_int_eq:4}
\end{equation}

By moving the exponent to the left side, the \ref{OS_general_int_eq:2}
could be further expressed into:
\begin{equation}
 \label{OS_general_int_eq:5}
\chi(r,\alpha,l+\iota_{i},R) =  
\frac{1}{2\alpha}\frac{\partial \chi(r,\alpha,l,R)}{\partial R_{i}}
+ \frac{N_{i}(l)}{2\alpha}\chi(r,\alpha,l-\iota_{i},R)
\end{equation} 

%%%%%%%%%%%%%%%%%%%%%%%%%%%%%%%%%%%%%%%%%%%%%%%%%%%%%%%%%%%%%%%%%%%%%%%%%%%%%%%%
\subsection{Three Center Overlap Integral}
%
%
%
The purpose for this section, is to employ the relation in the \ref{OS_general_int_eq:5}
to derive a recursive relation for evaluating the three center overlap integral
in terms of the corresponding lower angular momentum integrals. 

Now let's suggest a three body overlap integral between $\chi_{a}, \chi_{b}$
and $\chi_{c}$:
\begin{equation}
 \begin{split}
(a|b|c) &= \int \chi_{a}(r)\chi_{b}(r)\chi_{c}(r) dr \\
&= \int x^{l_{A}}_{A}y^{m_{A}}_{A}z^{n_{A}}_{A}e^{-\alpha r_{A}^{2}}
        x^{l_{B}}_{B}y^{m_{B}}_{B}z^{n_{B}}_{B}e^{-\beta  r_{B}^{2}} 
        x^{l_{C}}_{C}y^{m_{C}}_{C}z^{n_{C}}_{C}e^{-\gamma r_{C}^{2}}dr
 \end{split}
\label{OS_three_overlap_int_eq:1}
\end{equation}
A,B and C could be the same center, or different centers. From the previous
chapter to evaluate the overlap integral, we know that we could combine
$\chi_{a}$ and $\chi_{b}$ together and then combine the new Gaussian primitive
which centers at $p$ with the $\chi_{c}$. The integral could be generally expressed as:
\begin{equation}
\label{general_expression_for_3_overlap_os}
\begin{split}
 \int \chi_{a}(r)\chi_{b}(r)\chi_{c}(r) dr &= 
\kappa_{abc}I_{abc}^{x}I_{abc}^{y}I_{abc}^{z} \\
&= e^{-\frac{\alpha\beta}{\alpha+\beta}|AB|^{2}}
e^{-\frac{(\alpha+\beta)\gamma}{\alpha+\beta+\gamma}|PC|^{2}} \\
& \int x^{l_{A}}_{A}y^{m_{A}}_{A}z^{n_{A}}_{A}
       x^{l_{B}}_{B}y^{m_{B}}_{B}z^{n_{B}}_{B}
       x^{l_{C}}_{C}y^{m_{C}}_{C}z^{n_{C}}_{C} 
e^{-(\alpha+\beta+\gamma)|r_{G}|^{2}} dr        
\end{split}
\end{equation}
In the combination of Gaussian primitives, new centers generated are given as:
\begin{equation}
 \begin{split}
  \overrightarrow{P} &= \frac{\alpha \overrightarrow{A} + 
\beta \overrightarrow{B}}{\alpha+\beta}
\Rightarrow \\
  \overrightarrow{G} &= \frac{(\alpha+\beta)\overrightarrow{P} + \gamma
\overrightarrow{C}}{ \alpha+\beta +\gamma } \\
    &= \frac{\alpha \overrightarrow{A} + \beta \overrightarrow{B} + 
\gamma \overrightarrow{C}}{ \alpha+\beta +\gamma }
 \end{split}
\label{OS_three_overlap_int_eq:2}
\end{equation}
Here we intentionally do not expand the multiplication of 
$x^{l_{A}}_{A}y^{m_{A}}_{A}z^{n_{A}}_{A}$ etc. since we want to keep it
into some symmetrical form. The essence for deriving the three center of 
overlap integral, is not to really calculate it; but gain some recursive relationship
among the integrals. Here the symmetry of the integral \ref{general_expression_for_3_overlap_os}
is some key feature.

Firstly, for the pre-factor we could express it into:
\begin{equation}
\kappa_{abc} = e^{-\frac{\alpha\beta}{\alpha+\beta}|AB|^{2}}
e^{-\frac{(\alpha+\beta)\gamma}{\alpha+\beta+\gamma}|PC|^{2}}
\label{OS_three_overlap_int_eq:3}
\end{equation}
Now we expand the \ref{OS_three_overlap_int_eq:3} into (We use $A_{i}$, $B_{i}$
and $C_{i}$ to represent it's component on X, Y or Z direction, i could be
x, y, or z):
\begin{equation}
 \begin{split}
 &e^{-\frac{\alpha\beta}{\alpha+\beta}(A_{i}-B_{i})^{2}}
e^{-\frac{(\alpha+\beta)\gamma}{\alpha+\beta+\gamma}(P_{i}-C_{i})^{2}} \\
&=e^{-\frac{\alpha\beta}{\alpha+\beta}(A_{i}-B_{i})^{2}} 
e^{-\frac{\gamma}{(\alpha+\beta)(\alpha+\beta+\gamma)}
((\alpha+\beta)C_{i}-\alpha A_{i} - \beta B_{i})^{2}} \\
 \end{split}
\label{OS_three_overlap_int_eq:4}
\end{equation}
For the exponents we have:
\begin{equation}
 \begin{split}
 & \frac{\alpha\beta}{\alpha+\beta}(A_{i}-B_{i})^{2}
+ \frac{\gamma}{(\alpha+\beta)(\alpha+\beta+\gamma)}
((\alpha+\beta)C_{i} -\alpha A_{i} - \beta B_{i})^{2} \\
&=  \frac{\alpha\beta(\alpha+\beta+\gamma)(A_{i}-B_{i})^{2}
+ \gamma((\alpha+\beta)C_{i} -\alpha A_{i} - \beta B_{i})^{2}}
{(\alpha+\beta)(\alpha+\beta+\gamma)} \\
&= \frac{\alpha^{2}\beta(A_{i}-B_{i})^{2} 
+ \alpha\beta^{2}(A_{i}-B_{i})^{2}+\alpha\beta\gamma(A_{i}-B_{i})^{2}}
{(\alpha+\beta)(\alpha+\beta+\gamma)} \\
&+ \frac{\gamma(\alpha+\beta)^{2}C_{i}^{2}+\alpha^{2}\gamma A_{i}^{2} +
\beta^{2}\gamma B_{i}^{2} }{(\alpha+\beta)(\alpha+\beta+\gamma)} \\
&+ \frac{-2\alpha\gamma(\alpha+\beta)A_{i}C_{i}
- 2\beta\gamma(\alpha+\beta)B_{i}C_{i} + 2\alpha\beta\gamma A_{i}B_{i}}
{(\alpha+\beta)(\alpha+\beta+\gamma)}  \\
 \end{split}
\label{OS_three_overlap_int_eq:5}
\end{equation}
Now let's re-arrange the terms in terms of the order of $\alpha, \beta$
and $\gamma$:
\begin{equation}
 \begin{split}
  & \frac{\alpha\beta}{\alpha+\beta}(A_{i}-B_{i})^{2}
+ \frac{\gamma}{(\alpha+\beta)(\alpha+\beta+\gamma)}
((\alpha+\beta)C_{i}-\alpha A_{i} - \beta B_{i})^{2} \\
&= \frac{\alpha^{2}\beta(A_{i}-B_{i})^{2} 
+ \alpha\beta^{2}(A_{i}-B_{i})^{2}+
\alpha^{2}\gamma(A_{i}-C_{i})^{2} 
+ \beta^{2}\gamma(B_{i}-C_{i})^{2}}{(\alpha+\beta)(\alpha+\beta+\gamma)} \\
& +\frac{
  \alpha\beta\gamma(A_{i}^{2}+B_{i}^{2}+2C_{i}^{2}-2A_{i}C_{i}-2B_{i}C_{i})}
{(\alpha+\beta)(\alpha+\beta+\gamma)}  \\
&= \frac{\alpha^{2}\beta(A_{i}-B_{i})^{2} 
+ \alpha\beta^{2}(A_{i}-B_{i})^{2}+
\alpha^{2}\gamma(A_{i}-C_{i})^{2} 
+ \beta^{2}\gamma(B_{i}-C_{i})^{2}}{(\alpha+\beta)(\alpha+\beta+\gamma)} \\
&+\frac{
  \alpha\beta\gamma\left( (C_{i}-A_{i})^{2} + (C_{i}-B_{i})^{2}\right)}
{(\alpha+\beta)(\alpha+\beta+\gamma)} \\
&= \frac{\alpha\beta(A_{i}-B_{i})^{2}(\alpha+\beta) +
\alpha\gamma(A_{i}-C_{i})^{2}(\alpha+\beta) 
+ \beta\gamma(B_{i}-C_{i})^{2}(\alpha+\beta)}
{(\alpha+\beta)(\alpha+\beta+\gamma)} \\
&= \frac{\alpha\beta(A_{i}-B_{i})^{2} +
\alpha\gamma(A_{i}-C_{i})^{2} 
+ \beta\gamma(B_{i}-C_{i})^{2}}
{(\alpha+\beta+\gamma)}
 \end{split}
\label{OS_three_overlap_int_eq:6}
\end{equation}
The expression in \ref{OS_three_overlap_int_eq:6} is symmetric. Furthermore, we 
could re-form it by using the \ref{OS_three_overlap_int_eq:2} since $G$ will also
appears in the integral part:
\begin{equation}
 \begin{split}
 &\frac{\alpha\beta(A_{i}-B_{i})^{2} + \alpha\gamma(A_{i}-C_{i})^{2}  +
\beta\gamma(B_{i}-C_{i})^{2}}
 {(\alpha+\beta+\gamma)} \\
&= -(\alpha+\beta+\gamma)\left( G_{i}^{2} - \frac{\alpha A_{i}^{2} + \beta
B_{i}^{2} + \gamma C_{i}^{2}}
{\alpha+\beta+\gamma}\right)  
 \end{split}
\label{OS_three_overlap_int_eq:7}
\end{equation}
Therefore, for the pre-factor we have:
\begin{align}
\kappa_{abc} &= e^{-\frac{\alpha\beta}{\alpha+\beta}|A-B|^{2}}
e^{-\frac{(\alpha+\beta)\gamma}{\alpha+\beta+\gamma}|P-C|^{2}} \nonumber \\
&= e^{(\alpha+\beta+\gamma)\left( |G|^{2} - \frac{\alpha |A|^{2} + \beta |B|^{2} + \gamma |C|^{2}}
{\alpha+\beta+\gamma}\right)}
\label{OS_three_overlap_int_eq:8}
\end{align}
Its derivatives for the $R_{i}$ is:
\begin{equation}
 \label{OS_three_overlap_int_eq:9}
\frac{\partial \kappa_{abc}}{R_{i}} = (2\alpha G_{Ai} - 2\alpha
A_{i})\kappa_{abc}
\end{equation}

Now let's expand the integral of $I^{x}_{abc}$:
\begin{equation}
 \begin{split}
  I^{x}_{abc}(l_{A},l_{B},l_{C}) &= \sum_{l_{1}=0}^{l_{A}}\sum_{l_{2}=0}^{l_{B}}
\sum_{l_{3}=0}^{l_{C}}  
\binom{l_{A}}{l_{1}}\binom{l_{B}}{l_{2}}\binom{l_{C}}{l_{3}} \\
&(G_{x}-A_{x})^{(l_{A}-l_{1})} 
 (G_{x}-B_{x})^{(l_{B}-l_{2})}
 (G_{x}-C_{x})^{(l_{C}-l_{3})} \\
&\int (x-G_{x})^{l_{1}+l_{2}+l_{3}}
e^{-(\alpha+\beta+\gamma)(x-G_{x})^{2}} dx \\
&= \sqrt{\frac{\pi}{\alpha+\beta+\gamma}}\sum_{l_{1}=0}^{l_{A}}\sum_{l_{2}=0}^{l_{B}}
\sum_{l_{3}=0}^{l_{C}}  
\binom{l_{A}}{l_{1}}\binom{l_{B}}{l_{2}}\binom{l_{C}}{l_{3}} \\
&(G_{x}-A_{x})^{(l_{A}-l_{1})} 
 (G_{x}-B_{x})^{(l_{B}-l_{2})}
 (G_{x}-C_{x})^{(l_{C}-l_{3})} \\
&\frac{(l_{1}+l_{2}+l_{3}-1)!!}
{\left\lbrace 2(\alpha+\beta+\gamma)\right\rbrace^{l_{1}+l_{2}+l_{3}} }
 \end{split}
\label{OS_three_overlap_int_eq:10}
\end{equation}
We note that $l_{1}+l_{2}+l_{3}$ should be even number else the integral is zero. For
the integral on the y, z direction, we have the similar result, too.
It's derivative could be expressed as:
\begin{equation}
\begin{split}
 \frac{\partial I^{x}_{abc}}{\partial R_{x}} &= 
l_{A}\left( \frac{\alpha}{\alpha+\beta+\gamma} -1\right)I^{x}_{abc}(l_{A}-1,l_{B},l_{C}) \\
&+ l_{B}\left( \frac{\alpha}{\alpha+\beta+\gamma}\right)I^{x}_{abc}(l_{A},l_{B}-1,l_{C}) \\
&+ l_{C}\left( \frac{\alpha}{\alpha+\beta+\gamma}\right)I^{x}_{abc}(l_{A},l_{B},l_{C}-1) 
\end{split}
 \label{OS_three_overlap_int_eq:11}
\end{equation}
We note that for the derivatives of $I^{i}_{abc}$ in terms of $R_{j}$ ($j \neq i$), the result
is obviously zero. 

Now let's combine the result in \ref{OS_three_overlap_int_eq:9} and \ref{OS_three_overlap_int_eq:11},
and also employing the relation in \ref{OS_general_int_eq:5}; we can arrive at some symmetrical
recursive relation for the $(a|b|c)$:
\begin{equation}
 \begin{split}
 (a+\iota_{x}|b|c) &= \frac{1}{2\alpha}\frac{\partial }{\partial R_{Ax}}(a|b|c) 
+ \frac{l_{A}}{2\alpha}(a-\iota_{x}|b|c) \\
&= (G_{Ax} - A_{x})(a|b|c) + 
l_{A}\left(\frac{1}{2(\alpha+\beta+\gamma)} -\frac{1}{2\alpha}\right)(a-\iota_{x}|b|c) \\
&+ l_{B}\left(\frac{1}{2(\alpha+\beta+\gamma)}\right)(a|b-\iota_{x}|c)  \\
&+ l_{C}\left(\frac{1}{2(\alpha+\beta+\gamma)}\right)(a|b|c-\iota_{x}) +  
\frac{l_{A}}{2\alpha}(a-\iota_{x}|b|c) \\
&= (G_{Ax} - A_{x})(a|b|c) + 
l_{A}\left(\frac{1}{2(\alpha+\beta+\gamma)}\right)(a-\iota_{x}|b|c) \\
&+ 
l_{B}\left(\frac{1}{2(\alpha+\beta+\gamma)}\right)(a|b-\iota_{x}|c) \\
&+
l_{C}\left(\frac{1}{2(\alpha+\beta+\gamma)}\right)(a|b|c-\iota_{x}) 
 \end{split}
\label{OS_three_overlap_int_eq:12}
\end{equation}
Such relation could be generalized into the derivative for $R_{Xi}$ for all of 
center of X and all of components of $i=x,y,z$. The relation is the starting point for 
all of following derivations.

At last, let's consider that if one of primitive function is ``S'' type of 
function, how to reduce the recursive relation in
\ref{OS_three_overlap_int_eq:12}. For the S type of function it does not have
the angular momentum part, the pre-factor part part still exists; but for the
integral in \ref{OS_three_overlap_int_eq:10}, the derivatives for the S type 
function will be zero. For example, suggest that $a$ is the S type of function,
then it's obvious that:
\begin{equation}
\begin{split}
 \frac{\partial I^{x}_{abc}}{\partial R_{x}} &= 
l_{B}\left(
\frac{\alpha}{\alpha+\beta+\gamma}\right)I^{x}_{abc}(l_{A},l_{B}-1,l_{C}) \\
&+ l_{C}\left(
\frac{\alpha}{\alpha+\beta+\gamma}\right)I^{x}_{abc}(l_{A},l_{B},l_{C}-1) 
\end{split}
 \label{OS_three_overlap_int_eq:13}
\end{equation}
Then combined with \ref{OS_three_overlap_int_eq:12}, it gives:
\begin{equation}
 \begin{split}
 (0+\iota_{x}|b|c) &= \frac{1}{2\alpha}\frac{\partial }{\partial R_{Ax}}(0|b|c) 
\\
&= (G_{Ax} - A_{x})(0|b|c) 
+ l_{B}\left(\frac{1}{2(\alpha+\beta+\gamma)}\right)(0|b-\iota_{x}|c) \\ 
&+ l_{C}\left(\frac{1}{2(\alpha+\beta+\gamma)}\right)(0|b|c-\iota_{x}) 
 \end{split}
\label{OS_three_overlap_int_eq:14}
\end{equation}
For the other S type centers we have the similar relation, too.

From the above derivation, the bottom integral of three center 
overlap could be expressed as:
\begin{equation}
 \begin{split}
  (0_{A}|0_{B}|0_{C}) &= \int e^{-\alpha r_{A}^{2}}e^{-\beta  r_{B}^{2}} 
             e^{-\gamma r_{C}^{2}}dr \\
          &= e^{-\frac{\alpha\beta}{\alpha+\beta}|AB|^{2}}
e^{-\frac{(\alpha+\beta)\gamma}{\alpha+\beta+\gamma}|PC|^{2}} 
\int e^{-(\alpha+\beta+\gamma)|r_{G}|^{2}} dr \\
          &= e^{-\frac{\alpha\beta}{\alpha+\beta}|AB|^{2}}
e^{-\frac{(\alpha+\beta)\gamma}{\alpha+\beta+\gamma}|PC|^{2}} 
\left( \frac{\pi}{\alpha+\beta+\gamma}\right)^{\frac{3}{2}} 
 \end{split}
 \label{OS_bottom_three_overlap_int_1}
\end{equation}

We can also write the above integral in another form:
\begin{equation}
 \begin{split}
  (0_{A}|0_{B}|0_{C}) &= e^{-\frac{\alpha\beta}{\alpha+\beta}|AB|^{2}}
e^{-\frac{(\alpha+\beta)\gamma}{\alpha+\beta+\gamma}|PC|^{2}} 
\left( \frac{\pi}{\alpha+\beta+\gamma}\right)^{\frac{3}{2}} \\
&= (0_{A}|0_{B})\left( \frac{\alpha+\beta}{\alpha+\beta+\gamma}\right)^{\frac{3}{2}}
e^{-\frac{(\alpha+\beta)\gamma}{\alpha+\beta+\gamma}|PC|^{2}} 
 \end{split}
 \label{OS_bottom_three_overlap_int_2}
\end{equation}
where $(0_{A}|0_{B})$ is the overlap integral in \ref{overlap_direct_int_eq:1}.

%%%%%%%%%%%%%%%%%%%%%%%%%%%%%%%%%%%%%%%%%%%%%%%%%%%%%%%%%%%%%%%%%%%%%%%%%%%%%%%%
\subsection{Electron Repulsion Integrals}
\label{os_eri}
%
%
%
%
Now we begin to use the conclusion got in the previous section to solve the
real problems. This section we are going to tackle down the most difficult one,
the double electrons integral.

Let's consider some electronic repulsion integral over the
Gaussian primitive functions:
\begin{equation}
 \label{OS_ERI_eq:1}
(ab|cd) = \int dr \int dr^{'} \chi_{a}(r)\chi_{b}(r)\frac{1}{|r-r^{'}|}
\chi_{c}(r^{'})\chi_{d}(r^{'})
\end{equation}
a,b,c,d are just some general Gaussian primitive functions.

Firstly we do transformation to the $\frac{1}{|r-r^{'}|}$:
\begin{equation}
 \frac{1}{|r-r^{'}|} = \frac{2}{\pi^{1/2}}\int^{\infty}_{0}
e^{-(r-r^{'})^{2}u^{2}}du
\end{equation}
Here $(r-r^{'})^{2}$ is 
\begin{equation}
(r-r^{'})^{2} = (x_{r}-x_{r^{'}})^{2} +  (y_{r}-y_{r^{'}})^{2} +
(z_{r}-z_{r^{'}})^{2}
\end{equation}
We note the $r$ and $r^{'}$ are more like the A, B defined in the 
\ref{OS_three_overlap_int_eq:4}. Based on this transformation,
the integral could be reformed as:
\begin{equation}
 \begin{split}
 (ab|cd) &= \frac{2}{\pi^{1/2}}\int^{\infty}_{0} du 
\int dr \int dr^{'} \chi_{a}(r)\chi_{b}(r) e^{(r-r^{'})^{2}u^{2}}
\chi_{c}(r^{'})\chi_{d}(r^{'}) \\
&= \frac{2}{\pi^{1/2}}\int^{\infty}_{0} du
\int dr^{'}\chi_{c}(r^{'})\chi_{d}(r^{'}) 
\left( \int dr  \chi_{a}(r)\chi_{b}(r) e^{-(r-r^{'})^{2}u^{2}}\right) \\ 
&= \frac{2}{\pi^{1/2}}\int^{\infty}_{0} du
\int dr^{'}\chi_{c}(r^{'})\chi_{d}(r^{'})(a|0_{r^{'}}|b) \\
&= \frac{2}{\pi^{1/2}}\int^{\infty}_{0} du (ab|u|cd)
 \end{split}
\label{OS_ERI_eq:2}
\end{equation}
Therefore, the two electron integral now is converted into the three
center overlap integral.

Before we move on, it's useful to remind us the key idea of the derivation
we are going to. We are going to use the recursive relation for the three
center overlap integral, to derive the potential recursive relation for the
ERI. Furthermore, before we really move into the ERI section, actually there are
some very fancy formulas needed to be introduced first. Suggest that there are
three variables, $u$, $\epsilon$ and $\eta$, they are all independent with each
other; we can prove that:
\begin{equation}
 \frac{1}{\epsilon+u^{2}} = \frac{1}{\epsilon}\left( 1-\frac{\rho}{\epsilon}
\frac{u^{2}}{\rho+u^{2}}\right) - \frac{1}{\epsilon+\eta}
\frac{u^{2}}{\epsilon+u^{2}}\frac{u^{2}}{\rho+u^{2}} 
\label{OS_ERI_eq:3}
\end{equation}
Where the $\rho$ is:
\begin{equation}
 \rho = \frac{\epsilon\eta}{\epsilon+\eta}
\end{equation}
This intelligent identity is not so obvious by judging from its appearance.
However, this identity possesses a key path to the final form of expression.
Now let's go to see how to prove it.
\begin{equation}
 \begin{split}
 &\frac{u^{2}}{\rho+u^{2}}\left( \frac{\rho}{\epsilon^{2}} + 
\frac{1}{\epsilon+\eta}\frac{u^{2}}{\epsilon+u^{2}}\right) \\
&= \frac{u^{2}}{\rho+u^{2}}\left(
\frac{\epsilon\eta}{\epsilon^{2}(\epsilon+\eta)} + 
\frac{1}{\epsilon+\eta}\frac{u^{2}}{\epsilon+u^{2}}\right) \\
&= \frac{u^{2}}{\rho+u^{2}}\frac{1}{\epsilon+\eta}\left(
\frac{\epsilon\eta}{\epsilon^{2}} + 
\frac{u^{2}}{\epsilon+u^{2}}\right) \\
&= \frac{u^{2}}{\rho+u^{2}}\frac{1}{\epsilon+\eta}\left(
\frac{\eta}{\epsilon} + 
\frac{u^{2}}{\epsilon+u^{2}}\right) \\
&= \frac{u^{2}}{\rho+u^{2}}\frac{1}{\epsilon+\eta}
\frac{\epsilon\eta + \eta u^{2} + \epsilon u^{2}}{\epsilon(\epsilon+u^{2})} \\
&= \frac{u^{2}}{\rho+u^{2}}\frac{1}{\epsilon+\eta}
\frac{\epsilon\eta + (\eta+ \epsilon)u^{2}}{\epsilon(\epsilon+u^{2})} \\
&= \frac{u^{2}}{\rho+u^{2}}\frac{1}{\epsilon+\eta}
\frac{(\rho + u^{2})(\eta+ \epsilon)}{\epsilon(\epsilon+u^{2})} \\
&= \frac{u^{2}}{\epsilon(\epsilon+u^{2})}
 \end{split} 
\label{OS_ERI_eq:4}
\end{equation}
Now let's combine the result in (\ref{OS_ERI_eq:4}) with the last term in
\ref{OS_ERI_eq:3}, then it gives:
\begin{equation}
 \begin{split}
  \frac{1}{\epsilon}-\frac{u^{2}}{\epsilon(\epsilon+u^{2})}  
= \frac{1}{\epsilon}\left( \frac{\epsilon}{\epsilon+u^{2}}\right) 
= \frac{1}{\epsilon+u^{2}}
 \end{split}
\label{OS_ERI_eq:5}
\end{equation}
Which is the final result in \ref{OS_ERI_eq:3}. Here we note, that the result
in the \ref{OS_ERI_eq:3} will provide us a chance to link the primitives on 
$r$ and primitives on $r^{'}$ in \ref{OS_ERI_eq:1}.

Now let's firstly solve this integral of $(a|0_{r^{'}}|b)$. We would apply
the recursive relation we got in the \ref{OS_three_overlap_int_eq:12}:
\begin{equation}
 \begin{split}
 (a+\iota_{i}|b|c) 
&= (G_{Ai} - A_{i})(a|b|c) + 
N_{i}(A)\left(\frac{1}{2(\alpha+\beta+\gamma)}\right)(a-\iota_{i}|b|c) \\
&+ 
N_{i}(B)\left(\frac{1}{2(\alpha+\beta+\gamma)}\right)(a|b-\iota_{i}|c) \\
&+
N_{i}(C)\left(\frac{1}{2(\alpha+\beta+\gamma)}\right)(a|b|c-\iota_{i}) 
 \end{split}
\label{OS_ERI_eq:6}
\end{equation}
Here we extended it to the general form, and the $\iota_{i}$ and $N_{i}$ are just
the same as \ref{OS_general_int_eq:4} etc. $i$ could be $x,y, z$.

Let's consider to fit it to $(a|0_{r^{'}}|b)$. Here the $\gamma$ is $u^{2}$, and
for the center B (it's the $r^{'}$) we only have a S type of orbital so the
expression could be further expressed as:
\begin{equation}
 \begin{split}
  (a+\iota_{i}|0_{r^{'}}|b)
&=(G_{Ai} - A_{i})(a|0_{r^{'}}|b) +
N_{i}(A)\left(\frac{1}{2(\alpha+\beta+u^{2})}\right)(a-\iota_{i}|0_{r^{'}}|b) \\
&+
N_{i}(B)\left(\frac{1}{2(\alpha+\beta+u^{2})}\right)(a|0_{r^{'}}|b-\iota_{i}) \\
&=(G_{Ai} - A_{i})(a|0_{r^{'}}|b) +
N_{i}(A)\left(\frac{1}{2(\epsilon+u^{2})}\right)(a-\iota_{i}|0_{r^{'}}|b) \\
&+
N_{i}(B)\left(\frac{1}{2(\epsilon+u^{2})}\right)(a|0_{r^{'}}|b-\iota_{i})
 \end{split}
\label{OS_ERI_eq:7}
\end{equation}

According to the \ref{OS_ERI_eq:3}, $\dfrac{1}{\epsilon+u^{2}}$ could be
directly expanded; so the result could be reformed into:
\begin{equation}
 \begin{split}
  (a+\iota_{i}|0_{r^{'}}|b)
&=(G_{Ai} - A_{i})(a|0_{r^{'}}|b) \\
&+
\frac{N_{i}(A)}{2\epsilon}\left(1-\frac{\rho}{\epsilon}
\frac{u^{2}}{\rho+u^{2}}\right)(a-\iota_{i}|0_{r^{'}}|b) \\
&-\frac{N_{i}(A)}{2}\left(\frac{1}{\epsilon+\eta}
\frac{u^{2}}{\epsilon+u^{2}}\frac{u^{2}}{\rho+u^{2}}\right)(a-\iota_{i}|0_{r^{'}}|b) \\
&+ \frac{N_{i}(B)}{2\epsilon}\left(1-\frac{\rho}{\epsilon}
\frac{u^{2}}{\rho+u^{2}}\right)(a|0_{r^{'}}|b-\iota_{i}) \\
&-\frac{N_{i}(B)}{2}\left(\frac{1}{\epsilon+\eta}
\frac{u^{2}}{\epsilon+u^{2}}\frac{u^{2}}{\rho+u^{2}}\right)(a|0_{r^{'}}|b-\iota_{i})
 \end{split}
\label{OS_ERI_eq:8}
\end{equation}
The $\eta$ and corresponding $\rho$ could be referred later. Now they are only
arbitrary numbers ($\rho$ depends on $\eta$ and $\epsilon$). 

Now we are trying to wrap up some terms in the \ref{OS_ERI_eq:8} into a new
form:
\begin{equation}
 \begin{split}
  &(a|0_{r^{'}}+\iota_{i}|b) \\
  & =(0_{r^{'}}+\iota_{i}|a|b)  \\
  &= \left( \frac{\alpha A_{i} + \beta B_{i} + u^{2}r^{'}_{i}}
{\alpha +\beta + u^{2}} - r^{'}_{i}\right)(a|0_{r^{'}}|b)\\
&+
N_{i}(A)\left(\frac{1}{2(\epsilon+u^{2})}\right)(a-\iota_{i}|0_{r^{'}}|b) \\
&+
N_{i}(B)\left(\frac{1}{2(\epsilon+u^{2})}\right)(a|0_{r^{'}}|b-\iota_{i})
 \end{split}
\label{OS_ERI_eq:9}
\end{equation}
Since
\begin{equation}
 \begin{split}
  \frac{\alpha A_{i} + \beta B_{i} + u^{2}r^{'}_{i}}
{\alpha +\beta + u^{2}} - r^{'}_{i} = 
 \frac{\alpha A_{i} + \beta B_{i} -r^{'}_{i}(\alpha+\beta)}
{\epsilon + u^{2}} = \frac{P_{i}\epsilon -r^{'}_{i}\epsilon}
{\epsilon + u^{2}}
 \end{split}
\end{equation}
where 
\begin{equation}
 P_{i} = \frac{\alpha A_{i} + \beta B_{i}}{\alpha+\beta}
\end{equation}
and $\epsilon = \alpha+\beta$, therefore the \ref{OS_ERI_eq:9} could be
expressed as:
\begin{equation}
 \begin{split}
&(a|0_{r^{'}}+\iota_{i}|b) \\
&=-\frac{\epsilon}{\epsilon + u^{2}}(r^{'}_{i} - P_{i})(a|0_{r^{'}}|b) \\
&+
N_{i}(A)\left(\frac{1}{2(\epsilon+u^{2})}\right)(a-\iota_{i}|0_{r^{'}}|b) \\
&+
N_{i}(B)\left(\frac{1}{2(\epsilon+u^{2})}\right)(a|0_{r^{'}}|b-\iota_{i})  
 \end{split}
\label{OS_ERI_eq:10}
\end{equation}
Here we note that our purpose here is try to build some expression, which
could reduce the integral into some potential recursive forms.

Now we can use the $(a|0_{r^{'}}+\iota_{i}|b)$ to transform the
\ref{OS_ERI_eq:8}:
\begin{equation}
 \begin{split}
    (a+\iota_{i}|0_{r^{'}}|b)
&=(G_{Ai} - A_{i})(a|0_{r^{'}}|b) \\
&+
\frac{N_{i}(A)}{2\epsilon}\left(1-\frac{\rho}{\epsilon}
\frac{u^{2}}{\rho+u^{2}}\right)(a-\iota_{i}|0_{r^{'}}|b) \\
&+ \frac{N_{i}(B)}{2\epsilon}\left(1-\frac{\rho}{\epsilon}
\frac{u^{2}}{\rho+u^{2}}\right)(a|0_{r^{'}}|b-\iota_{i}) \\
&-\frac{u^{2}}{\epsilon+\eta}\frac{u^{2}}{\rho+u^{2}}(a|0_{r^{'}}+\iota_{i}|b) \\
&-\frac{\epsilon}{\epsilon+u^{2}}
\frac{u^{2}}{\epsilon+\eta}\frac{u^{2}}{\rho+u^{2}}(r^{'}_{i} - P_{i})((a|0_{r^{'}}|b)) 
 \end{split}
\label{OS_ERI_eq:11}
\end{equation}

Now let's use the relation defined in the \ref{OS_ERI_eq:3}
to expand the $(G_{Ai} - A_{i})$. We note that the $G_{Ai}$ is:
\begin{equation}
 G_{Ai} = \frac{\alpha A_{i} + \beta B_{i} + u^{2} r^{'}_{i}}{\alpha + \beta +
u^{2}}
 = \frac{\alpha A_{i} + \beta B_{i} + u^{2} r^{'}_{i}}{\epsilon + u^{2}}
\end{equation}
Hence, for $(G_{Ai} - A_{i})$ according to \ref{OS_ERI_eq:3} it's:
\begin{equation}
 \begin{split}
 G_{Ai} - A_{i} &= \left( \frac{\alpha A_{i} + \beta B_{i}}{\epsilon} -
A_{i}\right) +
\frac{u^{2} r^{'}_{i}}{\epsilon} \\
&-(\alpha A_{i} + \beta B_{i} + u^{2} r^{'}_{i})\left(\frac{\rho}{\epsilon^{2}}
\frac{u^{2}}{\rho+u^{2}} + \frac{1}{\epsilon+\eta}
\frac{u^{2}}{\epsilon+u^{2}}\frac{u^{2}}{\rho+u^{2}} 
\right)  \\
&= \left( P_{i} - A_{i}\right) + \frac{u^{2} r^{'}_{i}}{\epsilon} \\
&-\left( \frac{u^{2}r^{'}_{i}}{\epsilon} + P_{i}\right)
\left(\frac{\rho}{\epsilon}
\frac{u^{2}}{\rho+u^{2}} + \frac{\epsilon}{\epsilon+\eta}
\frac{u^{2}}{\epsilon+u^{2}}\frac{u^{2}}{\rho+u^{2}} 
\right) 
 \end{split}
\label{OS_ERI_eq:12}
\end{equation}

Now let's combine the terms for the $(a|0_{r^{'}}|b)$ in the \ref{OS_ERI_eq:11}
and \ref{OS_ERI_eq:12}. For the $P_{i}$ term, we can see that:
\begin{equation}
 \left( P_{i} - A_{i}\right)(a|0_{r^{'}}|b) 
-P_{i}\frac{\rho}{\epsilon}
\frac{u^{2}}{\rho+u^{2}}(a|0_{r^{'}}|b)
\label{OS_ERI_eq:13}
\end{equation}
the terms for the $\dfrac{\epsilon}{\epsilon+\eta}
\dfrac{u^{2}}{\epsilon+u^{2}}\dfrac{u^{2}}{\rho+u^{2}} $ canceled. 

For the $r_{i}^{'}$, the remaining terms are:
\begin{equation}
 \begin{split}
&\frac{u^{2} r^{'}_{i}}{\epsilon}
-\frac{u^{2}r^{'}_{i}}{\epsilon}
\left(\frac{\rho}{\epsilon}
\frac{u^{2}}{\rho+u^{2}} + \frac{\epsilon}{\epsilon+\eta}
\frac{u^{2}}{\epsilon+u^{2}}\frac{u^{2}}{\rho+u^{2}} 
\right) \\
&-\frac{\epsilon}{\epsilon+u^{2}}
\frac{u^{2}}{\epsilon+\eta}\frac{u^{2}}{\rho+u^{2}}r^{'}_{i} 
 \end{split}
\end{equation}
We do not include the integral of $(a|0_{r^{'}}|b)$ here since they are similar
terms in terms of $(a|0_{r^{'}}|b)$.

Here we have:
\begin{equation}
\frac{\epsilon}{\epsilon+\eta}
\frac{u^{2}}{\epsilon+u^{2}}\frac{u^{2}}{\rho+u^{2}}\left(
1+\frac{u^{2}}{\epsilon} \right)r^{'}_{i} = 
\frac{u^{2}}{\epsilon+\eta}
\frac{u^{2}}{\rho+u^{2}}r^{'}_{i} 
\end{equation}
Then
\begin{equation}
\begin{split}
&\frac{u^{2}}{\epsilon+\eta}
\frac{u^{2}}{\rho+u^{2}}r^{'}_{i} + 
\frac{u^{2}r^{'}_{i}}{\epsilon}
\frac{\rho}{\epsilon}
\frac{u^{2}}{\rho+u^{2}} = \\
&{u^{2}}r^{'}_{i}
\frac{u^{2}}{\rho+u^{2}}\left(\frac{1}{\epsilon+\eta} + 
\frac{\rho}{\epsilon^{2}} \right) \\
&= {u^{2}}r^{'}_{i}
\frac{u^{2}}{\rho+u^{2}}\left(\frac{1}{\epsilon+\eta} + 
\frac{1}{\epsilon+\eta}\frac{\eta}{\epsilon} \right) \\
&= {u^{2}}r^{'}_{i}
\frac{u^{2}}{\rho+u^{2}}\frac{1}{\epsilon}
\end{split}
\end{equation}
Finally
\begin{equation}
\frac{u^{2} r^{'}_{i}}{\epsilon}
-{u^{2}}r^{'}_{i}
\frac{u^{2}}{\rho+u^{2}}\frac{1}{\epsilon} = r^{'}_{i}
\frac{u^{2}}{\rho+u^{2}}\frac{\rho}{\epsilon}
\end{equation}
Surprisingly, this term is directly corresponding to the terms for $P_{i}$
in the \ref{OS_ERI_eq:13}, so totally we have the result that:
\begin{equation}
\begin{split}
&(G_{Ai} - A_{i})(a|0_{r^{'}}|b)
-\frac{\epsilon}{\epsilon+u^{2}}
\frac{u^{2}}{\epsilon+\eta}\frac{u^{2}}{\rho+u^{2}}(r^{'}_{i} -
P_{i})(a|0_{r^{'}}|b)  \\
&= \left( P_{i} - A_{i}\right)(a|0_{r^{'}}|b) + 
(r^{'}_{i}-P_{i})\frac{\rho}{\epsilon}
\frac{u^{2}}{\rho+u^{2}}(a|0_{r^{'}}|b)
\end{split}
\label{OS_ERI_eq:14}
\end{equation}
 
By using the \ref{OS_ERI_eq:14}, the \ref{OS_ERI_eq:8} could be finally
expressed as:
\begin{equation}
 \begin{split}
  (a+\iota_{i}|0_{r^{'}}|b)
&=(P_{i} - A_{i})(a|0_{r^{'}}|b) \\
&+
\frac{N_{i}(A)}{2\epsilon}\left(1-\frac{\rho}{\epsilon}
\frac{u^{2}}{\rho+u^{2}}\right)(a-\iota_{i}|0_{r^{'}}|b) \\
&+ \frac{N_{i}(B)}{2\epsilon}\left(1-\frac{\rho}{\epsilon}
\frac{u^{2}}{\rho+u^{2}}\right)(a|0_{r^{'}}|b-\iota_{i}) \\
&+(r^{'}_{i}-P_{i})\frac{\rho}{\epsilon}
\frac{u^{2}}{\rho+u^{2}}(a|0_{r^{'}}|b) 
-\frac{u^{2}}{\epsilon+\eta}\frac{u^{2}}{\rho+u^{2}}(a|0_{r^{'}}+\iota_{i}|b)
 \end{split}
\label{OS_ERI_eq:15}
\end{equation}

Now let's set the value of $\eta$ and $\rho$, so that to make the $(a+\iota_{i}|0_{r^{'}}|b)$
related to the outer integral on $\chi_{c}$ and $\chi_{d}$:
\begin{equation}
 \label{OS_ERI_eq:16}
\eta = \alpha^{'} + \beta^{'} 
\end{equation}
where $\alpha^{'}$ is the exponent factor for $\chi_{c}$, and $\beta^{'}$ is the exponent
factor for $\chi_{d}$. Therefore, the $\rho$ could be expressed as:
\begin{equation}
 \label{OS_ERI_eq:17}
\rho = \frac{(\alpha+\beta)(\alpha^{'} + \beta^{'})}{(\alpha+\beta)+(\alpha^{'} + \beta^{'})} 
\end{equation}

Let's consider the outer integral over $r^{'}$, it's easy to see that:
\begin{equation}
\label{OS_ERI_eq:18}
 -\int dr^{'} \chi_{c}(r^{'})\chi_{d}(r^{'})(a|0_{r^{'}} +\iota_{i} |b)
= \int dr \chi_{a}(r)\chi_{b}(r)(c|0_{r} + \iota_{i}|d)
\end{equation}
This is because that for the term of $0_{r} + \iota_{i}$, which is converted from
$r^{'}_{i} - r_{i}$ into $r_{i}-r^{'}_{i}$.Therefore, for the term of 
$\dfrac{u^{2}}{\epsilon+\eta}\dfrac{u^{2}}
{\rho+u^{2}}(a|0_{r^{'}}+\iota_{i}|b)$ in the integral it could be converted into:
\begin{equation}
 \label{OS_ERI_eq:19}
\begin{split}
&-\frac{u^{2}}{\epsilon+\eta}\frac{u^{2}}
{\rho+u^{2}}\int dr^{'} \chi_{c}(r^{'})\chi_{d}(r^{'})(a|0_{r^{'}}+\iota_{i}|b) \\
&=\frac{1}{\epsilon+\eta}\frac{u^{2}}
{\rho+u^{2}}\int dr \chi_{a}(r)\chi_{b}(r)u^{2}(c|0_{r}+\iota_{i}|d)
\end{split}
\end{equation}
While according to the \ref{OS_ERI_eq:10}, the term of $u^{2}(c|0_{r}+\iota_{i}|d)$
is:
\begin{equation}
 \begin{split}
  u^{2}(c|0_{r}+\iota_{i}|d) &= 
-\eta(r_{i} - Q_{i})(c|0_{r}|d) \\
&+
\left(\frac{N_{i}(C)}{2}\right)(c-\iota_{i}|0_{r}|d) \\
&+
\left(\frac{N_{i}(D)}{2}\right)(c|0_{r}|d-\iota_{i}) \\
&-\eta(c|0_{r}+\iota_{i}|d)
 \end{split}
\label{OS_ERI_eq:20}
\end{equation}
and 
\begin{equation}
 Q_{i} = \frac{\alpha^{'}C + \beta^{'}D}{\alpha^{'} + \beta^{'}}
\end{equation}

Now let's take the result of \ref{OS_ERI_eq:20} into \ref{OS_ERI_eq:19}, it's:
\begin{equation}
 \begin{split}
  &-\frac{u^{2}}{\epsilon+\eta}\frac{u^{2}}
{\rho+u^{2}}\int dr^{'} \chi_{c}(r^{'})\chi_{d}(r^{'})(a|0_{r^{'}}+\iota_{i}|b) \\
&=Q_{i}\frac{\eta}{\epsilon+\eta}\frac{u^{2}}
{\rho+u^{2}}\int dr \chi_{a}(r)\chi_{b}(r)(c|0_{r}|d) \\
&+
\left(\frac{N_{i}(C)}{2}\right)\frac{1}{\epsilon+\eta}\frac{u^{2}}
{\rho+u^{2}}\int dr \chi_{a}(r)\chi_{b}(r)(c-\iota_{i}|0_{r}|d) \\
&+
\left(\frac{N_{i}(D)}{2}\right)\frac{1}{\epsilon+\eta}\frac{u^{2}}
{\rho+u^{2}}\int dr \chi_{a}(r)\chi_{b}(r)(c|0_{r}|d-\iota_{i}) \\
&-\eta \frac{1}{\epsilon+\eta}\frac{u^{2}}
{\rho+u^{2}}\int dr \chi_{a}(r)\chi_{b}(r) r_{i} (c|0_{r}|d) \\
&-\eta\frac{1}{\epsilon+\eta}\frac{u^{2}}
{\rho+u^{2}}\int dr \chi_{a}(r)\chi_{b}(r)(c|0_{r}+\iota_{i}|d)
 \end{split}
\label{OS_ERI_eq:21}
\end{equation}

Now let's look into the details of \ref{OS_ERI_eq:21}. In terms of its last
two terms, actually we have:
\begin{equation}
 \begin{split}
 r_{i}(c|0_{r}|d) +(c|0_{r}+\iota_{i}|d) 
&=
\int dr^{'} \chi_{c}(r^{'})\chi_{d}(r^{'})
r_{i}e^{u^{2}(r^{'}-r)^{2}} \\
&+ 
\int dr^{'} \chi_{c}(r^{'})\chi_{d}(r^{'})
(r^{'}_{i}-r_{i})e^{u^{2}(r^{'}-r)^{2}}
 \end{split}
\label{OS_ERI_eq:22}
\end{equation}
Here according to our traditional notation (see \ref{OS_three_overlap_int_eq:4}),
$r$ and $r^{'}$ represent the points in the space, and $i$ denotes its components
on the X, Y or Z axis. Therefore, it's clear that:
\begin{equation}
  r_{i}(c|0_{r}|d) +(c|0_{r}+\iota_{i}|d) =
\int dr^{'} \chi_{c}(r^{'})\chi_{d}(r^{'})
r^{'}_{i}e^{u^{2}(r^{'}-r)^{2}}
\label{OS_ERI_eq:23}
\end{equation}  
If we go over integral by $r$, it yields:
\begin{equation}
\begin{split}
&\int dr \chi_{a}(r)\chi_{b}(r) \int dr^{'} \chi_{c}(r^{'})\chi_{d}(r^{'})
r^{'}_{i}e^{(r^{'}-r)^{2}} \\
&= \int dr^{'} \chi_{c}(r^{'})\chi_{d}(r^{'}) r^{'}_{i}
\int dr \chi_{a}(r)\chi_{b}(r) e^{u^{2}(r^{'}-r)^{2}} \\
&= \int dr^{'} \chi_{c}(r^{'})\chi_{d}(r^{'}) r^{'}_{i}(a|0_{r^{'}}|b)
\end{split}
\label{OS_ERI_eq:24}
\end{equation}
Therefore the \ref{OS_ERI_eq:21} could be converted into:
\begin{equation}
 \begin{split}
  &-\frac{u^{2}}{\epsilon+\eta}\frac{u^{2}}
{\rho+u^{2}}\int dr^{'} \chi_{c}(r^{'})\chi_{d}(r^{'})(a|0_{r^{'}}+\iota_{i}|b) \\
&=Q_{i}\frac{\eta}{\epsilon+\eta}\frac{u^{2}}
{\rho+u^{2}}\int dr \chi_{a}(r)\chi_{b}(r)(c|0_{r}|d) \\
&+
\left(\frac{N_{i}(C)}{2}\right)\frac{1}{\epsilon+\eta}\frac{u^{2}}
{\rho+u^{2}}\int dr \chi_{a}(r)\chi_{b}(r)(c-\iota_{i}|0_{r}|d) \\
&+
\left(\frac{N_{i}(D)}{2}\right)\frac{1}{\epsilon+\eta}\frac{u^{2}}
{\rho+u^{2}}\int dr \chi_{a}(r)\chi_{b}(r)(c|0_{r}|d-\iota_{i}) \\
&-\eta \frac{1}{\epsilon+\eta}\frac{u^{2}}
{\rho+u^{2}}\int dr^{'} \chi_{c}(r^{'})\chi_{d}(r^{'}) r^{'}_{i}(a|0_{r^{'}}|b)
 \end{split}
\label{OS_ERI_eq:25}
\end{equation}

Now let's combine the result in \ref{OS_ERI_eq:25} with the result in \ref{OS_ERI_eq:15}.
By multiplying $\chi_{c}$ and $\chi_{d}$ and integrate over $r^{'}$ in 
\ref{OS_ERI_eq:15}, we can get:
\begin{equation}
 \begin{split}
&\int dr^{'} \chi_{c}(r^{'})\chi_{d}(r^{'})(a+\iota_{i}|0_{r^{'}}|b) \\
&=(P_{i} - A_{i})\int dr^{'} \chi_{c}(r^{'})\chi_{d}(r^{'})(a|0_{r^{'}}|b) \\
&+
\frac{N_{i}(A)}{2\epsilon}\left(1-\frac{\rho}{\epsilon}
\frac{u^{2}}{\rho+u^{2}}\right)\int dr^{'} \chi_{c}(r^{'})\chi_{d}(r^{'})
(a-\iota_{i}|0_{r^{'}}|b) \\
&+ \frac{N_{i}(B)}{2\epsilon}\left(1-\frac{\rho}{\epsilon}
\frac{u^{2}}{\rho+u^{2}}\right)\int dr^{'} \chi_{c}(r^{'})\chi_{d}(r^{'})
(a|0_{r^{'}}|b-\iota_{i}) \\
&-P_{i}\frac{\rho}{\epsilon}
\frac{u^{2}}{\rho+u^{2}}\int dr^{'} \chi_{c}(r^{'})\chi_{d}(r^{'})(a|0_{r^{'}}|b) \\
&+\frac{\rho}{\epsilon}
\frac{u^{2}}{\rho+u^{2}}\int dr^{'} \chi_{c}(r^{'})\chi_{d}(r^{'})r^{'}_{i}(a|0_{r^{'}}|b) \\
&+Q_{i}\frac{\eta}{\epsilon+\eta}\frac{u^{2}}
{\rho+u^{2}}\int dr \chi_{a}(r)\chi_{b}(r)(c|0_{r}|d) \\
&+
\left(\frac{N_{i}(C)}{2}\right)\frac{1}{\epsilon+\eta}\frac{u^{2}}
{\rho+u^{2}}\int dr \chi_{a}(r)\chi_{b}(r)(c-\iota_{i}|0_{r}|d) \\
&+
\left(\frac{N_{i}(D)}{2}\right)\frac{1}{\epsilon+\eta}\frac{u^{2}}
{\rho+u^{2}}\int dr \chi_{a}(r)\chi_{b}(r)(c|0_{r}|d-\iota_{i}) \\
&-\eta \frac{1}{\epsilon+\eta}\frac{u^{2}}
{\rho+u^{2}}\int dr^{'} \chi_{c}(r^{'})\chi_{d}(r^{'}) r^{'}_{i}(a|0_{r^{'}}|b) \\
&= (P_{i} - A_{i})\int dr^{'} \chi_{c}(r^{'})\chi_{d}(r^{'})(a|0_{r^{'}}|b) \\
&+
\frac{N_{i}(A)}{2\epsilon}\left(1-\frac{\rho}{\epsilon}
\frac{u^{2}}{\rho+u^{2}}\right)\int dr^{'} \chi_{c}(r^{'})\chi_{d}(r^{'})
(a-\iota_{i}|0_{r^{'}}|b) \\
&+ \frac{N_{i}(B)}{2\epsilon}\left(1-\frac{\rho}{\epsilon}
\frac{u^{2}}{\rho+u^{2}}\right)\int dr^{'} \chi_{c}(r^{'})\chi_{d}(r^{'})
(a|0_{r^{'}}|b-\iota_{i}) \\
&+\left(W_{i} -P_{i}\right)
\frac{u^{2}}{\rho+u^{2}}\int dr^{'} \chi_{c}(r^{'})\chi_{d}(r^{'})(a|0_{r^{'}}|b) \\
&+
\left(\frac{N_{i}(C)}{2}\right)\frac{1}{\epsilon+\eta}\frac{u^{2}}
{\rho+u^{2}}\int dr \chi_{a}(r)\chi_{b}(r)(c-\iota_{i}|0_{r}|d) \\
&+
\left(\frac{N_{i}(D)}{2}\right)\frac{1}{\epsilon+\eta}\frac{u^{2}}
{\rho+u^{2}}\int dr \chi_{a}(r)\chi_{b}(r)(c|0_{r}|d-\iota_{i}) 
 \end{split}
\label{OS_ERI_eq:26}
\end{equation}
Where in this long expression, it's clear that the integral term containing
$r^{'}_{i}$ vanished, and we combine the terms of $P_{i}$ and $Q_{i}$ together
so to give $W_{i}$:
\begin{equation}
 W_{i} = \frac{\epsilon P_{i} + \eta Q_{i}}{\epsilon + \eta}
\end{equation}
We note, that this surprising result has very beautiful symmetry among the 
resulting integrals.

By using the expression of $(ab|u|cd)$ in \ref{OS_ERI_eq:2}, the expression for
the \ref{OS_ERI_eq:26} could be further simplified as:
\begin{equation}
 \begin{split}
((a+\iota_{i})b|u|cd) &= (P_{i} - A_{i})(ab|u|cd) +
\left(W_{i} -P_{i}\right)
\frac{u^{2}}{\rho+u^{2}}(ab|u|cd) \\
&+\frac{N_{i}(A)}{2\epsilon}\left(1-\frac{\rho}{\epsilon}
\frac{u^{2}}{\rho+u^{2}}\right)
((a-\iota_{i})b|u|cd) \\
&+\frac{N_{i}(B)}{2\epsilon}\left(1-\frac{\rho}{\epsilon}
\frac{u^{2}}{\rho+u^{2}}\right)
(a(b-\iota_{i})|u|cd) \\
&+\left(\frac{N_{i}(C)}{2}\right)\frac{1}{\epsilon+\eta}\frac{u^{2}}
{\rho+u^{2}}(ab|u|(c-\iota_{i})d) \\
&+\left(\frac{N_{i}(D)}{2}\right)\frac{1}{\epsilon+\eta}\frac{u^{2}}
{\rho+u^{2}}(ab|u|c(d-\iota_{i}))
\end{split}
\label{OS_ERI_eq:27}
\end{equation}

Now let's define some auxiliary integral function:
\begin{equation}
\label{OS_ERI_eq:28}
 (ab|cd)^{(m)} = \frac{2}{\sqrt{\pi}}\int^{\infty}_{0} du \left( \frac{u^{2}}
{\rho+u^{2}}\right)^{m}(ab|u|cd) 
\end{equation}
Then by multiplying with $\left( \dfrac{u^{2}}
{\rho+u^{2}}\right)^{m}$ and integrating over $u$
the above result could be written as:
\begin{equation}
 \begin{split}
((a+\iota_{i})b|cd)^{(m)} &= (P_{i} - A_{i})(ab|cd)^{(m)} +
\left(W_{i} -P_{i}\right)(ab|cd)^{(m+1)} \\
&+\frac{N_{i}(A)}{2\epsilon}\left(((a-\iota_{i})b|cd)^{(m)}-\frac{\rho}{
\epsilon }((a-\iota_{i})b|cd)^{(m+1)}\right)  \\
&+\frac{N_{i}(B)}{2\epsilon}\left((a(b-\iota_{i})|cd)^{(m)}-\frac{\rho}{
\epsilon }(a(b-\iota_{i})|cd)^{(m+1)}\right)  \\
&+\left(\frac{N_{i}(C)}{2}\right)\frac{1}{\epsilon+\eta}
(ab|(c-\iota_{i})d)^{(m+1)} \\
&+\left(\frac{N_{i}(D)}{2}\right)\frac{1}{\epsilon+\eta}
(ab|c(d-\iota_{i}))^{(m+1)}
\end{split}
\label{OS_ERI_result}
\end{equation}
We note that the true ERI is $(ab|cd)^{(0)}$. This is the final result 
for deriving the ERI in the OS framework.

Finally, let's make some complementary work. In the \ref{OS_ERI_result},
it's clear all of integrals could be derived from the basic integral of
$(00|00)^{(m)}$, which is in form of:
\begin{equation}
 \begin{split}
 (00|00)^{(m)} &= \frac{2}{\sqrt{\pi}} \int_{0}^{\infty} du \left( \frac{u^{2}}
{\rho+u^{2}}\right)^{m} (00|u|00) \\
&=\frac{2}{\sqrt{\pi}} \int_{0}^{\infty} du \left( \frac{u^{2}}
{\rho+u^{2}}\right)^{m} 
\int dr^{'}  e^{-\alpha^{'} (r^{'}_{C})^{2}}e^{-\beta^{'} (r^{'}_{D})^{2}} \\
&\int dr  e^{-\alpha r_{A}^{2}} e^{-\beta r_{B}^{2}} e^{-(r-r^{'})^{2}u^{2}} 
\end{split}
\end{equation}
As for the integral over $r$, by using the three body integral result in
\ref{OS_bottom_three_overlap_int_2}, we can rewrite it as:
\begin{equation}
 \begin{split}
  \int dr  e^{-\alpha r_{A}^{2}} e^{-\beta r_{B}^{2}} e^{-(r-r^{'})^{2}u^{2}} &=
  (0_{A}|0_{B})\left( \frac{\alpha+\beta}{\alpha+\beta+u^{2}}\right)^{\frac{3}{2}}
e^{-\frac{(\alpha+\beta)u^{2}}{\alpha+\beta+u^{2}}|r^{'}-P|^{2}} 
 \end{split}
 \label{OS_ERI_eq:30}
\end{equation}
This is equivalent to treat $e^{(r-r^{'})^{2}u^{2}}$ as a S type of Gaussian, where
the electron in $r$ centers on ``nuclei'' of $r^{'}$.

The RHS of \ref{OS_ERI_eq:30} is still like a Gaussian of $r^{'}$ centering
on $P$, so once again by using the \ref{OS_bottom_three_overlap_int_2};
it's able to integrate over $r^{'}$. Before we move on, let's do some prepare
work. Firstly, considering the combination of Gaussian on $P$ and Gaussian
on $C$ and $D$, for the exponent coefficient it gives:
\begin{equation}
 \begin{split}
  \frac{\frac{(\alpha+\beta)u^{2}}{\alpha+\beta+u^{2}}\times
  (\alpha^{'}+\beta^{'})}{\frac{(\alpha+\beta)u^{2}}{\alpha+\beta+u^{2}} 
  +\alpha^{'}+\beta^{'}} 
  &= \frac{(\alpha^{'}+\beta^{'})(\alpha+\beta)u^{2}}
  {(\alpha+\beta+u^{2})(\alpha^{'}+\beta^{'})+(\alpha+\beta)u^{2}} \\
  &= \frac{u^{2}}{1+\frac{u^{2}}{\alpha+\beta}+\frac{u^{2}}{\alpha^{'}+\beta^{'}}} \\
  &= \frac{u^{2}}{1+u^{2}\left( \frac{1}{\alpha+\beta}+\frac{1}{\alpha^{'}+\beta^{'}}
  \right) } \\
  &= \frac{u^{2}}{1+\frac{u^{2}}{\rho}} \\
  &= \frac{\rho u^{2}}{\rho+ u^{2}}
 \end{split}
 \label{OS_ERI_eq:31}
 \end{equation}
 Here we use the definition for variable of $\rho$ in \ref{OS_ERI_eq:17}.
 
 So the integral over $r^{'}$ becomes:
\begin{equation}
 \begin{split}
   &\int dr^{'}  e^{-\alpha^{'} (r^{'}_{C})^{2}}e^{-\beta^{'} (r^{'}_{D})^{2}} 
   \int dr  e^{-\alpha r_{A}^{2}} e^{-\beta r_{B}^{2}} e^{-(r-r^{'})^{2}u^{2}} \\
&= (0_{A}|0_{B})\left( \frac{\alpha+\beta}{\alpha+\beta+u^{2}}\right)^{\frac{3}{2}}   
  \int dr^{'}  e^{-\alpha^{'} (r^{'}_{C})^{2}}e^{-\beta^{'} (r^{'}_{D})^{2}} 
   e^{-\frac{(\alpha+\beta)u^{2}}{\alpha+\beta+u^{2}}|r^{'}-P|^{2}} \\
&= (0_{A}|0_{B})(0_{C}|0_{D})
\left( \frac{\alpha+\beta}{\alpha+\beta+u^{2}}\right)^{\frac{3}{2}} 
\left( \frac{\alpha^{'}+\beta^{'}}{\alpha^{'}+\beta^{'}+ 
\frac{(\alpha+\beta)u^{2}}{\alpha+\beta+u^{2}}}\right)^{\frac{3}{2}}
e^{-\frac{\rho u^{2}}{\rho+ u^{2}}|PQ|^{2}}
\end{split} 
\label{OS_ERI_eq:32}
\end{equation}
This is using the result of \ref{OS_ERI_eq:31}. Combing the term in \ref{OS_ERI_eq:32},
it gives:
\begin{equation}
\begin{split}
 &\left( \frac{\alpha+\beta}{\alpha+\beta+u^{2}}\right)^{\frac{3}{2}} 
\left( \frac{\alpha^{'}+\beta^{'}}{\alpha^{'}+\beta^{'}+ 
\frac{(\alpha+\beta)u^{2}}{\alpha+\beta+u^{2}}}\right)^{\frac{3}{2}} \\
&= \left( \frac{(\alpha+\beta)(\alpha^{'}+\beta^{'})}
{(\alpha+\beta+u^{2})(\alpha^{'}+\beta^{'})+(\alpha+\beta)u^{2}}\right)^{\frac{3}{2}}\\
&= \left(\frac{1}{1+\frac{u^{2}}{\rho}}\right)^{\frac{3}{2}} \\
&= \left(\frac{\rho}{\rho+u^{2}}\right)^{\frac{3}{2}}
\end{split}
 \label{OS_ERI_eq:33}
\end{equation}
So after integrate over $r$ and $r^{'}$ the bottom integral becomes:
\begin{equation}
 \begin{split}
 (00|00)^{(m)} &=\frac{2}{\sqrt{\pi}}(0_{A}|0_{B})(0_{C}|0_{D})
 \int_{0}^{\infty} du \left( \frac{u^{2}}{\rho+u^{2}}\right)^{m}
 \left(\frac{\rho}{\rho+u^{2}}\right)^{\frac{3}{2}}
 e^{-\frac{\rho u^{2}}{\rho+ u^{2}}|PQ|^{2}} 
 \end{split}
\label{OS_ERI_eq:34}
\end{equation}
If we set $t^{2} = \frac{u^{2}}{\rho+ u^{2}}$, and it's easy to know
that 
\begin{equation}
 du = (1-t^{2})^{-\frac{3}{2}}\rho^{1/2} dt
 \label{OS_ERI_eq:35}
\end{equation}
so \ref{OS_ERI_eq:34} becomes:
\begin{equation}
 \begin{split}
 (00|00)^{(m)} &= 2\left( \frac{\rho}{\pi}\right)^{\frac{1}{2}}(0_{A}|0_{B})
(0_{C}|0_{D})\int^{1}_{0} t^{2m} e^{-(\rho|PQ|^{2})t^{2}} dt 
 \end{split}
\label{OS_ERI_complementary_result}
\end{equation}

Where we have the arguments as:
\begin{align}
 \overrightarrow{P} &= \frac{\alpha \overrightarrow{A} + \beta
\overrightarrow{B}}{\alpha + \beta} \nonumber \\
\overrightarrow{Q} &= \frac{\alpha^{'} \overrightarrow{C} + \beta^{'}
\overrightarrow{D}}{\alpha^{'} + \beta^{'}} \nonumber \\
\rho &= \frac{(\alpha+\beta)(\alpha^{'}+\beta^{'})}
{(\alpha+\beta)+(\alpha^{'}+\beta^{'})}
\end{align}
    
%%%%%%%%%%%%%%%%%%%%%%%%%%%%%%%%%%%%%%%%%%%%%%%%%%%%%%%%%%%%%%%%%%%%%%%%%%%%%%%%
\subsection{Overlap Integral}
%
%
%
the basic overlap integral is:
\begin{equation}
 \label{OS_overlap_eq:1}
I = \int dr \chi_{a}(r) \chi_{b}(r)
\end{equation}

By using the recursive relation derived in the three center overlap
integral (see \ref{OS_three_overlap_int_eq:12}), set the middle Gaussian
primitive to be zero; we can get the recursive relation for overlap:
\begin{equation}
\begin{split}
(a+\iota_{i}|b) &= 
(P_{i} - A_{i})(a|b) + 
N_{i}(A)\left(\frac{1}{2(\alpha+\beta)}\right)(a-\iota_{i}|b) \\
&+ 
N_{i}(B)\left(\frac{1}{2(\alpha+\beta)}\right)(a|b-\iota_{i})  
\end{split}
\label{OS_overlap_result}
\end{equation}

%%%%%%%%%%%%%%%%%%%%%%%%%%%%%%%%%%%%%%%%%%%%%%%%%%%%%%%%%%%%%%%%%%%%%%%%%%%%%%%%
\subsection{Derivatives of Basis Sets in Overlap}
%
%
%
Before we derive the recursive relation for kinetic energy, let's firstly 
go to see another type of integral of 
$\left( \dfrac{\partial a}{\partial x}|b\right)$.

Firstly, we note that the derivatives can not directly be applied to the 
recursive relation like \ref{OS_overlap_result}:
\begin{equation}
\begin{split}
(\frac{\partial (a+\iota_{i})}{\partial x}|b) & \neq 
(P_{i} - A_{i})(\frac{\partial a}{\partial x}|b) + 
N_{i}(A)\left(\frac{1}{2(\alpha+\beta)}\right)(\frac{\partial(a-\iota_{i})}{\partial x}|b) \\
&+ 
N_{i}(B)\left(\frac{1}{2(\alpha+\beta)}\right)(\frac{\partial a}{\partial x}|b-\iota_{i})  
\end{split}
\end{equation}

Let's prove this with some simple example. Suggest we have two Gaussian 
primitives:
\begin{align}
 \chi_{i} &= x^{l_{i}}y^{m_{i}}z^{n_{i}} e^{-\alpha_{i} r^{2}} \nonumber \\
 \chi_{j} &= x^{l_{j}}y^{m_{j}}z^{n_{j}} e^{-\alpha_{j} r^{2}} \nonumber \\
 \gamma   &= \alpha_{i} + \alpha_{j}
 \end{align}

According to the \ref{derivative_overlap_direct_int_eq:2} and recursive relation
for the overlap integral, we can have:
\begin{equation}
 \begin{split}
 \left(\frac{\partial \chi_{i}}{\partial x}|\chi_{j}\right)  &= 
 l_{i}(\chi_{i}^{l-1mn}|\chi_{j}) - 2\alpha_{i} (\chi_{i}^{l+1mn}|\chi_{j}) \\
 &= l_{i}\left[ PA_{k}(\chi_{i}^{l-1mn}-\iota_{k}|\chi_{j}) + 
 \frac{N_{k}(A)}{2 \gamma}(\chi_{i}^{l-1mn}-2\iota_{k}|\chi_{j})  \right. \\ 
 &+ \left. \frac{N_{k}(B)}{2 \gamma}(\chi_{i}^{l-1mn}-\iota_{k}|\chi_{j}-\iota_{k})
 \right] \\
 &-2\alpha_{i} \left[ PA_{k}(\chi_{i}^{l+1mn}-\iota_{k}|\chi_{j}) + 
 \frac{N_{k}(A)}{2 \gamma}(\chi_{i}^{l+1mn}-2\iota_{k}|\chi_{j}) \right. \\ 
 &+ \left. \frac{N_{k}(B)}{2 \gamma}(\chi_{i}^{l+1mn}-\iota_{k}|\chi_{j}-\iota_{k})
 \right] 
 \end{split}
 \label{deriv_os_overlap:1}
\end{equation}
Here the index of $i$ and $j$ are omitted for l, m and n. If the l, m and n do not 
change, they are not listed neither.

It's easy to see that if $k = y$ or $k = z$ then the above equation could be 
written as:
\begin{equation}
 \begin{split}
  \left(\frac{\partial \chi_{i}}{\partial x}|\chi_{j}\right)  &= 
 l_{i}(\chi_{i}^{l-1mn}|\chi_{j}) - 2\alpha_{i} (\chi_{i}^{l+1mn}|\chi_{j}) \\ 
 &=  PA_{k}(\frac{\partial(\chi_{i}^{lmn}-\iota_{k})}{\partial x}|\chi_{j}) + 
 \frac{N_{k}(A)}{2 \gamma}(\frac{\partial(\chi_{i}^{lmn}-2\iota_{k})}{\partial x}|\chi_{j}) 
 \\ 
 &+ \frac{N_{k}(B)}{2 \gamma}
 (\frac{\partial (\chi_{i}^{lmn}-\iota_{k}) }{\partial x}|\chi_{j}-\iota_{k}) 
 \end{split}
\end{equation}
However, if $k = x$, the above equation does not hold.

For $k = x$, the \ref{deriv_os_overlap:1} could be expressed as:
\begin{equation}
 \begin{split}
 \left(\frac{\partial \chi_{i}}{\partial x}|\chi_{j}\right)  &= 
 l_{i}(\chi_{i}^{l-1mn}|\chi_{j}) - 2\alpha_{i} (\chi_{i}^{l+1mn}|\chi_{j}) \\
 &= l_{i}\left[ PA_{x}(\chi_{i}^{l-2mn}|\chi_{j}) + 
 \frac{N_{x}(A)}{2 \gamma}(\chi_{i}^{l-3mn}|\chi_{j})  \right. \\ 
 &+ \left. \frac{N_{x}(B)}{2 \gamma}(\chi_{i}^{l-2mn}|\chi_{j}^{l-1mn})
 \right] \\
 &-2\alpha_{i} \left[ PA_{x}(\chi_{i}|\chi_{j}) + 
 \frac{N_{x}(A)}{2 \gamma}(\chi_{i}^{l-1mn}|\chi_{j}) \right. \\ 
 &+ \left. \frac{N_{x}(B)}{2 \gamma}(\chi_{i}|\chi_{j}^{l-1mn})
 \right] \\
 &= PA_{x} \left( l_{i}(\chi_{i}^{l-2mn}|\chi_{j}) -2\alpha_{i} (\chi_{i}|\chi_{j}) 
 \right) \\
 &+ \frac{N_{x}(A)}{2 \gamma}\left(
 l_{i}(\chi_{i}^{l-3mn}|\chi_{j}) -2\alpha_{i} (\chi_{i}^{l-1mn}|\chi_{j})
 \right) \\
 &+ \frac{N_{x}(B)}{2 \gamma}\left(
 l_{i}(\chi_{i}^{l-2mn}|\chi_{j}^{l-1mn}) -2\alpha_{i} (\chi_{i}|\chi_{j}^{l-1mn})
 \right) \\
 &= PA_{x} (\frac{\partial \chi_{i}^{l-1mn}}{\partial x}|\chi_{j}) + 
 PA_{x} (\chi_{i}^{l-1mn}|\chi_{j}) \\
 &+ \frac{N_{x}(A)}{2 \gamma}(\frac{\partial \chi_{i}^{l-2mn}}{\partial x}|\chi_{j})
 + 2\frac{N_{x}(A)}{2 \gamma}(\chi_{i}^{l-2mn}|\chi_{j}) \\
 &+ \frac{N_{x}(B)}{2 \gamma}(\frac{\partial \chi_{i}^{l-1mn}}{\partial x}|\chi_{j}^{l-1mn})
 + \frac{N_{x}(B)}{2 \gamma}(\chi_{i}^{l-1mn}|\chi_{j}^{l-1mn}) \\
 &= PA_{x} (\frac{\partial \chi_{i}^{l-1mn}}{\partial x}|\chi_{j}) + 
 \frac{N_{x}(A)}{2 \gamma}(\frac{\partial \chi_{i}^{l-2mn}}{\partial x}|\chi_{j}) \\
 &+ \frac{N_{x}(B)}{2 \gamma}(\frac{\partial \chi_{i}^{l-1mn}}{\partial x}|\chi_{j}^{l-1mn}) \\
 &+ PA_{x} (\chi_{i}^{l-1mn}|\chi_{j}) + \frac{N_{x}(A)}{2 \gamma}(\chi_{i}^{l-2mn}|\chi_{j}) \\
 &+ \frac{N_{x}(B)}{2 \gamma}(\chi_{i}^{l-1mn}|\chi_{j}^{l-1mn}) \\
 &+ \frac{N_{x}(A)}{2 \gamma}(\chi_{i}^{l-2mn}|\chi_{j})
 \end{split}
 \label{deriv_os_overlap:2}
\end{equation}
Therefore, we have additional terms in \ref{deriv_os_overlap:2}.


%%%%%%%%%%%%%%%%%%%%%%%%%%%%%%%%%%%%%%%%%%%%%%%%%%%%%%%%%%%%%%%%%%%%%%%%%%%%%%%%
\subsection{Kinetic Energy Integral}
%
%
%
According to the \ref{kinetic_direct_int_eq:2}, the kinetic energy integral is:
\begin{align}
 \label{OS_kinetic_eq:1}
I &= \frac{1}{2}\int dr (\nabla\chi_{i}(r)) \cdotp (\nabla\chi_{j}(r)) \nonumber \\
  &= \frac{1}{2}\int d x \frac{\partial \chi_{i}(r)}{\partial x}
  \frac{\partial \chi_{j}(r)}{\partial x} 
  +  \frac{1}{2}\int d y \frac{\partial \chi_{i}(r)}{\partial y}
  \frac{\partial \chi_{j}(r)}{\partial y} 
  +  \frac{1}{2}\int d z \frac{\partial \chi_{i}(r)}{\partial z}
  \frac{\partial \chi_{j}(r)}{\partial z} 
\end{align}
So the kinetic integral is actually a linear combination of overlap integrals,
and the kinetic integral itself could be decomposed into three integrals sum:
\begin{equation}
 I = I_{x} + I_{y} + I_{z}
\end{equation}
In the following derivation, we will concentrate on one component, which is 
in general named as $I_{i}$.

For a general Gaussian primitive function of $\chi = x^{l}y^{m}z^{n}e^{-\alpha r^{2}}$, who 
centers on $A$; we can express its derivatives as:
\begin{align}
 \frac{\partial \chi^{lmn}_{A}}{\partial v} &= 
 N_{i}(A)(\chi_{A}^{lmn}-\iota_{i}) - 2\alpha(\chi_{A}^{lmn}+\iota_{i}) \nonumber \\
 &= N_{i}(A)(-1)_{i} - 2\alpha(+1)_{i}
 \label{OS_kinetic_eq:0}
\end{align}
$v$ denotes the possible derivatives, could be x, y or z. 

Therefore, suggest that in \ref{OS_kinetic_eq:1} the $\chi_{i}$ resides 
on $A$ with $\alpha$ exponential factor, and $\chi_{j}$ is on $B$ with
$\beta$ exponential factor; we can have:
\begin{equation}
\begin{split}
\left(  \frac{\partial \chi_{A}}{\partial v}|\frac{\partial \chi_{B}}{\partial v}
\right)  &= N_{i}(A)N_{i}(B)(-1|-1)_{i} -2\alpha N_{i}(B)(+1|-1)_{i} \\
&-2\beta N_{i}(A)(-1|+1)_{i} + 4\alpha\beta(+1|+1)_{i}
\end{split}
\label{OS_kinetic_eq:2}
\end{equation}
the $i$ is determined from the v. If v is x, then $N_{i}(A) = l_{A}$ and 
$N_{i}(B) = l_{B}$; if v is y, then $N_{i}(A) = m_{A}$ and 
$N_{i}(B) = m_{B}$ and if v is z, then $N_{i}(A) = n_{A}$ and 
$N_{i}(B) = n_{B}$. 

By expanding all of overlap integrals into its RR form, we can have:
\begin{equation}
 \begin{split}
 \left(  \frac{\partial \chi_{A}}{\partial v}|\frac{\partial \chi_{B}}{\partial v}
\right)  &=  N_{i}(A)N_{i}(B)
\left[ PA_{k}(-1-\iota_{k}|-1)_{i} + \frac{N_{k}(A)}{2 \gamma}(-1-2\iota_{k}|-1)_{i} \right. \\ 
 &+ \left. \frac{N_{k}(B)}{2 \gamma}(-1-\iota_{k}|-1-\iota_{k})_{i} \right] \\
&- 2\alpha N_{i}(B)
\left[ PA_{k}(+1-\iota_{k}|-1)_{i} + \frac{N_{k}(A)}{2 \gamma}(+1-2\iota_{k}|-1)_{i} \right. \\ 
 &+ \left. \frac{N_{k}(B)}{2 \gamma}(+1-\iota_{k}|-1-\iota_{k})_{i} \right] \\ 
&- 2\beta  N_{i}(A)
\left[ PA_{k}(-1-\iota_{k}|+1)_{i} + \frac{N_{k}(A)}{2 \gamma}(-1-2\iota_{k}|+1)_{i} \right. \\ 
 &+ \left. \frac{N_{k}(B)}{2 \gamma}(-1-\iota_{k}|+1-\iota_{k})_{i} \right] \\  
&+ 4\alpha\beta
\left[ PA_{k}(+1-\iota_{k}|+1)_{i} + \frac{N_{k}(A)}{2 \gamma}(+1-2\iota_{k}|+1)_{i} \right. \\ 
 &+ \left. \frac{N_{k}(B)}{2 \gamma}(+1-\iota_{k}|+1-\iota_{k})_{i} \right] 
 \end{split}
 \label{OS_kinetic_eq:3}
\end{equation}
where $\gamma = \alpha + \beta$. $k$ denotes possible RR direction, which is 
also x, y or z. $k$ is different from the $v$ here.

Firstly, let's go to see a simple case that the derivatives direction $i$ is not
same with RR direction $k$. It's easy to see, in this case the above result
turns into:
\begin{equation}
 \begin{split}
 \left(  \frac{\partial \chi_{A}}{\partial v}|\frac{\partial \chi_{B}}{\partial v}
\right)  &= PA_{k}
 \left(\frac{\partial (\chi_{A}-\iota_{k})}{\partial v}|
 \frac{\partial \chi_{B}}{\partial v} \right) + 
\frac{N_{k}(A)}{2 \gamma}
 \left(\frac{\partial (\chi_{A}-2\iota_{k})}{\partial v}|
 \frac{\partial \chi_{B}}{\partial v} \right) \\
&+\frac{N_{k}(B)}{2 \gamma}
 \left(\frac{\partial (\chi_{A}-\iota_{k})}{\partial v}|
 \frac{\partial(\chi_{B}-\iota_{k})}{\partial v} \right) 
 \end{split}
 \label{OS_kinetic_eq:4} 
\end{equation}

For the case that RR direction $k$ is same with derivatives direction of $i$, according
to the equation of \ref{OS_kinetic_eq:3}, we have:
\begin{equation}
 \begin{split}
 \left(  \frac{\partial \chi_{A}}{\partial v_{i=k}}|
 \frac{\partial \chi_{B}}{\partial v_{i=k}} \right)  &=  N_{i}(A)N_{i}(B)
\left[ PA_{i}(-2|-1)_{i} + \frac{N_{i}(A)-2}{2 \gamma}(-3|-1)_{i} \right. \\ 
 &+ \left. \frac{N_{i}(B)-1}{2 \gamma}(-2|-2)_{i} \right] \\
&- 2\alpha N_{i}(B)
\left[ PA_{i}(0|-1)_{i} + \frac{N_{i}(A)}{2 \gamma}(-1|-1)_{i} \right. \\ 
 &+ \left. \frac{N_{i}(B)-1}{2 \gamma}(0|-2)_{i} \right] \\ 
&- 2\beta  N_{i}(A)
\left[ PA_{i}(-2|+1)_{i} + \frac{N_{i}(A)-2}{2 \gamma}(-3|+1)_{i} \right. \\ 
 &+ \left. \frac{N_{i}(B)+1}{2 \gamma}(-2|0)_{i} \right] \\  
&+ 4\alpha\beta
\left[ PA_{i}(0|+1)_{i} + \frac{N_{i}(A)}{2 \gamma}(-1|+1)_{i} \right. \\ 
 &+ \left. \frac{N_{i}(B)+1}{2 \gamma}(0|0)_{i} \right] \\
&= PA_{i} \Bigg\{   
(N_{i}(A)-1)N_{i}(B)(-2|-1)_{i} - 2\alpha N_{i}(B)(0|-1)_{i}  \\
&- 2\beta(N_{i}(A)-1)(-2|+1)_{i} +4\alpha\beta (0|+1)_{i} \Bigg\}  \\
&+ PA_{i}\Bigg\{ N_{i}(B)(-2|-1)_{i} -2\beta(-2|+1)_{i} \Bigg\} \\
&+\frac{N_{i}(A)}{2 \gamma} \Bigg\{ 
(N_{i}(A)-2)N_{i}(B)(-3|-1)_{i} - 2\alpha N_{i}(B)(-1|-1)_{i}  \\
&-2\beta(N_{i}(A)-2)(-3|+1)_{i} + 4\alpha\beta(-1|+1)_{i} \Bigg \} \\
&+\frac{N_{i}(B)}{2 \gamma} \Bigg\{ 
(N_{i}(A)-1)(N_{i}(B)-1)(-2|-2)_{i} - 2\alpha (N_{i}(B)-1)(0|-2)_{i} \\ 
&- 2\beta(N_{i}(A)-1)(-2|0)_{i} + 4\alpha\beta(0|0)_{i} \Bigg\} \\
&+\frac{N_{i}(B)(N_{i}(B)-1)}{2 \gamma}(-2|-2)_{i} - 
\frac{2\beta N_{i}(A)}{2 \gamma} (-2|0)_{i} \\
&-\frac{2\beta N_{i}(B)}{2 \gamma} (-2|0)_{i}
\end{split}
 \label{OS_kinetic_eq:5}
\end{equation}

The above expression could be re-fomulated into more clearly way:
\begin{equation}
 \begin{split}
 \left(  \frac{\partial \chi_{A}}{\partial v_{i=k}}|
 \frac{\partial \chi_{B}}{\partial v_{i=k}} \right)  &=
 PA_{i}
 \left(\frac{\partial (\chi_{A}-\iota_{i})}{\partial v}|
 \frac{\partial \chi_{B}}{\partial v} \right) + 
\frac{N_{i}(A)}{2 \gamma}
 \left(\frac{\partial (\chi_{A}-2\iota_{i})}{\partial v}|
 \frac{\partial \chi_{B}}{\partial v} \right) \\
&+\frac{N_{i}(B)}{2 \gamma}
 \left(\frac{\partial (\chi_{A}-\iota_{i})}{\partial v}|
 \frac{\partial(\chi_{B}-\iota_{i})}{\partial v} \right) \\
&+ PA_{i}\left[ N_{i}(B)(-2|-1)_{i} -2\beta(-2|+1)_{i} \right] \\
&+\frac{N_{i}(B)(N_{i}(B)-1)}{2 \gamma}(-2|-2)_{i} - 
\frac{2\beta N_{i}(A)}{2 \gamma} (-2|0)_{i} \\
&-\frac{2\beta N_{i}(B)}{2 \gamma} (-2|0)_{i}
\end{split}
 \label{OS_kinetic_eq:6}
\end{equation}


\begin{comment}


%

\begin{equation}
 \begin{split}
 \left(  \frac{\partial \chi_{A}}{\partial v_{i=k}}|
 \frac{\partial \chi_{B}}{\partial v_{i=k}} \right)  &=  N_{i}(A)N_{i}(B)
\left[ PA_{i}(-2|-1)_{i} + \frac{N_{i}(A)-2}{2 \gamma}(-3|-1)_{i} \right. \\ 
 &+ \left. \frac{N_{i}(B)-1}{2 \gamma}(-2|-2)_{i} \right] \\
&- 2\alpha N_{i}(B)
\left[ PA_{i}(0|-1)_{i} + \frac{N_{i}(A)}{2 \gamma}(-1|-1)_{i} \right. \\ 
 &+ \left. \frac{N_{i}(B)-1}{2 \gamma}(0|-2)_{i} \right] \\ 
&- 2\beta  N_{i}(A)
\left[ PA_{i}(-2|+1)_{i} + \frac{N_{i}(A)-2}{2 \gamma}(-3|+1)_{i} \right. \\ 
 &+ \left. \frac{N_{i}(B)+1}{2 \gamma}(-2|0)_{i} \right] \\  
&+ 4\alpha\beta
\left[ PA_{i}(0|+1)_{i} + \frac{N_{i}(A)}{2 \gamma}(-1|+1)_{i} \right. \\ 
 &+ \left. \frac{N_{i}(B)+1}{2 \gamma}(0|0)_{i} \right] \\
&= PA_{i}
 \left(\frac{\partial (\chi_{A}-\iota_{i})}{\partial v_{i}}|
 \frac{\partial \chi_{B}}{\partial v_{i}} \right) + 
\frac{N_{i}(A)}{2 \gamma}
 \left(\frac{\partial (\chi_{A}-2\iota_{i})}{\partial v_{i}}|
 \frac{\partial \chi_{B}}{\partial v_{i}} \right) \\
&+\frac{N_{i}(B)}{2 \gamma}
 \left(\frac{\partial (\chi_{A}-\iota_{i})}{\partial v_{i}}|
 \frac{\partial(\chi_{B}-\iota_{i})}{\partial v_{i}} \right) \\
&+\Biggl( 
-\frac{2 N_{i}(A)N_{i}(B)}{2\gamma}(-3|-1)_{i} - 
\frac{N_{i}(A)N_{i}(B)}{2\gamma}(-2|-2)_{i} \\
&-
\frac{2 \alpha N_{i}(B)}{2\gamma}(0|-2)_{i} +
\frac{4 \beta N_{i}(A)}{2\gamma}(-3|+1)_{i} \\
&-
\frac{2 \beta N_{i}(A)}{2\gamma}(-2|0)_{i}  +
\frac{4 \alpha\beta}{2\gamma}(0|0)_{i}
\Biggr)
 \end{split}
 \label{OS_kinetic_eq:5}
\end{equation}

\end{comment}

%%%%%%%%%%%%%%%%%%%%%%%%%%%%%%%%%%%%%%%%%%%%%%%%%%%%%%%%%%%%%%%%%%%%%%%%%%%%%%%%
\subsection{Three Body Kinetic Integral}
%
%
%
According to the \ref{kinetic_direct_int_eq:2}, the three body kinetic enegy
could also be expressed as:
\begin{equation}
\frac{1}{2}\sum_{v = x, y, z}\int \chi_{A}(r)\chi_{B}(r)
\frac{\partial^{2}}{\partial v^{2}} \chi_{C}(r) dr
= \frac{1}{2}\int \nabla(\chi_{A}(r)\chi_{B}(r)) \cdot \nabla\chi_{C}(r) dr
 \label{three_body_kinetic_integral_eq:1}
\end{equation}

According to the \ref{OS_kinetic_eq:0}, we have:
\begin{align}
 \frac{\partial \chi^{lmn}_{A}}{\partial v} &= 
 N_{i}(A)(\chi_{A}^{lmn}-\iota_{i}) - 2\alpha(\chi_{A}^{lmn}+\iota_{i}) \nonumber \\
 &= N_{i}(A)(-1)_{i} - 2\alpha(+1)_{i}
 \label{three_body_kinetic_integral_eq:2}
\end{align}
We can further generate derivtives for \ref{three_body_kinetic_integral_eq:1}
as:
\begin{equation}
 \begin{split}
&\int \nabla(\chi_{A}(r)\chi_{B}(r)) \cdot \nabla\chi_{C}(r) dr \\
&= \int \left[ (\nabla\chi_{A}(r)\chi_{B}(r) + \chi_{A}(r)\nabla\chi_{B}(r)\right] 
\cdot \nabla\chi_{C}(r) dr \\
&= \sum_{i = x, y, z} \int
[ N_{i}(A)(\chi_{A}-\iota_{i}) - 2\alpha(\chi_{A}+\iota_{i})] \chi_{B} \\
&[ N_{i}(C)(\chi_{C}-\iota_{i}) - 2\gamma(\chi_{C}+\iota_{i})] dr \\
&+ \sum_{i = x, y, z} \int 
\chi_{A}(r) \left[ 
N_{i}(B)(\chi_{B}-\iota_{i}) - 2\beta(\chi_{B}+\iota_{i})\right] \\
&\left[ N_{i}(C)(\chi_{C}-\iota_{i}) - 2\gamma(\chi_{C}+\iota_{i})\right]  dr
 \end{split}
\label{three_body_kinetic_integral_eq:3}
\end{equation}
$\alpha$ is the expotential factor for $\chi_{A}$, $\beta$ is for $\chi_{B}$
and $\gamma$ is for $\chi_{C}$.

Now let's expand all the terms:
\begin{equation}
 \begin{split}
&\int \nabla(\chi_{A}(r)\chi_{B}(r)) \cdot \nabla\chi_{C}(r) dr \\
&= \sum_{i = x, y, z} \int
[ N_{i}(A)(\chi_{A}-\iota_{i}) - 2\alpha(\chi_{A}+\iota_{i})] \chi_{B} \\
&[ N_{i}(C)(\chi_{C}-\iota_{i}) - 2\gamma(\chi_{C}+\iota_{i})] dr \\
&+ \sum_{i = x, y, z} \int 
\chi_{A}(r) \left[ 
N_{i}(B)(\chi_{B}-\iota_{i}) - 2\beta(\chi_{B}+\iota_{i})\right] \\
&\left[ N_{i}(C)(\chi_{C}-\iota_{i}) - 2\gamma(\chi_{C}+\iota_{i})\right]  dr \\
&= \sum_{i = x, y, z} \int 
 N_{i}(A)N_{i}(C)(\chi_{A}-\iota_{i})\chi_{B}(\chi_{C}-\iota_{i}) dr \\
&- \sum_{i = x, y, z} \int 
 2\alpha N_{i}(C)(\chi_{A}+\iota_{i})\chi_{B}(\chi_{C}-\iota_{i}) dr \\
&- \sum_{i = x, y, z} \int 
 2\gamma N_{i}(A)(\chi_{A}-\iota_{i})\chi_{B}(\chi_{C}+\iota_{i}) dr \\
&+ \sum_{i = x, y, z} \int 
 4\alpha\gamma   (\chi_{A}+\iota_{i})\chi_{B}(\chi_{C}+\iota_{i}) dr \\ 
&+ \sum_{i = x, y, z} \int 
 N_{i}(B)N_{i}(C)\chi_{A}(\chi_{B}-\iota_{i})(\chi_{C}-\iota_{i}) dr \\
&- \sum_{i = x, y, z} \int 
 2\beta  N_{i}(C)\chi_{A}(\chi_{B}+\iota_{i})(\chi_{C}-\iota_{i}) dr \\
&- \sum_{i = x, y, z} \int 
 2\gamma N_{i}(B)\chi_{A}(\chi_{B}-\iota_{i})(\chi_{C}+\iota_{i}) dr \\
&+ \sum_{i = x, y, z} \int 
 4\beta\gamma    \chi_{A}(\chi_{B}+\iota_{i})(\chi_{C}+\iota_{i}) dr  
 \end{split}
\label{three_body_kinetic_integral_eq:4}
\end{equation}

Particularly, for the bottom $SSS$ type integrals, it has the expression as:
\begin{equation}
\begin{split}
 &\int \nabla(0_{A}0_{B}) \cdot \nabla 0_{C} dr \\
 &= \sum_{i = x, y, z} \int 
 4\alpha\gamma   (0_{A}+\iota_{i})0_{B}(0_{C}+\iota_{i}) dr \\ 
 &+ \sum_{i = x, y, z} \int 
 4\beta\gamma    0_{A}(0_{B}+\iota_{i})(0_{C}+\iota_{i}) dr \\
\end{split}
\label{three_body_kinetic_integral_eq:5}
\end{equation}
By using the RR expression for three body overlap integral, it becomes:
\begin{equation}
 \begin{split}
 &\int \nabla(0_{A}0_{B}) \cdot \nabla 0_{C} dr \\
 &= \sum_{i = x, y, z} \int 
 4\alpha\gamma   (0_{A}+\iota_{i})0_{B}(0_{C}+\iota_{i}) dr \\ 
 &+ \sum_{i = x, y, z} \int 
 4\beta\gamma    0_{A}(0_{B}+\iota_{i})(0_{C}+\iota_{i}) dr \\
 &= \sum_{i = x, y, z} \int 
 4\alpha\gamma\left\lbrace (G_{i}-C_{i})(0_{A}+\iota_{i})0_{B}0_{C}
 +\frac{1}{2(\alpha+\beta+\gamma)}0_{A}0_{B}0_{C} \right\rbrace dr \\
 &+ \sum_{i = x, y, z} \int 
 4\beta\gamma \left\lbrace (G_{i}-C_{i})0_{A}(0_{B}+\iota_{i})0_{C}
 +\frac{1}{2(\alpha+\beta+\gamma)}0_{A}0_{B}0_{C} \right\rbrace dr \\
 &= \sum_{i = x, y, z} \int 
 4\alpha\gamma\left\lbrace (G_{i}-C_{i})(G_{i}-A_{i})0_{A}0_{B}0_{C}
 +\frac{1}{2(\alpha+\beta+\gamma)}0_{A}0_{B}0_{C} \right\rbrace dr \\
 &+ \sum_{i = x, y, z} \int 
 4\beta\gamma \left\lbrace (G_{i}-C_{i})(G_{i}-B_{i})0_{A}0_{B}0_{C}
 +\frac{1}{2(\alpha+\beta+\gamma)}0_{A}0_{B}0_{C} \right\rbrace dr \\ 
 &= \sum_{i = x, y, z} \left\lbrace 
 4\gamma(G_{i}-C_{i}) \left[ \alpha(G_{i}-A_{i}) + \beta(G_{i}-B_{i})\right] 
 + \frac{2\gamma(\alpha+\beta)}{\alpha+\beta+\gamma}
 \right\rbrace \int 0_{A}0_{B}0_{C} dr
\end{split}
\label{three_body_kinetic_integral_eq:6}
\end{equation}


\begin{comment}
 &+ \sum_{i = x, y, z} \int \Bigl{ 
\chi_{A}(r) \left[ 
N_{i}(B)(\chi_{B}-\iota_{i}) - 2\beta(\chi_{B}+\iota_{i})\right]
\left[ N_{i}(C)(\chi_{C}-\iota_{i}) - 2\gamma(\chi_{C}+\iota_{i})\right] \Bigr} dr
\end{comment}




%%%%%%%%%%%%%%%%%%%%%%%%%%%%%%%%%%%%%%%%%%%%%%%%%%%%%%%%%%%%%%%%%%%%%%%%%%%%%%%%
\subsection{Nuclear Attraction Integral}
%
%
%
In the section of \ref{direct_NAI_derivation}, we have derived the nuclear
attraction integral explicitly. Here in this section, we will use the recursive
relation to derive a simple algorithm to calculate the NAI.

As we know, a general NAI is expressed as:
\begin{equation}
 \begin{split}
  (a|\Lambda_{c}|b) &= \int \chi_{i}(r)\frac{1}{r_{C}}\chi_{j}(r) dr \\
&= \int x^{l_{A}}_{A}y^{m_{A}}_{A}z^{n_{A}}_{A}e^{-\alpha r_{A}^{2}}
        \frac{1}{r_{C}}
        x^{l_{B}}_{B}y^{m_{B}}_{B}z^{n_{B}}_{B}e^{-\beta  r_{B}^{2}} dr
 \end{split}
\end{equation}
$r_{A}, r_{B}$ and $r_{c}$ denotes the distance between electron position and
the given nuclear (A, B or C).

Through transformation, the $\frac{1}{r_{C}}$ becomes:
\begin{equation}
\label{OS_nuclear_eq:1}
\frac{1}{r_{C}} = \frac{2}{\pi^{1/2}}\int^{\infty}_{0}e^{-u^{2}(r-C)^{2}} du
\end{equation}
Then it's fully converted into a S type of primitive function.

Therefore, for the $(a|0_{c}|b)$ it could be expressed as:
\begin{equation}
 \label{OS_nuclear_eq:2}
\begin{split}
(a|\Lambda_{c}|b) &= \frac{2}{\pi^{1/2}}\int^{\infty}_{0} du
 \int dr x^{l_{A}}_{A}y^{m_{A}}_{A}z^{n_{A}}_{A}e^{-\alpha r_{A}^{2}}
        e^{-u^{2}(r-C)^{2}}
        x^{l_{B}}_{B}y^{m_{B}}_{B}z^{n_{B}}_{B}e^{-\beta  r_{B}^{2}} \\
&= \frac{2}{\pi^{1/2}}\int^{\infty}_{0} du  (a|0_{c}|b)
\end{split}
\end{equation}
Then the NAI is converted back to the three center overlap integral.

According to the recursive relation in \ref{OS_three_overlap_int_eq:12},
the $(a|0_{c}|b)$ could be expressed as:
\begin{equation}
 \begin{split}
  (a+\iota_{i}|0_{c}|b) 
&= \left( \frac{\alpha A_{i} + \beta B_{i} + u^{2}C_{i}}{\alpha+\beta+u^{2}} -
A_{i}\right) (a|0_{c}|b)  \\
&+ 
N_{i}(A)\left(\frac{1}{2(\alpha+\beta+u^{2})}\right)(a-\iota_{i}|0_{c}|b) \\
&+
N_{i}(C)\left(\frac{1}{2(\alpha+\beta+u^{2})}\right)(a|0_{c}|b-\iota_{i}) \\
&= \left( P_{i}  - A_{i}\right) (a|0_{c}|b) - (P_{i} - C_{i})
\frac{u^{2}}{\alpha+\beta+u^{2}}(a|0_{c}|b)  \\
&+ 
\frac{N_{i}(A)}{2(\alpha+\beta)}\left(1-\frac{u^{2}}{(\alpha+\beta+u^{2})}
\right)(a-\iota_{i}|0_{ c }|b) \\
&+
\frac{N_{i}(C)}{2(\alpha+\beta)}\left(1-\frac{u^{2}}{(\alpha+\beta+u^{2})}
\right)(a|0_{c}|b-\iota_{i}) \\
 \end{split}
\label{OS_nuclear_eq:3}
\end{equation}
Where $P_{i}$ is:
\begin{equation}
 P_{i} = \frac{\alpha A_{i} + \beta B_{i}}{\alpha+\beta}
\end{equation}
Here, we are carefully modifying the expression so that to make the result
similar to the one we derived in ERI(see \ref{OS_ERI_eq:27}). Therefore,
here for \ref{OS_nuclear_eq:3} a similar auxiliary function (see
\ref{OS_ERI_eq:28}) could be defined($\epsilon = \alpha+\beta$):
\begin{equation}
\label{OS_nuclear_eq:4}
  (a|0_{c}|b)^{(m)} = \frac{2}{\sqrt{\pi}}\int du \left( \frac{u^{2}}
{\epsilon+u^{2}}\right)^{m}(ab|u|cd) 
\end{equation}
and we have:
\begin{equation}
 \begin{split}
  (a+\iota_{i}|0_{c}|b)^{(m)} &=  
\left( P_{i}  - A_{i}\right) (a|0_{c}|b)^{(m)} - (P_{i} - C_{i})
(a|0_{c}|b)^{(m+1)}  \\
&+ 
\frac{N_{i}(A)}{2\epsilon}\left(
(a-\iota_{i}|0_{ c }|b)^{(m)}-
(a-\iota_{i}|0_{ c }|b)^{(m+1)}
\right) \\
&+
\frac{N_{i}(C)}{2\epsilon}\left(
(a|0_{c}|b-\iota_{i})^{(m)}-
(a|0_{c}|b-\iota_{i})^{(m+1)}
\right) \\
 \end{split}
\label{OS_nuclear_eq:5}
\end{equation}
The corresponding integral of $(0_{a}|0_{c}|0_{b})^{(m)}$ is:
\begin{equation}
\label{OS_nuclear_eq:6}
 (0_{a}|0_{c}|0_{b})^{(m)} = 2\left(
\frac{\epsilon}{\pi}\right)^{1/2}(0_{a}|0_{b})
\int^{1}_{0} t^{2m} e^{-(\epsilon|PC|^{2})t^{2}} dt 
\end{equation}


%%%%%%%%%%%%%%%%%%%%%%%%%%%%%%%%%%%%%%%%%%%%%%%%%%%%%%%%%%%%%%%%%%%%%%%%%%%%%%%%
\begin{comment}
%%%
%%%  the description here is not quite what we want. So just screen the
%    contents.
%%%

\subsection{Implementation Considerations}
%
%
%
Now let's give some deep discussion on how to realize the algorithm. For the
OS and its derived algorithm, the first thing we can note is that the recursive
relationship is actually expressed based on ``shell'' rather than the basis set
orders (for example, shell P has three basis set orders; Px, Py and Pz).

This is easy to understand. Suggest for an arbitrary overlap integral,
$(a|b)$, there will be $n_{a}n_{b}$ basis set pairs ($n$ is the number of
Cartesian basis set). For the $(a+\iota_{i}|b)$, depending on the literal 
meaning the basis set pairs would be $3n_{a}n_{b}$ since $\iota_{i}$ loop
over x, y and z. However, if shell a is incremented to the shell a+1 (like
from shell P to shell D), the basis set pairs would be:
\begin{equation}
 N_{1} = n_{b}\left(\frac{(L+2)(L+3)(L+4)-(L+1)(L+2)(L+3)}{6} \right) 
\end{equation}
$L$ is the angular momentum number for the shell a. If we expand the
$3n_{a}n_{b}$ in terms of $L$, it's:
\begin{equation}
N_{2} = 3n_{b}\left(\frac{(L+1)(L+2)(L+3)-(L+1)(L+2)L}{6} \right) 
\end{equation}
Therefore $N_{1} - N_{2}$ gives:
\begin{equation}
 \begin{split}
 N_{1} - N_{2} &= n_{b}\left( \frac{(L+2)(L+3)(L+4)-(L+1)(L+2)(L+3)}{6}\right.
\\
&-\left. \frac{3(L+1)(L+2)(L+3)-3(L+1)(L+2)L}{6}\right) \\
&=  n_{b}(L+2)\left( \frac{(L+3)(L+4)-(L+1)(L+3)}{6}\right. \\
&-\left.  \frac{3(L+1)(L+3)-3(L+1)L}{6}\right) \\
&= n_{b}(L+2)\left( \frac{(L+3)(L+4)+3(L+1)L-4(L+1)(L+3)}{6}\right) \\
&= n_{b}(L+2)\left( \frac{(L+3)(-3L)+3(L+1)L}{6}\right) \\
&= n_{b}(L+2)L\left( \frac{-3(L+3)+3(L+1)}{6}\right) \\
&= n_{b}(L+2)L\left( \frac{-6}{6}\right) \\
&< 0
 \end{split}
\end{equation}
Therefore, The incremental on the integral could be fully expressed by shell
rather than basis set orders. What's more, it also indicates that for all of 
recursive relations in the OS framework, the integral is meaningful to the
``shells'', we need not to consider it in the basis set order level.

\end{comment}


