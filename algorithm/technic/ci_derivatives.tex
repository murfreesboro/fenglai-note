\chapter{The Analytical Gradient for CI Methods}
%
%
%
%
%%%%%%%%%%%%%%%%%%%%%%%%%%%%%%%%%%%%%%%%%%%%%%%%%%%%%%%%
\section{CI Energy}
%
%
%
%
In this chapter, we are going to consider the analytical gradients for general
CI method. Here we follow the general convention that to use
$C$ (capital C) to represent the Slater determinant amplitude in CI state,
to use $c$ (lowercase c) to designate the MO coefficients, $\Psi$ is used to
refer to CI wave function, $\Phi$ means the corresponding Slater
determinant, and $\varphi$  is refer to MO.

For the index, all the capital letters such as $I,J,K,L$ etc. indicates it's
CI wave functions, and the lowercase letter is used to refer to MO. Again, we
use a, b, c etc. to designate the virtual orbitals, the i, j, k etc. to refer to
the occupied orbitals; and p, q, r etc. to specify the general orbitals.

Suppose that we have gained a set of reference MO from HF calculation, then we
are able to form the corresponding slater determinants:
\begin{equation}
 \label{CI_energy_CI_derivatives:1}
\Psi_{K} = \sum_{I}C_{IK}\Phi_{I}
\end{equation}
Where the normalization condition for the wave functions requires:
\begin{equation}
 \label{CI_energy_CI_derivatives:2}
\langle\Psi_{J}|\Psi_{K}\rangle = \delta_{JK} \rightarrow \sum_{I}C_{IJ}C_{IK} 
= \delta_{JK}
\end{equation}

From the Schrodinger equation, we know:
\begin{align}
 \label{CI_energy_CI_derivatives:3}
\hat{H}\Psi_{K} &= E_{K}\Psi_{K} \Longrightarrow \nonumber \\
E_{K} &= \langle\Psi_{K}|\hat{H}|\Psi_{K}\rangle   \nonumber \\
&=\sum_{I}\sum_{J}C_{IK}C_{JK}\langle\Phi_{I}|\hat{H}|\Phi_{J}\rangle \nonumber
\\
&= \sum_{I}\sum_{J}C_{IK}C_{JK}H_{IJ}
\end{align}

For $H_{IJ}$, if we express it into the MO representation, it becomes:
\begin{equation}
 \label{CI_energy_CI_derivatives:4}
H_{IJ} = \sum_{pq}^{MO}Q_{pq}^{IJ}h_{pq} + \sum_{pqrs}^{MO}G_{pqrs}^{IJ}(pq|rs)
\end{equation}
From Slater rules, it turns out that $Q_{pq}^{IJ}$ and $G_{pqrs}^{IJ}$ are some
numbers, which could be $0$, $\pm 1$ or $\pm 2$. 

By further considering the CI amplitudes, we have:
\begin{equation}
 \label{CI_energy_CI_derivatives:5}
E_{K} = \sum_{pq}^{MO}Q^{K}_{pq}h_{pq} + \sum_{pqrs}^{MO}G^{K}_{pqrs}(pq|rs)
\end{equation}
Where
\begin{align}
 \label{CI_energy_CI_derivatives:6}
Q^{K}_{pq} &= \sum_{I}\sum_{J}C_{IK}C_{JK}Q_{pq}^{IJ} \nonumber \\
G^{K}_{pqrs} &= \sum_{I}\sum_{J}C_{IK}C_{JK}G_{pqrs}^{IJ}
\end{align}


%%%%%%%%%%%%%%%%%%%%%%%%%%%%%%%%%%%%%%%%%%%%%%%%%%%%%%%%
\section{The First Order Derivatives for CI Method}
%
%
%
%
Now according to the (\ref{CI_energy_CI_derivatives:3}), the first order energy
derivatives can be expressed as:
\begin{equation}
  \label{CI_1st_CI_derivatives:1}
\begin{split}
\frac{\partial E_{K}}{ \partial \bm{R}_{a}} &=
\sum_{I}\sum_{J}\frac{\partial C_{IK}}{\partial \bm{R}_{a}}C_{JK}H_{IJ} + 
\sum_{I}\sum_{J}C_{IK}\frac{\partial C_{JK}}{\partial \bm{R}_{a}}H_{IJ} \\
&+\sum_{I}\sum_{J}C_{IK}C_{JK}\frac{\partial H_{IJ} }{\partial \bm{R}_{a}}
\\ 
&= 2\sum_{I}\sum_{J}\frac{\partial C_{IK}}{\partial \bm{R}_{a}}C_{JK}H_{IJ} +
\sum_{I}\sum_{J}C_{IK}C_{JK}\frac{\partial H_{IJ} }{\partial \bm{R}_{a}}
\end{split}
\end{equation}
In this derivation, we have used the relation that $I$ and $J$ are symmetrical.

According to Winger's theorem (see \ref{Winger_theorem}), the derivatives for
the variational parameters should not appear in the first order derivatives
expression, this is because the $C_{I}$ has made the energy minimum. Hence by
applying the CI equation in (\ref{CIeq:17}), we have:
\begin{align}
  \label{CI_1st_CI_derivatives:2}
HC &= EC \Longrightarrow \nonumber \\
\sum_{J}H_{IJ}C_{JK} &= E_{K}C_{IK} 
\end{align}
Here $C_{JK}$ indicates the amplitudes for the $K$th CI wave function. By bring
(\ref{CI_1st_CI_derivatives:2}) into (\ref{CI_1st_CI_derivatives:1}), we have:
\begin{align}
 \label{CI_1st_CI_derivatives:3}
\frac{\partial E_{K}}{ \partial \bm{R}_{a}} &= 2E_{K}\sum_{I}\frac{\partial
C_{IK}}{\partial \bm{R}_{a}}C_{IK} +
\sum_{I}\sum_{J}C_{IK}C_{JK}\frac{\partial H_{IJ} }{\partial \bm{R}_{a}}
\nonumber \\
&= \sum_{I}\sum_{J}C_{IK}C_{JK}\frac{\partial H_{IJ} }{\partial \bm{R}_{a}}
\end{align}
This is because that from the normalization condition, we have:
\begin{equation}
 \begin{split}
  \frac{\partial \sum_{I}C_{IK}^{2}}{\partial \bm{R}_{a}} &= 0 \Longrightarrow
\\
2\sum_{I}\frac{\partial
C_{IK}}{\partial \bm{R}_{a}}C_{IK} &= 0
 \end{split}
\label{CI_1st_CI_derivatives:4}
\end{equation}

Expanding the (\ref{CI_1st_CI_derivatives:3}) in terms of
(\ref{CI_energy_CI_derivatives:5}), we can get that:
\begin{equation}
\label{CI_1st_CI_derivatives:5}
 \frac{\partial E_{K}}{ \partial \bm{R}_{a}} =
\sum_{pq}^{MO}Q^{K}_{pq}\frac{\partial h_{pq}}{\partial \bm{R}_{a}} +
\sum_{pqrs}^{MO}G^{K}_{pqrs}\frac{\partial (pq|rs)}{\partial \bm{R}_{a}}
\end{equation}

Now the problem is back to the MO derivatives. From
(\ref{core_Hamiltonian_MO_INT_gradient_eq:1}) and
(\ref{two_electron_MO_INT_gradient_eq:1}), which are:
\begin{align}
 \label{CI_1st_CI_derivatives:6}
\frac{\partial h_{pq}}{\partial \bm{R}_{a}} &= \sum_{t}\left[U^{a}_{tp}H_{qt} +
U^{a}_{tq}H_{pt}\right]  + H^{a}_{pq} \nonumber \\
(pq|rs)^{[a]} &= \Pi_{pqrs}^{[a]} =\sum_{t}\left[ 
U^{a}_{tp}\Pi_{tqrs} +
U^{a}_{tq}\Pi_{ptrs} + 
U^{a}_{tr}\Pi_{pqts} + 
U^{a}_{ts}\Pi_{pqrt}  
\right] + \Pi^{a}_{pqrs}
\end{align}
the (\ref{CI_1st_CI_derivatives:5}) can be further expressed as:
\begin{align}
  \label{CI_1st_CI_derivatives:7}
 \frac{\partial E_{K}}{ \partial \bm{R}_{a}} &=
\sum_{pq}^{MO}Q^{K}_{pq}\left\lbrace \sum_{t}\left[U^{a}_{tp}H_{qt} +
U^{a}_{tq}H_{pt}\right]  + H^{a}_{pq} \right\rbrace \nonumber \\
&+
\sum_{pqrs}^{MO}G^{K}_{pqrs}\left\lbrace \sum_{t}\left[ 
U^{a}_{tp}\Pi_{tqrs} +
U^{a}_{tq}\Pi_{ptrs} + 
U^{a}_{tr}\Pi_{pqts} + 
U^{a}_{ts}\Pi_{pqrt}  
\right] + \Pi^{a}_{pqrs}\right\rbrace \nonumber \\
&= \sum_{pq}^{MO}Q^{K}_{pq}H^{a}_{pq} +
\sum_{pqrs}^{MO}G^{K}_{pqrs}\Pi^{a}_{pqrs} \nonumber \\
&+ \sum_{pq}^{MO}Q^{K}_{pq}\left\lbrace \sum_{t}\left[U^{a}_{tp}H_{qt} +
U^{a}_{tq}H_{pt}\right]  \right\rbrace \nonumber \\
&+ \sum_{pqrs}^{MO}G^{K}_{pqrs}\left\lbrace \sum_{t}\left[ 
U^{a}_{tp}\Pi_{tqrs} +
U^{a}_{tq}\Pi_{ptrs} + 
U^{a}_{tr}\Pi_{pqts} + 
U^{a}_{ts}\Pi_{pqrt}  
\right] \right\rbrace \nonumber \\
\end{align}

Now let's try to understand the symmetry behind the
(\ref{CI_1st_CI_derivatives:7}). Compared with the Hatree-Fock derivatives,
here a important distinguishment is that the MO index are ``sumed up'', which
means, the $p$ and $q$ are identical with each other in the core Hamiltonian
derivatives, and the four indices of $p$, $q$, $r$, $s$ are identical with each
other in the double electron integral derivatives. That is to say:
\begin{equation}
 \begin{split}
  \sum_{pq}^{MO}\sum_{t}Q^{K}_{pq}U^{a}_{tp}H_{qt} &=
  \sum_{pq}^{MO}\sum_{t}Q^{K}_{pq}U^{a}_{tq}H_{pt} \\
\sum_{pqrs}^{MO} \sum_{t}G^{K}_{pqrs}
U^{a}_{tp}\Pi_{tqrs}  
&= 
\sum_{pqrs}^{MO} \sum_{t}G^{K}_{pqrs} 
U^{a}_{tq}\Pi_{ptrs} \\ 
&= 
\sum_{pqrs}^{MO} \sum_{t}G^{K}_{pqrs} 
U^{a}_{tr}\Pi_{pqts}  \\
&=
\sum_{pqrs}^{MO} \sum_{t}G^{K}_{pqrs} 
U^{a}_{ts}\Pi_{pqrt}  
 \end{split}
\label{CI_1st_CI_derivatives:8}
\end{equation}
This can be further explained as below: since in the summation
$\sum_{pq}^{MO}\sum_{t}$, $Q^{K}_{pq} = Q^{K}_{qp}$ (see its original
definition); then if we exchange the index of $p$ and $q$, then everything
should be same for the core integral:
\begin{equation}
  \sum_{pq}^{MO}\sum_{t}Q^{K}_{pq}U^{a}_{tq}H_{pt} \xrightarrow{p
\leftrightarrow q} \sum_{pq}^{MO}\sum_{t}Q^{K}_{qp}U^{a}_{tp}H_{qt}
\end{equation}
Then the second integral changes into the first one, hence they should be
identical with each other. For the double electron integrals, we can get the
similar proof by remembering that $Q^{K}_{pqrs}$ does not change if we
exchange the label of $p$, $q$, $r$ and $s$.

By bringing the (\ref{CI_1st_CI_derivatives:8}) into the
(\ref{CI_1st_CI_derivatives:7}), then we can get:
\begin{align}
 \label{CI_1st_CI_derivatives:9}
\frac{\partial E_{K}}{ \partial \bm{R}_{a}} &=
\sum_{pq}^{MO}Q^{K}_{pq}H^{a}_{pq} +
\sum_{pqrs}^{MO}G^{K}_{pqrs}\Pi^{a}_{pqrs} \nonumber \\
&+ 2\sum_{pq}^{MO}\sum_{t}Q^{K}_{pq}U^{a}_{tp}H_{qt} + 
4\sum_{pqrs}^{MO}\sum_{t} G^{K}_{pqrs}
U^{a}_{tp}\Pi_{tqrs} \nonumber \\
&= \sum_{pq}^{MO}Q^{K}_{pq}H^{a}_{pq} +
\sum_{pqrs}^{MO}G^{K}_{pqrs}\Pi^{a}_{pqrs} \nonumber \\
&+ 2\sum_{tp}U^{a}_{tp}\left\lbrace 
\sum_{q}^{MO}Q^{K}_{pq}H_{qt} + 
2\sum_{qrs}^{MO}G^{K}_{pqrs}
\Pi_{tqrs}
\right\rbrace \nonumber \\
&= \sum_{pq}^{MO}Q^{K}_{pq}H^{a}_{pq} +
\sum_{pqrs}^{MO}G^{K}_{pqrs}\Pi^{a}_{pqrs}
+ 2\sum_{tp}U^{a}_{tp}X_{tp}
\end{align}
Where $X_{tp}$ is:
\begin{equation}
 \label{CI_1st_CI_derivatives:10}
X_{tp} = 
\sum_{q}^{MO}Q^{K}_{pq}H_{qt} + 
2\sum_{qrs}^{MO}G^{K}_{pqrs}
\Pi_{tqrs}
\end{equation}
Now here the only thing left we need to solve is the MO response matrix of $U$
in (\ref{CI_1st_CI_derivatives:9}). 

%%%%%%%%%%%%%%%%%%%%%%%%%%%%%%%%%%%%%%%%%%%%%%%%%%%%%%%%
\section{Z-Vector method}
%
%
%
%
\cite{handy1984evaluation}
 

%%%%%%%%%%%%%%%%%%%%%%%%%%%%%%%%%%%%%%%%%%%%%%%%%%%%%%%%


%%%%%%%%%%%%%%%%%%%%%%%%%%%%%%%%%%%%%%%%%%%%%%%%%%%%%%%%

%%% Local Variables: 
%%% mode: latex
%%% TeX-master: "../../main"
%%% End: 


