%
%
% firstly initialized at Feb. 16th; 2009
% That's my doctor main project to finish
%
%
%
%
%
\chapter{Geometric Derivatives for Time Dependent
Density Functional Theory}
%
%
%
%%%%%%%%%%%%%%%%%%%%%%%%%%%%%%%%%%%%%%%%%%%%%%%%%%%%%%%%%%%%%%%%%%%%%%%%%%%%%%%%%%%%%%%%%%%%%%%%%%
\section{Introduction}
%
% what does the gradients of TDDFT for?
%
%
Now we are going to penetrate into the area of geometrical
derivatives for the time-dependent density functional theory. So far
there's only several papers publishing for the realization of this
function, two of them is given by Amos\cite{VanCaillie2000159,
VanCaillie1999249}, the others is given by Furche\cite{furche:7433,
rappoport:064105} and implemented in the Turbomole
program\cite{turbomole}. The papers given by Amos actually is a bit
of hard to handle, so we prefer to follow the method devised by
Furche, and it has been commented that this method has some
advantages compared with the Amos method\cite{chiba:144106}.



%%%%%%%%%%%%%%%%%%%%%%%%%%%%%%%%%%%%%%%%%%%%%%%%%%%%%%%%%%%%%%%%%%%%%%%%%%%%%%%%%%%%%%%%%%%%%%%%%%
\section{Theory}
%
%
\subsection{Excitation state in TDDFT}
%
% 1  the old linear response equation
% 2  to change the linear response equation into the
%    functional
%
%
As what we have been derived before, the linear response equation
used to calculate the excitation energy can be expressed as:
\begin{equation}\label{TDDFTGRADeq:1}
\begin{bmatrix}
  A & B \\
  B^{*} & A^{*} \\
\end{bmatrix}
\begin{bmatrix}
  X \\
  Y \\
\end{bmatrix}
=\omega
\begin{bmatrix}
  1 &  0 \\
  0 & -1 \\
\end{bmatrix}
\begin{bmatrix}
  X \\
  Y \\
\end{bmatrix}
\end{equation}
Where the matrix element are:
\begin{align}\label{}
(A+B)_{ia\sigma, jb\sigma^{'}} &= (\epsilon_{a} -
\epsilon_{i})\delta_{ij}\delta_{ab}\delta_{\sigma\sigma^{'}} +
2(ia\sigma|jb\sigma^{'}) + 2(ia\sigma|f_{xc}|jb\sigma^{'}) \nonumber
\\
& -c_{x}\delta_{\sigma\sigma^{'}}[(ja\sigma|ib\sigma^{'}) +
(ab\sigma|ij\sigma^{'})] \nonumber \\
(A-B)_{ia\sigma, jb\sigma^{'}} &= (\epsilon_{a} -
\epsilon_{i})\delta_{ij}\delta_{ab}\delta_{\sigma\sigma^{'}} +
c_{x}\delta_{\sigma\sigma^{'}}[(ja\sigma|ib\sigma^{'}) -
(ab\sigma|ij\sigma^{'})]
\end{align}

The $(ia\sigma|jb\sigma^{'})$ is labeled as two electron integrals:
\begin{multline}\label{}
(ia\sigma|jb\sigma^{'}) = \delta_{\sigma\sigma^{'}}\int d^{3}r^{'}
\int
d^{3}r^{''}\chi^{*}_{i\sigma}(r^{'})\chi_{a\sigma}(r^{'}) \\
\frac{1}{|r^{'}-r^{''}|}
\chi^{*}_{j\sigma^{'}}(r^{''})\chi_{b\sigma^{'}}(r^{''})
\end{multline}
The $(ia\sigma|f_{xc}|jb\sigma^{'})$ denotes the
exchange-correlation part in the adiabatic approximation:
\begin{multline}\label{}
(ia\sigma|f_{xc}|jb\sigma^{'}) = \delta_{\sigma\sigma^{'}}\int
d^{3}r^{'} \int
d^{3}r^{''}\chi^{*}_{i\sigma}(r^{'})\chi_{a\sigma}(r^{'}) \\
f_{XC}(r^{'},r^{''})
\chi^{*}_{j\sigma^{'}}(r^{''})\chi_{b\sigma^{'}}(r^{''})
\end{multline}
Where the $f_{XC}(r^{'},r^{''})$ is the exchange-correlation kernel
in the adiabatic approximation:
\begin{equation}\label{}
f_{XC}(r^{'},r^{''}) = \frac{\delta^{2} E_{XC}}{
\delta\rho_{\sigma}(r)\delta\rho_{\sigma^{'}}(r^{'})}
\end{equation}
Here the $\chi$ is the ground state Kohn-Sham orbitals before the
perturbation, and again we follow the convention that the subscript
of $i,j$ etc. indicate the occupied orbitals for Kohn-Sham time
independent states, $a, b$ etc. represent the virtual orbitals; and
$p, q$ etc. refer to the general orbitals. Usually the KS orbitals
in the molecular calculation is always chosen to be real, and we can
have:
\begin{align}\label{}
&(ia\sigma|jb\sigma^{'})  = (ia\sigma|bj\sigma^{'}) =
(ai\sigma|jb\sigma^{'})  = (ai\sigma|bj\sigma^{'}) \nonumber \\
&(ia\sigma|f_{xc}|jb\sigma^{'})  = (ia\sigma|f_{xc}|bj\sigma^{'}) =
(ai\sigma|f_{xc}|jb\sigma^{'})  = (ai\sigma|f_{xc}|bj\sigma^{'})
\end{align}

On the other hand, the hybrid mixing parameter of $c_{x}$ is
introduced by Becke which allows us to interpolate between the pure
functionals ($c_{x} = 0$) and TDHF theory ($c_{x} = 1, E_{XC} = 0$).

For the vector related to $X$, $Y$ in the equation of
(\ref{TDDFTGRADeq:1}), it actually represents the perturbed density
matrix (more details see the section \ref{TDDFT:1}):
\begin{align}\label{}
X_{ia} &= P_{ia} \nonumber \\
Y_{ia} &= P_{ai}
\end{align}

In the (\ref{TDDFTGRADeq:1}), $\omega$ corresponds to the frequency
of the excitation; in atom unit where $\hbar = 1$, we have $E =
\hbar\omega = \omega$.

For the linear response equation in the (\ref{TDDFTGRADeq:1}),
however, it can be considered as some stationary point from some
more general functional:
\begin{equation}\label{TDDFTGRADeq:2}
G[X,Y,\Omega] =\langle X,Y|\Lambda|X,Y\rangle - \Omega(\langle
X,Y|\Delta|X,Y\rangle - 1)
\end{equation}
Here the vector of $|X,Y\rangle$ corresponds to:
\begin{equation}\label{}
|X,Y\rangle \Leftrightarrow
\begin{bmatrix}
  X \\
  Y \\
\end{bmatrix}
\end{equation}
and the ``super operator'' $\Delta$ and $\Lambda$ are:
\begin{align}\label{}
\Lambda &=
\begin{bmatrix}
  A & B \\
  B^{*} & A^{*} \\
\end{bmatrix} \nonumber \\
\Delta &=
\begin{bmatrix}
  1 &  0 \\
  0 & -1 \\
\end{bmatrix}
\end{align}
$\Omega$ here is set to be some Lagrangian factor, obviously it
corresponds to the frequency in the (\ref{TDDFTGRADeq:1}).

By applying the variational principle with the constraint that:
\begin{align}\label{TDDFTGRADeq:3}
\frac{\delta G}{\delta \langle X,Y|} &= 0 \\
\frac{\delta G}{\delta \Omega} &= \langle X,Y|\Delta|X,Y\rangle - 1
 = 0
\end{align}
Through the first equation in (\ref{TDDFTGRADeq:3}), we can get the
same time-dependent Kohn-Sham equation, and the constraint condition
in the (\ref{TDDFTGRADeq:3}) yields the excitation energy of
$\Omega$.

Now we can calculate the geometrical derivatives through the
functional of $G$. Suggested by Pulay(please see the chapter of
\ref{PULAY:1}) we can know:
\begin{equation}\label{TDDFTGRADeq:4}
G^{\xi} = G^{\xi}[X,Y,\Omega] = \langle X,Y|\Lambda^{\xi}|X,Y\rangle
\end{equation}
Here $\xi$ refers to the nuclear coordinates, and $G^{\xi}$ denotes
the geometrical derivative for the functional of $G$. In the
(\ref{TDDFTGRADeq:4}) there's something important to note that the
fist order property does not require derivatives of excitation
vector of $|X,Y\rangle$, therefore the second term related to the
functional $G$ is vanished as making derivatives.

However, to calculate the geometrical derivatives for the $G$
through the (\ref{TDDFTGRADeq:4}) is still something difficult to
perform. The difficulties is arising from the fact that the super
operator of $\Lambda$ still involves of MO coefficients so that it's
difficult to handle (the MO coefficients are implicitly some
function of geometrical parameters, see the chapter of \ref{PULAY:1}
for more details). In the following content,
Furche\cite{furche:7433} introduced some novel method to bypass such
difficult evaluation; now let's go to see how to do this job.

%%%%%%%%%%%%%%%%%%%%%%%%%%%%%%%%%%%%%%%%%%%%%%%%%%%%%%%%%%%%%%%%%%%%%%%%%%%%%%%%%%%%%%%%%%%%%%%%%%
\subsection{Lagrangian of the excitation energy}
%
%
%
%
Firstly, in the original functional of $G$, let's replace the energy
expression with the form below:
\begin{equation}\label{}
(\epsilon_{a\sigma} -
\epsilon_{i\sigma^{'}})\delta_{ab}\delta_{ij}\delta_{\sigma\sigma^{'}}
=(F_{ab\sigma}\delta_{ij} -
F_{ij\sigma^{'}}\delta_{ab})\delta_{\sigma\sigma^{'}}
\end{equation}
Here the $F_{pq}$ stands for the KS Hamiltonian matrix element:
\begin{align}\label{TDDFTGRADeq:5}
F_{pq} &= \langle\chi_{p}|\hat{h}|\chi_{q}\rangle +
\sum_{i\sigma^{'}} \Bigg\{(pq\sigma|ii\sigma^{'}) -
c_{x}\delta_{\sigma\sigma^{'}}(pi\sigma|qi\sigma^{'}) \Bigg\}
\nonumber \\
& + V^{XC}_{pq\sigma}
\end{align}
$V^{XC}_{pq\sigma}$ is the exchange-correlation potential for the
ground state:
\begin{equation}\label{}
V^{XC}_{pq\sigma} = \frac{\delta E^{XC}}{\delta \rho_{\sigma}(r)}
\end{equation}

In the basis of canonical KS molecular orbitals, $F$ is diagonal:
\begin{equation}\label{TDDFTGRADeq:6}
F_{pq\sigma} = \delta_{\sigma\sigma^{'}}\delta_{pq}\epsilon_{p}
\end{equation}
However, the selection of canonical molecular orbitals is somewhat
arbitrary because all the physical properties are invariant under
transformations of occupied and virtual orbitals among themselves.
Such transformation corresponds to the so call ``unitary
transformation'', from the discussion in quantum mechanics
\ref{OPERATOR:2} and the similar discussion in Hatree-Fock chapter
\ref{HFT:3} we can directly get this point: the unitary
transformation among the basis sets does not alter any physical
properties for the quantum chemistry calculation.

All in all, by inserting the (\ref{TDDFTGRADeq:5}) into the
functional of $G$, then the constraint of the canonical sets between
the molecular orbitals can be dropped out. For clearness here we
note that in the (\ref{TDDFTGRADeq:3}), the constraint condition
between the $|X,Y\rangle$ is some kind of restrict condition between
the virtual orbitals and the occupied orbitals which is required by
the linear response equation (see the section of \ref{TDDFT:3}),
this is different with the canonical sets requirement shown in
(\ref{TDDFTGRADeq:6}).

Now we can define some auxiliary functional of $L$:
\begin{equation}\label{}
L [X, Y, \Omega, C, Z ,W]  = G[X, Y, \Omega] +
\sum_{ia\sigma}Z_{ia\sigma}F_{ia\sigma} - \sum_{pq\sigma, p\leq
q}W_{pq\sigma}(S_{pq\sigma} - \delta_{pq})
\end{equation}

The functional $L$ is required to be stationary with respect to all
it's parameters: $X, Y, \Omega, C, Z$ and $W$; therefore $L$ is
capable of representing the full variational functional for
calculating excitation states. Here, the Lagrangian multiplier $Z$
and $W$ is required to satisfy the condition:
\begin{align}\label{TDDFTGRADeq:7}
\frac{\partial L}{\partial Z_{ia\sigma}} &= F_{ia\sigma} = 0
\nonumber \\
\frac{\partial L}{\partial W_{pq\sigma}} &= S_{pq\sigma} -
\delta_{pq} = 0
\end{align}
Through the conditions defined in the (\ref{TDDFTGRADeq:7}), the MO
is constrained to satisfy the static KS equation up to some unitary
transformation. The $Z$ and $W$ themselves are determined through
the condition that:
\begin{equation}\label{}
\frac{\partial L}{\partial C} = 0
\end{equation}

How to understand the functional $L$? Physically, the $L$ can be
considered as the functional that combines the KS equation as well
as the TDDFT equation together. The variation with respect to the
$C$, $Z$ and $W$ gives the KS equation and the variation with
respect to the $|X,Y\rangle$ and $\Omega$ gives the TDDFT equation.
Compared with the $G$ which is implicitly depending on the MO
coefficients of $C$, the functional of $L$ is explicitly relaying on
the $C$ thus in the $\frac{\partial L}{\partial \xi}$ the MO
coefficients does not enter into the evaluation. That's the crucial
advantages for using the $L$ rather than the $G$. Once the $X, Y,
\Omega, C, Z$ and $W$ have been determined through the stationary
condition, the derivative of the excitation energy is given as:
\begin{align}\label{}
\Omega^{\xi} &= L^{\xi}[X, Y, \Omega, C, Z ,W] \nonumber \\
&=G^{(\xi)}[X, Y, \Omega] +
\sum_{ia\sigma}Z_{ia\sigma}F^{(\xi)}_{ia\sigma} - \sum_{pq\sigma,
p\leq q}W_{pq\sigma}S^{\xi}_{pq\sigma}
\end{align}
Here the superscript of $(\xi)$ indicates that in the geometrical
derivatives evaluation the MO coefficients are irrelevant; that is
to say, they are constant and not involved in the derivatives.





%%%%%%%%%%%%%%%%%%%%%%%%%%%%%%%%%%%%%%%%%%%%%%%%%%%%%%%%%%%%%%%%%%%%%%%%%%%%%%%%%%%%%%%%%%%%%%%%%%


%%% Local Variables: 
%%% mode: latex
%%% TeX-master: "../../main"
%%% End: 
