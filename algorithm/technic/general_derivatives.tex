
%%%%%%%%%%%%%%%%%%%%%%%%%%%%%%%%%%%%%%%%%%%%%%%%%%%%%%%%%
\chapter{General Discussion for Gradient theory in Quantum chemistry}
%
%
%
In quantum chemistry, the evaluating for the energy derivatives with
respect to the perturbation of system is overwhelming important. Many
important application; such as energy surface study, the vibration etc. are all
have to employ such technique.

There's a book that gives detailed discussion for this area, it's written by 
Yukio Yamaguchi, John D. Goddard, Yoshihiro Osamura and Henry
Schaefer\cite{New_Dimension_for_Derivatives_Calculation}\footnote{Actually,
we are just following the same procedure used in that book}.  The author
recommends the interesting reader to further refer to this book. On the other
hand, the initiator of this theory; Peter Pulay also has many
reference\cite{Pulay1, Pulay2, Pulay3, Pulay4, Pulay5, Pulay6, pulay:5043} that
can be further studied. Now let's concentrate on one aspect of energy
derivatives, which is to locate the potential energy surface.


%%%%%%%%%%%%%%%%%%%%%%%%%%%%%%%%%%%%%%%%%%%%%%%%%%%%%%%%%%
\section{Introduction}
%
%
%
Most of the quantum chemistry study is based on the Born-Oppenheimer
approximation. In this approximation, the nucleus are considered to
move on the classic trajectory, so its motion is continuous and
smooth; where the motion of electrons is treated quantized, and able
to adapt simultaneously to the position change of nucleus.  Therefore,
the total energy of the whole molecular system is some function of the
nuclear coordinates; such dependence is called potential energy
hyper-surface (later shorted as PES).

Consequently, to determine the PES is vital in the quantum
chemistry. Such determination requires to get the energy derivatives
for an arbitrary nuclear configuration of \vect{R_{0}}, that is to
say, the PES around \vect{R_{0}} can be adequately characterized by
its Taylor series expansion:
\begin{equation}\label{PULAYeq:1}
  \begin{split}
    E(\mathbf{R}) &= E(\mathbf{R_{0}}) +
    \sum_{a}E^{a}\Delta\mathbf{R_{a}}
    +  \frac{1}{2!}\sum_{a,b}E^{ab}\Delta\mathbf{R_{a}}\Delta\mathbf{R_{b}} \\
    & +
    \frac{1}{3!}\sum_{a,b,c}E^{abc}\Delta\mathbf{R_{a}}\Delta\mathbf{R_{b}}
    \Delta\mathbf{R_{c}} + \cdots
  \end{split}
\end{equation}
Here the $E^{a}, E^{ab}$ etc. denote the first derivatives, second
derivatives etc. with respect to the total energy in the \vect{R_{0}};
and the $\Delta\mathbf{R_{a}}, \Delta\mathbf{R_{b}}$ etc. express the
small changes around the reference point of \vect{R_{0}} in the given
direction: $\Delta\mathbf{R_{a}} =
(\mathbf{R_{a}}-\mathbf{R_{0}})_{a}$.

The solution to this problem was firstly suggested by professor Peter
Pulay around $1969$. In this chapter, we will provide a thorough and
detailed analysis for this topic.

%%%%%%%%%%%%%%%%%%%%%%%%%%%%%%%%%%%%%%%%%%%%%
\section{General Derivatives
  Expressions}\label{General_Derivatives_Expressions_hf_derivatives} 
%
%
In this section, we firstly introduce some general formulas for the
derivatives of variational energy expressions\footnote{This expression is
generally taken from Peter Pulay's review. Personally I like it,
since it's very compact expression and also very heuristic for understanding
the. variational process. However, as a symbolic expression, it's not the best
one to express the gradient.}.

First let's consider the energy functional:
\begin{equation}\label{PULAYeq:13}
  E = E(\bm{c},\bm{R})
\end{equation}
Here the $\bm{R}$ represents the nuclear coordinates, and $\bm{c}$ is the set of
variational parameter in the wave function; e.g.; in the case of SCF
wave function, $\bm{c}$ is the MO coefficients. In the later content, we use
$\bm{c}_{i}$ to indicate each of $\bm{c}$'s vector, e.g; in HFR equation
$\bm{c}_{i}$ corresponds to the $i$th MO coefficients. On the other hand,
In CI calculation, $\bm{c}$ is the Slater determinant coefficients. Here we note
that in the (\ref{PULAYeq:13}) the parameters should be gotten through variation
condition:
\begin{equation}\label{PULAYeq:14}
  \frac{\partial E}{\partial \bm{c}_{i}} = 0 \quad i=1,2,\cdots
\end{equation}
Thus for each reference coordinate of $\bm{R}$ we can get a set of
variational parameters of $\bm{c}$, so we can see that the $\bm{c}$ is some
function of nuclear coordinates of $\bm{R}$; $\bm{c} = c(\bm{R})$:
\begin{equation}\label{}
  \bm{R} \Rightarrow E \Rightarrow \bm{c} \Rightarrow \bm{c}= c(\bm{R})
\end{equation}
Therefore we can express the energy functional as:
\begin{equation}\label{PULAYeq:2}
  E = E(c(\bm{R}), \bm{R})
\end{equation}
This is some simpler model for discussing the derivatives of the total
energy, for we have not imposed any further constraint conditions on
the $\bm{c}$. Usually in the quantum chemistry, in the variation process of
(\ref{PULAYeq:14}) the MO coefficients should keep orthogonal with
each other. Latter we will study this case.

Our aim is to derive the formulas for the expression of
(\ref{PULAYeq:2}) with respect to the nuclear coordinates of $\bm{R}$.  For
the change of $R$ in the $a$ direction, according to the chain rule of
differentiation we have:
\begin{align}\label{PULAYeq:3}
  E_{a} &= \frac{\partial E}{\partial R_{a}} +\sum_{i}\frac{\partial
    E}{\partial \bm{c}_{i}}\frac{\partial
    \bm{c}_{i}}{\partial R_{a}} \nonumber \\
  &= \frac{\partial E}{\partial R_{a}}
\end{align}

In this derivation we have used the relation in the
(\ref{PULAYeq:14}). The results shows that the derivative of the
variational parameters (here is the $\bm{c}$) do not enter into the
gradient formula. In the later content, as in the derivation of the
constraint variational process, we can see that the same fact holds,
too.

For the second derivatives, by differentiation of (\ref{PULAYeq:3}),
we can get:
\begin{align}\label{}
  E_{ab} &=\frac{\partial}{\partial R_{b}}\left(\frac{\partial
      E}{\partial
      R_{a}}\right) \nonumber \\
  &=\frac{\partial^{2} E}{\partial R_{a}\partial R_{b}}
  +\sum_{i}\frac{\partial^{2} E}{\partial R_{b}\partial
    \bm{c}_{i}}\frac{\partial \bm{c}_{i}}{\partial R_{a}}
\end{align}

So far, the discussion is all about the variational process without
constraints. However, it's usually required that in the variational
process, the orbitals or the approximated wave functions should keep
to be orthogonal with each other. Therefore, it's appropriate to
introduce the general functional with constraint conditions:
\begin{equation}\label{PULAYeq:8}
  W(\bm{c}, \bm{R}, \lambda) = E(\bm{c}, \bm{R}) - \sum_{m}f_{m}(\bm{c},
\bm{R})\lambda_{m}
\end{equation}
Here the $W$ stands for the general Lagrangian functional. The
$f_{m}(\bm{c}, \bm{R})$ generally denote the $m$ constraint conditions, normally
the constraints are the orthogonality conditions for the molecular
orbitals or the configurations. The $\lambda_{m}$ is simply the
Lagrangian factor. We note that this term is usually related to the energy, such
as the orbital energy in Hatree-Fock theory. The term $f_{m}$ is also very
meaningful, in Hatree-Fock theory, it corresponds to the overlap matrix. In the
later content, we will see more direct applications.

The variation process now becomes:
\begin{equation}\label{PULAYeq:5}
  \frac{\partial W}{\partial \bm{c}_{i}} = 0 \quad (i=1,2, \cdots)
\end{equation}
On condition that:
\begin{equation}\label{PULAYeq:15}
  f_{m}(\bm{c}, \bm{R}) =  0 \quad (m=1,2, \cdots)
\end{equation}

Such conditions are sufficient to determine the unknown $\bm{c}_{i}$ and
$\lambda_{m}$, hence we can consider the $\bm{c}_{i}$ and $\lambda_{m}$ are
some functions of $\bm{R}$:
\begin{equation}\label{}
  \left\{
    \begin{array}{ll}
      \bm{c}_{i} = c(\bm{R}),             & i=1,2,\cdots  \\
      \lambda_{m} = \lambda(\bm{R}), & m=1,2,\cdots
    \end{array}
  \right.
\end{equation}

If the conditions in (\ref{PULAYeq:5}) have been satisfied, then from
the general functional in (\ref{PULAYeq:8}) we can get the total
energy:
\begin{equation}\label{PULAYeq:4}
  W(\bm{c}, \lambda, \bm{R}) \Rightarrow E(\bm{c}, \bm{R})
\end{equation}

Now we begin to seek the expressions for the first derivative and
second derivatives related to the $E(\bm{c}, \bm{R})$.

By the differentiation of $E(\bm{c}, \bm{R})$, we have:
\begin{equation}\label{PULAYeq:6}
  E_{a} =\frac{\partial E}{\partial R_{a}} + \sum_{i}\frac{\partial
    E}{\partial \bm{c}_{i}}\frac{\partial \bm{c}_{i}}{\partial R_{a}}
\end{equation}
This is same with the (\ref{PULAYeq:3}). However, here we do not
simply have the $\frac{\partial E}{\partial \bm{c}_{i}} = 0$, thus we have
to do some transformations.

For the constraint variation condition, we can further express the
(\ref{PULAYeq:5}) as:
\begin{align}\label{}
  \frac{\partial W}{\partial \bm{c}_{i}} &= \frac{\partial E}{\partial
   \bm{c}_{i}} - \sum_{m}\frac{\partial f_{m}(\bm{c}, \bm{R})}{\partial
    \bm{c}_{i}}\lambda_{m} \nonumber \\
  &= 0
\end{align}
By multiply the $\frac{\partial \bm{c}_{i}}{\partial R_{a}}$ to the above
expression and sum over all the label of $i$, we can get:
\begin{equation}\label{PULAYeq:17}
  \sum_{i}\frac{\partial E}{\partial \bm{c}_{i}}\frac{\partial
    \bm{c}_{i}}{\partial R_{a}} - \sum_{m}\lambda_{m}\sum_{i}\frac{\partial
    f_{m}(\bm{c}, \bm{R})}{\partial \bm{c}_{i}}\frac{\partial
\bm{c}_{i}}{\partial R_{a}} = 0
\end{equation}

On the other hand, if we differentiate the constraint conditions of
(\ref{PULAYeq:15}) with respect to the $R_{a}$, we can have:
\begin{align}\label{PULAYeq:16}
  \frac{\partial f_{m}(\bm{c}, \bm{R})}{\partial R_{a}} + \sum_{i}\frac{\partial
    f_{m}(\bm{c}, \bm{R})}{\partial \bm{c}_{i}}\frac{\partial
\bm{c}_{i}}{\partial R_{a}} =
  0
\end{align}
If we multiply the (\ref{PULAYeq:16}) with $\lambda_{m}$ and sum all
of label of $m$ together, we can get:
\begin{equation}\label{PULAYeq:18}
  \sum_{m}\lambda_{m}\frac{\partial f_{m}(\bm{c}, \bm{R})}{\partial R_{a}} +
  \sum_{m}\lambda_{m}\sum_{i}\frac{\partial f_{m}(\bm{c}, \bm{R})}{\partial
    \bm{c}_{i}}\frac{\partial \bm{c}_{i}}{\partial R_{a}} = 0
\end{equation}

By virtual of the results indicated in the (\ref{PULAYeq:17}) and
(\ref{PULAYeq:18}), we can have:
\begin{equation}\label{}
  \sum_{m}\lambda_{m}\frac{\partial f_{m}(\bm{c}, \bm{R})}{\partial R_{a}} = -
  \sum_{i}\frac{\partial E}{\partial \bm{c}_{i}}\frac{\partial
    \bm{c}_{i}}{\partial R_{a}}
\end{equation}

Finally for the (\ref{PULAYeq:6}) we can have:
\begin{align}\label{PULAYeq:7}
  E_{a} &= \frac{\partial E}{\partial R_{a}} - \sum_{m}\frac{\partial
    f_{m}(\bm{c}, \bm{R})}{\partial
    R_{a}}\lambda_{m} \nonumber \\
  &= \frac{\partial W}{\partial R_{a}}
\end{align}
Here the result shows the same important meaning with the
(\ref{PULAYeq:3}) that the derivatives of the variational parameters
do not enter the gradient formula. For example, in the HF calculation
(or the DFT calculation, which use the single determinant); the
evaluation of the MO coefficients can be avoided.  In the CI
calculation, the evaluation with respect to the coefficients for the
determinants can be neglected. In such scheme, only the derivatives of
the basis function integrals appearing in the energy formula need to
be calculated.

By differentiating the expression of (\ref{PULAYeq:7}) with respect to
the $R_{b}$, we can get:
\begin{align}\label{}
  E_{ab} &=\frac{\partial}{\partial R_{b}}\left(\frac{\partial
      W}{\partial
      R_{a}}\right) \nonumber \\
  &=\frac{\partial^{2} E}{\partial R_{a}\partial R_{b}} +
  \sum_{i}\frac{\partial^{2} E}{\partial R_{a}\partial
    \bm{c}_{i}}\frac{\partial \bm{c}_{i}}{\partial R_{b}} -
  \sum_{m}\frac{\partial^{2} f_{m}(\bm{c}, \bm{R})}
  {\partial R_{a}\partial R_{b}}\lambda_{m} - \nonumber \\
  &\sum_{m}\sum_{i}\frac{\partial^{2} f_{m}(\bm{c}, \bm{R})}{\partial
    R_{a}\partial \bm{c}_{i}}\frac{\partial \bm{c}_{i}}{\partial
    R_{b}}\lambda_{m} - \sum_{m}\frac{\partial f_{m}(\bm{c}, \bm{R})}{\partial
    R_{a}}\frac{\partial \lambda_{m}}{\partial R_{b}}
\end{align}

%%%%%%%%%%%%%%%%%%%%%%%%%%%%%%%%%%%%%%%%%%%%%%%%%%%
\section{Hamiltonian Expansion}
%
%
%
Firstly, let's give some general discussion for the dependence between the
Hamiltonian operator, the total energy and other parameters.

For the $\hat{H}$, which is actually the function of $\bm{R}$:
$\hat{H}(\bm{R})$; in terms of some arbitrary geometry variation:
\begin{equation}
 \bm{R} = \bm{R}_{0} + \bigtriangleup\bm{R}
\end{equation}
the Hamiltonian can be expressed as:
\begin{equation}
\label{GD_gradient_eq:1} 
\begin{split}
 \hat{H}(\bm{R}_{0} + \Delta\bm{R}) &= \hat{H}(\bm{R}_{0}) + 
\sum_{a}\Delta\bm{R}_{a}\frac{\partial \hat{H}}{\partial \bm{R}_{a}} + 
\frac{1}{2!}\sum_{a,b}\Delta\mathbf{R_{a}}\Delta\mathbf{R_{b}}\frac{\partial^{2}
\hat{H}}{\partial \bm{R}_{a}\partial \bm{R}_{b}} \\
    & +
    \frac{1}{3!}\sum_{a,b,c}\Delta\mathbf{R_{a}}\Delta\mathbf{R_{b}}
    \Delta\mathbf{R_{c}}\frac{\partial^{3}
\hat{H}}{\partial \bm{R}_{a}\partial \bm{R}_{b}\partial \bm{R}_{c}} + \cdots
\end{split}
\end{equation}
Here $\Delta\mathbf{R_{a}}$ etc. denote the geometrical displacement on the
nuclear $a$. 

If we turn to the perturbation viewpoint, it's found that:
\begin{align}
 \label{GD_gradient_eq:2}
\hat{H}^{'}_{a} = \frac{\partial \hat{H}}{\partial \bm{R}_{a}} &\quad
\lambda_{a} = \Delta\bm{R}_{a} \nonumber \\
\hat{H}^{''}_{ab} = \frac{\partial^{2} \hat{H}}{\partial \bm{R}_{a}\partial
\bm{R}_{b}} &\quad \lambda_{b} =
\Delta\bm{R}_{b} \nonumber \\
\cdots
\end{align}
transform the above Taylor expansion in (\ref{GD_gradient_eq:1}) into:
\begin{equation}
 \label{GD_gradient_eq:3}
 \hat{H} = \hat{H}_{0} + \sum_{a}\lambda_{a}\hat{H}^{'}_{a} +
\frac{1}{2!}\sum_{a,b}\lambda_{a}\lambda_{b}\hat{H}^{''}_{ab} + \cdots
\end{equation}
This is identical to the form of 3.1 in Schaefer's
book\cite{New_Dimension_for_Derivatives_Calculation}. If we only consider the
nuclear $a$'s displacement, then the whole expression will come down to:
\begin{equation}
\label{GD_gradient_eq:7}
 \hat{H} = \hat{H}_{0} + \lambda\hat{H}^{'}
\end{equation}

%%%%%%%%%%%%%%%%%%%%%%%%%%%%%%%%%%%%%%%%%%%%%%%%%%%
\section{General Dependence between Energy and Other
Parameters}\label{General_dependence_gradient}
%
%
%
%
As the $\hat{H}$ changes, the corresponding molecular orbital as well as the
wave functions will alter accordingly. In quantum chemistry, we usually employ
the LCAO method to express the MO; hence the MO is fully characterized by its
coefficients. That is to say, the mo coefficients of $\bm{c}$ is some function
of $\bm{R}$:
\begin{equation}
 \label{GD_gradient_eq:4}
\bm{c} = c(\bm{R})
\end{equation}
Obviously such dependence is indirect. On the other hand, the wave function is
also fully characterized by its amplitude; for example; the wave function for
CIS equation is determined by the coefficients $a_{i}^{b}$ of the Slater
determinants $\ket{\Psi_{i}^{b}}$, so it's also some indirect function of
$\bm{R}$:
\begin{equation}
\label{GD_gradient_eq:5}
 \bm{a} = a(\bm{R})
\end{equation}
Finally, there's explicit dependence between the basis set functions and the
nuclear coordinates. Because the Gaussian type basis functions or the
Slater type of basis functions used almost exclusively in quantum chemistry are
all atomic type functions, they are clearly the function of nuclear coordinates
of $\bm{R}$; thus in total the energy dependence can be generally expressed as:
\begin{equation}
 \label{GD_gradient_eq:6}
E = E(\bm{R}, \bm{c}, \bm{a}, \phi)
\end{equation}
 
%%%%%%%%%%%%%%%%%%%%%%%%%%%%%%%%%%%%%%%%%%%%%%%%%%%
\section{Winger's $2n+1$ Theorem}
\label{Winger_theorem}
%
%
%
Winger's theorem is some fundamental theorem in perturbation theory. It says
that if the wave function is determined up to $n$th order, the expectation
value, such as energy and energy derivatives; can be determined to $2n+1$th
order. This simplest theorem is called $2n+1$ theorem.

Giving by this theorem, if the wave function is determined in zero order, then
the gradient of energy, which is first order property; can be resolved. This is
reflected by the result we got in (\ref{PULAYeq:7}). For the second derivatives
of the energy, as well as the third order; giving by the theorem it can see
that we only need first order perturbation of the wave function. For the
Hatree-Fock case, it means that we only need first order perturbation to the MO
to determined the second and third derivatives of energy.


%%%%%%%%%%%%%%%%%%%%%%%%%%%%%%%%%%%%%%%%%%%%%%%%%%%
\section{Molecular Orbital Response}\label{MO_response_gradient}
%
%
%
%
Let's consider some geometrical perturbation to the $\hat{H}$:
\begin{equation}
 \hat{H} = \hat{H}_{0} + \lambda\hat{H}^{'}
\end{equation}
This is the expression in (\ref{GD_gradient_eq:7}), it means that now we only
consider the nuclear displacement in nuclear $a$.

Then according to the perturbation treatment in (\ref{PTIQMeq:5}), the MO
expanded into first order is:
\begin{equation}
 \label{orbital_response_gradient_eq:1}
\varphi_{p}^{per} = \varphi_{p}^{(0)} + \lambda\varphi_{p}^{(1)}
\end{equation}
the perturbed part can always expressed over the unperturbed sets:
\begin{align}
 \label{orbital_response_gradient_eq:2}
\varphi_{p}^{(1)} &= \sum_{q}^{ALL MO}U_{qp}\varphi_{q}^{(0)}
\end{align}
Here we follow the convention that $a,b,c$ etc. are used to designate the
virtual orbitals, $i,j,k$ etc. used to refer to occupied orbitals; and $p,q,r$
etc. corresponds to general orbitals. 

Because in quantum chemistry we usually employ the LCAO method, hence it's
necessary to know how to express the perturbation in terms of the MO
coefficients:
\begin{align}
 \label{orbital_response_gradient_eq:3}
&\varphi_{p}^{per}  = \varphi_{p}^{(0)} + \lambda\varphi_{p}^{(1)} \nonumber \\
&\varphi_{p}^{per}  = \varphi_{p}^{(0)} + \lambda\sum_{q}^{ALL
MO}U_{qp}\varphi_{q}^{(0)} \Longrightarrow \nonumber \\
&\sum_{\mu}c^{per}_{\mu p}\phi_{\mu} = \sum_{\mu}c^{(0)}_{\mu p}\phi_{\mu} + 
\lambda\sum_{q}^{ALL
MO}\sum_{\mu}c^{(0)}_{\mu q}U_{qp}\phi_{\mu} \Longrightarrow \nonumber \\
&\sum_{\mu}\left\lbrace c^{per}_{\mu p} - [c^{(0)}_{\mu p} +
\lambda\sum_{q}^{ALL
MO}c^{(0)}_{\mu q}U_{qp}]\right\rbrace \phi_{\mu}  = 0
\end{align}
Because the $\phi_{\mu}$ is some arbitrary function, the term within the brace
should be zero. then by dropping the unnecessary superscript $(0)$, it turns
into:
\begin{equation}
 \label{orbital_response_gradient_eq:4}
c^{per}_{\mu p} = c_{\mu p} + \lambda\sum_{q}^{ALL
MO}c_{\mu q}U_{qp}
\end{equation}
By using the matrix notion, it becomes:
\begin{equation}
 \label{orbital_response_gradient_eq:5}
C^{per} = C + \lambda C_{n\times n}U_{n\times n}
\end{equation}
Here we assume that the number of basis set is $n$. $C$ is the MO coefficient
matrix that the row index is AO, and column index is MO.

It's interesting to see that under the first order perturbation, the $U$ matrix
defined in the (\ref{orbital_response_gradient_eq:5}) is actually unitary:
\begin{align}
  \label{orbital_response_gradient_eq:6}
\langle\varphi_{p}^{(1)}|\varphi_{q}^{(1)}\rangle &= \delta_{pq} \nonumber \\
\sum_{r}^{ALL MO}\sum_{s}^{ALL
MO}U_{pr}^{+}U_{sq}\langle\varphi_{r}^{(0)}|\varphi_{s}^{(0)}\rangle &=
\delta_{pq} \nonumber \\
\sum_{r}^{ALL MO}\sum_{s}^{ALL
MO}U_{pr}^{+}U_{sq}\delta_{rs} &= \delta_{pq} \nonumber \\
\sum_{r}^{ALL MO}U_{pr}^{+}U_{rq} &= \delta_{pq} \Longrightarrow\nonumber \\
U^{+}U &= I
\end{align}
Hence we can imagine that the matrix $U$ corresponds to some ``rotation'' to
the original MO space, in this case; the $U$ is also called ``MO rotation''.

Finally, let's go to see how to understand the 
(\ref{orbital_response_gradient_eq:1}) in another way. First let's rewrite the
Hamiltonian into:
\begin{equation}
 \hat{H}(\bm{R}_{0} + \Delta \bm{R}) = \hat{H}(\bm{R}_{0}) + \Delta
\bm{R}_{a}\frac{\partial \hat{H}}{\partial \bm{R}_{a}}
\end{equation}
This is derived in (\ref{GD_gradient_eq:7}). According to this expression,
the $\lambda$ in the original Hamiltonian is actually $\Delta\bm{R}_{a}$. Since
the MO coefficient is also some indirect function of $\bm{R}$, then it's able
to get a Taylor expansion for the $\bm{c}$ over the $\bm{R}$:
\begin{equation}
  \label{orbital_response_gradient_eq:7}
c_{\mu p}(\bm{R}_{0} + \Delta \bm{R}) =  c_{\mu p}(\bm{R}_{0}) + \Delta
\bm{R}_{a}\frac{\partial c_{\mu p}}{\partial \bm{R}_{a}}
\end{equation}
 Let's compare the (\ref{orbital_response_gradient_eq:7}) with the
(\ref{orbital_response_gradient_eq:1}), it turns out that:
\begin{equation}
   \label{orbital_response_gradient_eq:8}
\frac{\partial c_{\mu p}}{\partial \bm{R}_{a}} = \sum_{q}^{ALL
MO}c_{\mu q}U^{a}_{qp}
\end{equation}
In the following content, so we will use $U^{a}_{qp}$ to represent the rotation
matrix with respect to the derivatives on $\bm{R}_{a}$.


Now let's extend the expression of (\ref{orbital_response_gradient_eq:7}):
\begin{align}
    \label{orbital_response_gradient_eq:9}
c_{\mu p}(\bm{R}_{0} + \Delta \bm{R}) &=  c_{\mu p}(\bm{R}_{0}) + \Delta
\bm{R}_{a}\frac{\partial c_{\mu p}}{\partial \bm{R}_{a}} + \Delta
\bm{R}_{b}\frac{\partial c_{\mu p}}{\partial \bm{R}_{b}} \nonumber \\
& + \Delta\bm{R}_{a}\Delta\bm{R}_{a}\frac{\partial^{2} c_{\mu p}}{\partial
\bm{R}_{a}^{2}} +  \Delta\bm{R}_{b}\Delta\bm{R}_{b}\frac{\partial^{2} c_{\mu
p}}{\partial \bm{R}_{b}^{2}} +
\Delta\bm{R}_{a}\Delta\bm{R}_{b}\frac{\partial^{2} c_{\mu
p}}{\partial\bm{R}_{a}\partial\bm{R}_{b}}
\end{align}
In this expression, $\Delta\bm{R}_{a}$(it can be named as $\lambda_{a}$) and
$\Delta\bm{R}_{b}$ (named as $\lambda_{b}$) are equivalent to the first order
perturbation multipliers, and
$\Delta\bm{R}_{a}\Delta\bm{R}_{a}$($\lambda_{a}\lambda_{a}$),
$\Delta\bm{R}_{a}\Delta\bm{R}_{b}$($\lambda_{a}\lambda_{b}$) and
$\Delta\bm{R}_{b}\Delta\bm{R}_{b}$($\lambda_{b}\lambda_{b}$) are equivalent to
the second order multipliers. Hence according to the general perturbation
theory, it's known that:
\begin{align}
    \label{orbital_response_gradient_eq:10}
\frac{\partial^{2} c_{\mu
p}}{\partial\bm{R}_{a}\partial\bm{R}_{b}} &= \varphi^{(2)}_{ab} \nonumber \\
&= \sum_{q}^{ALL MO}c_{\mu q}U^{ab}_{qp}
\end{align}
The final expression in (\ref{orbital_response_gradient_eq:10}) can be easily
derived from the same manner in (\ref{orbital_response_gradient_eq:3}). The
$U^{ab}$ represents the responding matrix for the second order perturbation. 

Now let's try to see the relation between $U^{ab}$ and $U$:
\begin{equation}
 \begin{split}
  \frac{\partial}{\partial\bm{R}_{b}} \left( \frac{\partial c_{\mu
p}}{\partial\bm{R}_{a}}\right)  &= \sum_{q}^{ALL MO}\frac{\partial c_{\mu
q}}{\partial\bm{R}_{b}}U^{a}_{qp} + \sum_{q}^{ALL MO}c_{\mu q}\frac{\partial
U^{a}_{qp}}{\partial\bm{R}_{b}}  \\
&= \sum_{q}^{ALL MO}\sum_{r}^{ALL MO}c_{\mu r}U^{b}_{rq}U^{a}_{qp} +
\sum_{q}^{ALL
MO}c_{\mu q}\frac{\partial U^{a}_{qp}}{\partial\bm{R}_{b}}
 \end{split}
\label{orbital_response_gradient_eq:11}
\end{equation}
Here it's worthy to note that although $U^{b}_{rq}$ and $U^{a}_{qp}$ are
the same order perturbation, they are totally different. Physically the
$U^{b}_{rq}$ represents the responding matrix for MO coefficients caused by the
$\bm{R}_{a}$ change, and $U^{b}_{rq}$ represents the responding matrix for
MO coefficients caused by the $\bm{R}_{a}$ change; they are totally different.
That's the reason why label them separately as $a$ or $b$. 

Since the $q$ and $r$ index are all run over the MO space, we can exchange the
$q$ and $r$ label in the $\sum_{q}^{ALL MO}\sum_{r}^{ALL MO}$ without changing
anything; hence it gives:
\begin{equation}
 \begin{split}
  &\frac{\partial}{\partial\bm{R}_{b}} \left( \frac{\partial c_{\mu
p}}{\partial\bm{R}_{a}}\right)  = \sum_{q}^{ALL MO}\sum_{r}^{ALL MO}c_{\mu
q}U^{b}_{qr}U^{a}_{rp} + \sum_{q}^{ALL
MO}c_{\mu q}\frac{\partial U^{a}_{qp}}{\partial\bm{R}_{b}} \Longrightarrow \\
 &\sum_{q}^{ALL MO}c_{\mu q}U^{ab}_{qp} = \sum_{q}^{ALL MO}\left[ 
\sum_{r}^{ALL MO}U^{b}_{qr}U^{a}_{rp} + \frac{\partial
U^{a}_{qp}}{\partial\bm{R}_{b}}
\right] c_{\mu q} \Longrightarrow \\
&\sum_{q}^{ALL MO}\left\lbrace U^{ab}_{qp} - \left[ 
\sum_{r}^{ALL MO}U^{b}_{qr}U^{a}_{rp} + \frac{\partial
U^{a}_{qp}}{\partial\bm{R}_{b}}
\right]\right\rbrace  c_{\mu q}  = 0 \Longrightarrow \\
&U^{ab}_{qp}  =
\sum_{r}^{ALL MO}U^{b}_{qr}U^{a}_{rp} + \frac{\partial
U^{a}_{qp}}{\partial\bm{R}_{b}}
 \end{split}
\label{orbital_response_gradient_eq:12}
\end{equation}

This is the relation we will use in the future, while deriving the second
derivatives for the system energy.

%%%%%%%%%%%%%%%%%%%%%%%%%%%%%%%%%%%%%%%%%%%%%%%%%%%
\section{Analytical Derivatives for Integrals in AO Form}
%
%
%
%
In traditional wave function methods, we only encounter three kind of integrals
in AO form:
\begin{itemize}
 \item overlap integral $S_{\mu\nu}$: $\langle\phi_{\mu}|\phi_{\nu}\rangle$ 
 \item single electron integral $H_{\mu\nu}$:
$\langle\phi_{\mu}|\hat{H}_{1}|\phi_{\nu}\rangle$
 \item double electrons integral $\Pi_{\mu\nu\lambda\eta}$:
$(\mu\nu|\lambda\eta)$\footnote{ In another form, it can also expressed
as $\langle\phi_{\mu}\phi_{\lambda}|\frac{1}{r_{12}}|\phi_{\nu}\phi_{\eta}
\rangle$. See \ref{General_Notion_Hatree-Fock}. Here we note that the
derivatives for the integrals, is independent with the index arrangement.
That is to say,  no matter what kind of index arrangement we use, the gradient
expression will be same.}.
\end{itemize}

The analytical gradients for these AO integrals are very simple, what we need
to do is only to calculate the derivatives for the basis set functions. For
example, for the $S_{\mu\nu}$:
\begin{align}
 \label{AO_INT_gradient_eq:1}
S_{\mu\nu}^{a} &= \frac{\partial S_{\mu\nu}}{\partial \bm{R}_{a}} =
\langle\frac{\partial\phi_{\mu}}{\partial \bm{R}_{a}}|\phi_{\nu}\rangle + 
\langle\phi_{\mu}|\frac{\partial\phi_{\nu}}{\partial \bm{R}_{a}}\rangle
\nonumber \\
S_{\mu\nu}^{ab} &= \frac{\partial^{2} S_{\mu\nu}}{\partial \bm{R}_{a}\partial
\bm{R}_{b}} = \langle\frac{\partial^{2}\phi_{\mu}}{\partial
\bm{R}_{a}\partial\bm{R}_{b}}|\phi_{\nu}\rangle + \langle\frac{
\partial\phi_{\mu}}{\partial\bm{R}_{a}}|\frac{\partial\phi_{\nu}}{\partial\bm{R}
_{b}}\rangle \nonumber \\
&+
\langle\frac{\partial\phi_{\mu}}{\partial\bm{R}_{b}}|\frac{\partial\phi_{\nu}}{
\partial\bm{R}
_{a}}\rangle + \langle\phi_{\mu}|\frac{\partial^{2}\phi_{\nu}}{\partial
\bm{R}_{a}\partial\bm{R}_{b}}\rangle 
\end{align}
For the other AO integrals, such as $\Pi$ etc. we can always derive their
analytical gradient expressions in a like manner, and label it as $\Pi^{a}$
and $\Pi^{ab}$ etc. Hence in the following chapter, we will only use such
abbreviated form to express their analytical gradients\footnote{The detailed
expression can see the chapter 3 in Schaefer
book\cite{New_Dimension_for_Derivatives_Calculation}}.



%%%%%%%%%%%%%%%%%%%%%%%%%%%%%%%%%%%%%%%%%%%%%%%%%%%
\section{Analytical Gradients for Integrals in MO Form}
%
%
%
%
Before we step into the Hatree-Fock analytical gradients, it's necessary for us
to study the integrals in MO form; they are the bricks for building
analytical gradients for all traditional wave function methods.

In quantum chemistry, the complexity of the energy expression, in my opinion;
is partially attributed to its multi-dimensional index character. For example,
the total energy for Hatree-Fock framework is depending on six indices:
\begin{equation}
 \label{MO_INT_gradient_general_eq:1}
E \rightarrow E(i,j,\mu,\nu,\lambda,\eta)
\end{equation}
$i$ and $j$ are indices for occupied orbitals, and $\mu,\nu,\lambda,\eta$ are
four indices for AO. It's very common in quantum chemistry, that to switch an
expression from MO to AO or from AO to MO; in such process; the usual procedure
is to ``sum over'' the AO or MO index so that to hide the information (For
example, the density matrix $P = \sum_{i}^{occ}c_{\mu i}c_{\nu i}$, the
expression runs over the occupied MO so that $P$ only has two AO indices).

So now we come to the question that why we want to study the analytical
gradients for the integrals in the MO form? As what we can see in the energy
expression, we can express it either in MO form or in AO form, they are
identical with each other. So here it's natural to put forward such a question,
that it it necessary to study the gradient theory in MO form?

In my opinion, the answer is ``Yes''. For all the gradient theory, since it's
either based on HF framework or based on the Kohn-sham framework, basically
it's necessary to introduce the MO as the foundation. If MO is introduced, then
we have to express the MO response to the perturbation, which is reflected in
the MO rotation matrix of $U$ as what we have shown above. In the energy is
expressed totally in AO form, which means to hide the MO index; then it's
impossible to find a way to express the MO response for perturbation.  This
fact implies that the MO response matrix of $U$ is the key for understanding
the gradient theory, on the other hand; we can understand it in this way that
the MO coefficients should appear as some fundamental and independent unit for
the gradient theory.

Now let's give the HF equation as an example, the energy in AO form can be
generally expressed in matrix way:
\begin{equation}
 \label{MO_INT_gradient_general_eq:2}
E = P\cdot H + \frac{1}{2} P\cdot H \cdot P^{'}
\end{equation}
Here $P$ and $P^{'}$ are density matrices:
\begin{align}
 \label{MO_INT_gradient_general_eq:3}
P = C_{o}C_{o}^{+},  \quad  P_{\mu\nu} &=   \sum_{i}^{occ}c_{\mu i}c_{\nu i}
\quad P^{'}_{\lambda\eta} =   \sum_{j}^{occ}c_{\lambda j}c_{\eta j}
\rightarrow\nonumber \\
P\cdot H  &= \sum_{\mu\nu}P_{\mu\nu}h_{\mu\nu} \nonumber \\
P\cdot H \cdot P^{'} &=
\sum_{\mu\nu}\sum_{\lambda\eta}P_{\mu\nu}P^{'}_{\lambda\eta}\Pi_{
\mu\nu\lambda\eta}
\end{align}

In the (\ref{MO_INT_gradient_general_eq:2}), the advantage is that the integral
and MO coefficient are separated, so the gradient expression for $E$ can
thoroughly built from the matrix notion:
\begin{equation}
\label{MO_INT_gradient_general_eq:4}
E^{[a]} = P^{[a]}\cdot H + P\cdot H^{[a]} + \frac{1}{2} P^{[a]}\cdot H \cdot
P^{'} + \frac{1}{2} P\cdot H^{[a]} \cdot P^{'} + \frac{1}{2} P\cdot H^{[a]}
\cdot P^{'[a]}
\end{equation}
Here we introduce the superscript of $[a]$, which indicates the gradient
expression for that term. This is different from the superscript of $a$ we used
in the (\ref{AO_INT_gradient_eq:1}). Actually we only use ``$a$'' to represent
the AO integral gradient expression, and use ``$[a]$'' in all the other cases
to signal it's a gradient expression.

However, $P^{[a]}$ is definitely introducing another MO dimension for the
expression, which is reflected by the $U$ matrix. According to Winger's
theorem, that $E^{[a]}$ will not contain any $U$ matrix (because it represents
the first order perturbation), so how can we eliminate it?

To eliminate this additional MO dimension, in fact we have to usher into other
MO dimensions so that to apply the orthogonality condition. The new dimensions
should be closed related to the $\Pi$ and $H$ matrix so that additional
transformation can be applied. However, for the $\Pi$ and $H$ there's no more
MO information left so the transformation is difficult to occur.

It's under such consideration, it's believed that to study the analytical
expression of integrals in MO form is necessary. The use of MO
orthogonality condition will bring us potential benefit for working out the
final analytical gradient expression. 

%%%%%%%%%%%%%%%%%%%%%%%%%%%%%%%%%%%%%%%%%%%%%%%%%%
\subsection{Overlap Integral}\label{overlap_int_general_derivatives}
%
%
Let's start from the overlap integral\footnote{From this section, we will not
mark clearly that the label is going through ``ALL MO''. However, as what we
know; the response rotation matrix always go over all the MO space.}:
\begin{align}
\label{overlap_MO_INT_gradient_eq:1}
\frac{\partial S_{pq}}{\partial \bm{R}_{a}} &= \frac{\partial}{\partial
\bm{R}_{a}}\left( \sum_{\mu\nu}c_{\mu p}c_{\nu q}S_{\mu\nu}\right) \nonumber\\
&= \sum_{\mu\nu}\frac{\partial c_{\mu p}}{\partial \bm{R}_{a}}c_{\nu
q}S_{\mu\nu} + \sum_{\mu\nu}c_{\nu p}\frac{\partial c_{\nu  q}}{\partial
\bm{R}_{a}}S_{\mu\nu} + \sum_{\mu\nu}c_{\mu p}c_{\nu q}\frac{\partial
S_{\mu\nu}}{\partial \bm{R}_{a}} \nonumber\\
&= \sum_{\mu\nu}\sum_{r}c_{\mu r}U^{a}_{rp}c_{\nu
q}S_{\mu\nu} + \sum_{\mu\nu}\sum_{r}c_{\mu p}c_{\nu r}U^{a}_{rq}S_{\mu\nu} +
\sum_{\mu\nu}c_{\mu p}c_{\nu q}S^{a}_{\mu\nu} \nonumber \\
&= \sum_{r}\left[U^{a}_{rp} \left( \sum_{\mu\nu}c_{\mu r}c_{\nu
q}S_{\mu\nu}\right)  + U^{a}_{rq}\left( \sum_{\mu\nu}c_{\mu p}c_{\nu
r}S_{\mu\nu}\right) \right]  + \sum_{\mu\nu}c_{\mu p}c_{\nu q}S^{a}_{\mu\nu}
\nonumber \\
&= \sum_{r}\left[U^{a}_{rp}S_{qr} + U^{a}_{rq}S_{pr}\right]  + S^{a}_{pq}
\end{align}
Giving by the orthogonality between the MO, we have:
\begin{align}
 \label{overlap_MO_INT_gradient_eq:2}
S_{pq} &= \delta_{pq} \Longrightarrow \nonumber \\
\frac{\partial S_{pq} }{\partial \bm{R}_{a}} &= \frac{\partial^{q} S_{pq}
}{\partial \bm{R}_{a}\partial \bm{R}_{b}} = 0
\end{align}
Hence we have:
\begin{equation}
 \label{overlap_MO_INT_gradient_eq:3}
S_{pq}^{[a]} = \frac{\partial S_{pq}}{\partial \bm{R}_{a}} = U^{a}_{qp} +
U^{a}_{pq} + S^{a}_{pq} = 0
\end{equation} 
It seems that this expression just links the U matrix and the overlap integral
derivatives together. 

On the other hand, let's understand this term in physical
way. Compared with the results shown in (\ref{1st_approximation_WT}),
it's known that the $a^{1}_{kk}$ is unable to be determined, and here;
we can see that $U^{a}_{pp}$ just equals to $-\dfrac{1}{2}S^{a}_{pp}$,
it can not be determined, too.

For the second derivatives:
\begin{align}
  \label{overlap_MO_INT_gradient_eq:4}
S_{pq}^{[ab]} &= \frac{\partial^{2} S_{pq}}{\partial \bm{R}_{a}\partial
\bm{R}_{b}}  = \frac{\partial U^{a}_{qp}}{\partial \bm{R}_{b}} +
\frac{\partial U^{a}_{pq}}{\partial \bm{R}_{b}} + 
\frac{\partial S^{a}_{pq}}{\partial \bm{R}_{b}}
\xrightarrow{\text{given by \ref{orbital_response_gradient_eq:12}}}\nonumber \\
&= (U^{ab}_{qp} - \sum_{r}U^{b}_{qr}U^{a}_{rp}) + (U^{ab}_{pq} -
\sum_{r}U^{b}_{pr}U^{a}_{rq}) + S^{ab}_{pq}  \nonumber \\
&+\sum_{r}\sum_{\mu\nu}
c_{\mu r}U^{b}_{rp}c_{\nu q}S^{a}_{\mu\nu} + \sum_{r}\sum_{\mu\nu} c_{\mu
p}c_{\nu r}U^{b}_{rq}S^{a}_{\mu\nu} \nonumber \\
&= U^{ab}_{qp}  + U^{ab}_{pq} + S^{ab}_{pq} \nonumber \\
&+ \sum_{r}\left[ U^{b}_{rp}S^{a}_{qr} + U^{b}_{rq}S^{a}_{pr} -
U^{b}_{qr}U^{a}_{rp} - U^{b}_{pr}U^{a}_{rq} \right] 
\end{align}
Now let's use the relation in (\ref{overlap_MO_INT_gradient_eq:3}), that gives:
\begin{equation}
 \label{overlap_MO_INT_gradient_eq:5}
U^{b}_{rp} +
U^{b}_{pr} + S^{b}_{pr} = 0 \Longrightarrow U^{b}_{rp} = -S^{b}_{pr} -
U^{b}_{pr} 
\end{equation}
We are going to use this relation to shave off the $U$ inside the $US$ term:
\begin{align}
 \label{overlap_MO_INT_gradient_eq:6}
S_{pq}^{[ab]} &= U^{ab}_{qp}  + U^{ab}_{pq} + S^{ab}_{pq} \nonumber \\
&+ \sum_{r}\left[ \left( -U^{b}_{pr} - S^{b}_{pr}\right) S^{a}_{qr} +
\left( -U^{b}_{qr} - S^{b}_{qr}\right) S^{a}_{pr} -U^{b}_{qr}U^{a}_{rp} -
U^{b}_{pr}U^{a}_{rq} \right] \nonumber \\
&=  U^{ab}_{qp}  + U^{ab}_{pq} + S^{ab}_{pq} \nonumber \\
&+ \sum_{r}\left[ -S^{b}_{pr} S^{a}_{qr}  - S^{b}_{qr} S^{a}_{pr}
-U^{b}_{pr} S^{a}_{qr} -U^{b}_{qr}S^{a}_{pr}
 -U^{b}_{qr}U^{a}_{rp} - U^{b}_{pr}U^{a}_{rq} \right] \nonumber \\
&=  U^{ab}_{qp}  + U^{ab}_{pq} + S^{ab}_{pq} \nonumber \\
&+ \sum_{r}\left[ -S^{b}_{pr} S^{a}_{qr}  - S^{b}_{qr} S^{a}_{pr}
-U^{b}_{pr}\left( S^{a}_{qr} + U^{a}_{rq} \right) 
-U^{b}_{qr}\left( S^{a}_{pr} + U^{a}_{rp}\right) \right]
\end{align}
Because the overlap matrix of $S$ is hermitian, and obviously its derivative
matrix $S^{a}$ is also hermitian; so we can exchange the label of $S$ without
changing its value:
\begin{equation}
 \label{overlap_MO_INT_gradient_eq:7}
S^{a}_{qr} = S^{a}_{rq} \quad S^{a}_{pr} = S^{a}_{rp}
\end{equation}
Hence combined with (\ref{overlap_MO_INT_gradient_eq:3}), the
(\ref{overlap_MO_INT_gradient_eq:6}) will further converted to:
\begin{align}
 \label{overlap_MO_INT_gradient_eq:8}
S_{pq}^{[ab]} &=  U^{ab}_{qp}  + U^{ab}_{pq} + S^{ab}_{pq} \nonumber \\
&+ \sum_{r}\left[ -S^{b}_{pr} S^{a}_{qr}  - S^{b}_{qr} S^{a}_{pr}
+U^{b}_{pr}U^{a}_{qr} +U^{b}_{qr}U^{a}_{pr}\right]
\end{align}
This is the final result.

%%%%%%%%%%%%%%%%%%%%%%%%%%%%%%%%%%%%%%%%%%%%%%%%%%
\subsection{Core Hamiltonian Integral}
%
%
Now let's turn to the single electron integral(it's also named as core
Hamiltonian integral). By simply copying the process in
(\ref{overlap_MO_INT_gradient_eq:1}), we can have the first order derivatives
for $H_{pq}$:
\begin{align}
\label{core_Hamiltonian_MO_INT_gradient_eq:1}
\frac{\partial H_{pq}}{\partial \bm{R}_{a}} &= \frac{\partial}{\partial
\bm{R}_{a}}\left( \sum_{\mu\nu}c_{\mu p}c_{\nu q}H_{\mu\nu}\right) \nonumber\\
&= \sum_{\mu\nu}\frac{\partial c_{\mu p}}{\partial \bm{R}_{a}}c_{\nu
q}H_{\mu\nu} + \sum_{\mu\nu}c_{\nu p}\frac{\partial c_{\nu  q}}{\partial
\bm{R}_{a}}H_{\mu\nu} + \sum_{\mu\nu}c_{\mu p}c_{\nu q}\frac{\partial
H_{\mu\nu}}{\partial \bm{R}_{a}} \nonumber\\
&= \sum_{\mu\nu}\sum_{r}c_{\mu r}U^{a}_{rp}c_{\nu
q}H_{\mu\nu} + \sum_{\mu\nu}\sum_{r}c_{\mu p}c_{\nu r}U^{a}_{rq}H_{\mu\nu} +
\sum_{\mu\nu}c_{\mu p}c_{\nu q}H^{a}_{\mu\nu} \nonumber \\
&= \sum_{r}\left[U^{a}_{rp} \left( \sum_{\mu\nu}c_{\mu r}c_{\nu
q}H_{\mu\nu}\right)  + U^{a}_{rq}\left( \sum_{\mu\nu}c_{\mu p}c_{\nu
r}H_{\mu\nu}\right) \right]  + \sum_{\mu\nu}c_{\mu p}c_{\nu q}H^{a}_{\mu\nu}
\nonumber \\
&= \sum_{r}\left[U^{a}_{rp}H_{qr} + U^{a}_{rq}H_{pr}\right]  + H^{a}_{pq}
\end{align}
Compared with overlap integral case, here we do not have orthogonality
condition to simply the final result.

For the second derivatives, it gives:
\begin{align}
\label{core_Hamiltonian_MO_INT_gradient_eq:2}
H_{pq}^{[ab]} &= \frac{\partial^{2} H_{pq}}{\partial \bm{R}_{a}\partial
\bm{R}_{b}} =  \frac{\partial}{\partial
\bm{R}_{b}}\left( \sum_{r}\left[U^{a}_{rp}H_{qr} + U^{a}_{rq}H_{pr}\right]  +
H^{a}_{pq}\right)  \nonumber \\
&= \underbrace{\sum_{r}\left[ 
\frac{\partial U^{a}_{rp} }{\partial \bm{R}_{b}}H_{qr} + 
\frac{\partial U^{a}_{rq} }{\partial \bm{R}_{b}}H_{pr}
\right]}_{\text{term 1}}  +
\underbrace{\sum_{r}\left[
\frac{\partial H_{qr} }{\partial \bm{R}_{b}}U^{a}_{rp} + 
\frac{\partial H_{pr} }{\partial \bm{R}_{b}}U^{a}_{rq}
\right]}_{\text{term 2}}  \nonumber \\
&+\underbrace{\frac{\partial H^{a}_{pq} }{\partial \bm{R}_{b}}}_{\text{term 3}}
\end{align}

Now let's compute each of the three terms. For the term 3, it's the most
easiest one:
\begin{align}
 \label{core_Hamiltonian_MO_INT_gradient_eq:3}
\frac{\partial H^{a}_{pq} }{\partial \bm{R}_{b}} &=
\sum_{r}\left[U^{b}_{rp}H^{a}_{qr} + U^{b}_{rq}H^{a}_{pr}\right]  + H^{ab}_{pq}
\end{align}

For the term 2:
\begin{align}
  \label{core_Hamiltonian_MO_INT_gradient_eq:4}
& \sum_{r}\left[
\frac{\partial H_{qr} }{\partial \bm{R}_{b}}U^{a}_{rp} + 
\frac{\partial H_{pr} }{\partial \bm{R}_{b}}U^{a}_{rq}
\right] \nonumber \\
&=\sum_{r} \left\lbrace\sum_{s} \left[H_{qs} U^{b}_{sr} +
H_{rs}U^{b}_{sq}\right]U^{a}_{rp} +  H^{b}_{qr}U^{a}_{rp} \right\rbrace
\nonumber \\
&+
\sum_{r} \left\lbrace\sum_{s} \left[H_{ps} U^{b}_{sr} +
H_{rs}U^{b}_{sp}\right]U^{a}_{rq} +  H^{b}_{pr}U^{a}_{rq} \right\rbrace
\nonumber \\
&= \sum_{r}\left[  H^{b}_{qr}U^{a}_{rp} + H^{b}_{pr}U^{a}_{rq}\right] +
\underbrace{\sum_{rs}\left[ H_{qs} U^{b}_{sr}U^{a}_{rp} + H_{ps}
U^{b}_{sr}U^{a}_{rq}\right]}_{\text{term 4}} \nonumber \\
&+  \sum_{rs}\left[ H_{rs}U^{b}_{sq}U^{a}_{rp} +
H_{rs}U^{b}_{sp}U^{a}_{rq}\right]
\end{align}

Finally, for the term 1:
\begin{align}
 \label{core_Hamiltonian_MO_INT_gradient_eq:5}
&\sum_{r}\left[ 
\frac{\partial U^{a}_{rp} }{\partial \bm{R}_{b}}H_{qr} + 
\frac{\partial U^{a}_{rq} }{\partial \bm{R}_{b}}H_{pr}
\right] \nonumber \\
& = \sum_{r}\left[ U^{ab}_{rp}  -
\sum_{s}U^{b}_{rs}U^{a}_{sp} \right]H_{qr} +
\sum_{r}\left[ U^{ab}_{rq}  -
\sum_{s}U^{b}_{rs}U^{a}_{sq} \right]H_{pr}
 \nonumber \\
&= \sum_{r}\left[  U^{ab}_{rp}H_{qr} + U^{ab}_{rq}H_{pr}\right] -
\underbrace{\sum_{rs}\left[ 
U^{b}_{rs}U^{a}_{sp} H_{qr} +
U^{b}_{rs}U^{a}_{sq} H_{pr}\right]}_{\text{term 5}} 
\end{align}

Now let's compare the term 4 and term 5, it's easy to see that they are
actually equal to each other. So in the whole summation, they are canceled.
Finally, the second derivatives for the core Hamiltonian can be expressed as:
\begin{align}
  \label{core_Hamiltonian_MO_INT_gradient_eq:6}
H_{pq}^{[ab]} &= \sum_{r}\left[ H_{qr} U^{ab}_{rp} + H_{pr}U^{ab}_{rq}\right]
\nonumber \\
&+ \sum_{r}\left[  H^{b}_{qr}U^{a}_{rp} + H^{b}_{pr}U^{a}_{rq}\right]  + 
\sum_{rs}\left[ H_{rs}U^{b}_{sq}U^{a}_{rp} +
H_{rs}U^{b}_{sp}U^{a}_{rq}\right] \nonumber \\
&+ \sum_{r}\left[H^{a}_{qr}U^{b}_{rp} + H^{a}_{pr}U^{b}_{rq}\right]  +
H^{ab}_{pq}
\end{align}

%%%%%%%%%%%%%%%%%%%%%%%%%%%%%%%%%%%%%%%%%%%%%%%%%%
\subsection{Two Electrons Integral}
%
%
For the first order derivatives, we follow the same procedure
above\footnote{In this section, we will meet many labels. So we use
$p,q,r,s,t,u,v$ etc. to represent the general MO orbitals.}:
\begin{align}
\label{two_electron_MO_INT_gradient_eq:1}
\Pi_{pqrs}^{[a]} &=\frac{\partial \Pi_{pqrs}}{\partial \bm{R}_{a}} =
\frac{\partial}{\partial \bm{R}_{a}}\left( \sum_{\mu\nu\lambda\eta}c_{\mu
p}c_{\nu q}c_{\lambda r}c_{\eta s}\Pi_{\mu\nu\lambda\eta}\right) \nonumber\\
&=  \sum_{\mu\nu\lambda\eta}
\frac{\partial c_{\mu p}}{\partial\bm{R}_{a}}
c_{\nu q}c_{\lambda r}c_{\eta s}
\Pi_{\mu\nu\lambda\eta} + 
       \sum_{\mu\nu\lambda\eta}
\frac{\partial c_{\nu q}}{\partial \bm{R}_{a}} 
c_{\mu p}c_{\lambda r}c_{\eta s}
\Pi_{\mu\nu\lambda\eta} \nonumber \\
&+  \sum_{\mu\nu\lambda\eta}
\frac{\partial c_{\lambda r}}{\partial\bm{R}_{a}}
c_{\mu p}c_{\nu q}c_{\eta s}
\Pi_{\mu\nu\lambda\eta} + 
       \sum_{\mu\nu\lambda\eta}
\frac{\partial c_{\eta s}}{\partial \bm{R}_{a}} 
c_{\mu p}c_{\nu q}c_{\lambda r}
\Pi_{\mu\nu\lambda\eta}   \nonumber \\
&+\sum_{\mu\nu\lambda\eta}
c_{\mu p}c_{\nu q}c_{\lambda r}c_{\eta s}
\Pi^{a}_{\mu\nu\lambda\eta}\nonumber \\
&= \sum_{\mu\nu\lambda\eta}
      \sum_{t}c_{\mu t}U^{a}_{tp}
c_{\nu q}c_{\lambda r}c_{\eta s}
\Pi_{\mu\nu\lambda\eta} + 
       \sum_{\mu\nu\lambda\eta}
       \sum_{t}c_{\nu t}U^{a}_{tq}
c_{\mu p}c_{\lambda r}c_{\eta s}
\Pi_{\mu\nu\lambda\eta} \nonumber \\
&+  \sum_{\mu\nu\lambda\eta}
      \sum_{t}c_{\lambda t}U^{a}_{tr}
c_{\mu p}c_{\nu q}c_{\eta s}
\Pi_{\mu\nu\lambda\eta} + 
       \sum_{\mu\nu\lambda\eta}
      \sum_{t}c_{\eta t}U^{a}_{ts} 
c_{\mu p}c_{\nu q}c_{\lambda r}
\Pi_{\mu\nu\lambda\eta} \nonumber \\
&+\sum_{\mu\nu\lambda\eta}
c_{\mu p}c_{\nu q}c_{\lambda r}c_{\eta s}
\Pi^{a}_{\mu\nu\lambda\eta}\nonumber \\
&= \sum_{t}\left[ 
U^{a}_{tp}\Pi_{tqrs} +
U^{a}_{tq}\Pi_{ptrs} + 
U^{a}_{tr}\Pi_{pqts} + 
U^{a}_{ts}\Pi_{pqrt}  
\right] + \Pi^{a}_{pqrs}
\end{align}

For the second order derivatives, it will become more complicated:
\begin{align}
 \label{two_electron_MO_INT_gradient_eq:2}
\Pi_{pqrs}^{[ab]} &=\frac{\partial^{2} \Pi_{pqrs}}{\partial \bm{R}_{a}\partial
\bm{R}_{b}} \nonumber \\
&= \frac{\partial}{\partial\bm{R}_{b}}\left(
\underbrace{\sum_{t}U^{a}_{tp}\Pi_{tqrs}}_{\text{term 1}} +
\underbrace{\sum_{t}U^{a}_{tq}\Pi_{ptrs}}_{\text{term 2}} + 
\underbrace{\sum_{t}U^{a}_{tr}\Pi_{pqts}}_{\text{term 3}} + 
\underbrace{\sum_{t}U^{a}_{ts}\Pi_{pqrt}}_{\text{term 4}} \right)  \nonumber \\
 &+
\frac{\partial}{\partial\bm{R}_{b}}\left(
\underbrace{\Pi^{a}_{pqrs}}_{\text{term
5}}\right) 
\end{align}
In this expression, we have five terms to differentiate. Now let's do it one by
one.

For the first term (term 1), we have:
\begin{align}
  \label{two_electron_MO_INT_gradient_eq:3}
&\frac{\partial}{\bm{R}_{b}}\left( \sum_{t}U^{a}_{tp}\Pi_{tqrs}\right)
\nonumber \\
&=\sum_{t}\frac{\partial U^{a}_{tp}}{\bm{R}_{b}}\Pi_{tqrs} +
\sum_{t}U^{a}_{tp}\frac{\partial \Pi_{tqrs}}{\bm{R}_{b}} \nonumber \\
&= \sum_{t}\left( U^{ab}_{tp} - \sum_{u}U^{b}_{tu}U^{a}_{up}\right) \Pi_{tqrs}
\nonumber \\
&+ \sum_{t}U^{a}_{tp}\sum_{u}\left( 
U^{b}_{ut}\Pi_{uqrs} +
U^{b}_{uq}\Pi_{turs} + 
U^{b}_{ur}\Pi_{tqus} + 
U^{b}_{us}\Pi_{tqru}  
\right) + \sum_{t}U^{a}_{tp}\Pi^{b}_{tqrs}
\end{align}
Here in the (\ref{two_electron_MO_INT_gradient_eq:3}), we can observe that:
\begin{align}
   \label{two_electron_MO_INT_gradient_eq:4}
\sum_{t}\sum_{u}U^{b}_{tu}U^{a}_{up}\Pi_{tqrs} &=
\sum_{t}\sum_{u}U^{a}_{tp}U^{b}_{ut}\Pi_{uqrs}
\end{align}
Since the label of $u$ and $t$ are arbitrary, they can exchange with each other
without changing the value. Hence finally, the term 1 becomes:
\begin{align}
 \label{two_electron_MO_INT_gradient_eq:5}
 \frac{\partial}{\bm{R}_{b}}\left( \sum_{t}U^{a}_{tp}\Pi_{tqrs}\right)
&= \sum_{t} U^{ab}_{tp}\Pi_{tqrs} +
\sum_{t}U^{a}_{tp}\Pi^{b}_{tqrs} \nonumber \\
&+ \sum_{t}\sum_{u}U^{a}_{tp}\left(
U^{b}_{uq}\Pi_{turs} + 
U^{b}_{ur}\Pi_{tqus} + 
U^{b}_{us}\Pi_{tqru}  
\right) \nonumber \\
&=   \sum_{t} \left[ U^{ab}_{tp}\Pi_{tqrs} + U^{a}_{tp}\Pi^{b}_{tqrs}\right] 
\nonumber \\
&+ \sum_{t}\sum_{u}U^{a}_{tp}\left(
U^{b}_{uq}\Pi_{turs} + 
U^{b}_{ur}\Pi_{tqus} + 
U^{b}_{us}\Pi_{tqru}  
\right)
\end{align}

Similarly, for the term 2, 3, 4 by repeating the same process we can get the
similar result:
\begin{align}
  \label{two_electron_MO_INT_gradient_eq:6}
&\frac{\partial}{\bm{R}_{b}} \left( \sum_{t}U^{a}_{tq}\Pi_{ptrs}\right)
\nonumber \\
&=\sum_{t}\left[ U^{ab}_{tq}\Pi_{ptrs} + U^{a}_{tq}\Pi^{b}_{ptrs}\right] 
\nonumber \\
&+ \sum_{t}\sum_{u}U^{a}_{tq}\left(
U^{b}_{up}\Pi_{utrs} + 
U^{b}_{ur}\Pi_{ptus} + 
U^{b}_{us}\Pi_{ptru}  
\right)
\end{align}
This is for the term 2.

For the term 3, it gives:
\begin{align}
  \label{two_electron_MO_INT_gradient_eq:7}
&\frac{\partial}{\bm{R}_{b}} \left( \sum_{t}U^{a}_{tr}\Pi_{pqts}\right)
\nonumber \\
&=\sum_{t}\left[ U^{ab}_{tr}\Pi_{pqts} + U^{a}_{tr}\Pi^{b}_{pqts}\right] 
\nonumber \\
&+ \sum_{t}\sum_{u}U^{a}_{tr}\left(
U^{b}_{up}\Pi_{uqts} + 
U^{b}_{uq}\Pi_{puts} + 
U^{b}_{us}\Pi_{pqtu}  
\right)
\end{align}

Term 4 gives:
\begin{align}
  \label{two_electron_MO_INT_gradient_eq:8}
&\frac{\partial}{\bm{R}_{b}} \left( \sum_{t}U^{a}_{ts}\Pi_{pqrt}\right)
\nonumber \\
&=\sum_{t}\left[ U^{ab}_{ts}\Pi_{pqrt} + U^{a}_{ts}\Pi^{b}_{pqrt}\right] 
\nonumber \\
&+ \sum_{t}\sum_{u}U^{a}_{ts}\left(
U^{b}_{up}\Pi_{uqrt} + 
U^{b}_{uq}\Pi_{purt} + 
U^{b}_{ur}\Pi_{pqut}  
\right)
\end{align}

For the last term, we have:
\begin{align}
 \label{two_electron_MO_INT_gradient_eq:9}
& \frac{\partial}{\partial\bm{R}_{b}}\left(\Pi^{a}_{pqrs}\right) \nonumber \\
&= \sum_{t}\left[ 
U^{b}_{tp}\Pi^{a}_{tqrs} +
U^{b}_{tq}\Pi^{a}_{ptrs} + 
U^{b}_{tr}\Pi^{a}_{pqts} + 
U^{b}_{ts}\Pi^{a}_{pqrt}  
\right] + \Pi^{ab}_{pqrs}
\end{align}

By collecting all the terms, we can finally get the second derivatives for the
integral $(pq|rs)$:
\begin{equation}
 \begin{split}
  \Pi_{pqrs}^{[ab]} &=(pq|rs)^{[ab]} = \frac{\partial^{2} (pq|rs)}{\partial
\bm{R}_{a}\partial\bm{R}_{b}} \\
&= \Pi^{ab}_{pqrs} + 
\sum_{t} \left[ 
U^{ab}_{tp}\Pi_{tqrs} + 
U^{ab}_{tq}\Pi_{ptrs} +
U^{ab}_{tr}\Pi_{pqts} +
U^{ab}_{ts}\Pi_{pqrt}                      
\right] \\
&+ 
\sum_{t} \left[ 
U^{a}_{tp}\Pi^{b}_{tqrs}  +
U^{a}_{tq}\Pi^{b}_{ptrs}  +
U^{a}_{tr}\Pi^{b}_{pqts}  +
U^{a}_{ts}\Pi^{b}_{pqrt}
\right]  \\
&+
\sum_{t}\left[ 
U^{b}_{tp}\Pi^{a}_{tqrs} +
U^{b}_{tq}\Pi^{a}_{ptrs} + 
U^{b}_{tr}\Pi^{a}_{pqts} + 
U^{b}_{ts}\Pi^{a}_{pqrt}  
\right]  \\
&+
\sum_{t}\sum_{u} \\
&\Big\{U^{a}_{tp}U^{b}_{uq}\Pi_{turs} + 
U^{a}_{tp}U^{b}_{ur}\Pi_{tqus} + 
U^{a}_{tp}U^{b}_{us}\Pi_{tqru}  \\
&+ 
U^{a}_{tq}U^{b}_{up}\Pi_{utrs} + 
U^{a}_{tq}U^{b}_{ur}\Pi_{ptus} + 
U^{a}_{tq}U^{b}_{us}\Pi_{ptru}   \\
&+
U^{a}_{tr}U^{b}_{up}\Pi_{uqts} + 
U^{a}_{tr}U^{b}_{uq}\Pi_{puts} + 
U^{a}_{tr}U^{b}_{us}\Pi_{pqtu}    \\
&+
U^{a}_{ts}U^{b}_{up}\Pi_{uqrt} + 
U^{a}_{ts}U^{b}_{uq}\Pi_{purt} + 
U^{a}_{ts}U^{b}_{ur}\Pi_{pqut}  \Big\} 
 \end{split}
 \label{two_electron_MO_INT_gradient_eq:10}
\end{equation}


Again, let's reorganize the terms in the last sum, which gives:
\begin{equation}
 \begin{split}
&sum =\\
&\sum_{t}\sum_{u}
\left( U^{a}_{tp}U^{b}_{uq}\Pi_{turs} + 
U^{a}_{tq}U^{b}_{up}\Pi_{utrs}\right) 
+\sum_{t}\sum_{u}
\left( U^{a}_{tp}U^{b}_{ur}\Pi_{tqus} + 
U^{a}_{tr}U^{b}_{up}\Pi_{uqts}\right)\\
& +\\ 
&\sum_{t}\sum_{u}
\left(U^{a}_{tp}U^{b}_{us}\Pi_{tqru}  +
U^{a}_{ts}U^{b}_{up}\Pi_{uqrt}\right) 
+\sum_{t}\sum_{u}
\left(U^{a}_{tq}U^{b}_{ur}\Pi_{ptus}  +
U^{a}_{tr}U^{b}_{uq}\Pi_{puts}\right) \\   
&+\\
&\sum_{t}\sum_{u}
\left(U^{a}_{tq}U^{b}_{us}\Pi_{ptru}  +
U^{a}_{ts}U^{b}_{uq}\Pi_{purt}\right)  
+\sum_{t}\sum_{u}
\left(U^{a}_{tr}U^{b}_{us}\Pi_{pqtu}  +
U^{a}_{ts}U^{b}_{ur}\Pi_{pqut}\right) 
\end{split}
\label{two_electron_MO_INT_gradient_eq:12}
\end{equation}
Here we indicate that since the label $t$ and $u$ are arbitrary, then we have
such relation:
\begin{align}
& \sum_{t}\sum_{u}
\left( U^{a}_{tp}U^{b}_{uq}\Pi_{turs} + 
U^{a}_{tq}U^{b}_{up}\Pi_{utrs}\right)   \nonumber \\
&=
\sum_{t}\sum_{u}
\left( U^{a}_{tp}U^{b}_{uq}\Pi_{turs} + 
U^{a}_{uq}U^{b}_{tp}\Pi_{turs}\right)  \nonumber  \\
&\sum_{t}\sum_{u}
\left( 
U^{a}_{tp}U^{b}_{uq} + 
U^{b}_{tp}U^{a}_{uq}
\right) \Pi_{turs}
\end{align}

Hence finally the sum becomes:
\begin{equation}
 \begin{split}
&sum =\\
&\sum_{t}\sum_{u}
\left(U^{a}_{tp}U^{b}_{uq} + 
U^{b}_{tp}U^{a}_{uq}\right) \Pi_{turs}
+\sum_{t}\sum_{u}
\left(U^{a}_{tp}U^{b}_{ur} + 
U^{b}_{tp}U^{a}_{ur}\right)\Pi_{tqus}   +\\ 
&\sum_{t}\sum_{u}
\left(U^{a}_{tp}U^{b}_{us} +
U^{b}_{tp}U^{a}_{us}\right) \Pi_{tqru}
+\sum_{t}\sum_{u}
\left(U^{a}_{tq}U^{b}_{ur}  +
U^{b}_{tq}U^{a}_{ur}\right)\Pi_{ptus}    +\\
&\sum_{t}\sum_{u}
\left(U^{a}_{tq}U^{b}_{us}  +
U^{b}_{tq}U^{a}_{us}\right)\Pi_{ptru}  
+\sum_{t}\sum_{u}
\left(U^{a}_{tr}U^{b}_{us}  +
U^{b}_{tr}U^{a}_{us}\right)\Pi_{pqtu} 
\end{split}
\label{two_electron_MO_INT_gradient_eq:13}
\end{equation} 

At last, the second derivatives for the $(pq|rs)$ is:
\begin{equation}
 \begin{split}
  \Pi_{pqrs}^{[ab]} &=(pq|rs)^{[ab]} = \frac{\partial^{2} (pq|rs)}{\partial
\bm{R}_{a}\partial\bm{R}_{b}} \\
&= \Pi^{ab}_{pqrs} + 
\sum_{t} \left[ 
U^{ab}_{tp}\Pi_{tqrs} + 
U^{ab}_{tq}\Pi_{ptrs} +
U^{ab}_{tr}\Pi_{pqts} +
U^{ab}_{ts}\Pi_{pqrt}                      
\right] \\
&+ 
\sum_{t} \left[ 
U^{a}_{tp}\Pi^{b}_{tqrs}  +
U^{a}_{tq}\Pi^{b}_{ptrs}  +
U^{a}_{tr}\Pi^{b}_{pqts}  +
U^{a}_{ts}\Pi^{b}_{pqrt}
\right]  \\
&+
\sum_{t}\left[ 
U^{b}_{tp}\Pi^{a}_{tqrs} +
U^{b}_{tq}\Pi^{a}_{ptrs} + 
U^{b}_{tr}\Pi^{a}_{pqts} + 
U^{b}_{ts}\Pi^{a}_{pqrt}  
\right]  \\
&+
\sum_{t}\sum_{u}
\left(U^{a}_{tp}U^{b}_{uq} + 
U^{b}_{tp}U^{a}_{uq}\right) \Pi_{turs}
+\sum_{t}\sum_{u}
\left(U^{a}_{tp}U^{b}_{ur} + 
U^{b}_{tp}U^{a}_{ur}\right)\Pi_{tqus}   \\ 
&+\sum_{t}\sum_{u}
\left(U^{a}_{tp}U^{b}_{us} +
U^{b}_{tp}U^{a}_{us}\right) \Pi_{tqru}
+\sum_{t}\sum_{u}
\left(U^{a}_{tq}U^{b}_{ur}  +
U^{b}_{tq}U^{a}_{ur}\right)\Pi_{ptus}    \\
&+\sum_{t}\sum_{u}
\left(U^{a}_{tq}U^{b}_{us}  +
U^{b}_{tq}U^{a}_{us}\right)\Pi_{ptru}  
+\sum_{t}\sum_{u}
\left(U^{a}_{tr}U^{b}_{us}  +
U^{b}_{tr}U^{a}_{us}\right)\Pi_{pqtu} 
 \end{split}
 \label{two_electron_MO_INT_gradient_eq:14}
\end{equation}


%%%%%%%%%%%%%%%%%%%%%%%%%%%%%%%%%%%%%%%%%%%%%%%%%%%
\section{Analytical Derivatives for Numerical Integrals}
%
%
%
%
The numerical integrals commonly used in quantum chemistry mostly concentrate
on the Density functional Theory; where the exchange-correlation integral can
be generally expressed as:
\begin{equation}
\begin{split}
 E_{XC} &= \int d^{3}r F[\rho(r), \nabla(r), \cdots] \\            
&=\sum_{n}^{atoms}\sum_{g}^{grids}A_{g}\omega_{n}(r_{g})F[\rho(r_{g}),
\nabla(r_{g}), \cdots] \\
&= \sum_{n}^{atoms}\sum_{g}^{grids}\varpi_{n}(r_{g})F[\rho(r_{g}),
\nabla(r_{g}), \cdots]
\end{split}
 \label{NUM_AO_INT_gradient_eq:1} 
\end{equation}
This expression can be got in the (\ref{NQM_general_eq}). Here inside the
(\ref{NUM_AO_INT_gradient_eq:1}), the $A_{g}$ is some series expansion constant
in Gauss-Lebedev grid expansion and Euler-Maclaurin grid expansion method, the
former is used for evaluation of angular grid integral, the later is used for
radical grid integral).

The $\omega_{n}(r_{g})$ determines the wight for atoms in terms of the given
grid. Commonly the grid point could be able to belong to several
atoms, so $\omega_{n}(r_{g})$ characterizes the ``weight'' for the atoms, and
it's summation is $1$ for some fixed grid point:
\begin{equation}
 \sum_{n}^{atoms}\omega_{n}(r_{g}) = 1
\end{equation}
So it's also called ``relative weight function''. The $\omega_{n}(r_{g})$ is the
explicit function of atom coordinates, so it should appear in the gradient
theory (more detailed please see \ref{weight_function_dft}).

If we multiply the relative weight function and the $A_{g}$ together, it
becomes the $\varpi_{n}(r_{g})$, we call it as ``weight function". This value
has some physical meaning, that
compared with the common integral expression:
\begin{equation}
 I = \int^{a_{1}}_{a_{2}} f(x) dx \approx \sum_{i}^{n}\frac{(a_{2} -
a_{1})i}{n}f(x_{i})
\end{equation}
Here in three-dimension, the $\varpi_{n}(r_{g})$ just characterizes the
``volumn'' of the infinitesimal. However, obviously it's not the real meaning
of ``volumn'', but the meaning is similar. 

Finally, the $F[\rho(r_{g}), \nabla(r_{g}), \cdots]$ is the
corresponding XC functional, which depends on variety of variables:
\begin{equation}
 \label{NUM_AO_INT_gradient_eq:2} 
 E_{XC} = \int  F(\rho_{\alpha}, \rho_{\beta}, \nabla\rho_{\alpha},
\nabla\rho_{\beta}, \nabla^{2}\rho_{\alpha},
\nabla^{2}\rho_{\beta}, \tau_{\alpha}, \tau_{\beta})d^{3}r
\end{equation}
More details related to the variables can be referred to 
\ref{sec:examples_functional_derivative}. The XC functional is implicitly
depending on the atom coordinates, which means it's explicitly depending on
variables, and the variables depend on atom coordinates. 

%%%%%%%%%%%%%%%%%%%%%%%%%%%%%%%%%%%%%%%%%%%%%%%%%%%
\subsection{The First Order Gradient Expression for $E_{XC}$}
\label{1st_XC_general_derivative}
%
%
%
%
Now let's begin to study the first order gradient expression. As what has shown
in (\ref{NUM_AO_INT_gradient_eq:1}), the exchange-correlation energy can be
expressed as:
\begin{equation}
\label{NUM_AO_INT_gradient_1st_eq:1}
E_{XC} = \sum\omega(r)F_{XC}(r) 
\end{equation}
Then the gradient is:
\begin{equation}
  E_{XC}^{[x]} =  \sum\omega^{[x]}(r)F_{XC}(r) +
\sum\omega(r)F^{[x]}_{XC}(r) 
\label{NUM_AO_INT_gradient_1st_eq:2}
\end{equation}
Where the $F^{[x]}_{XC}(r)$ is:
\begin{align}
F^{[x]}_{XC}(r) &= \sum_{\xi}\sum_{\sigma}\frac{\partial F}{\partial
\xi_{\sigma}} \frac{\partial \xi_{\sigma}}{\partial \bm{R}_{x}} \nonumber\\
&=  \sum_{\xi}\sum_{\sigma} V_{XC}(\xi_{\sigma})\xi_{\sigma}^{[x]}
 \label{NUM_AO_INT_gradient_1st_eq:3}
\end{align}
Here we have to note, that since variable is expressed on the density matrix:
\begin{equation}
 \xi_{\sigma} = \sum_{\mu\nu}P_{\mu\nu\sigma}f(\phi_{\mu}, \phi_{\nu}) = 
\sum_{i}^{occ}\sum_{\mu\nu}c_{\mu i}c_{\nu i}f(\phi_{\mu}, \phi_{\nu})
\end{equation}
For example, the Laplacian variable is:
\begin{equation}
 \nabla^{2}\rho = \sum_{\mu\nu}P_{\mu\nu}\nabla^{2}(\phi_{\mu}\phi_{\nu})
\end{equation}
Then the $\xi^{[x]}$ can be divided into two parts, one is the contributions
from MO rotation matrix, the other part is the AO gradient for the variable:
\begin{align}
\label{NUM_AO_INT_gradient_1st_eq:4}
 \xi^{[x]}  &=  \sum_{i}^{occ}\sum_{r}^{ALL MO}\sum_{\mu\nu}c_{\mu
r}U_{ri}c_{\nu i}f(\phi_{\mu}, \phi_{\nu}) 
+   \sum_{i}^{occ}\sum_{r}^{ALL MO}\sum_{\mu\nu}c_{\nu
r}U_{ri}c_{\mu i}f(\phi_{\mu}, \phi_{\nu}) \nonumber \\
&+ \sum_{i}^{occ}\sum_{\mu\nu}c_{\mu i}c_{\nu i}f^{x}(\phi_{\mu}, \phi_{\nu})
\nonumber \\
&= 2\sum_{i}^{occ}\sum_{r}^{ALL MO}\sum_{\mu\nu}c_{\mu
r}U_{ri}c_{\nu i}f(\phi_{\mu}, \phi_{\nu}) +
 \sum_{i}^{occ}\sum_{\mu\nu}c_{\mu i}c_{\nu i}f^{x}(\phi_{\mu}, \phi_{\nu})
\nonumber \\
&= 2\sum_{i}^{occ}\sum_{r}^{ALL MO}U_{ri}\xi_{ri} + \xi^{x}  
\end{align}
Here we have used the symmetrical relation between the $\mu$ and $\nu$ in the
summation.

Finally, the $F^{[x]}_{XC}(r)$ can be expressed as:
\begin{equation}
 F^{[x]}_{XC}(r) =  2\sum_{i}^{occ}\sum_{r}^{ALL MO}\sum_{\xi}\sum_{\sigma}
V_{XC}(\xi_{\sigma})U_{ri}\xi_{ri} + \sum_{\xi}\sum_{\sigma}
V_{XC}(\xi_{\sigma})\xi_{\sigma}^{x}
\end{equation}
More details about how to calculate $f^{x}(\phi_{\mu}, \phi_{\nu})$ can
be found in \ref{Func_Deriv_gradient_variable_GGA}
and \ref{Func_Deriv_gradient_variable_META_GGA}.


%%%%%%%%%%%%%%%%%%%%%%%%%%%%%%%%%%%%%%%%%%%%%%%%%%%
\subsection{The First Order Gradient Expression for XC Matrix}
%
%
%
%
The XC matrix part inside Fock matrix which corresponds to the Kohn-Sham
equation is defined as:
\begin{equation}
F^{XC}_{pq} = \sum_{\xi}\int  V_{XC}(\xi) \xi_{pq} d^{3}r
\label{NUM_AO_INT_gradient_fock_eq:1} 
\end{equation}
Where $ \xi_{pq}$ is given as:
\begin{equation}
 \xi_{pq} = \sum_{\mu\nu}c_{\mu p}c_{\nu q}f(\phi_{\mu},\phi_{\nu})
\end{equation}

Actually this expression can be got through the original Fock matrix
definition, which can be got from (\ref{functional_mega_gga_eq:3}):
\begin{align}
\label{NUM_AO_INT_gradient_fock_eq:2}
F^{XC}_{\mu\nu}  &= \frac{\partial E_{XC}}{\partial P_{\mu\nu}} \nonumber \\
&= \sum_{\xi}\int \frac{\partial F(r)}{\partial
\xi} \frac{\partial \xi}{\partial P_{\mu\nu}} d^{3}r  \nonumber \\
&= \sum_{\xi}\int \frac{\partial F(r)}{\partial
\xi} f(\phi_{\mu},\phi_{\nu}) d^{3}r 
\Rightarrow \nonumber \\
F^{XC}_{pq}   &= \sum_{\mu\nu}\sum_{\xi}\int \frac{\partial F(r)}{\partial
\xi} c_{\mu p}c_{\nu q} f(\phi_{\mu},\phi_{\nu}) d^{3}r  \nonumber \\
&= \sum_{\xi}\int  V_{XC}(\xi) \xi_{pq} d^{3}r
\end{align}
More details can be found in (\ref{sec:examples_functional_derivative}). 

The derivatives for the XC matrix, is actually used in the CP-SCF equation so
that to derive the MO rotation matrix of $U$. The MO rotation matrix of $U$ is
used for calculating the frequencies or some properties like NMR shift within
DFT framework.

The general integral in (\ref{NUM_AO_INT_gradient_fock_eq:1}) can be numerically
expressed as:
\begin{equation}
\begin{split}
F^{XC}_{pq} &= \sum_{\xi}\int  V_{XC}(\xi) \xi_{pq} d^{3}r  \\ 
 &= \sum_{\xi}\sum_{r}\omega(r)V_{XC}(\xi) \xi_{pq}(r)
\end{split}
\label{NUM_AO_INT_gradient_fock_eq:3}
\end{equation}
Here the $\omega$ is just the weight function defined in
(\ref{NUM_AO_INT_gradient_eq:1}).

Based on the expression (\ref{NUM_AO_INT_gradient_fock_eq:3}), The first order
derivative for $F^{XC}_{pq}$ can be expressed as:
\begin{align}
\label{NUM_AO_INT_gradient_fock_eq:3}
 (F^{XC}_{pq})^{[x]} &=  
\sum_{\xi}\sum_{r}\omega^{[x]}(r)V_{XC}(\xi) \xi_{pq}(r) +
\sum_{\xi}\sum_{r}\omega(r)V^{[x]}_{XC}(\xi) \xi_{pq}(r) \nonumber \\
&+
\sum_{\xi}\sum_{r}\omega(r)V_{XC}(\xi) \xi^{[x]}_{pq}(r) \nonumber \\
&= \sum_{\xi}\sum_{r}\omega^{[x]}(r)V_{XC}(\xi) \xi_{pq}(r) +
\sum_{\xi}\sum_{\zeta}\sum_{r}\omega(r)\frac{\partial 
V_{XC}(\xi)}{\partial \zeta}\zeta^{[x]} \xi_{pq}(r) \nonumber \\
&+
\sum_{\xi}\sum_{r}\omega(r)V_{XC}(\xi) \xi^{[x]}_{pq}(r)
\end{align}
Now let's understand each term inside (\ref{NUM_AO_INT_gradient_fock_eq:3}).

The $\dfrac{\partial V_{XC}(\xi)}{\partial \zeta}$ is actually the derivatives
for the potential of $V_{XC}$:
\begin{equation}
\label{NUM_AO_INT_gradient_fock_eq:4}
 \frac{\partial V_{XC}(\xi)}{\partial \zeta} = \frac{\partial^{2} F(r)}{
\partial \xi \partial\zeta} = F^{\xi\zeta}
\end{equation}

For the $\xi^{[x]}_{pq}(r)$, by repeating the procedure in
(\ref{NUM_AO_INT_gradient_1st_eq:4}) we can have:
\begin{align}
\label{NUM_AO_INT_gradient_fock_eq:4}
 \xi^{[x]}_{pq}  &=  \sum_{s}^{ALL MO}\sum_{\mu\nu}c_{\mu
 s}U_{sp}c_{\nu q}f(\phi_{\mu}, \phi_{\nu}) 
+ \sum_{s}^{ALL MO}\sum_{\mu\nu}c_{\nu
s}U_{sq}c_{\mu p}f(\phi_{\mu}, \phi_{\nu}) \nonumber \\
&+ \sum_{\mu\nu}c_{\mu p}c_{\nu q}f^{x}(\phi_{\mu}, \phi_{\nu})
\nonumber \\
&= \sum_{s}^{ALL MO}U_{sp}\xi_{sq} + 
\sum_{s}^{ALL MO}U_{sq}\xi_{sp}  + \xi^{x}_{pq}  
\end{align}


In this expression, we can sum the explicit derivatives together:
\begin{align}
(F^{XC}_{pq})^{x} &=  \sum_{r}\omega(r)\psi_{p}^{x}(r) V_{XC}(r) \psi_{q}(r) 
+  \sum_{r}\omega(r)\psi_{p}(r) V_{XC}(r) \psi_{q}^{x}(r) \nonumber \\ 
&+  \sum_{r}\omega(r)^{[x]}\psi_{p}(r) V_{XC}(r)\psi_{q}(r) 
\end{align}
Here the first two terms are the derivative on AO part, and the third term is
the derivative on weight function.

By using the expression for $\xi^{[x]}$ in
(\ref{NUM_AO_INT_gradient_1st_eq:4}), we can further have:
\begin{align}
\label{NUM_AO_INT_gradient_fock_eq:5}
&\sum_{\xi}\sum_{\sigma}\sum_{r}\omega(r)\psi_{p}(r)\frac{\partial 
V_{XC}(r)}{\partial \xi_{\sigma}}\xi_{\sigma}^{[x]} \psi_{q}(r) \nonumber \\
&=
2\sum_{\xi}\sum_{\sigma}\sum_{r}\omega(r)\psi_{p}(r)\frac{\partial 
V_{XC}(r)}{\partial \xi_{\sigma}}(\xi_{\sigma}^{U})^{x} \psi_{q}(r) \nonumber
\\
&+ \sum_{\xi}\sum_{\sigma}\sum_{r}\omega(r)\psi_{p}(r)\frac{\partial 
V_{XC}(r)}{\partial \xi_{\sigma}}\xi_{\sigma}^{x} \psi_{q}(r) 
\end{align}

Finally, we can further rewrite the (\ref{NUM_AO_INT_gradient_fock_eq:3}) by
dividing all the terms into two groups; the first group contains the terms
related to MO rotation matrix of $U$, and the left terms are driven into the
other group:
\begin{equation}
 \begin{split}
 (F^{XC}_{pq})^{[x]} &= (F^{XC}_{pq})^{x} + 
\sum_{\xi}\sum_{\sigma}\sum_{r}\omega(r)\psi_{p}(r)\frac{\partial 
V_{XC}(r)}{\partial \xi_{\sigma}}\xi_{\sigma}^{x} \psi_{q}(r)   \\
&+  \sum_{r}\sum_{s}^{ALL MO}U_{sp}\omega(r)\psi_{p}(r)V_{XC}(r) \psi_{q}(r) \\
&+  \sum_{r}\sum_{s}^{ALL MO}U_{sq}\omega(r)\psi_{p}(r)V_{XC}(r) \psi_{q}(r) 
 \\
&+ 2\sum_{\xi}\sum_{\sigma}\sum_{r}\omega(r)\psi_{p}(r)\frac{\partial 
V_{XC}(r)}{\partial \xi_{\sigma}}(\xi_{\sigma}^{U})^{x} \psi_{q}(r) \\
&= (F^{XC}_{pq})^{x} + 
\sum_{\xi}\sum_{\sigma}\sum_{r}\omega(r)\psi_{p}(r)\frac{\partial 
V_{XC}(r)}{\partial \xi_{\sigma}}\xi_{\sigma}^{x} \psi_{q}(r)   \\
&+ \sum_{s}^{ALL MO}U_{sp}F^{XC}_{pq} +  \sum_{s}^{ALL MO}U_{sq}F^{XC}_{pq} \\
&+ 2\sum_{\xi}\sum_{\sigma}\sum_{r}\omega(r)\psi_{p}(r)\frac{\partial 
V_{XC}(r)}{\partial \xi_{\sigma}}(\xi_{\sigma}^{U})^{x} \psi_{q}(r)
 \end{split}
\label{NUM_AO_INT_gradient_fock_eq:6}
\end{equation}

%%%%%%%%%%%%%%%%%%%%%%%%%%%%%%%%%%%%%%%%%%%%%%%%%%%
\subsection{The Second Order Gradient Expression for $E_{XC}$}
%
%
%
%
Following the first order gradient expansion in
(\ref{NUM_AO_INT_gradient_1st_eq:2}), the second order gradient expansion can
be expressed as:
\begin{equation}
\begin{split}
  E_{XC}^{[xy]} &=  
\sum\omega^{[xy]}(r)F_{XC}(r) + 
\sum\omega^{[x]}(r)F^{[y]}_{XC}(r) \\
&+ 
\sum\omega^{[y]}(r)F^{[x]}_{XC}(r) +
\sum\omega(r)F^{[xy]}_{XC}(r)  
\end{split}
 \label{NUM_AO_INT_gradient_2ed_eq:1}
\end{equation}
Here in these four terms,  the $\omega^{[xy]}(r)$ can be directly got because
it's some explicit function for $r$. Hence the only unknown term is the
$F^{[xy]}_{XC}(r)$:
\begin{align}
 F^{[xy]}_{XC}(r) &= \frac{\partial}{\partial \bm{R}_{x}}\sum_{\xi}\sum_{\sigma}
V_{XC}(\xi_{\sigma})\xi_{\sigma}^{[x]} \nonumber \\
&= \sum_{\xi}\sum_{\sigma} \frac{\partial
V_{XC}(\xi_{\sigma})}{\partial \bm{R}_{x}}\xi_{\sigma}^{[x]} +
\sum_{\xi}\sum_{\sigma}
V_{XC}(\xi_{\sigma})\xi_{\sigma}^{[xy]} \nonumber \\
&= \sum_{\xi}\sum_{\sigma}\sum_{\zeta}\sum_{\sigma^{'}} \frac{\partial
V_{XC}(\xi_{\sigma})}{\partial
\zeta_{\sigma^{'}}}\zeta_{\sigma^{'}}^{[y]}\xi_{\sigma}^{[x]} +
\sum_{\xi}\sum_{\sigma}
V_{XC}(\xi_{\sigma})\xi_{\sigma}^{[xy]} \nonumber \\
\end{align}

 

