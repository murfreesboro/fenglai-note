%
% begin at Jan. 2009, finished at Jan 29th; 2009 problems left behind:
% 1 for the integral force: the definition of the pt for the integral
% force is clear, but how can we do it in the real calculation? That
% means, how to convert the two principles below into the codes?  1.1
% the exponents of the basis sets are fixed to constant.  1.2 the
% position vectors in the basis sets are linear combination of the
% nuclear coordinates.
%
%
%
%

\chapter{Force Methods in Quantum Chemistry}\label{PULAY:1}
%
%
%
%%%%%%%%%%%%%%%%%%%%%%%%%%%%%%%%%%%%%%%%%%%%%%%%%%%%%%%%%%%%%%%%%%%%%%%%%%%%
\section{Introduction}
%
% how to understand it
%
%
Most of the quantum chemistry study is based on the Born-Oppenheimer
approximation. In this approximation, the nucleus are considered to
move on the classic trajectory, so its motion is continuous and
smooth; where the motion of electrons is treated quantized, and able
to adapt simultaneously to the position change of nucleus.  Therefore,
the total energy of the whole molecular system is some function of the
nuclear coordinates; such dependence is called potential energy
hyper-surface (later shorted as PES).

Consequently, to determine the PES is vital in the quantum
chemistry. Such determination requires to get the energy derivatives
for an arbitrary nuclear configuration of \vect{R_{0}}, that is to
say, the PES around \vect{R_{0}} can be adequately characterized by
its Taylor series expansion:
\begin{equation}\label{PULAYeq:1}
  \begin{split}
    E(\mathbf{R}) &= E(\mathbf{R_{0}}) +
    \sum_{a}E^{a}\Delta\mathbf{R_{a}}
    +  \frac{1}{2!}\sum_{a,b}E^{ab}\Delta\mathbf{R_{a}}\Delta\mathbf{R_{b}} \\
    & +
    \frac{1}{3!}\sum_{a,b,c}E^{abc}\Delta\mathbf{R_{a}}\Delta\mathbf{R_{b}}
    \Delta\mathbf{R_{c}} + \cdots
  \end{split}
\end{equation}
Here the $E^{a}, E^{ab}$ etc. denote the first derivatives, second
derivatives etc. with respect to the total energy in the \vect{R_{0}};
and the $\Delta\mathbf{R_{a}}, \Delta\mathbf{R_{b}}$ etc. express the
small changes around the reference point of \vect{R_{0}} in the given
direction: $\Delta\mathbf{R_{a}} =
(\mathbf{R_{a}}-\mathbf{R_{0}})_{a}$.

The solution to this problem was firstly suggested by professor Peter
Pulay around $1969$. In this chapter, we will provide a thorough and
detailed analysis for this topic. The materials we are referring to,
are mainly from professor Pulay\cite{Pulay1, Pulay2, Pulay3, Pulay4,
  Pulay5}.

%%%%%%%%%%%%%%%%%%%%%%%%%%%%%%%%%%%%%%%%%%%%%%%%%%%%%%%%%%%%%%%%%%%%%%%%%%%%%
\section{General derivatives expressions}
%
% derive the first and second order of derivatives in the general form
% the benefit to do this, is to make the relations in the derivatives
% more simpler.  1 non-constraint type 1.1 general functional
% expression of E 1.2 C can be seen as the function of R 1.3 the first
% and second derivatives 2 constraint type 2.1 the general Lagrangian
% functional of W under the constraint conditions 2.2 to express the
% variational conditions on W 2.3 derive the gradients, and the second
% derivatives
%
In this section, we firstly introduce some general formulas for the
derivatives of variational energy expressions.

First let's consider the energy functional:
\begin{equation}\label{PULAYeq:13}
  E = E(C,R)
\end{equation}
Here the $R$ represents the nuclear coordinates, and $C$ is the set of
variational parameter in the wave function; e.g.; in the case of SCF
wave function, $C$ is the SCF coefficients. Here we note that in the
(\ref{PULAYeq:13}) the parameters should be gotten through variation
condition:
\begin{equation}\label{PULAYeq:14}
  \frac{\partial E}{\partial C_{i}} = 0
\end{equation}
Thus for each reference coordinate of $R$ we can get a set of
variational parameters of $C$, so we can see that the $C$ is some
function of nuclear coordinates of $R$; $C = C(R)$:
\begin{equation}\label{}
  R \Rightarrow E \Rightarrow C \Rightarrow C= C(R)
\end{equation}
Therefore we can express the energy functional as:
\begin{equation}\label{PULAYeq:2}
  E = E(C(R), R)
\end{equation}
This is some simpler model for discussing the derivatives of the total
energy, for we have not imposed any further constraint conditions on
the $C$. Usually in the quantum chemistry, in the variation process of
(\ref{PULAYeq:14}) the SCF coefficients should keep orthogonal with
each other. Latter we will study this case.

Our aim is to derive the formulas for the expression of
(\ref{PULAYeq:2}) with respect to the nuclear coordinates of $R$.  For
the change of $R$ in the $a$ direction, according to the chain rule of
differentiation we have:
\begin{align}\label{PULAYeq:3}
  E_{a} &= \frac{\partial E}{\partial R_{a}} +\sum_{i}\frac{\partial
    E}{\partial C_{i}}\frac{\partial
    C_{i}}{\partial R_{a}} \nonumber \\
  &= \frac{\partial E}{\partial R_{a}}
\end{align}

In this derivation we have used the relation in the
(\ref{PULAYeq:14}). The results shows that the derivative of the
variational parameters (here is the $C$) do not enter into the
gradient formula. In the later content, as in the derivation of the
constraint variational process, we can see that the same fact holds,
too.

For the second derivatives, by differentiation of (\ref{PULAYeq:3}),
we can get:
\begin{align}\label{}
  E_{ab} &=\frac{\partial}{\partial R_{b}}\left(\frac{\partial
      E}{\partial
      R_{a}}\right) \nonumber \\
  &=\frac{\partial^{2} E}{\partial R_{a}\partial R_{b}}
  +\sum_{i}\frac{\partial^{2} E}{\partial R_{a}\partial
    C_{i}}\frac{\partial C_{i}}{\partial R_{b}}
\end{align}

So far, the discussion is all about the variational process without
constraints. However, it's usually required that in the variational
process, the orbitals or the approximated wave functions should keep
to be orthogonal with each other. Therefore, it's appropriate to
introduce the general functional with constraint conditions:
\begin{equation}\label{PULAYeq:8}
  W(C, R, \lambda) = E(C,R) - \sum_{m}f_{m}(C,R)\lambda_{m}
\end{equation}
Here the $W$ stands for the general Lagrangian functional. The
$f_{m}(C,R)$ generally denote the $m$ constraint conditions, normally
the constraints are the orthogonality conditions for the molecular
orbitals or the configurations. The $\lambda_{m}$ is simply the
Lagrangian factor. Here we note that the variation process defined in
the (\ref{PULAYeq:14}) does not hold true anymore, instead we have:
\begin{equation}\label{PULAYeq:5}
  \frac{\partial W}{\partial C_{i}} = 0 \quad (i=1,2, \cdots)
\end{equation}
On condition that:
\begin{equation}\label{PULAYeq:15}
  f_{m}(C,R) =  0 \quad (m=1,2, \cdots)
\end{equation}

Such conditions are sufficient to determine the unknown $C_{i}$ and
$\lambda_{m}$, hence we can consider the $C_{i}$ and $\lambda_{m}$ are
some functions of $R$:
\begin{equation}\label{}
  \left\{
    \begin{array}{ll}
      C_{i} = C(R),             & i=1,2,\cdots  \\
      \lambda_{m} = \lambda(R), & m=1,2,\cdots
    \end{array}
  \right.
\end{equation}

If the conditions in (\ref{PULAYeq:5}) have been satisfied, then from
the general functional in (\ref{PULAYeq:8}) we can get the total
energy:
\begin{equation}\label{PULAYeq:4}
  W(C, \lambda, R) \Rightarrow E(C, R)
\end{equation}

Now we begin to seek the expressions for the first derivative and
second derivatives related to the $E(C,R)$.

By the differentiation of $E(C,R)$, we have:
\begin{equation}\label{PULAYeq:6}
  E_{a} =\frac{\partial E}{\partial R_{a}} + \sum_{i}\frac{\partial
    E}{\partial C_{i}}\frac{\partial C_{i}}{\partial R_{a}}
\end{equation}
This is same with the (\ref{PULAYeq:3}). However, here we do not
simply have the $\frac{\partial E}{\partial C_{i}} = 0$, thus we have
to do some transformations.

For the constraint variation condition, we can further express the
(\ref{PULAYeq:5}) as:
\begin{align}\label{}
  \frac{\partial W}{\partial C_{i}} &= \frac{\partial E}{\partial
    C_{i}} - \sum_{m}\frac{\partial f_{m}(C,R)}{\partial
    C_{i}}\lambda_{m} \nonumber \\
  &= 0
\end{align}
By multiply the $\frac{\partial C_{i}}{\partial R_{a}}$ to the above
expression and sum over all the label of $i$, we can get:
\begin{equation}\label{PULAYeq:17}
  \sum_{i}\frac{\partial E}{\partial C_{i}}\frac{\partial
    C_{i}}{\partial R_{a}} - \sum_{m}\lambda_{m}\sum_{i}\frac{\partial
    f_{m}(C,R)}{\partial C_{i}}\frac{\partial C_{i}}{\partial R_{a}} = 0
\end{equation}

On the other hand, if we differentiate the constraint conditions of
(\ref{PULAYeq:15}) with respect to the $R_{a}$, we can have:
\begin{align}\label{PULAYeq:16}
  \frac{\partial f_{m}(C,R)}{\partial R_{a}} + \sum_{i}\frac{\partial
    f_{m}(C,R)}{\partial C_{i}}\frac{\partial C_{i}}{\partial R_{a}} =
  0
\end{align}
If we multiply the (\ref{PULAYeq:16}) with $\lambda_{m}$ and sum all
of label of $m$ together, we can get:
\begin{equation}\label{PULAYeq:18}
  \sum_{m}\lambda_{m}\frac{\partial f_{m}(C,R)}{\partial R_{a}} +
  \sum_{m}\lambda_{m}\sum_{i}\frac{\partial f_{m}(C,R)}{\partial
    C_{i}}\frac{\partial C_{i}}{\partial R_{a}} = 0
\end{equation}

By virtual of the results indicated in the (\ref{PULAYeq:17}) and
(\ref{PULAYeq:18}), we can have:
\begin{equation}\label{}
  \sum_{m}\lambda_{m}\frac{\partial f_{m}(C,R)}{\partial R_{a}} = -
  \sum_{i}\frac{\partial E}{\partial C_{i}}\frac{\partial
    C_{i}}{\partial R_{a}}
\end{equation}

Finally for the (\ref{PULAYeq:6}) we can have:
\begin{align}\label{PULAYeq:7}
  E_{a} &= \frac{\partial E}{\partial R_{a}} - \sum_{m}\frac{\partial
    f_{m}(C,R)}{\partial
    R_{a}}\lambda_{m} \nonumber \\
  &= \frac{\partial W}{\partial R_{a}}
\end{align}
Here the result shows the same important meaning with the
(\ref{PULAYeq:3}) that the derivatives of the variational parameters
do not enter the gradient formula. For example, in the HF calculation
(or the DFT calculation, which use the single determinant); the
evaluation of the MO coefficients can be avoided.  In the CI
calculation, the evaluation with respect to the coefficients for the
determinants can be neglected. In such scheme, only the derivatives of
the basis function integrals appearing in the energy formula need to
be calculated.

By differentiating the expression of (\ref{PULAYeq:7}) with respect to
the $R_{b}$, we can get:
\begin{align}\label{}
  E_{ab} &=\frac{\partial}{\partial R_{b}}\left(\frac{\partial
      W}{\partial
      R_{a}}\right) \nonumber \\
  &=\frac{\partial^{2} E}{\partial R_{a}\partial R_{b}} +
  \sum_{i}\frac{\partial^{2} E}{\partial R_{a}\partial
    C_{i}}\frac{\partial C_{i}}{\partial R_{b}} -
  \sum_{m}\frac{\partial^{2} f_{m}(C,R)}
  {\partial R_{a}\partial R_{b}}\lambda_{m} - \nonumber \\
  &\sum_{m}\sum_{i}\frac{\partial^{2} f_{m}(C,R)}{\partial
    R_{a}\partial C_{i}}\frac{\partial C_{i}}{\partial
    R_{b}}\lambda_{m} - \sum_{m}\frac{\partial f_{m}(C,R)}{\partial
    R_{a}}\frac{\partial \lambda_{m}}{\partial R_{b}}
\end{align}

%%%%%%%%%%%%%%%%%%%%%%%%%%%%%%%%%%%%%%%%%%%%%%%%%%%%%%%%%%%%%%%%%%%%%%%%%%%%%
\section{Force calculation from SCF wave functions}
%
%
\subsection{Structure of the wave functions}
%
% 1 how to express the MCSCF wave functions 2 some gradient relations
%
%
In this section let's begin to derive the energy derivatives for the
MCSCF wave functions. Firstly, let's note that why we choose MCSCF
wave functions for calculation. In MCSCF calculation, both of the
coefficients of the determinants and the molecular orbitals are
assumed to vary, so if we fix the coefficients of the molecular
orbitals, then the whole method will recede to be the pure CI method;
if we eliminate the determinants, then it will be back to be the
Hatree-Fock method. Thus the discussion of MCSCF method will provide
us some flexibility for further discussion.

Let $\Phi$ be some normalized linear combination of orthogonal
configurations of $\Psi_{K}$:
\begin{equation}\label{}
  \Phi = \sum_{K}A_{K}\Psi_{K}
\end{equation}
With the constraint condition that:
\begin{equation}\label{}
  \langle\Psi_{K}|\Psi_{L}\rangle = \delta_{KL} \quad
  \sum_{K}A_{K}^{2} = 1
\end{equation}

Here the $\Psi_{K}$ are some fixed linear combinations of slater
determinants constructed from a set of orthogonal orbitals of
$\psi_{i}$; which is in turn $m$ linear combinations of the basis
functions of $\chi_{r}$:
\begin{equation}\label{PULAYeq:25}
  \psi_{i} = \sum_{r=1}^{m}c_{ri}\chi_{r}
\end{equation}

Since that the orbitals of $\psi$ should maintain the orthogonality
between each other, therefore the coefficients of the basis sets
should satisfy that:
\begin{align}\label{}
  \int\psi_{i}\psi_{j}d\tau &= \delta_{ij} \Rightarrow \nonumber \\
  \sum_{r}\sum_{s}c^{*}_{ri}c_{sj}\int\chi^{*}_{r}\chi_{s}d\tau &=
  \delta_{ij}
\end{align}
Sometimes it's more convenient to write it into the matrix form:
\begin{align}\label{PULAYeq:19}
  \sum_{i}\sum_{j}\sum_{r}\sum_{s}c^{*}_{ri}c_{sj}\int\chi^{*}_{r}\chi_{s}d\tau
  &= \sum_{i}\sum_{j}\delta_{ij} = n \Rightarrow \nonumber \\
  \sum_{i}\sum_{r}\sum_{s}c^{*}_{ri}c_{si}\int\chi^{*}_{r}\chi_{s}d\tau
  &= n \Rightarrow \nonumber \\
  Tr(C^{+}SC) &= Tr(I_{n})
\end{align}
Where $I_{n}$ is the unit matrix, $S_{pq} =
\langle\chi_{p}|\chi_{q}\rangle$. The $C$ represents the vectors of
the molecular orbitals:
\begin{equation}\label{}
  C = \begin{bmatrix}
    c_{11} & c_{12} & \cdots & c_{1n} \\
    c_{21} & c_{22} & \cdots & c_{2n} \\
    \cdots &\cdots  & \cdots & \cdots \\
    c_{n1} & c_{n2} & \cdots & c_{nn} \\
  \end{bmatrix}
\end{equation}
Here each column denotes one composition of molecular orbital, and the
row characterizes the weight of the specific $\chi_{i}$ in all the
molecular orbitals. $C^{+}$ is the unitary transformation of the $C$:
$C^{+}_{ij} = C_{ji}^{*}$. However, usually in the quantum chemistry
we use the real orbitals and the coefficients, thus we have that
$C^{+}_{ij} = C_{ji}$. Here below we follow such convention.

Now let's consider the expectation value for the $\Phi$:
\begin{equation}\label{PULAYeq:24}
  E = \bra{\Phi}\hat{H}\ket{\Phi}
\end{equation}
Here the $E$ has the same meaning with it's general form of $E(C,R)$
defined in (\ref{PULAYeq:8}). Furthermore, based on the constraint
conditions the variation process can be expressed as:
\begin{equation}\label{PULAYeq:10}
  \begin{split}
    \frac{\partial}{\partial A_{K}} \Big\{
    E -  \lambda(\sum_{K}A_{K}^{2} - 1)\Big\}&=  0  \\
    \frac{\partial}{\partial c_{ri}} \Big\{ E - Tr[\epsilon(C^{+}SC-
    I_{n})]\Big\}&=  0  \\
  \end{split}
\end{equation}
Hence we can express the Lagrangian functional defined in
(\ref{PULAYeq:8}) as:
\begin{equation}\label{PULAYeq:9}
  W = E - \lambda(\sum_{K}A_{K}^{2} - 1) - Tr[\epsilon(C^{+}SC -
  I_{n})]
\end{equation}

From the (\ref{PULAYeq:10}), we can obtains some relations concerned
with the gradients expression. They are useful in the following
contents. For the differentiation with respect to the $A_{X}$, we
have:
\begin{equation}\label{PULAYeq:11}
  \begin{split}
    \frac{\partial E}{\partial A_{X}} &=
    \lambda\frac{\partial (\sum_{K}A_{K}^{2} - 1)}{\partial A_{X}} \\
    &=2\lambda A_{X}
  \end{split}
\end{equation}

The derivation for the differentiation with respect to the $c_{xy}$ is
something more complicated, so below I try to give the redundant
details in the calculation. Hope that we can make it as self-clear as
it can.
\begin{align}\label{PULAYeq:12}
  \frac{\partial E}{\partial c_{xy}} &= \frac{\partial
    Tr[\epsilon(C^{+}SC - I_{n})]}{\partial c_{xy}} \nonumber \\
  &=\frac{\partial}{\partial c_{xy}}\left\{ \sum_{i}\sum_{r}\sum_{s}
    \epsilon_{ii}c_{ri}c_{si}S_{rs} -
    \sum_{i}\epsilon_{ii} \right\}\nonumber \\
  &=\frac{\partial}{\partial c_{xy}} \left\{\cdots + \sum_{s}
    \epsilon_{yy}c_{xy}c_{sy}S_{xs} + \cdots + \right.\nonumber \\
  &\left.\sum_{r} \epsilon_{yy}c_{ry}c_{xy}S_{rx} + \cdots -
    \sum_{i}\epsilon_{ii}\right\} \nonumber \\
  &=\sum_{s} \epsilon_{yy}c_{sy}S_{xs} + \sum_{r}
  \epsilon_{yy}c_{ry}S_{rx} \nonumber \\
  &=2\sum_{r} \epsilon_{yy}c_{ry}S_{xr} \nonumber \\
  &=2(SC\epsilon)_{xy}
\end{align}
In the derivation above the $S_{rs} =
\langle\chi_{r}|\chi_{s}\rangle$, and it's hermitian matrix so that
$S_{rs} = S_{sr}$.


%%%%%%%%%%%%%%%%%%%%%%%%%%%%%%%%%%%%%%%%%%%%%%%%%%%%%%%%%%%%%%%%%%%%%%%%%%%%%
\subsection{Dependent variables related to nuclear coordinates}
%
% how to understand the dependence between the other variables and the
% nuclear coordinates, the most important is the basis functions
% discussion
%
After setting up the structure of the wave functions, the next
question is: as the nuclear coordinate changes, in the energy
functional what kind of arguments have to make the corresponding
change?

Obviously, as the nuclear coordinate changes; the coefficients for the
determinants and the orbitals will all have to change coordinately:
\begin{equation}\label{}
  \delta R \Rightarrow \left\{
    \begin{array}{ll}
      \delta A_{K}   &   \\
      \delta c_{ri}  &
    \end{array}
  \right.
\end{equation}

On the other hand, we note that there's implicit dependence between
the basis functions and the nuclear coordinates. Because that the
Gaussian type basis functions or the Slater type of basis functions
used almost exclusively in quantum chemistry are all atomic type
functions, that is to say; they are all centered around the nucleus,
and decays in exponent way away from the nucleus; So it's reasonable
to rationalize that there's dependence between basis functions and the
nuclear coordinates. However, how can we construct the mathematical
dependent relationship between them?

Historically there had been a lot of arguments over this
issue\cite{meyer:2109, bishop:3515}. As Pulay had
suggested\cite{Pulay5}, that the basis functions are defined by their
general functional form which regraded as fixed, and by a set of
nonlinear parameters $p_{t}$, the most important orbital centers of
$\chi_{r}$ and the orbital exponents of $\varepsilon_{r}$. It's
assumed that basis functions depend explicitly on the nuclear
coordinates of $R_{a}$ through the parameter of $p_{t}$. The further
relationship will be further discussed in the following content.

Therefore, we can write that:
\begin{equation}\label{PULAYeq:12}
  E = E(R_{a}, A_{K}, c_{ri}, p_{t})
\end{equation}

In the next passage, we will further investigate the dependence by the
(\ref{PULAYeq:9}).

%%%%%%%%%%%%%%%%%%%%%%%%%%%%%%%%%%%%%%%%%%%%%%%%%%%%%%%%%%%%%%%%%%%%%%%%%%%%%
\subsection{First derivative of the SCF energy: general expression}
%
% 1 consider the general derivatives with respect to the 4 variables
% 1.1 the R_{a}: only in Hamiltonian, and nuclear-electron term 1.2
% deal with A_{k} 1.3 density force 1.4 integral force
%
%
As for the first derivative of the MCSCF wave functions, let's further
investigate the (\ref{PULAYeq:12}). We will discuss the
differentiation with respect to the $Ra$, $A_{K}$, $c_{ri}$ and
$p_{t}$; separately:
\begin{equation}\label{}
  E(R_{a}, A_{K}, c_{ri}, p_{t}) = \bra{\Phi}\hat{H}\ket{\Phi}
\end{equation}

For the $R_{a}$, it's only explicitly contained in the Hamiltonian,
thus the differentiation in terms of the $R_{a}$ only appear in the
operator of $\hat{H}$. Since that in the Hamiltonian only the
nuclear-electron term is related to the nuclear coordinates, therefore
the differentiation can be simply expressed as:
\begin{align}\label{}
  \frac{\partial \hat{H}}{\partial R_{a}} = \frac{\partial}{\partial
    R_{a}}\left[-\sum_{\alpha}\sum_{i}\frac{Z_{\alpha}}{r_{i\alpha}}\right]
  &=-\sum_{i}\frac{\partial}{\partial
    R_{a}}\left[\frac{Z_{a}}{r_{ia}}\right] \nonumber \\
  &=\sum_{i}\frac{Z_{a}}{(r_{ia})^{2}} \nonumber \\
  &=-Z_{a}\sum_{i}\frac{R_{i} - R_{a}}{(r_{ia})^{3}}
\end{align}
Here $r_{i\alpha}$ is defined as the distance between the electron $i$
and the nuclear $\alpha$: $r_{i\alpha} = R_{\alpha} - R_{i}$.
Moreover, we can define the Hellmann-Feynman force as:
\begin{equation}\label{}
  -f^{HF}_{a} = \left\langle\Phi|\frac{\partial \hat{H}}{\partial
      R_{a}}|\Phi\right\rangle
\end{equation}
We do so is because this expression is similar to the Hellmann-Feynman
theorem discussed in the quantum mechanics part.

For the differentiation related to the CI coefficients of $A_{K}$,
according to the (\ref{PULAYeq:11}); we have:
\begin{align}\label{}
  \sum_{K}\frac{\partial E}{\partial A_{K}}\frac{\partial
    A_{K}}{\partial R_{a}} &=2\lambda\sum_{K}A_{K}\frac{\partial
    A_{K}}{\partial R_{a}} \nonumber \\
  &=\lambda\frac{\partial}{\partial R_{a}}\sum_{K}
  \left(A_{K}^{2}\right)\nonumber \\
  &= 0
\end{align}
The result turns out that the variational parameters of $A_{K}$ does
not enter into the gradients expression, just coincide with the
conclusion we got from the (\ref{PULAYeq:7}).

For the term of differentiation with respect to the coefficients of
$c_{xy}$, according to the (\ref{PULAYeq:12}) we can have:
\begin{equation}\label{PULAYeq:21}
  \sum_{x}\sum_{y}\frac{\partial E}{\partial c_{xy}}\frac{\partial
    c_{xy}}{\partial R_{a}} =
  2\sum_{x}\sum_{y}\sum_{r}S_{xr}c_{ry}\epsilon_{yy}\frac{\partial
    c_{xy}}{\partial R_{a}}
\end{equation}

However, such expression is not easy to evaluate. Thus we wish to
transform it into the matrix expression. Before it, we are going to
investigate some similar process.

In the expression of (\ref{PULAYeq:19}), we have derived some relation
between the matrix expression and the sum of matrix elements:
\begin{align}\label{}
  Tr(C^{+}SC) &=
  Tr(I_{n}) \Rightarrow \nonumber \\
  \sum_{i}\sum_{r}\sum_{s}c_{ri}c_{si}S_{rs} &= n
\end{align}
Again, we note that $C^{+}_{ir} = C_{ri}$, and $S_{rs} = \int
\chi_{r}\chi_{s}d\tau$(all the coefficients and the basis functions
are assumed to be real).

We can see that if we multiply the $\epsilon_{ii}$ to the left side of
the above equation (here we have to remember that $\epsilon$ is some
diagonal matrix, with its diagonal element denotes the orbital
energy), we can get that:
\begin{align}\label{PULAYeq:20}
  \sum_{i}\sum_{r}\sum_{s}\epsilon_{ii}c_{ri}c_{si}S_{rs}
  &= \sum_{i}\epsilon_{ii} \nonumber \\
  \sum_{i}\sum_{r}\sum_{s}\epsilon_{ii}c_{ri}S_{rs}c_{si}
  &= \sum_{i}\epsilon_{ii} \Rightarrow \nonumber \\
  Tr(\epsilon C^{+}SC) &= Tr(\epsilon I_{n})
\end{align}

Analog to the above case, we can set up such one-to-one correspondence
between the elements in the (\ref{PULAYeq:20}) and elements in the
(\ref{PULAYeq:21}):
\begin{equation}\label{}
  \begin{split}
    \epsilon_{ii} &\Leftrightarrow S_{xr}  \\
    c_{ri}        &\Leftrightarrow c_{ry}  \\
    S_{rs}        &\Leftrightarrow \epsilon_{yy}  \\
    c_{si} &\Leftrightarrow \frac{\partial c_{xy}}{\partial R_{a}}
  \end{split}
\end{equation}
Therefore, we can write the (\ref{PULAYeq:21}) as:
\begin{equation}\label{}
  \sum_{x}\sum_{y}\frac{\partial E}{\partial c_{xy}}\frac{\partial
    c_{xy}}{\partial R_{a}} = 2Tr[SC\epsilon C^{a+}]
\end{equation}
Where the $C^{a+} = \frac{\partial C^{+}}{\partial R_{a}}$.

Now by the matrix expression, we can prove that the differentiation
related to the coefficients of $C$ can be transformed to the
differentiation related to the matrix density of $S$. For the
constraint condition related to the $C$, which is defined as:
\begin{equation}\label{}
  C^{+}SC = I_{n}
\end{equation}
By differentiating the above equation with respect to the $R_{a}$, we
have:
\begin{equation}\label{PULAYeq:23}
  C^{a+}SC + C^{+}SC^{a} = -C^{+}S^{a}C
\end{equation}
Multiply the matrix of $\epsilon$ and take the trace of the above
equation we obtain:
\begin{multline}\label{}
  2Tr[\epsilon C^{a+}SC] = 2Tr[C^{a+}SC\epsilon] = 2Tr[SC\epsilon
  C^{a+}] =\\
  -Tr[\epsilon C^{+}S^{a}C] = -Tr[C\epsilon C^{+}S^{a}]
\end{multline}
Here we have used the identities that $Tr[AB] = Tr[BA]$, and $Tr[C] =
Tr[C^{+}]$ in the above transformations. Therefore finally we have:
\begin{equation}\label{PULAYeq:22}
  \sum_{x}\sum_{y}\frac{\partial E}{\partial c_{xy}}\frac{\partial
    c_{xy}}{\partial R_{a}} = -Tr[C\epsilon C^{+}S^{a}]
\end{equation}

(\ref{PULAYeq:22}) shows that we do not need to calculate the
gradients of the coefficients of $C$ in terms of $R$, instead the
$S^{a}$ and $C\epsilon C^{+}$ are sufficient. According to
Pulay\cite{Pulay5}, this step is called the ``key step in deriving the
analytical form of the gradient''. Because of its connection with the
change in the density matrix as the nuclear coordinates change, so
this part is also called ``density force''.

Finally, it's very interesting to compare the (\ref{PULAYeq:22}) with
the (\ref{PULAYeq:7}). The (\ref{PULAYeq:7}) indicates that the
variational parameters do not enter into the gradient expression,
however; the results shows in (\ref{PULAYeq:22}) does not coincide
with such conclusion. In my opinion, this can be attributed from the
implicit dependence between the basis functions and the nuclear
coordinates. As indicated in the (\ref{PULAYeq:23}), the derivatives
on the $C$ with respect to the $R_{a}$ can be converted in to the
derivatives on the density; thus if the basis functions are irrelevant
to the change of nuclear coordinates; then the term on the right side
of (\ref{PULAYeq:23}) will be zero, and the the derivatives on the $C$
with respect to the $R_{a}$ will accordingly be zero, too.

Therefore, we can write the $S^{a}$ in the density force as:
\begin{equation}\label{}
  S^{a} = \sum_{t}\frac{\partial S}{\partial p_{t}}\frac{\partial
    p_{t}}{\partial R_{a}}
\end{equation}

Similarly, since that in the expression of $E$ there's integrals on
the basis functions, hence we can define the integral force in terms
of the differentiation on basis sets integral with respect to the
$R_{a}$:
\begin{equation}\label{}
  \sum_{t}\frac{\partial \bra{\Phi}\hat{H}\ket{\Phi}}{\partial p_{t}}
  \frac{\partial p_{t}}{\partial R_{a}}
\end{equation}


%%%%%%%%%%%%%%%%%%%%%%%%%%%%%%%%%%%%%%%%%%%%%%%%%%%%%%%%%%%%%%%%%%%%%%%%%%%%
\subsection{First derivative of the SCF energy: final expression}
%
% 1 general expression 2 the concrete expansion for the E 3 apply the
% general gradient expression to the E 4 Hatree-Fock case, close shell
%
So far we can derive the gradient of MCSCF energy, it's simply the
addition of the four terms discussed in the above section:
\begin{align}\label{PULAYeq:26}
  E_{a} &= \frac{\partial E}{\partial R_{a}} + \sum_{K}\frac{\partial
    E}{\partial A_{K}}\frac{\partial A_{K}}{\partial R_{a}} +
  \sum_{x}\sum_{y}\frac{\partial E}{\partial c_{xy}}\frac{\partial
    c_{xy}}{\partial R_{a}} +\sum_{t}\frac{\partial E}{\partial p_{t}}
  \frac{\partial p_{t}}{\partial R_{a}} \nonumber
  \\
  &=\left\langle\Phi|\hat{H}^{a}|\Phi\right\rangle -Tr[C\epsilon
  C^{+}S^{a}] + \sum_{t}\frac{\partial
    \bra{\Phi}\hat{H}\ket{\Phi}}{\partial p_{t}} \frac{\partial
    p_{t}}{\partial R_{a}}
\end{align}

Now let's expand the total energy of $E$ into practical expression.
Firstly by spreading the coefficients $A_{K}$ of the determinants, the
total energy in the (\ref{PULAYeq:24}) can be:
\begin{equation}\label{}
  E = \sum_{K}\sum_{L}A_{K}A_{L}H_{KL}
\end{equation}
And the $H_{KL}$ is:
\begin{equation}\label{}
  H_{KL} =\bra{\Psi_{K}}\hat{H}\ket{\Psi_{L}}
\end{equation}

On the other hand, if we further expand the $H_{KL}$ with the
molecular orbitals, it can be expressed as:
\begin{equation}\label{}
  H_{KL} = \sum_{ij}q^{ij}_{KL}h_{ij} +
  \frac{1}{2}\sum_{ijkl}Q^{ijkl}_{KL}(ij|kl)
\end{equation}
With
\begin{align}\label{}
  h_{ij} &= \int \psi_{i}\hat{h}\psi_{j}d\tau \nonumber \\
  (ij|kl) &=\int \psi_{i}(1)\psi_{j}(1)\frac{1}{r_{12}}
  \psi_{k}(2)\psi_{l}(2)d\tau_{1}d\tau_{2}
\end{align}
This is some general form between two arbitrary slater determinants.
According to the Slater rules, for a close shell case if the
determinants of $K$ and $L$ are identical; then we have $q^{ij}_{KL} =
\delta_{ij}$, $Q^{ijkl}_{KL} = 4\delta_{ij}\delta_{kl} -
2\delta_{ik}\delta_{jl}$. Other cases can be derived in the similar
way.

It's helpful to introduce the ``density matrix'' based on the CI
coefficients:
\begin{align}\label{}
  \gamma_{ij} &= \sum_{KL}q^{ij}_{KL}A_{K}A_{L} \nonumber \\
  \Gamma_{ijkl} &=\sum_{KL}Q^{ijkl}_{KL}A_{K}A_{L}
\end{align}
With such notion, the total energy can be expressed as:
\begin{equation}\label{}
  E = \sum_{ij}\gamma_{ij}h_{ij} +
  \frac{1}{2}\sum_{ijkl}\Gamma_{ijkl}(ij|kl)
\end{equation}

Finally, the molecular orbitals should be expanded over the basis sets
functions. By inserting the (\ref{PULAYeq:25}) into the above
equation, we have:
\begin{equation}\label{PULAYeq:27}
  E = \sum_{pq}d_{pq}h_{pq} + \frac{1}{2}\sum_{pqrs}D_{pqrs}(pq|rs)
\end{equation}
In which the density matrices are given as:
\begin{align}\label{}
  d_{pq} &=\sum_{ij}c_{pi}c_{qj}\gamma_{ij} \nonumber \\
  D_{pqrs} &=\sum_{ijkl}c_{pi}c_{qj}c_{rk}c_{sl}\Gamma_{ijkl}
\end{align}

Now we can apply the result in the (\ref{PULAYeq:26}) to the final
energy expression of (\ref{PULAYeq:27}):
\begin{equation}\label{PULAYeq:28}
  E_{a} = \sum_{pq}d_{pq}h^{a}_{pq} +
  \frac{1}{2}\sum_{pqrs}D_{pqrs}(pq|rs)^{a} -Tr[C\epsilon C^{+}S^{a}]
\end{equation}
With
\begin{align}\label{}
  h^{a}_{pq} &= \bra{\chi_{p}}\hat{h}^{a}\ket{\chi_{q}} + \sum_{t}
  \bra{\frac{\partial \chi_{p}}{\partial
      p_{t}}}\hat{h}\ket{\chi_{q}}\frac{\partial p_{t}}{\partial
    R_{a}} + \cdots
  \nonumber \\
  (pq|rs)^{a} &=\sum_{t} \left\langle\frac{\partial
      \chi_{p}(1)}{\partial
      p_{t}}\chi_{q}(1)\left|\frac{1}{r_{12}}\right|
    \chi_{r}(2)\chi_{s}(2)\right\rangle\frac{\partial p_{t}}{\partial
    R_{a}} + \cdots
\end{align}
Here we note that we have omitted some terms, in which the
differentiation is on the $\chi_{q}$, $\chi_{r}$ etc.


For the close shell Hatree-Fock calculation, its total energy can be
expressed as:
\begin{equation}\label{}
  E = \sum_{pq}2D_{pq}h_{pq} + \sum_{pqrs}D_{pq}D_{rs}\left\{2(pq|rs)
    -(pr|qs)\right\}
\end{equation}
Where the $D_{pq}$ is the density matrix:
\begin{equation}\label{}
  D_{pq} = \sum_{i}c_{pi}c_{qi}
\end{equation}

Therefore, the gradients of the close Hatree-Fock energy can be:
\begin{multline}\label{}
  E_{a} =\sum_{pq}2D_{pq}
  \left\{\bra{\chi_{q}}\hat{h}^{a}\ket{\chi_{q}} +
    \left\langle\frac{\partial \chi_{p}}{\partial
        p_{t}}\left|\hat{h}\right|\chi_{q}\right\rangle\frac{\partial
      p_{t}}{\partial R_{a}} +
    \left\langle\chi_{p}\left|\hat{h}\right|\frac{\partial
        \chi_{q}}{\partial p_{t}}\right\rangle\frac{\partial
      p_{t}}{\partial
      R_{a}}\right\} \\
  +\sum_{pqrs}D_{pq}D_{rs}\left\{2(p^{a}q|rs) -(p^{a}r|qs) +
    2(pq^{a}|rs) -(pr^{a}|qs) \right. \\
  \left.+ 2(pq|r^{a}s) -(pr|q^{a}s) +
    2(pq|rs^{a})-(pr|qs^{a})\right\} \\
  -Tr[C\epsilon C^{+}S^{a}]
\end{multline}

%%%%%%%%%%%%%%%%%%%%%%%%%%%%%%%%%%%%%%%%%%%%%%%%%%%%%%%%%%%%%%%%%%%%%%%%%%%%
\subsection{Dependence between the basis sets and the integral force}
%
% this section is to illustrate that how to calculate the integral
% force 1 two ways to tackle this problem 2 the deficiency for the
% second method 3 how to use the first method
%
There are some different ways to define the dependence of the
non-linear parameter of $p_{t}$ over the nuclear
coordinates\cite{Pulay5, meyer:2109}:
\begin{itemize}
\item Define an arbitrary but physically reasonable and simple
  functional dependence of $p_{t}$ on the nuclear coordinates and
  evaluate the integral force with this definition of $\frac{\partial
    p_{t}}{\partial R_{a}}$.
\item Optimize the parameter of $p_{t}$ so that to make the energy as
  a minimum. Since we have to keep the orthogonality between the
  orbitals, the functional of $W = E - Tr[\epsilon(C^{+}SC- I_{n})]$
  is selected to make $\frac{\partial W}{\partial p_{t}} = 0$; then we
  can have:
  \begin{align}\label{}
    \frac{\partial E}{\partial p_{t}} &=
    Tr\left[\epsilon\left(C^{+}\frac{\partial S}{\partial
          p_{t}}C\right)\right] =Tr\left[C\epsilon C^{+}\frac{\partial
        S}{\partial p_{t}}\right] \Rightarrow \nonumber \\
    \sum_{t}\frac{\partial E}{\partial p_{t}}\frac{\partial
      p_{t}}{\partial R_{a}} &=\sum_{t}Tr\left[C\epsilon
      C^{+}\frac{\partial S}{\partial p_{t}}\right]\frac{\partial
      p_{t}}{\partial R_{a}}
  \end{align}
  Therefore, the integral force has been transformed into the density
  force. From the general force expression in (\ref{PULAYeq:26}), we
  can see that the density force and the integral force have
  compensated with each other (they have different sign), thus in
  (\ref{PULAYeq:26}) only the Hellmann-Feynman force left behind.
\end{itemize}

The second method is mathematically clear, and able to derive a very
simple expression for evaluating the first derivative for the given
system. However, the second method requires that the basis sets should
be highly flexible so that it's parameters can be varied in the SCF
optimization process. Such basis sets are called ``floating
functions''\cite{Pulay5}. Unfortunately, it turns out that such method
will cause too much computational efforts, which is not proportional
to the benefits it has brought in. Therefore, even this method was
frequently used at the early stage, now it's seldom used anymore.

On the other hand, the first method to evaluate the integral force has
been proved to be some good method which is as efficient as the second
one, but only cause the comparable computation efforts.

This method contains two points:
\begin{itemize}
\item the exponents of the basis sets are fixed to constant.
\item the position vectors in the basis sets are linear combination of
  the nuclear coordinates:
  \begin{equation}\label{PULAYeq:29}
    \chi_{r} = \sum_{a}\mu_{ra}R_{a}
  \end{equation}
\end{itemize}
Since that the basis sets are usually placed on the atom center or the
bonding line between the nucleus, thus as the $R_{a}$ varies there
will be at most two non-vanishing terms in the (\ref{PULAYeq:29}). For
example, if the basis sets are placed on the atom center (all the GTO
and STO belong to this type); then the integral force can be evaluated
as:
\begin{equation}\label{}
  \sum_{t}\frac{\partial E}{\partial p_{t}}\frac{\partial
    p_{t}}{\partial R_{a}} =\sum_{t}\frac{\partial E}{\partial
    p_{t}}\mu_{ra}
\end{equation}


%%%%%%%%%%%%%%%%%%%%%%%%%%%%%%%%%%%%%%%%%%%%%%%%%%%%%%%%%%%%%%%%%%%%%%%%%%%%%



%%% Local Variables: 
%%% mode: latex
%%% TeX-master: "../../main"
%%% End: 
