%
% OK
%
% setting up at Aug. third, 2009
% finished at Aug 5th, 2009
% seems to be OK
%
%
%
%%%%%%%%%%%%%%%%%%%%%%%%%%%%%%%%%%%%%%%%%%%%%%%%%%%%%%%%%%%%%%%%%%%%%%%%%%%%%%%

\chapter{Introduction to delta function}
\label{sec:delta_function}

As what we can see, in the discussion of quantum mechanics or even the
quantum chemistry, we always meet the situation that the delta
function has to be used (for example, the normalization of free
particle wave function, the discussion of position and momentum
operator in \ref{sec:PRAMR_in_position_representation} etc. For the
case of the continuous basis functions, the delta function is
frequently used). Hence in this part, we are going to give some
mathematical introduction for the delta function.


%%%%%%%%%%%%%%%%%%%%%%%%%%%%%%%%%%%%%%%%%%%%%%%%%%%%%%%%%%%%%%%%%%%%%%%%%%%%%%%
\section{Definition of delta function}
\label{sec:definition_delta_function}
%
%
%
%
The delta function can be considered as some ``peculiar'' function
which has two features below:
\begin{equation}
\label{DELTAeq:1}
\delta (x) = \begin{cases}
    \infty & x = 0 \\
    0 & x \neq 0
  \end{cases}
\end{equation}

\begin{equation}
\label{DELTAeq:2}
 \int_{-\infty}^{+\infty}\delta (x)dx =
\int_{-\epsilon}^{+\epsilon}\delta (x)dx = 1
\end{equation}
Here $(-\epsilon, \epsilon)$ is some ``small enough'' open interval
around the $0$. The (\ref{DELTAeq:1}) and (\ref{DELTAeq:2}) are also
the definition for the delta function, which indicates that the
delta function is infinity at $0$, but quickly tends to $0$ as is
apart from the $0$ point.

Now let's give some examples. Firstly, consider the function below:
\begin{equation}\label{}
\Psi_{n}(x) = \frac{\sin nx}{\pi x} \quad \text{$n$ is some natural
number }
\end{equation}
Firstly, we can prove that:
\begin{align}\label{}
\lim_{x \rightarrow 0}\Psi_{n}(x) = \lim_{x \rightarrow 0}\frac{\sin
nx}{\pi x} = \lim_{x \rightarrow 0}\frac{\sin nx}{n x}\frac{n}{\pi}
= \frac{n}{\pi}
\end{align}
So we can think that $\Psi_{n}(0) = \frac{n}{\pi}$. Here as $n
\rightarrow +\infty$, $\Psi_{\infty}(0) \rightarrow +\infty$. Hence
the (\ref{DELTAeq:1}) is satisfied.

Secondly, from the improper integral theory, we can know that:
\begin{equation}\label{}
\int_{0}^{+\infty}\frac{\sin nx}{x}dx = \frac{\pi}{2}
\end{equation}
Hence we have the integral as:
\begin{align}\label{}
\int_{-\infty}^{+\infty}\frac{\sin nx}{\pi x}dx &=
2\frac{1}{\pi}\int_{0}^{+\infty}\frac{\sin nx}{x}dx \quad
 \text{here the $\Psi_{n}(x)$ is even function}\nonumber \\
&=2\frac{1}{\pi}\times\frac{\pi}{2} \nonumber \\
&=1
\end{align}
So the (\ref{DELTAeq:2}) is satisfied. Thus we can have
$\Psi_{\infty}(x) = \delta(x)$.

From the definition of $\Psi_{n}(x)$, we can even prove that the
function below is also the delta function:
\begin{equation}\label{}
\Phi(x) = \frac{1}{2\pi}\int_{-\infty}^{+\infty} e^{ikx}dk
\end{equation}

This function is very important in discussing the property related
to plane wave functions. Firstly, as $x = 0$, it's easy to see tat
the $\Phi(0) = \frac{1}{2\pi}\int_{-\infty}^{+\infty} dk \Rightarrow
\infty$, so the (\ref{DELTAeq:1}) is satisfied.

As for the (\ref{DELTAeq:2}) since we have (suggest that $a$ is some
natural number):
\begin{align}\label{}
\Phi_{a}(x) = \frac{1}{2\pi}\int_{-a}^{+a} e^{ikx}dk &=
\frac{1}{i2x\pi}e^{ikx}\Big|^{a}_{-a} \nonumber \\
&=\frac{1}{i2x\pi}2i\sin ax \quad \cos(a) = \cos(-a)\nonumber \\
&=\frac{\sin ax}{x\pi}
\end{align}
Hence the integral for the $\Phi(x)$ in the whole real space has
been resorted to the $\Psi_{a\rightarrow \infty}(x)$; so we have:
\begin{equation}\label{}
\Phi(x) = \delta(x)
\end{equation}

Finally, let's consider the Gauss distribution function:
\begin{equation}\label{}
\Omega_{\sigma}(x) =
\frac{1}{\sqrt{2\pi\sigma}}e^{-\frac{x^{2}}{2\sigma}}
\end{equation}

Firstly, we can see that:
\begin{align}\label{}
\Omega_{\sigma}(0) &= \frac{1}{\sqrt{2\pi\sigma}}\times 1 \quad
\underrightarrow{\sigma \rightarrow 0} \nonumber
\\
\Omega_{\sigma \rightarrow 0 }(0) &= \infty
\end{align}

On the other hand, since from the improper integral theory, we have:
\begin{equation}\label{DELTAeq:20}
\int_{0}^{+\infty}e^{-x^{2}}dx = \frac{\sqrt{\pi}}{2}
\end{equation}
Then we have the integral for the $\Omega_{\sigma}(x)$ as:
\begin{align}\label{}
\int_{-\infty}^{+\infty}
\frac{1}{\sqrt{2\pi\sigma}}e^{-\frac{x^{2}}{2\sigma}} dx &=
\frac{1}{\sqrt{2\pi\sigma}}\int_{-\infty}^{+\infty}
e^{-\frac{x^{2}}{2\sigma}} dx \nonumber \\
&= \frac{2}{\sqrt{2\pi\sigma}}\int_{0}^{+\infty}\sqrt{2\sigma}
e^{-(\frac{x}{\sqrt{2\sigma}})^{2}} d\frac{x}{\sqrt{2\sigma}} \quad
\text{$\Omega_{\sigma}(x)$ is even} \nonumber \\
&=\frac{2}{\sqrt{\pi}}\times\frac{\sqrt{\pi}}{2} \quad
\text{From the \ref{DELTAeq:20}}\nonumber \\
&= 1
\end{align}
Hence we have $\Omega_{0}(x) = \delta(x)$.

%%%%%%%%%%%%%%%%%%%%%%%%%%%%%%%%%%%%%%%%%%%%%%%%%%%%%%%%%%%%%%%%%%%%%%%%%%%%%%%
\section{Characters of delta function}
\label{sec:character_delta_function}
%
%
%
%
In this section let's state some characters of delta function.

\begin{theorem}
$\delta(x)$ is some even function: $\delta(x) = \delta(-x)$
\end{theorem}

\begin{proof}
the proof is straightforward. For (\ref{DELTAeq:1}), the
$\delta(-x)$ is naturally satisfied, and for the integral in the
(\ref{DELTAeq:2}) we have:
\begin{equation}\label{}
\int_{-\infty}^{+\infty}\delta(-x)dx =
-\int_{-\infty}^{+\infty}\delta(-x)d(-x) =
\int_{-\infty}^{+\infty}\delta(t)dt = 1
\end{equation}
Hence  $\delta(x) = \delta(-x)$. \qedhere
\end{proof}

\begin{theorem}
$\delta(ax) = \frac{1}{|a|}\delta(x)$
\end{theorem}

\begin{proof}
Firstly suggest that $a>0$, then:
\begin{equation}\label{DELTAeq:3}
\begin{split}
  \int_{-\infty}^{+\infty}\delta(ax)dx &=
  \frac{1}{a}\int_{-\infty}^{+\infty}\delta(ax)d(ax) \\
    &= \frac{1}{a}\delta(x)
\end{split}
\end{equation}

For the $a < 0$ (in this case, if we simply repeat the progress in
\ref{DELTAeq:3}; the upper limit and the lower limit in integral
should change), since we have $\delta(ax) = \delta(-ax)$; then we
can replace the $a$ with $|a|$ to repeat the demonstration in
(\ref{DELTAeq:3}). So finally we can get the conclusion. \qedhere
\end{proof}

\begin{theorem}
\label{DELTA:4}
$\int_{-\infty}^{+\infty}f(x)\delta(x)dx = f(0)$.
\end{theorem}

\begin{proof}
\begin{equation}\label{}
\begin{split}
  \int_{-\infty}^{+\infty}f(x)\delta(x)dx &=
  \int_{-\epsilon}^{+\epsilon}f(x)\delta(x)dx \\
    &= f(0)\int_{-\epsilon}^{+\epsilon}\delta(x)dx \\
    &= f(0)
\end{split}
\end{equation} \qedhere
\end{proof}

The theorem (\ref{DELTA:4}) is some important expression for delta
function. Now consider the delta function $\delta (x - x^{'})$,
where $x^{'}$ is some arbitrary number; it can be easily understood
that the (\ref{DELTAeq:1}) and (\ref{DELTAeq:2}) are converted as:
\begin{equation}
\label{DELTAeq:5} \delta (x - x^{'}) = \begin{cases}
    \infty & x = x^{'} \\
    0 & x \neq x^{'}
  \end{cases}
\end{equation}

\begin{equation}
\label{DELTAeq:6}
 \int_{-\infty}^{+\infty}\delta (x - x^{'})dx =
\int_{-\epsilon}^{+\epsilon}\delta (x - x^{'})dx = 1
\end{equation}
The (\ref{DELTAeq:1}) and (\ref{DELTAeq:2}) can be considered as the
special case while $x^{'} = 0$. Through this expansion, by using the
theorem (\ref{DELTA:4}) we can prove that the relation below is
true:
\begin{theorem}
\begin{equation}\label{DELTAeq:7}
\int_{-\infty}^{+\infty}f(x)\delta(x - x^{'})dx =
\int_{-\infty}^{+\infty}f(x^{'})\delta(x - x^{'})dx^{'} =f(x)
\end{equation}
\end{theorem}

\begin{proof}
For the first equation in the (\ref{DELTAeq:7}), we can directly use
the even function character to prove it. For the second equation in
(\ref{DELTAeq:7}), we have:
\begin{equation}\label{}
\begin{split}
  \int_{-\infty}^{+\infty}f(x)\delta(x - x^{'})dx &=
   \int_{-\infty}^{+\infty}f(t+x^{'})\delta(t)dt
   \quad t = x - x^{'}\\
    &= f(t+x^{'})\Big|_{t=0} \quad \text{From theorem \ref{DELTA:4}}\\
    &= f(x^{'})
\end{split}
\end{equation}
 \qedhere
\end{proof}

\begin{theorem}
$\delta(x) = \delta^{*}(x)$
\end{theorem}

\begin{proof}
Suggest that we have some arbitrary real function of $f(x)$,
\begin{equation}\label{}
\begin{split}
  \left[\int_{-\infty}^{+\infty}f(x)\delta(x)dx\right]^{*} &=
   \int_{-\infty}^{+\infty}f(x)\delta^{*}(x)dx \\
    &=f^{*}(0) = f(0) \\
    &= \int_{-\infty}^{+\infty}f(x)\delta(x)dx \Rightarrow \\
    &= \int_{-\infty}^{+\infty}f(x)
    \Big(\delta(x)-\delta^{*}(x)\Big)dx = 0
\end{split}
\end{equation}
Hence we have $\delta(x) = \delta^{*}(x)$. \qedhere
\end{proof}

\begin{theorem}
$\int_{-\infty}^{+\infty}\delta(x - x^{'}) \delta(x - x^{''})dx =
\delta(x^{'} - x^{''})$. Here $x^{'}$ and $x^{''}$ are both of
arbitrary real number.
\end{theorem}

\begin{proof}
From  (\ref{DELTAeq:7}), we set the $f(x) = \delta(x)$ so according
to the (\ref{DELTAeq:7}):
\begin{equation}\label{}
\begin{split}
  \int_{-\infty}^{+\infty}f(x- x^{''})\delta(x - x^{'})dx
  &= f(x- x^{''})\Big|_{x =x^{'}} \\
    &= f(x^{'}- x^{''})
\end{split}
\end{equation}
\qedhere
\end{proof}

%%%%%%%%%%%%%%%%%%%%%%%%%%%%%%%%%%%%%%%%%%%%%%%%%%%%%%%%%%%%%%%%%%%%%%%%%%%%%%%
\section{Example to use delta function:
the normalization of plane wave functions}
\label{sec:normalization_delta_function}
%
%
%
%
Now by using the delta function let's solve some important problem
in quantum mechanics: how to normalize the plane wave function?

According to the (\ref{BASICeq:14}), the plane wave function can be
expressed as:
\begin{equation}\label{}
\Psi_{p}(x) = e^{\frac{ipx}{\hbar}}
\end{equation}
Actually it's the eigen state for the $\hat{p}$:
\begin{equation}\label{}
\hat{p}\Psi_{p}(x) = p\Psi_{p}(x)
\end{equation}

Now let's go to see how to normalize this function:
\begin{equation}\label{}
\begin{split}
  \langle \Psi_{p}(x^{'})|\Psi_{p}(x)\rangle
  =\int_{-\infty}^{+\infty}\Psi^{*}(x')\Psi(x)dp
  &= \int_{-\infty}^{+\infty} e^{\frac{ip(x-x^{'})}{\hbar}}dp \\
  &= \hbar\int_{-\infty}^{+\infty} e^{\frac{ip(x-x^{'})}{\hbar}}
  d\left(\frac{p}{\hbar}\right) \\
  &= 2\pi\hbar\delta(x-x^{'})
\end{split}
\end{equation}

Hence if we integrate over all the $x$:
\begin{equation}\label{}
\begin{split}
  \int_{-\infty}^{+\infty}
  \langle \Psi_{p}(x^{'})|\Psi_{p}(x)\rangle dx
  &= \int_{-\infty}^{+\infty} 2\pi\hbar\delta(x-x^{'}) dx \\
    &= 2\pi\hbar
\end{split}
\end{equation}

Thus finally we can express the plane wave function as:
\begin{equation}\label{}
\Psi_{p}(x) = \frac{1}{\sqrt{2\pi\hbar}}e^{\frac{ipx}{\hbar}}
\end{equation}


%%%%%%%%%%%%%%%%%%%%%%%%%%%%%%%%%%%%%%%%%%%%%%%%%%%%%%%%%%%%%%%%%%%%%%%%%%%%%%%






%%% Local Variables:
%%% mode: latex
%%% TeX-master: "../main"
%%% End:
