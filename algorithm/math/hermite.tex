%
% setting up at Nov. 2011
%
%
%
%%%%%%%%%%%%%%%%%%%%%%%%%%%%%%%%%%%%%%%%%%%%%%%%%%%%%%%%%%%%%%%%%%%%%%%%%%%%%%%

\chapter{Introduction to Hermite Polynomials}
\label{sec:hermite_polynomials}

In the mathematics, Hermite polynomials is a series of orthogonal functions
that form an orthogonal basis of the Hilbert space, it's also a complete 
orthogonal function system so that we can use it to expand a lot of functions.
In quantum chemistry, the Hermite polynomials are used in the integral
calculation. Hence here a brief summary for its properties will be given.

%%%%%%%%%%%%%%%%%%%%%%%%%%%%%%%%%%%%%%%%%%%%%%%%%%%%%%%%%%%%%%%%%%%%%%%%%%%%%%%
\section{Definition}
The Hermite polynomials are defined as:
\begin{equation}
 \label{hermite_definition_eq:1}
H_{n}(x) = (-1)^{n}e^{x^{2}}\frac{d^{n}}{dx^{n}}e^{-x^{2}}
\end{equation}
This relation could be derived from a generation function:
\begin{equation}
\label{hermite_definition_eq:2}
 e^{2xt-t^{2}} = \sum_{t=0}^{\infty}H_{n}(x)\frac{t^{n}}{n!}
\end{equation}
when we make $n$th derivatives for the $e^{2xt-t^{2}}$, $H_{n}(x)$
could be generated:
\begin{equation}
 \label{hermite_definition_eq:3}
H_{n}(x) =  \left. \frac{\partial^{n} (e^{2xt-t^{2}}) }{\partial
t^{n}}\right|_{t=0}
\end{equation}
By setting the $e^{2xt-t^{2}} = e^{-(t-x)^{2} + x^{2}}$, then through
the $n$th derivatives we can get the \ref{hermite_definition_eq:1}.

For this polynomial we have some recursive relation from its definition:
\begin{equation}
 \label{hermite_definition_eq:4}
H_{n}^{'}(x) = 2xH_{n}(x) - H_{n+1}(x)
\end{equation}
By using the relation that $\frac{\partial 
e^{2xt-t^{2}} }{\partial x} = 2te^{2xt-t^{2}}$, then we can bring this form
into \ref{hermite_definition_eq:3} and expand the higher derivatives for
$2te^{2xt-t^{2}}$; we get:
\begin{equation}
 \label{hermite_definition_eq:5}
H_{n}^{'}(x) = 2nH_{n-1}(x)
\end{equation}
Therefore we arrive at some final recursive expression:
\begin{equation}
 \label{hermite_definition_eq:6}
2nH_{n-1}(x) = 2xH_{n}(x) - H_{n+1}(x)
\end{equation}
we note that such derivative relation could also be expressed in terms of 
its derivatives. 

%%%%%%%%%%%%%%%%%%%%%%%%%%%%%%%%%%%%%%%%%%%%%%%%%%%%%%%%%%%%%%%%%%%%%%%%%%%%%%%
\section{Orthogonality} 
%
%
%
Hermite polynomials has one important property, is that they are orthogonal with 
each other by a weight function:
\begin{equation}
\label{hermite_orthogonality_eq:1}
 \int^{+\infty}_{-\infty} H_{m}(x)H_{n}(x)e^{-x^{2}} = 
 2^{n}n!\delta_{mn}  
\end{equation}

The orthogonality could be derived through partial integral(suppose that $n\neq m$):
\begin{align}
 \label{hermite_orthogonality_eq:2}
 \int^{+\infty}_{-\infty} H_{m}(x)H_{n}(x)e^{-x^{2}} dx &= 
(-1)^{n} \int^{+\infty}_{-\infty} H_{m}(x) \frac{d^{n}}{dx^{n}}e^{-x^{2}}dx \nonumber \\
&=  (-1)^{n-1} 2m\int^{+\infty}_{-\infty} H_{m-1}(x)\frac{d^{n-1}}{dx^{n-1}}e^{-x^{2}}dx 
\nonumber \\
&= (-1)^{n-m} 2^{m}m!\int^{+\infty}_{-\infty} H_{0}(x)\frac{d^{n-m}}{dx^{n-m}}e^{-x^{2}}dx 
\nonumber \\
&= 0
\end{align}
Such derivation uses the relation in the \ref{hermite_definition_eq:5}.
 

