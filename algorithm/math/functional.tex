%
% originally set up at Sep. 17th, 2009
% 
% working list:
% TDDFT energy expression
%
%

% problem list:
%  1  why we have the expression of eq:functional:45 established?
%   Why Can I use the chain rule of functional and delta function 
%   (just the above content) can not prove it?
%   By the way, if we replace the g(x^{'}) with the f(x^{'}) in the 
%   functional chain rule and use the delta function, we can not get
%   This result. Puzzling.....
%




    
%%%%%%%%%%%%%%%%%%%%%%%%%%%%%%%%%%%%%%%%%%%%%%%%%%%%%%%%%%%%%%%%%%%%%%%%%%%%
\chapter{Introduction to the functional analysis}
%
%
%
%
%
%
%%%%%%%%%%%%%%%%%%%%%%%%%%%%%%%%%%%%%%%%%%%%%%%%%%%%%%%%%%%%%%%%%%%%%%%%%%%%
\section{Original concepts for the functional and functional
  derivatives}
%
% 1 how to make define the functional, and examples 2 how to usher
% into the concept for the differential of functional from the limit
% of the finite space case
%  
%
%
In mathematical analysis, the definition for the function is very
clear; it's some relation that connects two number fields together:
\begin{equation}\label{eq:functional:1}
  x \xrightarrow{f}  y \quad \text{we can abbreviate the $y$ as $f(x)$}
\end{equation}
Here the $x$ and $y$ are varied in some given field, it's usually the
real number field or the complex number field. As an example, the $y =
\sin x$ just makes the $ (-\infty, +\infty)$ field correspond to the
$[-1, +1]$ field.

Furthermore, in mathematics a functional is also some ``mapping
relation'' between some function and a number, here function is some
basic variables and through the expression we can get some number:
\begin{equation}\label{eq:functional:2}
  f(x) \xrightarrow{F} y
\end{equation}
The $F$ is the functional for the functions of $f(x)$, here such
relation can be abbreviated as $F[f(x)] = y$.

there are many examples of functional that we can put forward in
physics. In fact, in traditional quantum mechanics, the average
expectation value of $\bra{\Psi}\hat{A}\ket{\Psi}$ is just some
functional of wave function of $\Psi$:
\begin{equation}\label{eq:functional:3}
  \Psi \xrightarrow{F} \bra{\Psi}\hat{A}\ket{\Psi}
\end{equation}
If the operator is Hamiltonian, then the functional is just transform
the wave function into it's energy.

On the other hand, let's prompt some mathematical examples:
\begin{equation}\label{eq:functional:4}
  A[y(x)] = \int^{1}_{0} y^{2}(x) d x
\end{equation}
Here the $ A[y(x)]$ corresponds to some number which is depending on
the function of $y(x)$. For example, if $y(x) = x^{2}$, then $A[y(x)]
= \frac{1}{5}$; if $y(x) = e^{x}$, then $A[y(x)]$ gives
$\frac{1}{2}(e^{2} - 1)$.

Now let's proceed to the functional derivatives, which is very
important in density functional theory. Firstly let's return back to
the math analysis to recall the differential definition for function.
Consider a function $F(y)$. If $y$ is making some infinitesimal change
of $\bigtriangleup y$, the corresponding change for the function of
$\bigtriangleup F(y)$ will be $\bigtriangleup F(y) =
F(y+\bigtriangleup y) - F(y) = F^{'}(y)\bigtriangleup y + \alpha
O(\bigtriangleup y)$.

Here, the $\alpha$ means higher derivatives for the function $F(y)$
with respect to the $y$. It can be made clear by the Talyor expansion
(see the standard mathematical books for more detail).
                                                                    
Hence, for the infinitesimal change of some function, it mainly
contains two parts; one is the linear part of $F^{'}(y)\bigtriangleup
y$, and the other is the non-linear part of $\alpha O(\bigtriangleup
y)$. The linear part means it's linear with $\bigtriangleup y$, so as
$\bigtriangleup y$ becomes smaller and smaller; then the difference
between the $\bigtriangleup F(y)$ and $F^{'}(y)\bigtriangleup y$ will
be very small, that means we can safely replace the $\bigtriangleup
F(y)$ with $F^{'}(y)\bigtriangleup y$. That's the key idea for the
differential of function.

For the functional differential, we have the similar case. When the
function of $y(x)$ is making some infinitesimal changes $\delta y(x)$;
then the corresponding changes for the functional can always be
expressed as:
\begin{equation}
  \delta A[f(x)] = A[f(x) + \delta f(x)] - A[f(x)] = M[f(x)]\delta
  f(x) + N[f(x)]O(\delta f(x))
\end{equation}
Here, the $O(\delta f(x))$ is just the higher order changes of $\delta
f(x)$, and compared with the definition of function differential, we
can see that the functional differential is just the linear part of
$M[f(x)]\delta f(x)$. Hence, the functional derivative is the
$M[f(x)]$.   

Now let's put forward the definition for the functional derivative
with respect to the functional in the form of $F[f(x)] = \int L[f(x)]
dx$. Firstly let's define the infinitesimal of the functional:
\begin{equation}
  \label{eq:functional:5}
 \delta F[ f(x)] = \int \left(\frac{\delta L}{\delta
      f(x)}\delta f(x)\right)dx
\end{equation}
Here the $\delta F[ f(x)]$ means the functional corresponding to
infinitesimal of $\delta f(x)$, and we can see that $\dfrac{\delta
L}{\delta f(x)}$ is made in certain $x$ point, for example; the
$x_{0}$ point; Then the integration over all the $x$ gives the
infinitesimal of the functional. 

The derivation of the functional derivative, should be made to fit for
the expression of (\ref{eq:functional:5}). On the other hand,
(\ref{eq:functional:5}) has some equivalent form that:
\begin{align}
  \label{eq:functional:6}
 \delta F[ f(x)] &= \lim_{\epsilon \rightarrow 0}\Bigg[ \frac{F[f(x)
    +\epsilon \delta f(x) ] - F[f(x)]}{\epsilon}\Bigg] \nonumber \\
  &= F^{'}_{\epsilon=0}[f(x) + \epsilon \delta f(x) ]
\end{align}
Hence in the (\ref{eq:functional:6}) the functional derivative
expression is degenerated into some common function derivatives
expression.


%%%%%%%%%%%%%%%%%%%%%%%%%%%%%%%%%%%%%%%%%%%%%%%%%%%%%%%%%%%%%%%%%%%%%%%%%%%%
\section{How to derive the functional derivative}
%
%
%
%
%
The preceding discussion gives a definition of the functional
derivative, but it does not give a useful method for calculating it.
Mathematically, We can start with (\ref{eq:functional:5}) as a
definition of the functional derivative, and use it to calculate.

As an example to show that how to perform this process, it's better
for us to start with an simple case; hence let's derive the functional
derivative for the kinetic energy gotten from the uniform electron gas
model: $T_{TF}[\rho] = c_{F}\int \rho^{\frac{5}{3}}(r) d^{3}r$:
\begin{equation}\label{eq:functional:9}
  \begin{split}
    T_{TF}[\rho] + \delta T_{TF}[\rho] &= c_{F}\int \left(\rho(r) +
      \delta\rho(r) \right)^{\frac{5}{3}} d^{3}r \\
    &= c_{F}\int \left(\rho^{\frac{5}{3}}(r) +
      \frac{5}{3}\rho^{\frac{2}{3}}(r)\delta\rho(r)
      + O(\delta\rho(r)) \right) d^{3}r  \\
    &= T_{TF}[\rho] + \frac{5}{3}c_{F}\int
    \rho^{\frac{2}{3}}(r)\delta\rho(r)d^{3}r
    + O(\delta\rho(r)) \Rightarrow \\
   \delta T_{TF}[\rho] &= \frac{5}{3}c_{F}\int\rho^{\frac{2}{3}}(r)
    \delta\rho(r)d^{3}r
  \end{split}
\end{equation}
Now if we compare the form in (\ref{eq:functional:9}) with the
functional derivative in (\ref{eq:functional:5}), then we can see that
the derivative for the $T_{TF}[\rho]$ is:
\begin{equation}
  \label{eq:functional:10}
  \frac{\delta T_{TF}[\rho]}{\delta\rho(r)} = \frac{5}{3}c_{F}
  \rho^{\frac{2}{3}}(r)   
\end{equation}
We note that in the (\ref{eq:functional:9}), we expand the
$\left(\rho(r) + \delta\rho(r) \right)^{\frac{5}{3}}$ according to the
Taylor expansion rule:
\begin{equation}
  \label{eq:functional:11}
  f(x) = f(x_{0}) +
  \sum_{n=1}^{\infty}\frac{\left.f^{(n)}(x)\right|_{x=x_{0}}}{n!}\times
  (x-x_{0})^{n}   
\end{equation}
Here $x$ should be close enough to the $x_{0}$ so that the
(\ref{eq:functional:11}) could be converged.

Compared with (\ref{eq:functional:11}), in (\ref{eq:functional:9}) the
variable of $x_{0}$ is replaced by the function of $\rho(r)$, and the
variable of $x$ corresponds to the $\rho(r) + \delta\rho(r)$; then the
similar expansion can be used. However, here we omit the convergence
discussion, so we assume that all the expansion are mathematically
converged.

This simple example reveals how to derive the functional derivative
for the more general expression. Suggest we have such more general
form:
\begin{equation}
  \label{eq:functional:12}
  F[f(x)] = \int A[f(x)] dx
\end{equation}
Here $A[f(x)]$ represents some analytical expression of the given
function $f(x)$, for example; $A[f(x)] = c_{F}f^{\frac{5}{3}}(x)$ in
the (\ref{eq:functional:9}). Then if $f(x)$ gets some infinitesimal
change $f(x) + \delta f(x)$, by using the Taylor expansion defined in
the (\ref{eq:functional:11}) we can expand the $A[f(x)] + \delta A[
f(x)]$
into the corresponding series and omitting all the higher order
infinitesimal terms of $\delta f(x)$, finally we can reach the
expression same with (\ref{eq:functional:5}) so that to obtain the
functional derivative for the $F[f(x)]$.

Finally we note that the discussion made for the
(\ref{eq:functional:12}) is only for the one dimensional function of
$f(x)$; however, it can be simply expanded into the multiple dimension
function case.

%%%%%%%%%%%%%%%%%%%%%%%%%%%%%%%%%%%%%%%%%%%%%%%%%%%%%%%%%%%%%%%%%%%%%%%%%%%%
\section{The general functional derivative expression}
%
%
%
%
%
When the functional is a simple integral (most of the functionals in
physics belong to this type), the equation below gives a powerful
formula for quick calculation of the functional derivative. Now let's
start with the simplest case:
\begin{equation}
  F[y(x)] = \int L(x, y(x)) dx
\end{equation}
According to the general rule described in the
(\ref{eq:functional:12}), we have:
\begin{multline}
  \label{eq:functional:13}
  F[y(x)] + \delta F[ y(x)] =  \\
  \int \left\{ L(x, y(x)) + \frac{\partial L(x, y(x))}{\partial y(x)}
    \delta y(x) + O(\delta y(x))\right\} dx
\end{multline}
Now we omit the term of $O(\delta y(x))$, then the
(\ref{eq:functional:13}) turns into:
\begin{equation}
  \label{eq:functional:14}
  \delta F[ y(x)] = \int \left(\frac{\partial
      L(x, y(x))}{\partial y(x)} \delta y(x)\right) dx
\end{equation}
Hence the functional derivative for the (\ref{eq:functional:13}) is:
\begin{equation}
  \label{eq:functional:15}
  \frac{\delta  F[y(x)]}{\delta y(x)} =  \frac{\partial
    L(x, y(x))}{\partial y(x)} 
\end{equation}

For example, the functional $T_{TF}[\rho]$ in (\ref{eq:functional:9})
just belongs to this type, hence we have:
\begin{equation}
  \label{eq:functional:16}
  \frac{\delta T_{TF}}{\delta \rho} =  c_{F}\frac{5}{3}\rho^{\frac{2}{3}} 
\end{equation}

Here we note something important, that the difference between the
$\delta$ symbol and the $\partial$ symbol. Here in the left side of
the equation (\ref{eq:functional:15}), the $\delta$ symbol means that
the infinitesimal change of $f(x)$ will cause the corresponding
functional change, too. However, in the right side, the $\partial$
symbol means that only the change of $f(x)$ for the functional is
considered, so it's partial derivative.

For the functional in form of $F[y(x)] = \int L(x, y(x)) dx$,
actually it's difficult to see their difference; however, as the
functional is composed by many variables; such as 
\begin{equation}
 F[f(x)] = \int L(f(x), f^{'}(x), f^{''}(x), \cdots) dx
\end{equation} 
Then the partial derivative will be more clear, for example;
$\dfrac{\partial L}{\partial f^{'}(x)}$ will only keep the $f^{'}(x)$
component change but keep the $f(x), f^{''}(x), \cdots$ constant.
However, since $\delta f(x)$ will also cause changes in the $f^{'}(x),
f^{''}(x), \cdots$, hence the functional derivative in the left side
of equation clearly demonstrates how much linear changes for the $F$
with respect to the change of $\delta f(x)$. All in all, they
possesses different meanings.

Next, let's go to see a little more complicated case, where the
functional has the general form as:
\begin{equation}
  \label{eq:functional:17}
  F[y(x)] = \int L(x, y(x), y^{'}(x)) dx
\end{equation}
This functional depends on two type of functions: $y(x)$ and
$y^{'}(x)$. If the $y(x)$ and $y^{'}(x)$ make some infinitesimal
change, in terms of the Taylor expansion the (\ref{eq:functional:17})
becomes:
\begin{multline}
  \label{eq:functional:18}
  F[y(x)] + \delta F[ y(x)] =  \\
  \int \left\{ L(x, y(x), y^{'}(x)) + \frac{\partial L(x, y(x),
      y^{'}(x))} {\partial y(x)} \delta y(x) + \right. \\
  \left. \frac{\partial L(x, y(x), y^{'}(x))} {\partial y^{'}(x)}
    \delta y^{'}(x) + O(\delta y(x)) + O(\delta y^{'}(x)) \right\} dx
\end{multline}

Omitting the higher order terms of infinitesimal, then we have:
\begin{multline}
  \label{eq:functional:19}
\delta  F[ y(x)] =  \\
  \int \left\{ \frac{\partial L(x, y(x), y^{'}(x))} {\partial y(x)}
    \delta y(x) + \frac{\partial L(x, y(x), y^{'}(x))} {\partial
      y^{'}(x)} \delta y^{'}(x) \right\} dx
\end{multline}

However, it can not simply compare the (\ref{eq:functional:19}) with
the (\ref{eq:functional:5}) to obtain the functional derivative, since
in (\ref{eq:functional:19}) we have $ \delta y^{'}(x)$ but not the $
\delta y(x)$; hence additional treatment must be appended, which is to
turn $\delta y^{'}(x)$ into $\delta y(x)$ in the last term; we do this
with an integration by parts:
\begin{equation}
  \label{eq:functional:20}
  \begin{split}
    \int^{b}_{a} \frac{\partial L(x, y(x), y^{'}(x))} {\partial
      y^{'}(x)} \delta y^{'}(x) dx &= \int^{b}_{a} \frac{\partial L(x,
      y(x), y^{'}(x))}
    {\partial y^{'}(x)} d (\delta y(x)) \\
    &= \left. \frac{\partial L(x, y(x), y^{'}(x))}
      {\partial y^{'}(x)} \delta y(x) \right |^{b}_{a} -\\
    &\int^{b}_{a} \frac{d}{dx} \left(\frac{\partial L(x, y(x),
        y^{'}(x))} {\partial y^{'}(x)} \right) \delta y(x) dx
  \end{split}
\end{equation}

Herein the first term in the (\ref{eq:functional:20}) is called
``boundary term'' (in quantum physics the $a$ and $b$ usually denote
the boundary condition for the wave functions), since it's
proportional to the $\delta y(x)$ in the interval of $[a,b]$, then
it's some infinitesimal hence we can safely omit it here.

Finally, the (\ref{eq:functional:20}) turns into:
\begin{equation}
  \label{eq:functional:21}
  \int^{b}_{a} \frac{\partial L(x, y(x), y^{'}(x))} {\partial
    y^{'}(x)} \delta y^{'}(x) dx = - \int^{b}_{a} \frac{d}{dx}  
  \left(\frac{\partial L(x, y(x), y^{'}(x))}
    {\partial y^{'}(x)} \right) \delta y(x)  dx
\end{equation}

The differential of functional for the (\ref{eq:functional:17})
becomes:
\begin{multline}
  \label{eq:functional:22}
 \delta F[ y(x)] =  \\
  \int \left\{ \frac{\partial L(x, y(x), y^{'}(x))} {\partial y(x)} -
    \frac{d}{dx} \left(\frac{\partial L(x, y(x), y^{'}(x))} {\partial
        y^{'}(x)} \right) \right\} \delta y(x) dx
\end{multline}

The functional derivative then becomes:
\begin{equation}
  \label{eq:functional:23}
  \frac{\delta F[y(x)]}{\delta y(x)} = 
  \frac{\partial L(x, y(x), y^{'}(x))} {\partial y(x)} -
  \frac{d}{dx} \frac{\partial L(x, y(x), y^{'}(x))}
  {\partial y^{'}(x)}
\end{equation}

Now the result in the (\ref{eq:functional:23}) are easily extended
into higher orders derivatives of $y(x)$. For example, if the function
is $L(x, y(x), y^{'}(x), y^{''}(x))$; then we have the functional as:
\begin{equation}
  \label{eq:functional:25}
  F[y(x)] = \int L(x, y(x), y^{'}(x), y^{''}(x)) dx
\end{equation}

through the same procedure we can get it's derivative:
\begin{multline}
  \label{eq:functional:24}
  \frac{\delta F[ y(x)]}{\delta y(x)} = \frac{\partial L(x, y(x),
    y^{'}(x), y^{''}(x))} {\partial y(x)} - \frac{d}{dx}
  \frac{\partial L(x, y(x), y^{'}(x), y^{''}(x))}
  {\partial y^{'}(x)} + \\
  \frac{d^{2}}{dx^{2}} \frac{\partial L(x, y(x), y^{'}(x), y^{''}(x))}
  {\partial y^{''}(x)}
\end{multline}
The second derivative comes from the need to integrate by parts twice
to deal with $y^{''}(x)$ in the (\ref{eq:functional:20}), and two
minus signs make a plus sign.

Finally, let's push the one dimension case to the three dimensional
situation. For the simplest case in the (\ref{eq:functional:13}), now
we have:
\begin{equation}
  \label{eq:functional:26}
  F[y(r)] = \int L(r, y(r)) d^{3}r
\end{equation}
It's functional derivative becomes:
\begin{equation}
  \label{eq:functional:27}
  \frac{\delta F}{\delta y(r)} =  \frac{\partial
    L(r, y(r))}{\partial y(r)} 
\end{equation}

Now for the functional containing the first derivative, now it's
general expression in three dimension is:
\begin{equation}
  \label{eq:functional:28}
  F[y(r)] = \int L(r, y(r), \nabla y(r)) d^{3}r
\end{equation}
Compared with (\ref{eq:functional:23}), it's functional derivative is:
\begin{equation}
  \label{eq:functional:29}
  \frac{\delta  F[y(r)]}{\delta y(r)} = 
  \frac{\partial L(r, y(r), \nabla y(r))} {\partial y(r)} -
  \nabla \cdot \frac{\partial L(r, y(r), \nabla y(r))}
  {\partial (\nabla y(r))}
\end{equation}

Finally, for the functional containing the second derivative:
\begin{equation}
  \label{eq:functional:30}
  F[y(r)] = \int L(r, y(r), \nabla y(r), \nabla^{2} y(r)) d^{3}r
\end{equation}
The (\ref{eq:functional:24}) becomes:
\begin{multline}
  \label{eq:functional:31}
  \frac{\delta F[y(r)]}{\delta y(r)} = \frac{\partial L(r, y(r),
    \nabla y(r), \nabla^{2} y(r))} {\partial y(r)} - \nabla \cdot
  \frac{\partial L(r, y(r), \nabla y(r), \nabla^{2} y(r))}
  {\partial (\nabla y(r))} \\
  + \nabla^{2} \cdot \frac{\partial L(r, y(r), \nabla y(r), \nabla^{2}
    y(r) )} {\partial (\nabla^{2} y(r))}
\end{multline}

In generality, for the functional which is in more general form, who
contains $N$ order of function derivatives:
\begin{equation}
  \label{eq:functional:32}
  F[y(r)] = \int L(r, y(r), \nabla y(r), \nabla^{2} y(r), \cdots
  \nabla^{N} y(r)) d^{3}r 
\end{equation}

It's corresponding functional derivative can be expressed as:
\begin{multline}
  \label{eq:functional:33}
  \frac{\delta  F[y(r)]}{\delta y(r)} = \frac{\partial L} {\partial
    y(r)} - \nabla \cdot \frac{\partial L} {\partial (\nabla y(r))} +
  \nabla^{2} \cdot \frac{\partial L} {\partial
    (\nabla^{2} y(r))} \\
  + \cdots + (-1)^{i} \nabla^{i} \cdot \frac{\partial L} {\partial
    (\nabla^{i} y(r))} + \cdots
\end{multline}
All of these formulas can be derived in the similar way which we used
for deriving the functional derivative for (\ref{eq:functional:23}).

Finally, let's add some important note with respect to the evaluation
of the expression below:
\begin{equation}
\int^{+\infty}_{-\infty}\frac{d L(f(x))}{dx} g(x) dx 
\end{equation}
If the $g(x)$ is vanishing at the boundary ($+\infty$ and $-\infty$);
then we can use integration by parts to transform the above
integration:
\begin{align}
 \int^{+\infty}_{-\infty}\frac{d L(f(x))}{dx} g(x) dx &= 
 \int^{+\infty}_{-\infty} g(x) dL(f(x)) \nonumber \\
&= g(x)L(f(x))|^{+\infty}_{-\infty} - \int^{+\infty}_{-\infty} L(f(x))
dg(x) \nonumber \\
&= - \int^{+\infty}_{-\infty} L(f(x)) dg(x)
\end{align}
Hence the differentiation on the $L(f(x))$ is dropped. Such
transformation is usually very useful because the
differentiation for $L(f(x))$ is always much more complicated then
the differentiation for $g(x)$. In quantum chemistry, the $g(x)$ is
the approximated wave function, which is the bound state so in the
$\infty$ it becomes zero. 

What's more, if it's not the first order but the $N$ times order of
differentiation on $L(f(x))$:
\begin{equation}
 \int^{+\infty}_{-\infty}\frac{d^{N} L(f(x))}{dx^{N}} g(x) dx 
\end{equation} 
By repeating the above procedure, we can get:
\begin{equation}
  \int^{+\infty}_{-\infty}\frac{d^{N} L(f(x))}{dx^{N}} g(x) dx  =
(-1)^{N} \int^{+\infty}_{-\infty} L(f(x)) d^{N}g(x)
\end{equation} 
Here we have assumed that the $g(x)$ can be differentiated up to $N$
times, and the derivatives for the $g(x)$ are all vanished at the
infinity.

Finally, the result can be extended to the three-dimensional case,
where it takes the form of:
\begin{equation}
 \label{eq:integration_rule_functional}
  \int^{+\infty}_{-\infty}\nabla^{N}L(f(r)) g(r) d^{3}r  =
(-1)^{N} \int^{+\infty}_{-\infty} L(f(r)) \nabla^{N}g(r) d^{3}r
\end{equation} 

%%%%%%%%%%%%%%%%%%%%%%%%%%%%%%%%%%%%%%%%%%%%%%%%%%%%%%%%%%%%%%%%%%%%%%
\subsection{More rigorous explanation for the expansion of functional
derivative}
%
%
%
%
In the above conetent, generally we drive the functional derivative
by expanding them to a series of partial derivatives according to
Talyor expansion. However, why we can do this? In the following
content, we will give a brief description for the reason. This is
come from the rigorous functional derivative shown in the
(\ref{eq:functional:6}). 

Now let's start from the gradient form in (\ref{eq:functional:28}),
but the other more complicated expressions are actually using the
same way:
\begin{equation}
   F[y(r)] = \int L(r, y(r), \nabla y(r)) d^{3}r
\end{equation}

then according to the (\ref{eq:functional:6}), we have:
\begin{equation}
\begin{split}
 \left.\frac{d F[y(r) + \epsilon \delta
y(r)]}{d\epsilon}\right|_{\epsilon=0} &= \dfrac{d}{d\epsilon}\int L(r,
y(r) + \epsilon \delta y(r), \nabla y(r) +
\epsilon\nabla\delta y(r)) d^{3}r \\
&= \int \left\lbrace \dfrac{\partial L}{\partial
y}\delta y + \dfrac{\partial L}{\partial\nabla y}\nabla\delta
y \right\rbrace d^{3}r \\
&=  \int \left\lbrace \dfrac{\partial L}{\partial
y}\delta y + \left[ \nabla \cdot\left( \dfrac{\partial
L}{\partial\nabla y} \delta y\right) \right.\right. \\
&-\left.\left.  \nabla \cdot\left( \dfrac{\partial f}{\partial\nabla
y}\right) \delta y \right]\right\rbrace  d^{3}r \\
&= \int \left\lbrace \dfrac{\partial L}{\partial
y}\delta y - \nabla \cdot\left( \dfrac{\partial
f}{\partial\nabla
y}\right) \delta y \right\rbrace  d^{3}r 
\end{split}
\end{equation}
Now we can see that it's same with the result in the
(\ref{eq:functional:29}). What's more, we have used the integration
by parts in the above derivation, which is same to the derivation
of (\ref{eq:functional:20}).

%%%%%%%%%%%%%%%%%%%%%%%%%%%%%%%%%%%%%%%%%%%%%%%%%%%%%%%%%%%%%%%%%%%%%%
\section{Other properties}
\label{general_functional_other_properties}
%
%
%
%
Here in this section, we are going to gather some common properties
related to the functional. They are very useful in calculation of the
functional derivatives.

The first one we wish to prompt is the chain rule in functional
derivatives. Suggest that we have a functional in expression of
$F[f(x)] = \int L(f(x))dx$, then for $f(x)$; at each point of $x$ it
turns to be some functional of $g(x^{'})$:
\begin{equation}
 f[g] = \int f[g(x^{'})] dx^{'}
\end{equation}

Then we can evaluate the
$\delta f$ as:
\begin{equation}
  \delta f = \int \frac{\delta f}{\delta g(x^{'})} \delta
g(x^{'}) dx^{'} 
\end{equation}

Now let's combine the functional derivative expression for the $f[g]$
and $F[f(x)]$ together, we can get:
\begin{equation}
\begin{split}
\delta F[ f(x)] &= \int \frac{\delta L}{\delta f(x)} \delta f(x) dx
\\ 
&= \int \int \frac{\delta L}{\delta f(x)} \frac{\delta f}{\delta
g(x^{'})} \delta g(x^{'}) dx dx^{'}
\end{split}
\end{equation}
On the other hand, since the variables of $x$ and $x^{'}$ are
only two dummy variables and they are independent with each other, we
can also have:
\begin{equation}
\delta F[ g(x^{'})] = \int \int \frac{\delta L}{\delta f(x)}
\frac{\delta f}{\delta
g(x^{'})} \delta g(x^{'}) dx dx^{'}
\end{equation}
then compared with the definition of the functional derivative, we
can get that:
\begin{equation}
 \label{eq:functional:42}
\frac{\delta F[ g(x^{'})]}{\delta g(x^{'})} = \int \frac{\delta
L}{\delta
f(x)}
\frac{\delta f(x)}{\delta
g(x^{'})} dx
\end{equation} 
This is the chain rule for the functional derivatives, and it can be
easily extended to the three-dimensional case.

By the way, there's something very interesting to see; that if we
derive the chain rule for partial derivative of the functional; what
is its expression? Actually it's same with the partial derivatives of
variables:
\begin{equation}
\label{partial_derivative_functional}
\frac{\partial F[ g(x^{'})]}{\partial g(x^{'})} = \frac{\partial
L}{\partial f(x^{'})}\frac{\partial f(x^{'})}{\partial
g(x^{'})}
\end{equation} 
Compared with (\ref{eq:functional:42}), we can see the difference
between the partial derivative of functional and the derivative of
the functional.

the second important character for the functional derivative, is the
use of delta function. the function itself can be expressed as some
functional: 
\begin{equation}
  \label{eq:functional:43}
  f(x) = \int f(x^{'})\delta(x-x^{'})dx^{'}
\end{equation}
In this sense, we can even calculate the functional derivative with
respect to the function itself by using the equation above:
\begin{align}
  \label{eq:functional:44}
\frac{\delta f(x)}{\delta f(x^{'})} = \frac{\partial
  [f(x^{'})\delta(x-x^{'})]}{\partial f(x^{'})} = \delta(x-x^{'})  
\end{align}
This relation is very important in evaluating the functional such
as $\dfrac{\delta F[f(x)]}{\delta f(x^{'})}$. Mathematically, it
means that as function $f(x)$ make infinitesimal change of
$\delta f(x)$ at the point of $x_{0}$, how can we calculate the
response of $\delta F[f(x)]$ at another point of $x$? In physics we
always meet such problems (for example; in TDDFT). For this case, we
have:
\begin{equation}
  \label{eq:functional:45}
  \frac{\delta F[f(x)]}{\delta f(x^{'})} = \frac{\delta
    F[f(x)]}{\delta f(x^{'})} \delta(x-x^{'})
\end{equation}

Sometimes if the functional $F[f(x)]$ indirectly depends on some
variable of $\lambda$, that is to say, $F[f(x)]$ depends on the
$f(x)$ and $f(x)$ depends on some variable of $\lambda$. This is the
usual case in variational method in quantum mechanics. In section
(\ref{SE:2}) we have described the Ritz variation process in
(\ref{SEeq:7}), there we have the basis functions fixed and the
changing of wave function is depending on the variables of $c_{1},
c_{2}, \cdots$. In practice, in the quantum chemistry we always meeet
such case so it's very important.

If we want to evaluate the functional derivative of for $F[f(x)]$
based on the $\lambda$, according to the chain rule in the
(\ref{eq:functional:42}) we can have:
\begin{equation}
 \label{lambda_variable:1}
  \frac{\partial F[f(x)]}{\partial \lambda} = \int \frac{\delta
F}{\delta f(x)}\frac{\partial f(x)}{\partial\lambda}dx
\end{equation} 
Here we have to mentioned that $\dfrac{\delta F}{\delta f(x)}$ is the
functional derivative for the $F[f(x)]$, and $\frac{\partial
f(x)}{\partial\lambda}$ is the partial derivative for the $f(x)$. We
should care about the difference meaning between the symbol of
$\partial$ and $\delta$.



%%%%%%%%%%%%%%%%%%%%%%%%%%%%%%%%%%%%%%%%
\section{Functional derivatives etc. in DFT}
\label{sec:examples_functional_derivative}
%
%
%
%
%%%%%%%%%%%%%%%%%%%%%%%%%%%%%%%%%%%%%%%
\subsection{Minimization with respect to density VS. density matrix}
\label{sec:minimization_density_vs_density_matrix}
%
%
%
%
Before our further discussion, let's firstly talk about the energy
minimization. In density functional theory, the energy is minimized
with respect to the density, then we can get the optimized density
and all the other physical properties related to the result density.

However, there's something difficult in the evaluation process, that
we have to use the functional derivatives expression. As the energy
functional becomes more complicated, for example; as it is in META
GGA form; to evaluate the functional derivatives through the
minimization of density becomes more and more difficult. Hence how can
we avoid this case?

Fortunately, In quantum chemistry field, things could be easiler to
solve. In this region, the density is actually solely depending on
the density matrix:
\begin{equation}
 \label{minimization_density_vs_density_matrix_eq:1}
\rho^{\sigma} = \sum_{\mu\nu}P_{\mu\nu}^{\sigma}\phi_{\mu}\phi_{\nu} 
\end{equation} 
Here the so called basis set functions, $\phi_{\mu}$ and
$\phi_{\nu}$, are fixed in the whole calculation; only the density
matrix are varied. Hence it's easy to see:
\begin{equation}
  \label{minimization_density_vs_density_matrix_eq:2}
\delta\rho^{\sigma} =
\sum_{\mu\nu}\delta P_{\mu\nu}^{\sigma}\phi_{\mu}\phi_{\nu} 
\end{equation} 
Hence as each $P_{\mu\nu}^{\sigma}$ is minimized, then the density is
minimized. The density is solely determined by some $n\times n$
matrix:
\begin{equation}
 \label{minimization_density_vs_density_matrix_eq:3}
\rho \Leftrightarrow 
\begin{bmatrix}
 P_{11} & P_{12} & \cdots & P_{1n} \\
 P_{21} & P_{22} & \cdots & P_{2n} \\
 \cdots & \cdots & \cdots & \cdots \\
 P_{n1} & P_{n2} & \cdots & P_{nn} \\
\end{bmatrix}
\end{equation}  
As each element is minimized, then the whole density is minimized.
Therefore, the minimization of energy with respect to the density
could be transformed to the minimization of energy with respect to
the density matrix:
\begin{equation}
 \label{minimization_density_vs_density_matrix_eq:4}
\frac{\delta E_{XC}}{\delta \rho} \Leftrightarrow \frac{\partial
E_{XC}}{\partial P_{\mu\nu}} (\mu,\nu = 1,2, \cdots, n)
\end{equation}  
Finally the energy functional of $F(\rho)$ is transformed back to the
common multiple dimension function of $F(P_{\mu\nu})$, then we can
use the partial derivatives to evaluate the function derivatives for
any giving expression! As we can see in the following content, this
is fairly important in the meta-GGA evaluation.

Furthermore, down from this way we can get more similarities for the
functional derivatives; as indicated below:
\begin{equation}
 \label{minimization_density_vs_density_matrix_eq:5}
\begin{split}
 \int d^{3}r\frac{\delta F}{\delta\rho(r)}\rho(r) 
 &\Leftrightarrow
 \sum_{\mu\nu}\frac{\partial E_{XC}}{\partial P_{\mu\nu}}P_{\mu\nu}\\
 \int d^{3}r\int d^{3}r^{'}
 \frac{\delta^{2} F}
      {\delta\rho(r)\delta\rho(r^{'})}
       \rho(r)\rho(r^{'})
&\Leftrightarrow
 \sum_{\lambda\eta}\sum_{\mu\nu}
\frac{\partial^{2} E_{XC}}
     {\partial P_{\mu\nu}\partial
P_{\lambda\eta}}P_{\mu\nu}P_{\lambda\eta} \\
&\cdots\cdots
\end{split}
\end{equation}  
This is some enlightening but not rigorous comparison. Later from
this clue, we will get more results. 

%%%%%%%%%%%%%%%%%%%%%%%%%%%%%%%%%%%%%%%%%%%%%%
\subsection{Functional derivatives for ground state DFT}
%
%
%
%
Consider the general functional expression for the density, it
represents the skeleton of the formulation of GGA in DFT:
\begin{equation}
  \label{eq:functional:34}
  E_{XC} = \int  F \Big(\rho_{\alpha}, \rho_{\beta},
  \gamma_{\alpha\alpha}, \gamma_{\beta\beta},
  \gamma_{\alpha\beta}\Big) d^{3}r
\end{equation}
Where we have:
\begin{align}
  \label{eq:functional:35}
  \gamma_{\alpha\alpha} &= |\nabla\rho_{\alpha}|^{2} =
\nabla\rho_{\alpha}\cdotp\nabla\rho_{\alpha}  \nonumber \\
  \gamma_{\beta\beta}  &= |\nabla\rho_{\beta}|^{2} = 
 \nabla\rho_{\beta}\cdotp\nabla\rho_{\beta}\nonumber \\
  \gamma_{\alpha\beta} &=\nabla\rho_{\alpha} \cdot \nabla\rho_{\beta}
\end{align}

According to the (\ref{eq:functional:33}), we can easily derive their
energy potential of $V_{XC}$ as:
\begin{align}
  \label{eq:functional:36}
  V_{\alpha}^{XC} &= \frac{\delta E_{XC}} {\delta \rho_{\alpha}} =
  \frac{\partial F} {\partial \rho_{\alpha}} - \nabla\cdot \left(
    \frac{\partial F}{\partial (\nabla\rho_{\alpha})}
  \right) \nonumber \\
  &= \frac{\partial F}{\partial \rho_{\alpha}} - \nabla\cdot \left(
    \frac{\partial F}{\partial \gamma_{\alpha\alpha}}\frac{\partial
      \gamma_{\alpha\alpha}} {\partial (\nabla\rho_{\alpha})} \right)-
  \nabla\cdot \left( \frac{\partial F}{\partial
      \gamma_{\alpha\beta}}\frac{\partial \gamma_{\alpha\beta}}
    {\partial (\nabla\rho_{\alpha})} \right)
  \nonumber \\
  &= \frac{\partial F}{\partial \rho_{\alpha}} - 2\nabla\cdot \left(
    \frac{\partial F} {\partial \gamma_{\alpha\alpha}}
    \nabla\rho_{\alpha}\right) - \nabla\cdot \left( \frac{\partial F}
    {\partial \gamma_{\alpha\beta}} \nabla\rho_{\beta} \right)
\end{align}
Here we mention that $\frac{\partial F} {\partial
\gamma_{\alpha\alpha}}$ is a number at given r point
($\gamma_{\alpha\alpha}$ is a number), hence we neither have dot
product nor cross product between this partial derivative and the
$\nabla\rho_{\alpha}$.

The potential for the beta electron density can be derived in the
similar way. Because of the symmetric formation between the
$\rho_{\alpha}$ and $\rho_{\beta}$ in the (\ref{eq:functional:34}), we
can see that we only need to exchange the $\rho_{\alpha}$ and
$\rho_{\beta}$ then we can get the $V_{\beta}^{XC}$. Finally, we
note that this form has been gotten in some
previous papers\cite{CPL_1992_6_557,johnson:5612}.

Now the next question is, how can we express the energy in terms of
the density matrix? In this sense, we can express the whole thing
into real calculation. By the chain rule, we have:
\begin{align}
  \label{eq:XC_functional.9}
F^{\alpha}_{\mu\nu} &= \frac{\partial E^{\alpha}_{XC}}{\partial
P^{\alpha}_{\mu\nu}}
\nonumber \\
&=
\int \frac{\delta F}{\delta
  \rho^{\alpha}(r)}\frac{\partial \rho^{\alpha}(r)}{\partial
P^{\alpha}_{\mu\nu}} d^{3}r
\nonumber \\
&= \int V^{\alpha}_{XC}(r) \phi^{*}_{\mu}(r)\phi_{\nu}(r) d^{3}r
\end{align}

Now let's discuss the formation of electron density. Generally we
consider the unrestricted case, where the alpha electron and beta
electron do not need to share the same spatial orbitals (the
restricted type can be seen as the special case of unrestricted
method). Hence we have:
\begin{align}
  \label{eq:XC_functional.5}
  \rho_{\alpha} &=
  \sum_{j}^{n}\sum_{k}^{n}P^{\alpha}_{jk}\phi_{j}^{*}\phi_{k}
  \nonumber
  \\
  \rho_{\beta} &=
  \sum_{j}^{n}\sum_{k}^{n}P^{\beta}_{jk}\phi_{j}^{*}\phi_{k}
\end{align}

The expression in (\ref{eq:XC_functional.9}) can be also got directly
from the Kohn-Sham equation:
\begin{equation}
\begin{split}
  \hat{H}_{KS}\varphi_{i} = \epsilon_{i}\varphi_{i} &\Rightarrow
(\cdots + \frac{\delta E_{XC}}{\delta \rho})\varphi_{i} =
\epsilon_{i}\varphi_{i} \Rightarrow 
\nonumber \\
\cdots + \int d^{3}r \varphi^{*}_{i}\frac{\delta E_{XC}}{{\delta
\rho}}\varphi_{i} = \epsilon_{i} &\Rightarrow  
\cdots + \sum_{\mu\nu}P_{\mu\nu}\int d^{3}r
\phi^{*}_{\mu}\frac{\delta E_{XC}}{{\delta \rho}}\phi_{\nu}
= \epsilon_{i} \Rightarrow 
\nonumber \\
& \cdots + \sum_{\mu\nu}P_{\mu\nu}F_{\mu\nu} = \epsilon_{i}
\end{split}
\end{equation} 

In evaluating the (\ref{eq:XC_functional.9}), we will meet some
difficulty in calculating the gradient part in
(\ref{eq:functional:36}); however, we have some simple method to
avoid it. Let's take a gradient part as example:
\begin{equation}
  \label{eq:XC_functional.7}
  \begin{split}
    & \int \phi^{*}_{\mu}(r) \Bigg\{2\nabla\cdot \left( \frac{\partial
        f} {\partial \gamma_{\alpha\alpha}}
      \nabla\rho_{\alpha}\right)\Bigg\}\phi_{\nu}(r) d^{3}r  \\
    &=\int \Bigg\{2\nabla\cdot \left( \frac{\partial f} {\partial
        \gamma_{\alpha\alpha}}
      \nabla\rho_{\alpha}\right)\Bigg\}\phi^{*}_{\mu}(r) \phi_{\nu}(r)
    d^{3}r  \\
    &= 2\int \phi^{*}_{\mu}(r) \phi_{\nu}(r)\Bigg\{
\triangle\cdot\left(
      \frac{\partial f} { \gamma_{\alpha\alpha}}
      \nabla\rho_{\alpha}\right) \Bigg\} \\
    &= 2\phi^{*}_{\mu}(r) \phi_{\nu}(r)\left. \left( \frac{\partial f}
        {\partial \gamma_{\alpha\alpha}}
        \nabla\rho_{\alpha}\right)\right|^{+\infty}_{-\infty} - 2\int
    \left( \frac{\partial f} {\partial \gamma_{\alpha\alpha}}     
\nabla\rho_{\alpha}\right)\cdotp\Big\{\triangle(\phi^{*}_{\mu}(r)
\phi_{\nu}(r))\Big\} \\
    &= - 2\int \left( \frac{\partial f} {\partial
        \gamma_{\alpha\alpha}}     
\nabla\rho_{\alpha}\right)\cdotp\Big\{\nabla(\phi^{*}_{\mu}(r)
    \phi_{\nu}(r))\Big\}d^{3}r
  \end{split}
\end{equation}
Here in this derivation, we have used the integration by parts, and
since the wave function should be zero at the $+\infty$ and $-\infty$,
hence it's clear the first term is going zero. Then we get the final
result. The $\triangle$ symbol means the differential in the three
dimensional space, which can be specified as: $\triangle
f(x,y,z) = f^{'}(x)d\vec{x} + f^{'}(y)d\vec{y} + f^{'}(z)d\vec{z}$.
Later we will come back to this derivation for further discussion.

In terms of the transformation in (\ref{eq:XC_functional.7}), we can
finally reach the result:
\begin{equation}
  \label{eq:XC_functional.8}
  \begin{split}
      F^{\alpha}_{\mu\nu} &= \int
  \phi^{*}_{\mu}(r)\hat{V}^{\alpha}_{XC}\phi_{\nu}(r) d^{3}r  \\
  &= \int \frac{\partial f}{\partial
      \rho_{\alpha}}\phi^{*}_{\mu}(r) \phi_{\nu}(r)d^{3}r + \\
  &\int\Bigg\{2\left( \frac{\partial f} {\partial
\gamma_{\alpha\alpha}}
    \nabla\rho_{\alpha}\right) + \left( \frac{\partial f} {\partial
      \gamma_{\alpha\beta}} \nabla\rho_{\beta} \right)
  \Bigg\}\cdot\nabla(\phi^{*}_{\mu}(r) \phi_{\nu}(r))d^{3}r
  \end{split}
\end{equation}
This is what we have got for calculating the ground state Kohn-Sham
equation by the GGA.

On the other hand, the energy for the GGA can be expressed in terms
of the variable form:
\begin{equation}
 E_{XC} = \int F(\rho_{\alpha}, \rho_{\beta}, \nabla\rho_{\alpha},
\nabla\rho_{\beta})d^{3}r
\end{equation} 
So we do not have the $\gamma$ terms.

According to the chain rule of the functional, we have:
\begin{align}\label{TDADDEDeq:2}
F_{\mu\nu} &= \sum_{\sigma}\frac{\partial E_{XC}}{\partial
P^{\sigma}_{\mu\nu}}
\nonumber \\
&=
\sum_{\sigma}\int \frac{\delta F}{\delta
  \rho^{\sigma}(r)}\frac{\partial \rho^{\sigma}(r)}{\partial
P^{\sigma}_{\mu\nu}} d^{3}r
\nonumber \\
&= \sum_{\sigma}\int V^{\sigma}_{XC}
\phi^{*}_{\mu}(r)\phi_{\nu}(r) d^{3}r
\end{align}

the $\hat{V}^{\sigma}_{XC}$ is the energy potential, which has the
form
that:
 \begin{align}
  \label{TDADDEDeq:3}
  V_{\sigma}^{XC} &= \frac{\delta E_{XC}} {\delta \rho_{\sigma}} =
  \frac{\partial F} {\partial \rho_{\sigma}} - \nabla\cdot \left(
    \frac{\partial F}{\partial (\nabla\rho_{\sigma})}
  \right) 
\end{align} 
These results are just coincided with the previous ones.

Combined with (\ref{TDADDEDeq:2}), we can get:
\begin{align}\label{TDADDEDeq:4}
F_{\mu\nu} &= 
\sum_{\sigma}\int \left\lbrace \frac{\partial f} {\partial
\rho_{\sigma}} - \nabla\cdot \left(
    \frac{\partial f}{\partial (\nabla\rho_{\sigma})}
  \right) \right\rbrace 
\phi^{*}_{\mu}(r)\phi_{\nu}(r) d^{3}r \nonumber \\
&=  \sum_{\sigma}\int \frac{\partial f} {\partial
\rho_{\sigma}} \phi^{*}_{\mu}(r)\phi_{\nu}(r) d^{3}r \nonumber \\
&- \sum_{\sigma}\int 
\nabla\cdot \left(
    \frac{\partial f}{\partial (\nabla\rho_{\sigma})}
  \right) 
\phi^{*}_{\mu}(r)\phi_{\nu}(r) d^{3}r 
\end{align}

Since we have the following transformation:
\begin{align}
 &\sum_{\sigma}\int 
\nabla\cdot \left(
    \frac{\partial F}{\partial (\nabla\rho_{\sigma})}
  \right) 
\phi^{*}_{\mu}(r)\phi_{\nu}(r) d^{3}r \nonumber \\
&= -\sum_{\sigma}\int 
 \left(
    \frac{\partial F}{\partial (\nabla\rho_{\sigma})}
  \right)\cdot 
\nabla(\phi^{*}_{\mu}(r)\phi_{\nu}(r)) d^{3}r 
\end{align} 
this can be easily got from the integration by parts method.

Hence we can get the expression in (\ref{TDADDEDeq:2}) as:
\begin{equation}
 F_{\mu\nu} = \sum_{\sigma}\int \frac{\partial F} {\partial
\rho_{\sigma}} \phi^{*}_{\mu}(r)\phi_{\nu}(r) d^{3}r 
+ \sum_{\sigma}\int 
 \left(
    \frac{\partial F}{\partial (\nabla\rho_{\sigma})}
  \right)\cdot 
\nabla(\phi^{*}_{\mu}(r)\phi_{\nu}(r)) d^{3}r 
\end{equation} 
This is for the ground state DFT evaluation, and it's in AO form.
Finally we note that this form is easily transform into the
$\gamma$ form. Let's take the case that $\sigma$ is $\alpha$ as an
example: 
\begin{align}
\label{TDADDEDeq:1} 
\frac{\partial F}{\partial (\nabla\rho_{\alpha})} &=
\frac{\partial F}{\partial\gamma_{\alpha\alpha}}
\frac{\partial\gamma_{\alpha\alpha}}{\partial(\nabla\rho_{\alpha})} 
+ 
\frac{\partial F}{\partial\gamma_{\alpha\beta}}
\frac{\partial\gamma_{\alpha\beta}}{\partial(\nabla\rho_{\alpha})}
\nonumber \\
&= 2\frac{\partial F} {\partial \gamma_{\alpha\alpha}}
   \nabla\rho_{\alpha} + \frac{\partial F}
   {\partial \gamma_{\alpha\beta}} \nabla\rho_{\beta} 
\end{align}
Hence in terms of the $\alpha$ component, we come back to the
expression in (\ref{eq:XC_functional.8}).

Because this method to derive the functional derivatives of DFT is
much more clear, hence in the following conent we will stick to this
form of expression.

%%%%%%%%%%%%%%%%%%%%%%%%%%%%%%%%%%%%%%
\subsection{Extention to the meta-GGA form}
%
%
%
%
%
Now let' extend the discussion to the meta-GGA form, while the energy
expression is expressed as:
\begin{equation}
 \label{functional_mega_gga_eq:1}
E_{XC} = \int  F(\rho_{\alpha}, \rho_{\beta}, \nabla\rho_{\alpha},
\nabla\rho_{\beta}, \nabla^{2}\rho_{\alpha},
\nabla^{2}\rho_{\beta}, \tau_{\alpha}, \tau_{\beta})d^{3}r
\end{equation} 
Here we note that the $ \nabla^{2}\rho_{\alpha}$ and
$\nabla^{2}\rho_{\beta}$ are some number rather than the vector
because of the operator form: $\nabla^{2} = \nabla\cdotp\nabla$. On
the other hand, the $\tau_{\sigma}$ is expressed as:
\begin{equation}
 \begin{split}
  \tau_{\sigma} 
&= \frac{1}{2}\sum_{i=1}^{occ}|\nabla\varphi_{i\sigma}|^{2} \\
&= \frac{1}{2}\sum_{i=1}^{occ}(\nabla\varphi_{i\sigma}\cdotp
\nabla\varphi_{i\sigma}) \\
&= \frac{1}{2}P_{\mu\nu}(\nabla\phi_{\mu\sigma}\cdotp
\nabla\phi_{\nu\sigma})
 \end{split}
\label{functional_mega_gga_eq:2}
\end{equation} 
Here in the above expression, the $\varphi$ express the KS orbitals,
and the $\phi$ is the AO basis set function, so the $P_{\mu\nu}$
corresponds the the density matrix. This term in meta-GGA is termed
as ``kinetic energy density''. 

However, here we meet some problems; that the kinetic energy density
is not directly associated with the density; thus if we just minimize
the energy with respect to the density, then it's difficult to deal
with the kinetic energy density term. However, as what we have stated
in the (\ref{sec:minimization_density_vs_density_matrix}), actually
we can do it in the other way; by seeking he minimization with
respect to the density matrix:
\begin{align}
\label{functional_mega_gga_eq:3}
F_{\mu\nu}^{\sigma}  &= \frac{\partial E_{XC}}{\partial P_{\mu\nu}}
\nonumber \\
  &= \sum_{\xi}\sum_{\sigma}\int \frac{\partial F}{\partial
\xi_{\sigma}} \frac{\partial \xi_{\sigma}}{\partial P_{\mu\nu}}
d^{3}r 
\end{align}
Since in quantum chemistry, the energy finally turns out to be the
function of density matrix, hence each of the variables in
(\ref{functional_mega_gga_eq:1}) is some function for density matrix;
then we can use the chain rule for multivariate function to evaluate
the final expression.

On the other hand, we note that the $\frac{\partial
F}{\partial \xi_{\sigma}}$ is not the functional derivatives for the 
(\ref{functional_mega_gga_eq:1}). Here the functional derivative for
$F$ should be minimized with respect to the density, that is to say:
\begin{equation}
 \label{not_the_functional_derivatives}
V_{XC} = \frac{\delta F(r)}{\delta \rho} \Leftrightarrow F_{pq} = \int V_{XC}
\varphi_{p}(r)\varphi_{q}(r) d^{3} r
\end{equation}
By simple mathematical procedure we can see that if we do not count into the
kinetic energy density, then the functional derivatives and the $\frac{\partial
F}{\partial \xi_{\sigma}}$ are equivalent with each other, but in terms of the
kinetic energy density, then they are not equal to each other anymore.
That's all because the kinetic energy density is not directly related to the
density.

Here we introduce the ``$\xi$'', which is called ``variables'', it
represents each term in the (\ref{functional_mega_gga_eq:1}):
\begin{equation}
\label{functional_mega_gga_eq:4}
 \xi_{\sigma} = \rho_{\sigma}, \nabla\rho_{\sigma},
\nabla^{2}\rho_{\sigma}, \tau_{\sigma}
\end{equation} 
the most important character for the variables is that they are all
linear with the density matrix, that is to say:
\begin{equation}
 \label{functional_mega_gga_eq:5}
\xi_{\sigma} = P_{\mu\nu}f(\phi_{\mu},\phi_{\nu})
\end{equation} 
The $f$ is determined by the variable form, e.g.; for the Laplacian
operator, $f(\phi_{\mu},\phi_{\nu}) =
\nabla^{2}(\phi_{\mu}\phi_{\nu})$, and for the kinetic energy
density; it's $f(\phi_{\mu},\phi_{\nu}) =
\nabla\phi_{\mu}\cdotp\nabla\phi_{\nu}$. However, as what we have
noted; the variable is only determined by density matrix, the basis
set functions are all fixed in variation.

By bringing the potential into the expression, we can finally get the
$F_{\mu\nu}^{\sigma}$ as:
\begin{equation}
\begin{split}
F_{\mu\nu}^{\sigma} &= 
\int \frac{\partial F} {\partial
\rho_{\sigma}} \phi^{*}_{\mu}(r)\phi_{\nu}(r) d^{3}r 
+ 
\int \left(
     \frac{\partial F}{\partial (\nabla\rho_{\sigma})}
     \right)\cdot 
\nabla(\phi^{*}_{\mu}(r)\phi_{\nu}(r)) d^{3}r \\ 
&+
\int
    \frac{\partial F}{\partial (\nabla^{2}\rho_{\sigma})}
\nabla^{2}(\phi^{*}_{\mu}(r)\phi_{\nu}(r)) d^{3}r + \frac{1}{2}
\int
    \frac{\partial F}{\partial \tau_{\sigma}}
(\nabla\phi^{*}_{\mu}(r)\cdotp\nabla\phi_{\nu}(r)) d^{3}r 
\end{split}
\label{functional_mega_gga_eq:6}
\end{equation}    
By further apply the linear relation between the variables and the
density matrix, we can have:
\begin{equation}
 \label{functional_mega_gga_eq:7}
\begin{split}
 E_{XC} 
&=\sum_{\sigma}\sum_{\mu\nu}F_{\mu\nu}^{\sigma}P_{\mu\nu}^{\sigma} \\
&= \sum_{\sigma}\int \frac{\partial F} {\partial
\rho_{\sigma}}\rho_{\sigma}d^{3}r 
+ 
\sum_{\sigma}\int \left(
     \frac{\partial F}{\partial (\nabla\rho_{\sigma})}
     \right)\cdot 
\nabla\rho_{\sigma} d^{3}r \\ 
&+
\sum_{\sigma}\int
    \frac{\partial F}{\partial (\nabla^{2}\rho_{\sigma})}
\nabla^{2}\rho_{\sigma} d^{3}r + \sum_{\sigma}
\int
    \frac{\partial F}{\partial \tau_{\sigma}}\tau_{\sigma}
 d^{3}r \\
&= \sum_{\sigma}\sum_{\xi}\int
    \frac{\partial F}{\partial \xi_{\sigma}}
\xi_{\sigma}d^{3}r
\end{split}
\end{equation} 
Hence we can get some very simplied term.

%%%%%%%%%%%%%%%%%%%%%%%%%%%%%%%%%%%%%%%%%%%%%%%%%%%%%%%%%%%%%%%%%%%%%%
\subsection{Variable expression for TDDFT}
%
%
%
%
%
In TDDFT, the exchange correlation part is defind as:
\begin{equation}
 \label{functional_derivatives_TDDFT_new_variable:1}
\int d^{3}r \int d^{3}r^{'}
\varphi_{p\sigma}^{*}(r)\varphi_{q\sigma}(r)
\frac{\delta^{2} F(r) }{\delta\rho_{\sigma}(r)
\delta\rho_{\tau}(r^{'})}
\varphi_{s\tau}^{*}(r^{'})\varphi_{t\tau}(r^{'})
\end{equation}
Moreover, we know that this expression is equivalent to the
expression below:
\begin{equation}
\label{functional_derivatives_TDDFT_new_variable:2}
\begin{split}
&\sum_{\mu\nu}\sum_{\lambda\eta}
c^{*}_{\mu p,\sigma}c^{*}_{\nu q, \sigma}
c^{*}_{\lambda s, \tau}c^{*}_ {\eta q, \tau} 
\frac{\partial F_{\mu\nu\sigma}}{\partial P_{\lambda\eta\tau}} \\
&=\sum_{\mu\nu}\sum_{\lambda\eta}
c^{*}_{\mu p,\sigma}c^{*}_{\nu q, \sigma}
c^{*}_{\lambda s, \tau}c^{*}_ {\eta q, \tau}
\int d^{3}r \frac{\delta F_{\mu\nu\sigma}}
{\delta\rho_{\tau}(r^{'})}
\frac{\partial\rho_{\tau}(r^{'})}
{\partial P_{\lambda\eta\tau}} \\ 
&=\sum_{\mu\nu}\sum_{\lambda\eta}
c^{*}_{\mu p,\sigma}c^{*}_{\nu q, \sigma}
c^{*}_{\lambda s, \tau}c^{*}_ {\eta q, \tau}
\int d^{3}r^{'} 
\frac{\delta F_{\mu\nu\sigma}}
{\delta\rho_{\tau}(r^{'})}
\phi^{*}_{\lambda\tau}(r^{'})\phi_{\eta\tau}(r^{'})
\\
&=\sum_{\mu\nu}\sum_{\lambda\eta}
c^{*}_{\mu p,\sigma}c^{*}_{\nu q, \sigma}
c^{*}_{\lambda s, \tau}c^{*}_ {\eta q, \tau}
\int d^{3}r\int d^{3}r^{'}
\phi^{*}_{\mu\sigma}(r)\phi_{\nu\sigma}(r)
\frac{\delta V^{\sigma}_{XC}(r)}{\delta\rho_{\tau}(r^{'})}
\phi_{\lambda\tau} (r^{'})\phi_{\eta\tau}(r^{'}) \\
&=\sum_{\mu\nu}\sum_{\lambda\eta}
c^{*}_{\mu p,\sigma}c^{*}_{\nu q, \sigma}
c^{*}_{\lambda s, \tau}c^{*}_ {\eta q, \tau} \\
&\int d^{3}r\int d^{3}r^{'}
\phi^{*}_{\mu\sigma}(r)\phi_{\nu\sigma}(r)
\frac{\delta^{2}
F(r)}{\delta\rho_{\sigma}(r)\delta\rho_{\tau}(r^{'})}
\phi_{\lambda\tau}(r^{'})\phi_{\eta\tau}(r^{'}) \\
&= \int d^{3}r \int d^{3}r^{'}
\varphi_{p\sigma}^{*}(r)\varphi_{q\sigma}(r)
\frac{\delta^{2} F(r) }{\delta\rho_{\sigma}(r)
\delta\rho_{\tau}(r^{'})}
\varphi_{s\tau}^{*}(r^{'})\varphi_{t\tau}(r^{'})
\end{split}
\end{equation} 
Actually this nasty long expression only demonstrates that we can
evaluate the integral back to the AO space, further more; we can use
new variable to calculate the Fock matrix element by avoding the
functional derivatives expression.

Now let's concentrate on the $\dfrac{\partial
F_{\mu\nu\sigma}}{\partial P_{\lambda\eta\tau}}$, this is the
centrail expression we need to express into variable form.
According to the (\ref{functional_mega_gga_eq:6}), this Fock matrix
element can be expressed as:
\begin{equation}
\begin{split}
 \frac{\partial F_{\mu\nu\sigma}}{\partial P_{\lambda\eta\tau}}
&= 
\frac{\partial \sum_{\xi} \int d^{3}r 
\frac{\partial F}{\partial \xi_{\sigma}}
f(\phi_{\mu}(r)\phi_{\nu}(r))}
{\partial P_{\lambda\eta\tau}} \\
&=\sum_{\xi}\sum_{\zeta}\int d^{3} r^{'}\int d^{3}r 
\frac{\partial^{2} F}
{\partial \xi_{\sigma} \partial \zeta_{\tau}}
f(\phi_{\mu}(r)\phi_{\nu}(r))
f(\phi_{\lambda}(r^{'})\phi_{\eta}(r^{'}))
\end{split}
\end{equation}
This is the kernel for the second functional derivatives, and by
further considering the spin, density matrix etc. we can finally get
the exchange correlation part expression in terms of variables.

%%%%%%%%%%%%%%%%%%%%%%%%%%%%%%%%%%%%%%%%%%%%%%%%%%%%%
\subsection{The functional derivative for the gradient of DFT}
\label{Func_Deriv_gradient_variable_GGA}
%
%
%
%
%
Now let's turn to the energy gradient of ground state DFT. Firstly,
the energy expression for the exchange-correlation part is still
expressed as:
\begin{equation}
\label{NUCLEAR_GRADIENT_DFTeq:1}
 E_{XC} = \int F(\rho_{\alpha}, \rho_{\beta}, \nabla\rho_{\alpha},
\nabla\rho_{\beta})d^{3}r
\end{equation} 
So we do not have the $\gamma$ terms.

The variation with respect to the nuclear coordinate of $r_{A}$ is:
\begin{align}
\label{NUCLEAR_GRADIENT_DFTeq:2}
 E_{XC}^{A} &=  \int  F^{A}(\rho_{\alpha}, \rho_{\beta},
\nabla\rho_{\alpha},\nabla\rho_{\beta}) d^{3}r \nonumber \\
  &= \sum_{\sigma}\int  \frac{\delta E_{XC} }{\delta\rho_{\sigma}
}\rho_{\sigma}^{A} d^{3}r \nonumber \\
  &=  \sum_{\sigma}\sum_{\mu\nu}\int 
V^{\sigma}_{XC}(r)P_{\mu\nu}^{\sigma}(\phi_{\mu}\phi_{\nu
})^{A} d^{3}r \nonumber \\
  &=  \sum_{\mu\nu}\sum_{\sigma}\int\left\lbrace \frac{\partial F}
{\partial \rho_{\sigma}} - \nabla\cdot \left(
    \frac{\partial F}{\partial (\nabla\rho_{\sigma})}
  \right) \right\rbrace  P_{\mu\nu}^{\sigma}(\phi_{\mu}\phi_{\nu
})^{A} d^{3}r \nonumber \\
  &= \sum_{\mu\nu} P_{\mu\nu}^{\sigma}\sum_{\sigma}\int\left\lbrace
\left(
\frac{\partial F}{\partial \rho_{\sigma}}(\phi_{\mu}\phi_{\nu
})^{A}\right) + \left(
    \frac{\partial F}{\partial (\nabla\rho_{\sigma})}
    \cdotp\nabla(\phi_{\mu}\phi_{\nu
})^{A}\right)\right\rbrace  d^{3}r
\end{align}

Now let's consider a little more about the $(\phi_{\mu}\phi_{\nu
})^{A}$, this is short for the variation for the $(\phi_{\mu}\phi_{\nu
})$ with respect to the $r_{A}$. Since the basis sets function is
generally taking the form that:
\begin{equation}
 \phi = f(r)e^{-ar^{2}}
\end{equation}
Where the $r$ is $r = r_{e} - r_{I}$, th $r_{I}$ is the nuclear
position, and the $r_{e}$ is the distance between the $r_{I}$ and the
electron position. Hence we can see that if the $r_{I}$ is not
$r_{A}$, then the differentiation for the $\phi$ will be zero, thus
only the basis set functions centering the nuclear A has the gradient
which is not zero. 

On the other hand, since the movement for the nuclear A can be
decomposd into three weights: $r_{Ax}$, $r_{Ay}$ and $r_{Az}$,
the variation for the nuclear A actually is taken like:
\begin{align}
\label{NUCLEAR_GRADIENT_DFTeq:6}
\frac{\partial \phi}{\partial r_{Ax}} &= \frac{\partial
\phi}{\partial r_{x}}\frac{\partial r_{x}}{\partial r_{Ax}}
=-\frac{\partial\phi}{\partial r_{x}}  \nonumber \\
\frac{\partial \phi}{\partial r_{Ay}} &= \frac{\partial
\phi}{\partial r_{y}}\frac{\partial r_{y}}{\partial r_{Ay}}
=-\frac{\partial\phi}{\partial r_{y}}  \nonumber \\\frac{\partial
\phi}{\partial r_{Az}} &= \frac{\partial
\phi}{\partial r_{z}}\frac{\partial r_{z}}{\partial r_{Az}}
=-\frac{\partial\phi}{\partial r_{z}} 
\end{align}
Where the $r_{i}$ ($i=x,y,z$) are the compnents for the the $r$:
$r^{2} = \sum_{i}^{3}r_{i}^{2}$. Furthermore, we note that in the
above process, the derivation has some important point that only
$r_{i}$ is connected to the $r_{Ai}$, hence in the chain rule of
the partial differentiation we only have the $r_{i}$ term.

Finally, we can see that the $\phi^{A}$ is expressed as some vector
has three components, and only the basis set functions centering the
nuclear A has the gradient which is not zero:
\begin{equation}
\label{NUCLEAR_GRADIENT_DFTeq:3}
\phi^{A} =
 \begin{cases}
 -\nabla\phi & r = r^{'} - r_{A} \\
 0           & r = r^{'} - r_{I}, I \neq A
 \end{cases}
\end{equation} 

Now we can apply the (\ref{NUCLEAR_GRADIENT_DFTeq:3}) to the
below expression:
\begin{equation}
  \sum_{\mu\nu}(\phi_{\mu}\phi_{\nu})^{A} = 
  -2\sum_{\mu}^{A}\sum_{\nu}(\nabla\phi_{\mu}\phi_{\nu})
\label{NUCLEAR_GRADIENT_DFTeq:4}
\end{equation} 
We suppose that the gradient is only taken on the $\phi_{\mu}$ who
are centering on the A. Since that the $\mu$ and $\nu$ are
symmetrical in the summation, hence we multiply factor of 2 before
the summation and restrict the nuclear gradient is only applied on
the $\phi_{\mu}$(see the \cite{CPL_1992_6_557,
johnson:5612} etc.).

Similarly, we have:
\begin{equation}
\label{NUCLEAR_GRADIENT_DFTeq:5}
\begin{split}
  \sum_{\mu\nu}\nabla(\phi_{\mu}\phi_{\nu})^{A} &= 
  2\sum_{\mu}^{A}\sum_{\nu}\nabla(\phi_{\mu}\phi_{\nu})^{A} \\ 
  &= 
  2\sum_{\mu}^{A}\sum_{\nu}(\nabla\phi_{\mu}\phi_{\nu} +
\phi_{\mu}\nabla\phi_{\nu})^{A} \\
   &= 
  2\sum_{\mu}^{A}\sum_{\nu}\left\lbrace
(\nabla\phi_{\mu})^{A}\phi_{\nu} +
\phi_{\mu}^{A}\nabla\phi_{\nu}\right\rbrace  \\
&=-2\sum_{\mu}^{A}\sum_{\nu}\left\lbrace
\nabla\cdotp(\nabla\phi_{\mu})^{T}\phi_{\nu} +
\nabla\phi_{\mu}\cdotp\nabla\phi_{\nu}^{T}\right\rbrace
\end{split}
\end{equation}
Where the supperscript of $T$ means ``tranpose'' the vector of
$\nabla\phi$, so that the final result we can see is actually some
number but not the vector. Hence we can label it as $X_{\mu\nu}$ by
following the tradition in the \cite{CPL_1992_6_557,
johnson:5612} etc.) 

Now by applying the (\ref{NUCLEAR_GRADIENT_DFTeq:4}) and
(\ref{NUCLEAR_GRADIENT_DFTeq:5}), we can improve the result in the
(\ref{NUCLEAR_GRADIENT_DFTeq:2}) as:
\begin{equation}
\label{NUCLEAR_GRADIENT_DFTeq:7}
\begin{split}
 E_{XC}^{A} &=  \sum_{\mu\nu}
P_{\mu\nu}^{\sigma}\sum_{\sigma}\int\left\lbrace
\left(
\frac{\partial F}{\partial \rho_{\sigma}}(\phi_{\mu}\phi_{\nu
})^{A}\right) + \left(
    \frac{\partial F}{\partial (\nabla\rho_{\sigma})}
    \cdotp\nabla(\phi_{\mu}\phi_{\nu
})^{A}\right)\right\rbrace  d^{3}r \\
&=  -2\sum_{\mu}^{A}\sum_{\nu}
P_{\mu\nu}^{\sigma}\sum_{\sigma}\int\left\lbrace
\left(
\frac{\partial F}{\partial \rho_{\sigma}}(\nabla\phi_{\mu}\phi_{\nu
})\right) + \left(
    X_{\mu\nu}\frac{\partial F}{\partial (\nabla\rho_{\sigma})}
    \right)\right\rbrace  d^{3}r
\end{split}
\end{equation}

%%%%%%%%%%%%%%%%%%%%%%%%%%%%%%%%%%%%%%%%%%
\subsection{Extension to the meta-GGA form}
\label{Func_Deriv_gradient_variable_META_GGA}
%
%
%
%
%
Now let's extend the gradient form into the meta-GGA by using the
variables, in this case; the energy expressed as:
\begin{equation}
 \label{functional_meta_gga_gradient_eq:1}
E_{XC} = \int  F(\rho_{\alpha}, \rho_{\beta}, \nabla\rho_{\alpha},
\nabla\rho_{\beta}, \nabla^{2}\rho_{\alpha},
\nabla^{2}\rho_{\beta}, \tau_{\alpha}, \tau_{\beta})d^{3}r
\end{equation} 
As suggested by the (\ref{NUCLEAR_GRADIENT_DFTeq:2}), the general
gradient form in terms of the variable should be:
\begin{equation}
 \label{functional_meta_gga_gradient_eq:2}
E_{XC}^{A} = \sum_{\mu\nu}\sum_{\sigma}\sum_{\xi}\int
\frac{\partial F}{\partial \xi_{\sigma}} \xi_{\sigma}^{A} d^{3}r
\end{equation} 
Since we have got the functional derivatives of $\dfrac{\partial
F}{\partial \xi_{\sigma}}$, hence the problem is how to obtain the
gradient form of variables.

For the tau, we have:
\begin{equation}
 \label{functional_meta_gga_gradient_eq:3}
\begin{split}
 \sum_{\mu\nu}(\nabla\phi_{\mu}\cdot\nabla\phi_{\nu})^{A}
&=2
\sum_{\mu}^{A}\sum_{\nu}(\nabla\phi_{\mu})^{A}\cdot(\nabla\phi_{\nu})
\\
&=-2
\sum_{\mu}^{A}\sum_{\nu}\nabla^{2}\phi_{\mu}(\nabla\phi_{\nu})
\end{split}
\end{equation} 
we note that this is some vector variable, we can label it as
$\xi_{tau}^{A}$.

For the laplacian variable, it gives:
\begin{equation}
 \begin{split}
  \sum_{\mu\nu}\nabla^{2}(\phi_{\mu}\phi_{\nu})^{A} &= 
  2\sum_{\mu}^{A}\sum_{\nu}\nabla^{2}(\phi_{\mu}\phi_{\nu})^{A} \\ 
  &= 
  2\sum_{\mu}^{A}\sum_{\nu}\left[
\nabla\cdotp(\nabla\phi_{\mu}\phi_{\nu} +
\phi_{\mu}\nabla\phi_{\nu})\right]^{A} \\
   &= 2\sum_{\mu}^{A}\sum_{\nu}\left[
(\nabla^{2}\phi_{\mu})\phi_{\nu} +
2\nabla\phi_{\mu}\cdotp\nabla\phi_{\nu} +
\phi_{\mu}(\nabla^{2}\phi_{\nu})\right]^{A} \\
&=2\sum_{\mu}^{A}\sum_{\nu}\left[
(\nabla^{2}\phi_{\mu})^{A}\phi_{\nu} +
2(\nabla\phi_{\mu})^{A}\cdotp\nabla\phi_{\nu} +
(\phi_{\mu})^{A}(\nabla^{2}\phi_{\nu})\right] \\
&=-2\sum_{\mu}^{A}\sum_{\nu}\left[
(\nabla^{3}\phi_{\mu})\phi_{\nu} +
2(\nabla^{2}\phi_{\mu})\cdotp\nabla\phi_{\nu} +
(\nabla\phi_{\mu})(\nabla^{2}\phi_{\nu})\right] 
 \end{split}
\label{functional_meta_gga_gradient_eq:4}
\end{equation} 
We note that this is also some vector form variable, since the
$\nabla^{3}$ and $\nabla$ are some vector operator, and the
$\nabla^{2}$ is a scalar operator. We label it as $\xi_{lap}^{A}$.

Finally, the gradient for the meta-GGA form can be expressed as:
\begin{equation}
 \label{functional_meta_gga_gradient_eq:4}
\begin{split}
E_{XC}^{A} &= -2\sum_{\mu}^{A}\sum_{\nu}\sum_{\sigma}
P_{\mu\nu}^{\sigma}\int\left\lbrace
\left(
\frac{\partial F}{\partial \rho_{\sigma}}(\nabla\phi_{\mu}\phi_{\nu
})\right) + \left(
    X_{\mu\nu}\frac{\partial F}{\partial (\nabla\rho_{\sigma})}
\right)
\right. \\
&\left. + 
 \frac{1}{2}\left(
    \xi_{tau}^{A}\frac{\partial F}{\partial \xi_{tau}}\right) +
 \left(
    \xi_{lap}^{A}\frac{\partial F}{\partial \xi_{lap}}\right)
    \right\rbrace  d^{3}r 
\end{split}
\end{equation}   


%%%%%%%%%%%%%%%%%%%%%%%%%%%%%%%%%%%%%%%%%%%%%%%%%%%%%%%%%%%%%%%%%%%%%%
\begin{comment}

%%%%%%%%%%%%%%%%%%%%%%%%%%%%%%%%%%%%%%%%%%%%%%%%%%%%%%%%%%%%%%
\subsection{The functional derivative for the time dependent DFT}
%
%
%
%
%
The XC term in the time dependent density functional theory is
expressed as:
\begin{equation}
 \label{eq:functional:38}
E_{TD} = \int d^{3}r \int d^{3}r^{'}
\varphi_{p}^{*}(r)\varphi_{q}(r)
f_{xc}\varphi_{s}^{*}(r^{'})\varphi_{t}(r^{'})
\end{equation}
Here the $f_{XC}$ is called exchange-correlation kernel, which is
expressed as:
\begin{equation}
 \label{eq:functional:39}
f_{XC} = \frac{\delta^{2} E_{XC} }{\delta\rho(r)
\delta\rho(r^{'})}
\end{equation} 
Hence we need to derive the second functional derivative for the
$E_{XC}$. Additionaly, compared with (\ref{eq:XC_functional.8}), this
term is expressed in the MO space. 

Let's start from the $\dfrac{\delta V_{\sigma}^{XC}(r)}{\delta
\rho_{\sigma^{'}}(r^{'})}$ (now the spin term is generally considered)
and assume that the exchange-correlation energy is generally expressed
as:
\begin{equation}
 E_{XC} = \int F(\rho_{\alpha}, \rho_{\beta}, \nabla\rho_{\alpha},
\nabla\rho_{\beta})d^{3}r
\end{equation} 
It's in GGA's form and express in variables. Furthermore, we can also
express it in the $\gamma$ form by slight changes.

Here the only expression need to evaluate is the second implicit
derivatives for the $E_{XC}$, hence with the result from
(\ref{TDADDEDeq:3}),
we can have:
\begin{equation}
 \begin{split}
\frac{\delta^{2}
E_{XC}}{\delta \rho_{\sigma}(r)\delta \rho_{\sigma^{'}}(r^{'})} 
&= \frac{\partial V_{\sigma}^{XC}(r)} {\partial
\rho_{\sigma^{'}}(r^{'})}
- \nabla_{r^{'}}\cdot \left(
    \frac{\partial V_{\sigma}^{XC}(r)} {\partial
(\nabla_{r^{'}}\rho_{\sigma^{'}}(r^{'}))}
  \right) \\
&= \frac{\partial^{2} E_{XC} }{\partial\rho_{\sigma}(r)
\partial\rho_{\sigma^{'}}(r^{'})} - \nabla_{r}\cdotp\frac{\partial^{2}
E_{XC} }{\partial(\nabla_{r}\cdotp\rho_{\sigma}(r))
\partial\rho_{\sigma^{'}}(r^{'})} \\
&-
\nabla_{r^{'}}\cdotp\frac{\partial^{2} E_{XC}}
{\partial\rho_{\sigma}(r)
\partial(\nabla_{r^{'}}\cdotp\rho_{\sigma^{'}}(r^{'}))} \\
&+ 
\nabla_{r}\nabla_{r^{'}}\cdotp\frac{\partial^{2} E_{XC}}
{\partial(\nabla_{r}\cdotp\rho_{\sigma}(r))
\partial(\nabla_{r^{'}}\cdotp\rho_{\sigma^{'}}(r^{'}))}
 \end{split}
\label{TDADDEDeq:5}
\end{equation} 
In this derivation, we just bring in the expression of exchange
correlation potential then get the result.

Finally, we can use the integration by parts method. Hence the
gradient operator in the (\ref{TDADDEDeq:5}) can be all dropped. We
note that for the $\nabla_{r}\nabla_{r^{'}}$ actually we have to use
twice integration by parts method. Finally we can get:
\begin{equation}
\begin{split}
  &XC_{ia, jb}   \\
&=\sum_{\mu\nu}\sum_{\lambda\eta}c^{*}_{i\mu\sigma}c_{a\nu\sigma} 
c^{*}_{j\lambda\sigma^{'}}c_{b\eta\sigma^{'}}\int d^{3}r \int
d^{3}r^{'} \\
&\phi_{i\mu}^{*}(r)\phi_{a\nu}(r) 
\frac{\partial^{2} E_{XC} }{\partial\rho_{\sigma}(r)
\partial\rho_{\sigma^{'}}(r^{'})}
\phi_{j\lambda}^{*}(r^{'})\phi_{b\eta}(r^{'}) \\
&+ \sum_{\mu\nu}\sum_{\lambda\eta}c^{*}_{i\mu\sigma}c_{a\nu\sigma} 
c^{*}_{j\lambda\sigma^{'}}c_{b\eta\sigma^{'}}
\int d^{3}r \int d^{3}r^{'} \\
&\nabla_{r}\cdotp(\phi_{i\mu}^{*}(r)\phi_{a\nu}(r)) 
\frac{\partial^{2}
E_{XC} }{\partial(\nabla_{r}\cdotp\rho_{\sigma}(r))
\partial\rho_{\sigma^{'}}(r^{'})}
\phi_{j\lambda}^{*}(r^{'})\phi_{b\eta}(r^{'}) \\
&+\sum_{\mu\nu}\sum_{\lambda\eta}c^{*}_{i\mu\sigma}c_{a\nu\sigma} 
c^{*}_{j\lambda\sigma^{'}}c_{b\eta\sigma^{'}}
\int d^{3}r \int d^{3}r^{'} \\
&\phi_{i\mu}^{*}(r)\phi_{a\nu}(r) 
\frac{\partial^{2} E_{XC}}
{\partial\rho_{\sigma}(r)
\partial(\nabla_{r^{'}}\cdotp\rho_{\sigma^{'}}(r^{'}))}
\nabla_{r^{'}}\cdotp\phi_{j\lambda}^{*}(r^{'})\phi_{b\eta}(r^{'}) \\
&+\sum_{\mu\nu}\sum_{\lambda\eta}c^{*}_{i\mu\sigma}c_{a\nu\sigma} 
c^{*}_{j\lambda\sigma^{'}}c_{b\eta\sigma^{'}}
\int d^{3}r \int d^{3}r^{'} \\
&\nabla_{r}\cdotp(\phi_{i\mu}^{*}(r)\phi_{a\nu}(r)) 
\frac{\partial^{2} E_{XC}}
{\partial(\nabla_{r}\cdotp\rho_{\sigma}(r))
\partial(\nabla_{r^{'}}\cdotp\rho_{\sigma^{'}}(r^{'}))}
\nabla_{r^{'}}\cdotp\phi_{j\lambda}^{*}(r^{'})\phi_{b\eta}(r^{'})
\end{split}
\label{TDADDEDeq:6}
\end{equation} 
Now in this final expression we just spread out the MO expression
into the AO expression. By using the (\ref{eq:functional:45}) and
delta function we can finally drop the $r^{'}$ integral and change
$r^{'}$ into $r$, then we get the final expression. 




%%%%%%%%%%%%%%%%%%%%%%%%%%%%%%%%%%%%%%%%%%%%%%%%%%%%%%%%%%%%%%%%%%%%%%


%%% Local Variables:
%%% mode: latex
%%% TeX-master: "../main"
%%% End: 


%
% here below is my first derivation for TDDFT functional expression.
% wrong and not accurate. So abandan it.
%
%



Now let's expand it in the careful way. Firstly, let's try to expand
the $\dfrac{\partial V_{\alpha}^{XC}(r)}
{\partial\rho_{\alpha}(r^{'})}$ by using the expression in
(\ref{eq:functional:36}):
\begin{equation}
  \label{eq:functional:7}
\begin{split}
\frac{\partial V_{\alpha}^{XC}(r)} {\partial
\rho_{\alpha}(r^{'})} &= 
\frac{\partial^{2} f(r)} {\partial \rho_{\alpha}(r)\partial
\rho_{\alpha}(r^{'})} 
- 2\nabla_{r}\cdot 
\left(\frac{ \partial
\left(  \frac{\partial f(r)} {\partial
\gamma_{\alpha\alpha}(r)}\nabla_{r}\rho_{\alpha}(r) \right) } 
    {\partial \rho_{\alpha}(r^{'})}
\right) \\
&- 
\nabla_{r}\cdot 
\left( 
\frac{\partial\left( \frac{\partial f(r)}
    {\partial \gamma_{\alpha\beta}(r)} \nabla_{r}\rho_{\beta}(r)
\right) }
{\partial \rho_{\alpha}(r^{'})}
\right) \\
&= \frac{\partial^{2} f(r)} {\partial \rho_{\alpha}(r)\partial
\rho_{\alpha}(r^{'})}  - 
2\nabla_{r}\cdot 
\left(
\frac{\partial^{2} f(r)}
{\partial\gamma_{\alpha\alpha}(r)\partial\rho_{\alpha}(r^{'})}
\nabla_{r}\rho_{\alpha}(r)\right) \\
&- 
\nabla_{r}\cdot 
\left(
\frac{\partial^{2} f(r)}
{\partial\gamma_{\alpha\beta}(r)\partial\rho_{\alpha}(r^{'})}
\nabla_{r}\rho_{\beta}(r)\right)
\end{split}
\end{equation}
Here in this derivation, we have to remember that if the partial
derivative is applied, the $\nabla\rho$ and the $\rho$ are
independent so if we differentiate with $\rho$, then $\nabla\rho$
should be viewed as some constant.

%Here in this derivation, we have omited some step:
%\begin{align}
% \label{eq:XC_functional.3}
%& 2\nabla_{r}\cdot
%\left(  \frac{\partial f(r)} {\partial
%\gamma_{\alpha\alpha}(r)}\frac{\partial(\nabla_{r}\rho_{\alpha}(r))}{
%\partial \rho_{\alpha}(r^{'})} \right) \nonumber \\
%&= 2\nabla_{r}\cdot
%\left(  \frac{\partial f(r)} {\partial
%\gamma_{\alpha\alpha}(r)}\nabla_{r}\cdot\left(
%\frac{\partial\rho_{\alpha}(r) } {
%\partial \rho_{\alpha}(r^{'})} \right) \right) \nonumber \\
%&= 2\nabla_{r}\cdot
%\left(  \frac{\partial f(r)} {\partial
%\gamma_{\alpha\alpha}(r)}\nabla_{r}\cdot\delta(r-r^{'}) \right)
%\nonumber \\
%&= 0
%\end{align}

Then we expand the second term in the (\ref{eq:functional:40}) again
by using the (\ref{eq:functional:36}):
\begin{equation}
  \label{eq:functional:8}
\begin{split}
&\nabla_{r^{'}}\cdot \left(
\frac{\partial V_{\alpha}^{XC}(r)} {\partial
(\nabla_{r^{'}}\rho_{\alpha}(r^{'}))}  
\right) \\
&= 
\nabla_{r^{'}}\cdot \left( \frac{\partial^{2} f(r)} {\partial
\rho_{\alpha} (r)
\partial(\nabla_{r^{'}}\rho_{\alpha}(r^{'}))}\right) 
- 2\nabla_{r^{'}}\nabla_{r}\cdot 
\left(\frac{ \partial
\left(  \frac{\partial f(r)} {\partial
\gamma_{\alpha\alpha}(r)}\nabla_{r}\rho_{\alpha}(r) \right) } 
    {\partial (\nabla_{r^{'}}\rho_{\alpha}(r^{'}))}
\right) \\ 
&- 
\nabla_{r^{'}}\nabla_{r}\cdot 
\left( 
\frac{\left( \frac{\partial f(r)}
    {\partial \gamma_{\alpha\beta}(r)} \nabla_{r}\rho_{\beta}(r)
\right) }
{\partial (\nabla_{r^{'}}\rho_{\alpha}(r^{'}))}
\right) \\
&= \nabla_{r^{'}}\cdot \left( \frac{\partial^{2} f(r)} {\partial
\rho_{\alpha}(r) \partial(\nabla_{r^{'}}\rho_{\alpha}(r^{'}))}\right) 
- 
2\nabla_{r^{'}}\nabla_{r}\cdot 
\left(
\frac{\partial f(r)}
{\partial\gamma_{\alpha\alpha}(r)}
\delta(r-r^{'})\right) \\
&- 
2\nabla_{r^{'}}\nabla_{r}\cdot 
\left(
\frac{\partial^{2} f(r)}
{\partial\gamma_{\alpha\alpha}(r)\partial(\nabla_{r^{'}}\rho_{\alpha}
(r^{'})) }
\nabla_{r}\rho_{\alpha}(r)\right) \\
&- 
\nabla_{r^{'}}\nabla_{r}\cdot 
\left(
\frac{\partial^{2} f(r)}
{\partial\gamma_{\alpha\beta}(r)\partial(\nabla_{r^{'}}\rho_{\alpha}
(r^{'}))}
\nabla_{r}\rho_{\beta}(r)\right)
\end{split}
\end{equation}
Additionally, we note that here the $\nabla_{r}\rho_{\beta}(r)$ is
actually independent with $\nabla_{r^{'}}\rho_{\alpha}(r^{'})$, since
the alpha electron density and beta electron density are assumed into
different and independent spatial space. For the colse shell case, it
can be viewed as a special case that by forcing all the alpha
electron density equals to the beta electron density, and the formula
generated from the unrestricted type is still useful.


Here we can use some old trick to drop the differentiation with
$\nabla_{r^{'}}\rho_{\alpha}$, which has been used in
(\ref{eq:functional:36}):
\begin{align}
 \label{eq:functional:52}
\frac{\partial L(r)}{\partial(\nabla_{r^{'}}\rho_{\alpha}
(r^{'}))} 
&= \int \frac{\partial
L(r)}{\partial\gamma_{\alpha\alpha}(r^{'})}
\frac{\partial \gamma_{\alpha\alpha}(r^{'})}
{\partial(\nabla_{r^{'}} \rho_{\alpha} (r^{'}))} 
d^{3}r^{'} \nonumber \\
&= \int \frac{\partial
L(r)}{\partial\gamma_{\alpha\alpha}(r^{'})}
2\nabla_{r^{'}} \rho_{\alpha} (r^{'})
d^{3}r^{'} \nonumber \\
&= \frac{\partial
L(r)}{\partial\gamma_{\alpha\alpha}(r^{'})}
2\nabla_{r^{'}} \rho_{\alpha} (r^{'}) \delta(r-r^{'})
\end{align}

By using the result in the (\ref{eq:functional:52}), the
result expression in the (\ref{eq:functional:8}) can be transformed
as:
\begin{equation}
 \begin{split}
  &\nabla_{r^{'}}\cdot \left(
\frac{\partial V_{\alpha}^{XC}(r)} {\partial
(\nabla_{r^{'}}\rho_{\alpha}(r^{'}))}  
\right) \\
&= 2\nabla_{r^{'}}\cdot \left( \frac{\partial^{2} f(r)} {\partial
\rho_{\alpha}(r) \partial\gamma_{\alpha\alpha}(r^{'})}
\nabla_{r^{'}}\rho_{\alpha}(r^{'})\delta(r-r^{'})\right) \\ 
&- 
2\nabla_{r^{'}}\nabla_{r}\cdot 
\left(
\frac{\partial f(r)}
{\partial\gamma_{\alpha\alpha}(r)}
\delta(r-r^{'})\right) \\
&-
4\nabla_{r^{'}}\nabla_{r}\cdot 
\left(
\frac{\partial^{2} f(r)}
{\partial\gamma_{\alpha\alpha}(r)
 \partial\gamma_{\alpha\alpha}(r^{'})}
\nabla_{r^{'}}\rho_{\alpha}(r^{'})
\nabla_{r}\rho_{\alpha}(r) \delta(r-r^{'}) \right) \\
&-
2\nabla_{r^{'}}\nabla_{r}\cdot 
\left(
\frac{\partial^{2} f(r)}
{\partial\gamma_{\alpha\beta}(r)
 \partial\gamma_{\alpha\alpha}(r^{'})}
\nabla_{r^{'}}\rho_{\alpha}(r^{'})
\nabla_{r}\rho_{\beta}(r) \delta(r-r^{'}) \right)
 \end{split}
 \label{eq:functional:53}
\end{equation} 

Now let's begin to deal with the integral form shown in
(\ref{eq:functional:38}). For the (\ref{eq:functional:53}), we can
apply the general method in the (\ref{eq:integration_rule_functional})
so that we can shift the $\nabla$ from the functional to the orbitals:
\begin{align}
  \label{eq:functional:54}
&\int \int f(r) \left[ \nabla_{r^{'}}\nabla_{r}\cdot
L(r,r^{'})\right] g(r^{'}) d^{3}r d^{3}r^{'} \nonumber \\
&= \int f(r)d^{3}r \int \left[ \nabla_{r^{'}}\nabla_{r}\cdot
L(r,r^{'})\right] g(r^{'}) d^{3}r^{'} \nonumber \\
&= \int f(r)d^{3}r \left[ -\int \nabla_{r}\cdot L(r,r^{'})
\nabla_{r^{'}}g(r^{'})
d^{3}r^{'} \right] \nonumber \\
&= -\int \nabla_{r^{'}}\cdot g(r^{'})
d^{3}r^{'} \int \nabla_{r} \cdot L(r,r^{'}) f(r)d^{3}r  \nonumber \\
&= -\int \nabla_{r^{'}}\cdot g(r^{'})
d^{3}r^{'} \left[ -\int  L(r,r^{'}) \nabla_{r}\cdot f(r) d^{3}r\right]
 \nonumber \\
&= \int \int \nabla_{r}\cdot f(r) L(r,r^{'})
\nabla_{r^{'}}\cdot g(r^{'})  
d^{3}r d^{3}r^{'}    
\end{align}
By this result, we can safely transform the integral for
(\ref{eq:functional:53}) as:
\begin{equation}
 \begin{split}
&-\int d^{3}r \int d^{3}r^{'}
\varphi_{p}^{*}(r)\varphi_{q}(r)
\left[ \nabla_{r^{'}}\cdot \left(
\frac{\partial V_{\alpha}^{XC}(r)} {\partial
(\nabla_{r^{'}}\rho_{\alpha}(r^{'}))}  
\right)\right] \varphi_{s}^{*}(r^{'})\varphi_{t}(r^{'}) \\
&=
-2\int d^{3}r \int d^{3}r^{'}
\left( \frac{\partial^{2} f(r)} {\partial
\rho_{\alpha}(r) \partial\gamma_{\alpha\alpha}(r^{'})}
\nabla_{r^{'}}\rho_{\alpha}(r^{'})\delta(r-r^{'})\right) \\
&\left(\varphi_{p}^{*}(r)\varphi_{q}(r)\right)
\nabla_{r^{'}}\cdot\left(
\varphi_{s}^{*}(r^{'})\varphi_{t}(r^{'})\right) \\
&+2
\int d^{3}r \int d^{3}r^{'}
\left(
\frac{\partial f(r)}
{\partial\gamma_{\alpha\alpha}(r)}
\delta(r-r^{'})\right) \\
&\nabla_{r}\cdot
\left(\varphi_{p}^{*}(r)\varphi_{q}(r)\right)
\nabla_{r^{'}}\cdot
\left(\varphi_{s}^{*}(r^{'})\varphi_{t}(r^{'})\right) \\
&+4
\int d^{3}r \int d^{3}r^{'}
\left(
\frac{\partial^{2} f(r)}
{\partial\gamma_{\alpha\alpha}(r)
 \partial\gamma_{\alpha\alpha}(r^{'})}
\nabla_{r^{'}}\rho_{\alpha}(r^{'})
\nabla_{r}\rho_{\alpha}(r) \delta(r-r^{'}) \right) \\
&\nabla_{r}\cdot
\left(\varphi_{p}^{*}(r)\varphi_{q}(r)\right)
\nabla_{r^{'}}\cdot
\left(\varphi_{s}^{*}(r^{'})\varphi_{t}(r^{'})\right) \\
&+2
\int d^{3}r \int d^{3}r^{'}
\left(
\frac{\partial^{2} f(r)}
{\partial\gamma_{\alpha\beta}(r)
 \partial\gamma_{\alpha\alpha}(r^{'})}
\nabla_{r^{'}}\rho_{\alpha}(r^{'})
\nabla_{r}\rho_{\beta}(r) \delta(r-r^{'}) \right) \\
&\nabla_{r}\cdot
\left(\varphi_{p}^{*}(r)\varphi_{q}(r)\right)
\nabla_{r^{'}}\cdot
\left(\varphi_{s}^{*}(r^{'})\varphi_{t}(r^{'})\right)
\end{split}
\label{eq:functional:55}
\end{equation} 

Finally, let's use the delta function so that generally we have:
\begin{equation}
 \label{eq:functional:56}
\int d^{3}r \int d^{3} r^{'} f(r^{'}) g(r) \delta (r-r^{'}) = 
\int d^{3}r f(r) g(r)
\end{equation} 
Hence in the result of (\ref{eq:functional:55}), what we are going to
do in presence of delta function is that to transform all the
functions and variables that related to $r^{'}$ into $r$, and then
drop the integration symbol of $\int d^{3} r^{'}$. After such
procedure, the result in (\ref{eq:functional:55}) finally becomes:
\begin{equation}
 \begin{split}
&-\int d^{3}r \int d^{3}r^{'}
\varphi_{p}^{*}(r)\varphi_{q}(r)
\left[ \nabla_{r^{'}}\cdot \left(
\frac{\partial V_{\alpha}^{XC}(r)} {\partial
(\nabla_{r^{'}}\rho_{\alpha}(r^{'}))}  
\right)\right] \varphi_{s}^{*}(r^{'})\varphi_{t}(r^{'}) \\
&=
-2\int d^{3}r 
\left( \frac{\partial^{2} f} {\partial
\rho_{\alpha} \partial\gamma_{\alpha\alpha}}
\nabla\rho_{\alpha}\right) 
\left(\varphi_{p}^{*}\varphi_{q}\right)
\left[ \nabla\cdot\left(
\varphi_{s}^{*}\varphi_{t}\right)\right]  \\
&+2
\int d^{3}r
\left(
\frac{\partial f}
{\partial\gamma_{\alpha\alpha}}\right)
\left[ \nabla\cdot
\left(\varphi_{p}^{*}\varphi_{q}\right)\right] 
\left[ \nabla\cdot
\left(\varphi_{s}^{*}\varphi_{t}\right)\right] \\
&+4
\int d^{3}r
\left(
\frac{\partial^{2} f}
{\partial\gamma_{\alpha\alpha}^{2}}
\nabla\rho_{\alpha}
\nabla\rho_{\alpha}\right)
\left[ \nabla\cdot
\left(\varphi_{p}^{*}\varphi_{q}\right)\right] 
\left[ \nabla\cdot
\left(\varphi_{s}^{*}\varphi_{t}\right)\right] \\
&+2
\int d^{3}r
\left(
\frac{\partial^{2} f}
{\partial\gamma_{\alpha\beta}
 \partial\gamma_{\alpha\alpha}}
\nabla\rho_{\alpha}
\nabla\rho_{\beta}\right) 
\left[ \nabla\cdot
\left(\varphi_{p}^{*}\varphi_{q}\right)\right] 
\left[ \nabla\cdot
\left(\varphi_{s}^{*}\varphi_{t}\right)\right] 
\end{split}
\label{eq:functional:57}
\end{equation}

Next let's repeat the same procedure to the first item in
(\ref{eq:functional:40}), and we have got some explicit formula to
express it in (\ref{eq:XC_functional.7}). We can just drive the
integration form by the method shown in (\ref{eq:functional:54}):
\begin{equation}
 \begin{split}
&\int d^{3}r \int d^{3}r^{'}
\varphi_{p}^{*}(r)\varphi_{q}(r)
\left[ \frac{\partial V_{\alpha}^{XC}(r)} {\partial
\rho_{\alpha}(r^{'})}\right] 
\varphi_{s}^{*}(r^{'})\varphi_{t}(r^{'}) \\
&=\int d^{3}r \int d^{3}r^{'}
\left( \varphi_{p}^{*}(r)\varphi_{q}(r)\right) 
\frac{\partial^{2} f(r)} {\partial \rho_{\alpha}(r)\partial
\rho_{\alpha}(r^{'})}
\left( \varphi_{s}^{*}(r^{'})\varphi_{t}(r^{'})\right)  \\
&+2\int d^{3}r \int d^{3}r^{'}
\left(
\frac{\partial^{2} f(r)}
{\partial\gamma_{\alpha\alpha}(r)\partial\rho_{\alpha}(r^{'})}
\nabla_{r}\rho_{\alpha}(r)\right)
\left[ \nabla\cdot\varphi_{p}^{*}(r)\varphi_{q}(r)\right] 
\left( \varphi_{s}^{*}(r^{'})\varphi_{t}(r^{'})\right) \\
&+\int d^{3}r \int d^{3}r^{'}
\left(
\frac{\partial^{2} f(r)}
{\partial\gamma_{\alpha\beta}(r)\partial\rho_{\alpha}(r^{'})}
\nabla_{r}\rho_{\beta}(r)\right)
\left[ \nabla\cdot\varphi_{p}^{*}(r)\varphi_{q}(r)\right]
\left( \varphi_{s}^{*}(r^{'})\varphi_{t}(r^{'})\right)
 \end{split}
\label{eq:functional:58}
\end{equation} 
According to the (\ref{eq:functional:45}), so we have:
\begin{equation}
 \label{eq:functional:59}
\frac{\partial^{2} f(r)} {\partial \rho_{\alpha}(r)\partial
\rho_{\alpha}(r^{'})} = \frac{\partial^{2} f(r)} {\partial
\rho_{\alpha}(r)\partial
\rho_{\alpha}(r^{'})}\delta(r-r^{'})
\end{equation} 
So we can drop all the $r^{'}$ then the (\ref{eq:functional:58})
becomes:
\begin{equation}
 \begin{split}
&\int d^{3}r \int d^{3}r^{'}
\varphi_{p}^{*}(r)\varphi_{q}(r)
\left[ \frac{\partial V_{\alpha}^{XC}(r)} {\partial
\rho_{\alpha}(r^{'})}\right] 
\varphi_{s}^{*}(r^{'})\varphi_{t}(r^{'}) \\
&=\int d^{3}r 
\left(
\varphi_{p}^{*}\varphi_{q}\varphi_{s}^{*}\varphi_{t}
\right) 
\frac{\partial^{2} f} 
{\partial^{2}\rho_{\alpha}} \\
&+2\int d^{3}r
\left(
\frac{\partial^{2} f}
{\partial\gamma_{\alpha\alpha}\partial\rho_{\alpha}}
\nabla\rho_{\alpha}\right)
\left[ \nabla\cdot\varphi_{p}^{*}\varphi_{q}\right] 
\left( \varphi_{s}^{*}\varphi_{t}\right) \\
&+\int d^{3}r
\left(
\frac{\partial^{2} f}
{\partial\gamma_{\alpha\beta}\partial\rho_{\alpha}}
\nabla\rho_{\beta}\right)
\left[ \nabla\cdot\varphi_{p}^{*}\varphi_{q}\right]
\left( \varphi_{s}^{*}\varphi_{t}\right)
 \end{split}
\label{eq:functional:60}
\end{equation}





From the (\ref{functional_mega_gga_eq:7}), it turns out that the
functional derivative takes a very simple expression, it equals to
the partial derivatives of the variables:
\begin{equation}
 \label{functional_mega_gga_functional_derivatives_eq:1}
\frac{\delta E_{XC}}{\delta \rho(r)} = \sum_{\sigma}\sum_{\xi}
    \frac{\partial F}{\partial \xi_{\sigma}}
\end{equation} 

Then from this expression we can easily get the higher order of
functional derivatives:
\begin{equation}
 \label{functional_mega_gga_functional_derivatives_eq:2}
\begin{split}
\frac{\delta^{2} E_{XC}}{\delta \rho(r) \delta \rho(r^{'})} &=
\sum_{\sigma^{'}}\sum_{\zeta}\left\lbrace \sum_{\sigma}\sum_{\xi}
   \frac{\partial\left( {\frac{\partial F(r)}{\partial
\xi_{\sigma}(r)}}\right)}{\partial\zeta_{\sigma^{'}}(r^{'})}
\right\rbrace  \\
&= \sum_{\sigma^{'}}\sum_{\zeta}\sum_{\sigma}\sum_{\xi}
   \frac{\partial^{2} F(r)}{\partial
\xi_{\sigma}(r)\partial\zeta_{\sigma^{'}}(r^{'})}
\end{split}
\end{equation}
Here the $\xi$ and $\zeta$ are both representing the variables.

Furthermore, even the third functional derivatives can be derived in
the similar way:
\begin{equation}
 \label{functional_mega_gga_functional_derivatives_eq:3}
\begin{split}
\frac{\delta^{3} E_{XC}}{\delta \rho(r) \delta \rho(r^{'})\delta
\rho(r^{''})} 
&=
\sum_{\sigma^{''}}\sum_{\kappa}
\sum_{\sigma^{'}}\sum_{\zeta}
\sum_{\sigma }\sum_{\xi}
   \frac{\partial^{3} F(r)}{\partial
\xi_{\sigma}(r)\partial\zeta_{\sigma^{'}}(r^{'})\partial\kappa_{
\sigma^{''}(r^{''}) } }
\end{split}
\end{equation}


% First, the expression for the
% $f_{XC}$ is related to two variables: $r$ and $r^{'}$. We can make
% use of (\ref{eq:functional:45}) to drop the additional $r^{'}$:
% \begin{align}
%   \label{eq:functional:41}
% f_{XC} &= \frac{\partial V_{\alpha}^{XC}(r)} {\partial
% \rho_{\alpha}(r^{'})}
% - \nabla\cdot \left(
%     \frac{\partial V_{\alpha}^{XC}(r)} {\partial
% (\nabla\rho_{\alpha}(r^{'}))}
%   \right) \nonumber \\
% &= \frac{\partial V_{\alpha}^{XC}(r)} {\partial
% \rho_{\alpha}(r)} \delta(r-r^{'}) - \nabla\cdot \left(
%     \frac{\partial V_{\alpha}^{XC}(r)} {\partial
% (\nabla\rho_{\alpha}(r))} \delta(r-r^{'}) \right)
% \end{align} 
% Combined with integration form in (\ref{eq:functional:38}), we can
% see that the integration for the $r^{'}$ is droped:
% \begin{equation}
%  \begin{split}
%   \label{eq:functional:46}
% E_{XC} &= \int d^{3}r \int d^{3}r^{'} \varphi_{p}^{*}(r)\varphi_{q}(r)
% \left\lbrace \frac{\partial V_{\alpha}^{XC}(r)} {\partial
% \rho_{\alpha}(r)}
% \delta(r-r^{'})\right\rbrace \varphi_{s}^{*}(r^{'})\varphi_{t}(r^{'})
% \\
% &- \int d^{3}r \int d^{3}r^{'} \varphi_{p}^{*}(r)\varphi_{q}(r)
% \left\lbrace \nabla\cdot \left(
%     \frac{\partial V_{\alpha}^{XC}(r)} {\partial
% (\nabla\rho_{\alpha}(r))}  \right)\delta(r-r^{'})\right\rbrace
% \varphi_{s}^{*}(r^{'})\varphi_{t}(r^{'})
% \\
% &= \int d^{3}r \varphi_{p}^{*}(r)\varphi_{q}(r)
% \left\lbrace \frac{\partial V_{\alpha}^{XC}(r)} {\partial
% \rho_{\alpha}(r)}
% \right\rbrace \varphi_{s}^{*}(r)\varphi_{t}(r) \\
% &- \int d^{3}r \varphi_{p}^{*}(r)\varphi_{q}(r)
% \left\lbrace \nabla\cdot \left(
%     \frac{\partial V_{\alpha}^{XC}(r)} {\partial
% (\nabla\rho_{\alpha}(r))}  \right)\right\rbrace
% \varphi_{s}^{*}(r)\varphi_{t}(r)
%  \end{split} 
% \end{equation} 
% We can get it because of the defintion of delta function. Then the
% $f_{XC}$ is transformed into:
% \begin{equation}
%  \label{eq:functional:47}
% f_{XC} = \frac{\partial V_{\alpha}^{XC}(r)} {\partial
% \rho_{\alpha}(r)} - \nabla\cdot \left(
%     \frac{\partial V_{\alpha}^{XC}(r)} {\partial
% (\nabla\rho_{\alpha}(r))}  \right)
% \end{equation} 
% In the later content, since it's merely functional on the $r$, hence
% we will omit it in the expression.


% Next let's evaluate the second term of $\nabla\cdot \left(
%     \dfrac{\partial V_{\alpha}^{XC}(r)} {\partial
% (\nabla\rho_{\alpha}(r))}  \right)$ in (\ref{eq:functional:47}):
% \begin{equation}
%   \label{eq:functional:49}
% \begin{split}
% \nabla\cdot \left(
% \frac{\partial V_{\alpha}^{XC}(r)} {\partial
% (\nabla\rho_{\alpha}(r))}  
% \right) &= 
% \nabla\cdot \left( \frac{\partial^{2} f} {\partial
% (\nabla\rho_{\alpha})\partial \rho_{\alpha}}\right) 
% - 2\nabla^{2}\cdot 
% \left(\frac{ \partial
% \left(  \frac{\partial f} {\partial
% \gamma_{\alpha\alpha}}\nabla\rho_{\alpha} \right) } 
%     {\partial (\nabla\rho_{\alpha})}
% \right) \\ 
% &- 
% \nabla^{2}\cdot 
% \left( 
% \frac{\left( \frac{\partial f}
%     {\partial \gamma_{\alpha\beta}} \nabla\rho_{\beta} \right) }
% {\partial (\nabla\rho_{\alpha})}
% \right) \\
% &= \nabla\cdot \left( \frac{\partial^{2} f} {\partial
% (\nabla\rho_{\alpha})\partial \rho_{\alpha}}\right)  - 
% 2\nabla^{2}\cdot 
% \left(
% \frac{\partial f}
% {\partial\gamma_{\alpha\alpha}}
% \right) \\
% &- 
% 2\nabla^{2}\cdot 
% \left(
% \frac{\partial^{2} f}
% {\partial\gamma_{\alpha\alpha}\partial(\nabla\rho_{\alpha})}
% \nabla\rho_{\alpha}\right) \\
% &- 
% \nabla^{2}\cdot 
% \left(
% \frac{\partial^{2} f}
% {\partial\gamma_{\alpha\beta}\partial(\nabla\rho_{\alpha})}
% \nabla\rho_{\beta}\right)
% \end{split}
% \end{equation}

% Now we begin to deal with the integration. For the first term in the
% (\ref{eq:functional:46}), according to the derivation in the
% (\ref{eq:integration_rule_functional}), its result is:
% \begin{equation}
%  \begin{split}
% &\int d^{3}r \varphi_{p}^{*}(r)\varphi_{q}(r)
% \left\lbrace \frac{\partial V_{\alpha}^{XC}(r)} 
% {\partial\rho_{\alpha}(r)}
% \right\rbrace 
% \varphi_{s}^{*}(r)\varphi_{t}(r) \\
% &=
% \int d^{3}r \varphi_{p}^{*}\varphi_{q}
% \left\lbrace 
% \frac{\partial^{2} f} {\partial \rho^{2}_{\alpha}}
% \right\rbrace 
% \varphi_{s}^{*}\varphi_{t} + \int \left[ 
% 2
% \left(
% \frac{\partial^{2} f}
% {\partial\gamma_{\alpha\alpha}\partial\rho_{\alpha}}
% \nabla\rho_{\alpha}\right) \right. \\
% &+
% \left. \left(
% \frac{\partial^{2} f}
% {\partial\gamma_{\alpha\beta}\partial\rho_{\alpha}}
% \nabla\rho_{\beta}\right) \right] \nabla
% (\varphi_{p}^{*}\varphi_{q}\varphi_{s}^{*}\varphi_{t})d^{3}r
%  \end{split}
%   \label{eq:functional:50}
% \end{equation} 

% For the second term, we apply the same idea; so the operator
% of $\nabla$ and the Laplacian operator can all be dropped from the
% functional. Hence we can get the result:
% \begin{equation}
%  \begin{split}
%   &\int d^{3}r \varphi_{p}^{*}(r)\varphi_{q}(r)
% \left\lbrace \nabla\cdot \left(
%     \frac{\partial V_{\alpha}^{XC}(r)} 
%          {\partial (\nabla\rho_{\alpha}(r))} \right)
% \right\rbrace
% \varphi_{s}^{*}(r)\varphi_{t}(r) \\
% &=- \int d^{3}r
% \frac{\partial^{2} f} {\partial
% (\nabla\rho_{\alpha})\partial \rho_{\alpha}} 
% \nabla (\varphi_{p}^{*}\varphi_{q}\varphi_{s}^{*}\varphi_{t}) \\
% &+ \int d^{3}r
% \left[ 
% -2\left(
% \frac{\partial f}
% {\partial\gamma_{\alpha\alpha}}
% \right) 
% -2 
% \left(
% \frac{\partial^{2} f}
% {\partial\gamma_{\alpha\alpha}\partial(\nabla\rho_{\alpha})}
% \nabla\rho_{\alpha}\right) \right. \\
% &-\left. 
% \left(
% \frac{\partial^{2} f}
% {\partial\gamma_{\alpha\beta}\partial(\nabla\rho_{\alpha})}
% \nabla\rho_{\beta}\right)
%  \right]
% \nabla^{2}(\varphi_{p}^{*}\varphi_{q}\varphi_{s}^{*}\varphi_{t})
%  \end{split}
% \label{eq:functional:51}
% \end{equation} 

\end{comment}

